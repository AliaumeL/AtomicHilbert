%! TEX program = xelatex
% WARNING: this is a generated file.
%
% Please do not edit this file directly. 
% - If you want to update the medatata of the paper (title, authors, abstract), please
%   edit the `paper-meta.yaml` file in the root of the repository.
% - If you want to update the content of the paper, please edit the latex files
%   in the `src` directory.
% - If you want to update the template itself (e.g., change the layout), please
%   edit the `templates/plain-article.tex` file instead.
\documentclass[11pt,a4paper,twosided]{article}

% we setup a custom geometry because the default one is too narrow
\usepackage{geometry}
\geometry{margin=3.5cm}

% utf-8 for old systems
\usepackage[utf8]{inputenc}
\usepackage[T1]{fontenc}

% babel for language settings
\usepackage[english]{babel}

% microtype for better typography
\usepackage{microtype}

\usepackage{todonotes}
\usepackage{lineno}


% math packages
\usepackage{amsmath,amsthm,amssymb,stmaryrd,thmtools,upgreek}

% configure some theorems
\newtheorem{theorem}{Theorem}
\newtheorem{lemma}[theorem]{Lemma}
\newtheorem{corollary}[theorem]{Corollary}
\newtheorem{proposition}[theorem]{Proposition}
\newtheorem{conjecture}[theorem]{Conjecture}
\theoremstyle{definition}
\newtheorem{definition}[theorem]{Definition}
\newtheorem{remark}[theorem]{Remark}
\newtheorem{example}[theorem]{Example}


% graphics packages
\usepackage{graphicx}
\usepackage[obeyclassoptions,mode=tex]{standalone}
\usepackage{tikz}
\usetikzlibrary{backgrounds}
\usetikzlibrary{shapes.geometric}
\usetikzlibrary{positioning}
\usetikzlibrary{automata}
\usetikzlibrary{tikzmark}
\usetikzlibrary{patterns}
\usetikzlibrary{arrows}
\tikzset{every state/.style={minimum size=1pt}}
\usepackage{tikz-cd}


% links inside the document
\usepackage{hyperref}
\usepackage[capitalise,noabbrev,nameinlink]{cleveref}
\usepackage[composition,hyperref,xcolor,cleveref]{knowledge}
\knowledgeconfigure{notion}

% Tables 
\usepackage{booktabs}
\usepackage{varwidth}

% Packages for macro definitions
\usepackage{xparse}
\usepackage{xpatch}
\usepackage{tokcycle}
\usepackage{ifthen}

% Proofs
\usepackage{bussproofs}

% Colors 
\usepackage{ensps-colorscheme}


% we include whatever the user wants to include in the header

% we include libraries (tex files) usually written in the `lib` directory
% Upgreek letters
\makeatletter
\newcommand\mathgr[1]{\tokcycle
  {\addcytoks{##1}}
  {\processtoks{##1}}
  {\ifcsname up\expandafter\@gobble\string##1\endcsname
   \addcytoks[1]{\csname up\expandafter\@gobble\string##1\endcsname}%
    \else\addcytoks{##1}\fi}
  {\addcytoks{##1}}{#1}%
  \expandafter\mathrm\expandafter{\the\cytoks}%
}
\makeatother


% Create a new macro proofof
% taking as input a label of a theorem
% and creating a proof with a reference to that
% label
\NewDocumentEnvironment{proofof}{ m O{appendix} }{
    % if the command \#1 exists, then 
    % call \#1* to restate the theorem
    \ifcsname #1\endcsname
        \def\isInsideRestatedTheorem{1}
        \csname #1\endcsname*
    \fi
    \begin{proof}[Proof of {\cref{#1}} as stated on page {\pageref{#1}}]
        \phantomsection
        \label{#1:proof}
}{
        % if the optional argument is "appendix" 
        % then printout a "backlink"
        % and otherwise do nothing
        \ifthenelse{\equal{#2}{appendix}}{
        % Some link to go back to the theorem
        \marginpar{\vspace{-2em}\texttt{\small{\hyperref[#1]{$\triangleright$ Back to p.\pageref{#1}}}}}
        }{}
    \end{proof}
}

% Create a new macro proofref
% that takes as input a label of a theorem
% and creates a reference to its proof
\NewDocumentCommand{\proofref}{ m }{
    % checks if the label #1:proof exists, if yes
    % it creates a link to it, otherwise it writes nothing
    \IfRefUndefinedExpandable{#1:proof}{}{
        % Checks if we are inside a restated theorem
        % if yes, we do not print anything
        \ifdefined\isInsideRestatedTheorem
        \else
            \marginpar{\vspace{0.6em}\texttt{\small{\hyperref[#1:proof]{$\triangleright$ Proven p.\pageref{#1:proof}}}}}
        \fi
    }
}



\newcommand{\circled}[2]{%
\hypertarget{#2}{}%
\tikz[baseline=(char.base),color=A2,thick]{%
\node[shape=circle,draw,inner sep=1pt,font=\tiny] (char) {#1};%
}}
\newcommand{\circleref}[2]{%
\hyperlink{#2}{%
\tikz[baseline=(char.base),color=A2,thick]{%
\node[shape=circle,draw,inner sep=1pt,font=\tiny] (char) {#1};}%
}}

% Little math macros
\NewDocumentCommand{\set}{ m }{\{ #1 \}}
\NewDocumentCommand{\setof}{ m m }{\{ #1 \mid #2 \}}
\NewDocumentCommand{\card}{ m }{\left| #1 \right|}
\NewDocumentCommand{\seqof}{ m O{n \in \N} }{\left( #1 \right)_{#2}}

\NewDocumentCommand{\defined}{ }{\triangleq}
\newcommand{\defiff}{\overset{\mathrm{def}}{\iff}}
\newcommand{\defeq}{\overset{\mathrm{def}}{=}}

\newcommand{\subfin}{\subset_{\text{fin}}}

\NewDocumentCommand{\range}{ O{1} m }{[#1, #2]}

% functions of all sorts (injective, partial, surjective)
\newcommand{\topartial}{\rightharpoonup}
\newcommand{\toinj}{\hookrightarrow}
\newcommand{\tosurj}{\twoheadrightarrow}
\newcommand{\tobij}{\stackrel{\simeq}{\longrightarrow}}


% Automate the creation of new orderings
% based on a given symbol.
% For instance,
% \NewDocumentOrdering{\pref}{\preceq}{\prec}
% will create the following commands:
% \prefleq and \preflt
% that will respectively expand to
% \mathrel{\kl[\pref]{\preceq}} and \mathrel{\kl[\pref]{\prec}}
\NewDocumentCommand{\NewDocumentOrdering}{ m m m }{
    \expandafter\newcommand\csname #1leq\endcsname{
        \mathrel{\kl[#1]{#2}}
    }
    \expandafter\newcommand\csname #1lt\endcsname{
        \mathrel{\kl[#1]{#3}}
    }
    \knowledge{#1}{notion}
}

% Order macros
\NewDocumentCommand{\upset}{ O{} m }{{\uparrow_{#1} #2}}
\NewDocumentCommand{\dwset}{ O{} m }{{\downarrow_{#1} #2}}


% Number theory
\NewDocumentCommand{\factorial}{ O{} m }{
    \if\relax\detokenize{#1}\relax
        #2!
    \else
        (#2)!
    \fi
}

\newcommand{\A}{\mathcal{A}}
\newcommand{\R}{\mathbb{R}}
\newcommand{\C}{\mathbb{C}}
\newcommand{\F}{\mathcal{F}}
\newcommand{\Q}{\mathbb{Q}}
\newcommand{\N}{\mathbb{N}}
\newcommand{\K}{\mathbb{K}}
\newcommand{\X}{\mathcal{X}}
\newcommand{\Y}{\mathcal{Y}}

% group actions
\newcommand{\actson}{\curvearrowright}

% orders 
\NewDocumentCommand{\divleq}{}{
    \mathrel{\sqsubseteq^{\mathrm{div}}}
}
\NewDocumentCommand{\gdivleq}{ O{\group} }{
    \mathrel{\sqsubseteq^{\mathrm{div}}_{#1}}
}


\NewDocumentCommand{\Basis}{O{B}}{\mathcal{#1}}
\NewDocumentCommand{\LBasis}{O{B} m}{\mathcal{#1}_{#2}}

\newcommand{\poly}[2]{#1[#2]}
\newcommand{\aut}[2][]{\mathsf{Aut}_{#1}{(#2)}}
\newcommand{\mon}[2][]{\mathsf{Mon}_{#1}(#2)}
\newcommand{\perm}[1]{\mathsf{Perm}(#1)}
\newcommand{\otu}[2]{#1^{(#2)}}
\newcommand{\group}{\mathcal{G}}
\newcommand{\gen}[2]{\langle #1\rangle_{#2}}
\newcommand{\radoG}{\mathbb{G}_{\mathsf{Rado}}}
\newcommand{\radoV}{\mathbb{V}_{\mathsf{Rado}}}
\newcommand{\radoE}{\mathbb{E}_{\mathsf{Rado}}}
\newcommand{\cycleSet}[1][]{\mathsf{Cycles}_{#1}}

\newcommand{\ordinal}{\eta}


\newcommand{\hbp}{\text{Hilbert's basis property}}
\newcommand{\orbit}[2][]{\mathsf{orbit}_{#1}{(#2)}}

\newcommand{\order}[1][]{\prec_{#1}}
\newcommand{\ordereq}[1][]{\preceq_{#1}}

\NewDocumentCommand{\sOrderLt}{O{S}}{\prec_{#1}}
\NewDocumentCommand{\sOrderLeq}{O{S}}{\preceq_{#1}}

\newcommand{\revlex}[1][]{<_{\mathsf{RevLex}}^{#1}}
\newcommand{\revlexeq}[1][]{\leq_{\mathsf{RevLex}}^{#1}}
\newcommand{\gr}{Gr\"obner}
\newcommand{\dom}{\mathsf{dom}}
\newcommand{\spoly}[2]{\mathsf{D}(#1,#2)}
\newcommand{\spolyset}{\mathsf{DSet}}
\newcommand{\spolytext}{$\mathsf{S}$-polynomial}
\newcommand{\lcm}{\mathsf{LCM}}
\newcommand{\lc}[1][]{\mathsf{LC}_{#1}}
\newcommand{\lt}[1][]{\mathsf{LT}_{#1}}
\newcommand{\reducstep}[1]{\to_{#1}}
\newcommand{\reduc}[1]{\to^*_{#1}}
\newcommand{\rem}[2]{\mathsf{Rem}_{#1}(#2)}
\newcommand{\closure}[1]{\widehat{#1}}
\newcommand{\wforder}{\triangleleft}

\newcommand{\lm}[1][]{\mathop{\mathsf{LM}_{#1}}}
\newcommand{\cm}[1][]{\mathop{\mathsf{CM}_{#1}}}

\newcommand{\probBasic}[4]
{
\begin{flalign*}
\quad
\begin{tabular}{l  l}
  \multicolumn{2}{l}{\mathsf{#1}}\\
  \textbf{Input:}    & #2 \\
  \textbf{#4} & #3
\end{tabular}
&&
\end{flalign*}
}

\newcommand{\prob}[3]
{
\probBasic{#1}{#2}{#3}{Question:}
}

\input{lib/knowledges.kl}

% We include the title and author information based on the 
% `paper-meta.yaml` file.
 
\title{Deciding the Equivariant Ideal Membership Problem}

\author{
Aliaume Lopez\thanks{University of Warsaw, Poland}
 \and
Arka Ghosh\thanks{Université de Bordeaux, France}
}

% For the date, we first check if the user has provided a date,
% and otherwise use the git meta inforamtion (if available).
\date{2025-01-23 17:04:08 +0100\footnote{f8bc65c3fda723c77c6db7fb510eb4b128bffb90 -- branch main at git@github.com:AliaumeL/AtomicHilbert.git}}

\newcommand{\repositoryUrl}{\url{https://github.com/AliaumeL/AtomicHilbert}}


% Now, we create the document itself.
\begin{document}
% Generate the title page
\maketitle
% Print the abstract
\begin{abstract}
    Polynomials with infinite number of variables do not have the Hilbert basis property. However, if the variables are considered up to the action of a group, one can focus on the equivariant ideals, for which it has recently been proven that they do have the Hilbert basis property, under mild assumptions on the group action. We extend this result by showing that the ideal membership problem (that is, given a polynomial, does it belong to a given equivariant ideal) is decidable under similar assumptions. This is significant because it paves the way towards decision procedures based on the Hilbert method in the presence of infinite data sturctures with symmetries.
\end{abstract}

% Include the content of the paper
%!TEX root = ../atomic.ieee.tex
% LTeX: language=en
\section{Even newer introduction}
%
Consider the system $P$ of polynomial equations
\[
x^2 + yz = 2\ , \quad
x\neq y\neq z\in\Indets
\]
over a set of variables $\Indets$.
If $\Indets$ is finite then weak nullstellensatz \cite[\S 1 Theorem 1]{Cox2015Dict} says that $P$ has a complex solution if and only if the ideal $\idl[P]$ generated by the set of polynomials
\[
P' = \setof{x^2 + yz - 2}{x\neq y\neq z\in\Indets}
\]
does not contain the constant polynomial $1$.
The ideal $\idl[P]$ contains $1$ if and only if any(each) of its \Grbs{} \cite[\S 5 Definition 1]{Cox2015Grb} contains $1$.  
So to check if $P$ has a solution we need to compute a \Grb{} of $\idl[P]$.
This poses no problem since by Hilbert's basis theorem every ideal has a \Grb{} \cite[\S 5 Corollary 6]{Cox2015Grb},
and given a finite set of polynomials $B$ one can use the Buchberger's algorithm  to find a \Grb{} of the ideal generated by $B$ \cite[\S 7 Theorem 2]{Cox2015Grb}.

\Grbs{} are also useful for other computational purposes as well,
the most important of which is the decidability of the ideal membership problem:
given a polynomial $f$ and a finite set of polynomials $B$,
is $f$ in the ideal generated by $B$.

If $\Indets$ is countably infinite, one can still use the version of weak nullstellensatz give in \cite[Page 1]{LangNull}.
However, ideals in $\poly{\K}{\Indets}$ can have arbitrary behaviours which makes difficult to extend the notion of \Grbs{} in an useful way.
For example,
if $\Indets$ is the set of configurations of a deterministic Turing machine,
then the ideal generated by
\[
\setof{x - y}{x\to y}
\]
contains a polynomial $u - v$ if and only if $u$ is reachable by $v$.
Thus one must restrict to a specific class of ideals that are better behaved.
The class we are interested in this article is the class of ideals that are \kl{equivariant}, i.e.\ invariant under certain permutations of the variables.
For example, consider the ideal $\idl[P]$ we defined above,
that is generated by the sets
\[
P' = \setof{x^2 + yz - 2}{x\neq y\neq z\in\Indets} \ .
\]
For any $x\neq y\neq z\in\Indets$, a permutation $\pi$ of $\Indets$ takes the polynomial $x^2 + yz - 2$ to the polynomial $\pi(x)^2 + \pi(y)\pi(z) - 2$.
Let $\perm{\Indets}$ denote the set of all permutations of $\Indets$.
The set $P'$ is equivariant w.r.t. $\perm{\Indets}$, i.e.invariant under the permutations of $\Indets$.
Thus $\idl[P]$ is also equivariant (w.r.t. $\perm{\Indets}$).
With some difficulty we can prove:
%
\begin{lemma}
The ideal $\idl[P]$ is not finitely generated.
\end{lemma}
%
However, once we allow application of permutations while generating the ideal,
$\idl[P]$ is generated by any(every) polynomial.
Stated differently, for any $x\neq y\neq z$, $\idl[P]$ is the smallest ideal containing the polynomial $x^2 + yz - 2$.
Thus $\idl[P]$ is finitely generated as an equivariant ideal.
In fact \cite[Proposition 2]{COHEN67} shows that every equivariant ideal in $\poly{\K}{\Indets}$ is finitely generated.
This line of research continues in \cite{Emmott87} which extends the notions of a \Grbs{} and Buchberger's algorithm to this setting.
The above-mentioned results were rediscovered in \cite{AH07,AH08,HKL18}.
In \cite{HS12} these results were used to prove the Independent Set Conjecture in
algebraic statistics.
In \cite{HS12}, the authors also showed that one can even take a submonoid $\calM$ of $\inc{<}$ and prove existence and computability of finite Gr\"{o}bner basis assuming that $\gdivleq[\calM]$ is a well-partial-order.
These results were independently generalised in \cite{GHOLAS24} and \cite{LauSno23} which study ideals that are equivariant w.r.t. a subgroup $\group$ of permutations of $\Indets$.
In particular \cite[Theorem 11 and 12, Lemma 13]{GHOLAS24} and \cite[Proposition 3.2, Proposition 3.3]{LauSno23} give a necessary and a sufficient condition on the group $\group$ for the \kl{Equivariant Hilbert basis property}:
ideals that are equivariant w.r.t. $\group$ are finitely generated. 
The necessary and sufficient conditions are equivalent up to a well-known conjecture by Pouzet \cite[Problems 12]{POUZ24}.

\arka{write using homogeneous languages.
Can specify contributions more clearly. And also specify how much it differs from before.
Add a comment that every oligomorphic structure is homogeneous with a (possibly infinite) relational language}

This results imply the Noetherian property of vector subspaces that are equivariant with respect to the action of a group $\group$ satisfying the sufficient conditions (\cite[Page 21]{BFKM24} and \cite[Theorem 27]{GHOLAS24}),
and also the decidability of the zero-ness problem of weighted register automata
(\cite[Theorem 30 and Remark 33]{GHOLAS24}).
%
\subsection{Contribution}
%
In this paper, we bridge the gap between the
theoretical understanding of the \intro{equivariant Hilbert basis property} \cite[Property 4]{GHOLAS24}, and the computational aspects of \kl{equivariant
ideals}, by showing that a mild strengthening of the  sufficient condition for the \kl{Equivariant Hilbert basis property} guarantees computability of \kl{equivariant Gröbner bases} of an \kl{equivariant ideal} (\Cref{thm:compute-egb}).
We also show that


This result along also implies we can solve equations,
but does not tell us how the solution looks

we rectify the situation by 

this implies wqo implies extremely amenable

%
\section{Introduction}
\label{sec:intro}

\AP For a field $\K$ and a non-empty set $\Indets$ of indeterminates, we use
$\poly{\K}{\Indets}$ to denote the ring of polynomials with coefficients from $\K$
and indeterminates/variables from $\Indets$. A fundamental result in commutative
algebra is \intro{Hilbert's basis theorem}, stating that when $\Indets$ is finite,
every ideal in $\poly{\K}{\Indets}$ is finitely generated \cite{HILB1890}, where an
\kl{ideal} is a non-empty subset of $\poly{\K}{\Indets}$ that is closed under
addition and multiplication by elements of $\poly{\K}{\Indets}$. This property can
be rephrased as the fact that the set of polynomials $\poly{\K}{\Indets}$ is
\intro{Noetherian}. \kl{Hilbert's basis theorem} extends to the case where $\K$
is a ring that is itself \kl{Noetherian} \cite[Theorem 4.1]{Lang02}.

\AP A \Grb\ is a specific kind of generating set of a polynomial ideal
which allows easy checking of membership of a given polynomial in that ideal.
\kl{Gr\"{o}bner bases} were introduced by Buchberger who showed when $\Indets$ is
finite, every ideal in $\poly{\K}{\Indets}$ has a finite \kl{Gr\"{o}bner basis} and
that, for a given a set of polynomials in $\poly{\K}{\Indets}$, one can compute a
finite \kl{Gröbner basis} of the ideal generated by them via the so-called
\intro{Buchberger algorithm} \cite{BUCH76}. The
existence and computability of \Grbs\ implies the decidability of the
\kl{ideal membership problem}: given a polynomial $f$ and set of polynomial
$H$, decide whether $f$ is in the ideal generated by $H$. The theory of
\kl{Gr\"{o}bner bases} has applications in very diverse areas of computer
science, including integer programming \cite{Sturmfels96}, algebraic proof
systems \cite{algProof}, geometric reasoning \cite{Cox2015chGeom}, fixed
parameter tractability \cite{ACDM22}, program analysis \cite{SSM04} and
constraint satisfaction problems \cite{Mas21}.
In automata theory it has been used for deciding zeroness of polynomial
automata \cite{BEDUSHWO17}, reachability in symmetric Petri nets \cite{MAME82},
equivalence for string-to-string transducers \cite{HONKALA00} and equivalence
of polynomial differential equations \cite{CLEMENTE24}. 

\AP There has been a growing interest in the last few years for computational
models that are manipulating infinite data structures in a finite way, for
instance an automaton reading words on the infinite alphabet $\N$, while
maintaining a finite number of states. While this idea can be traced back to
the 90s with the notion of register automata \cite{KAFR94}, it has been revived
in with the development of the theory of \emph{orbit finite sets}. In this
setting, one would like to consider an infinite set of variables $\Indets$. As an
example, let us consider the set $\Indets$ of variables $x_i$ for $i \in \N$, and
the \kl{ideal} $\idlZ$ generated by the set $\setof{x_i}{i \in \N}$. It is
clear that $\idlZ$ is not finitely generated, and we conclude that the
\kl{Hilbert's basis theorem} (and a fortiori, the \kl{Gr\"{o}bner basis}
theory) does not extend to the case of infinite sets of indeterminates.

\AP However, in the applications mentionned above, the infinite set of
variables (data) comes with an extra structure: the behaviour of the considered
systems are invariant under the action of a group $\group$ on $\Indets$. The action
of this $\group$ on $\Indets$ naturally induces an action on $\poly{\K}{\Indets}$, by
renaming the variables. The typical example is the group of all permutations of
$\Indets$, which corresponds to seeing $\Indets$ as a set of \emph{indistinguishable}
names: one is not interested in the ideal $\idlZ$ generated by the set
$\setof{x_i}{i \in \N}$, but rather in the \kl{equivariant ideal} generated by
the set $\setof{x_i}{i \in \N}$, which is the smallest ideal that contains it
and is invariant under the action of $\group$. In this case, this ideal is
finitely generated by a single indeterminate, e.g. $x_1$. Please note that
equivariance does not imply finite generation in general: for instance, the
ideal $\idlZ$ is not finitely generated as an equivariant ideal with respect to
the trivial group.


\subsection{Related Research}
The above-mentioned results were rediscovered in \cite{AH07,AH08,HKL18}. In
\cite{HS12} these results were used to prove the Independent Set Conjecture in
algebraic statistics. In \cite{HS12}, the authors also showed that one can even
take a submonoid $\calM$ of $\inc{<}$ and prove existence and computability of
finite Gr\"{o}bner basis assuming that $\gdivleq[\calM]$ is a
well-partial-order. These results were significantly generalised in
\cite{GHOLAS24}, which gives a necessary and a sufficient condition on the
actions $\group\actson\Indets$ for the \kl{Equivariant Hilbert basis property}
to hold \cite[Theorems 11 and 12, Lemma 13]{GHOLAS24}. The necessary and
sufficient conditions are equivalent up to a well-known conjecture by Pouzet
\cite[Problems 12]{POUZ24}. But to obtain decision procedures, one still lacks
a generalisation of \kl{Buchberger's algorithm} to the equivariant case, except
under artificial extra assumptions \cite[Section 6]{GHOLAS24}. Overall, a
general understanding of the decidability of the \kl{equivariant ideal
membership problem} is still missing, and \emph{a fortiori}, a generalisation
of \kl{Buchberger's algorithm} to the equivariant case is still an open
problem.

Our results are part of a larger research direction that aims at establishing
an algorithmic theory of computation with orbit-finite sets. For instance,
\cite{BFKM24} studies equivariant subspaces of vector spaces generated by
orbit-finite sets, \cite{GHL22,GHL25} study solvability of orbit-finite systems
of linear equations and inequalities, and \cite{BFKM24,GHL22,Prz23} study duals
of vector spaces generated by orbit-finite sets.

\subsection{Contributions.}
\AP In this paper, we bridge the gap between the
theoretical understanding of the \intro{equivariant Hilbert basis property}
 \cite[Property 4]{GHOLAS24}, and the computational aspects of \kl{equivariant
ideals}, by showing that under mild assumptions on the group action, one can
compute an \kl{equivariant Gröbner basis} of an \kl{equivariant ideal}, hence,
that one can decide the \kl{equivariant ideal membership problem}. In order to
compute such sets, we will need to introduce some classical \kl{computability
assumptions} on the group action $\group \actson \Indets$, and on the set of
indeterminates $\Indets$. These will be defined in
\cref{sec:preliminaries}, but informally, we assume
that one can compute representatives of the orbits of elements under the action
of $\group$ (this is called \kl{effective oligomorphism}), and that one has
access to a total ordering on $\Indets$ that is computable, and
\kl(ord){compatible} with the action of $\group$. Please note that the ordering
on $\Indets$ is not required to be well-founded, and a typical example of our
computable assumptions would be the set $\Q$ of rationals, equipped with the
natural ordering $\leq$ and the group $\group$ would be the group of all
monotone bijections from $\Q$ to itself.

\AP Let us now focus on the mild semantic assumption that we will need to make
on the set of indeterminates $\Indets$ and the group $\group$, that will
guarantee the termination of our procedures. We refer to our preliminaries
(\cref{sec:preliminaries}) for a more detailed
discussion on these assumptions, but again informally, we ask that the set of
\kl{monomials} $\mon{\Indets}$ is well-behaved with respect to divisibility up
to the action of $\group$, which we write as the fact that $(\mon{\Indets},
\gdivleq)$ is a \kl{well-quasi-ordering} (\kl{WQO}). It is known from that this
is a necessary condition for the \kl{equivariant Hilbert basis property}
\cref{thm:equiv-hilbert-property}, and we will rely on a slightly stronger
condition, namely that $(\mon[Y]{\Indets}, \gdivleq)$ is a \kl{WQO}, whenever
$(Y, \leq)$ is one, which is conjectured to be equivalent to the first
condition. Beware that \cref{thm:equiv-hilbert-property,thm:compute-egb}
are
incomparable: the former does not talk about decidability, while the latter 
only considers \kl{equivariant ideals} that are already finitely presented, and we 
will show in
\cref{ex:non-wqo-undecidable} an example where \kl{equivariant
Gröbner bases} are computable, but the \kl{Hilbert basis property} fails.

\begin{theorem}[name={\cite[Theorem 11 and 12]{GHOLAS24}}]
  \label{thm:equiv-hilbert-property}
  Let $\Indets$ be a totally ordered set of indeterminates
  equipped with a group action $\group \actson \Indets$ that is 
  \kl(ord){compatible} with the ordering on $\Indets$.
  Then, $(\mon[\om]{\Indets}, \gdivleq)$ is a \kl{WQO}, if and only if 
  the \kl{equivariant Hilbert basis property} holds for $\poly{\K}{\Indets}$.
\end{theorem}

\begin{theorem}[name={Equivariant Gröbner Basis},restate=thm:compute-equiv-gb]
  \label{thm:compute-egb}
  Let $\Indets$ be a totally ordered set of indeterminates
  equipped with a group action $\group \actson \Indets$, under our \kl{computability assumptions}.
  If $(\mon[Y]{\Indets}, \gdivleq)$ is a \kl{WQO} for every 
  \kl{well-quasi-ordered} set $(Y,\leq)$, then one can
  compute an \kl{equivariant Gröbner bases} of \kl{equivariant ideals}.
\end{theorem}

\AP To prove our \cref{thm:compute-egb}, we will first introduce a weaker
notion of \kl{weak equivariant Gröbner basis}, which characterises the results
obtained by naïvely adapting \kl{Buchberger's algorithm} to the equivariant
case. Then, we will show that under our \kl{computability assumptions}, one can
start from a finite set of generators $H$ of an \kl{equivariant ideal}, and
compute a well-chosen \kl{weak equivariant Gröbner basis}, which happens to be
an \kl{equivariant Gröbner basis} of the ideal generated by $H$. As a
consequence, we obtain effective representations of \kl{equivariant ideals},
over which one can check membership, inclusion, and compute the sum and
intersection of \kl{equivariant ideals}
(\cref{cor:equivariant-ideals-computations}).

\AP We then focus on providing undecidability results for the \kl{equivariant
ideal membership problem} in the case where our effective assumptions are
satisfied, but the \kl{well-quasi-ordering} condition is not. This aims at
illustrating the fact that our assumptions are close to optimal. One classical
way for a set of structures to not be \kl{well-quasi-ordered} (when labelled
using integers) is to have the ability to represent an \emph{infinite path} (a
formal definition will be given in
\cref{sec:undecidability}). We prove that
whenever one can (effectively) represent an infinite path in the set of
\kl{monomials} $\mon{\Indets}$, then the \kl{equivariant ideal membership
problem} is undecidable.

\begin{theorem}[name={Undecidability of Equivariant Ideal Membership},restate=thm:undecidable-paths]
  \label{thm:undecidable-paths}
  Let $\Indets$ be a totally ordered set of indeterminates
  equipped with a group action $\group \actson \Indets$, under our \kl{computability assumptions}.
  If $\Indets$ contain an \kl(of){infinite path}
  then the \kl{equivariant ideal membership problem} is undecidable.
\end{theorem}

Finally, we illustrate how our positive results find applications in numerous
situations. This is done by providing families indeterminates that satisfy our
\kl{computability assumptions}, and for which we can compute \kl{equivariant
Gröbner bases}, and also by showing how our results can be used in the context
of \kl{topological well-structured transition systems} \cite{JGL10}, with
applications do the verification of infinite state systems such as orbit
finite weighted automata \cite{BOKLMO21}, \kl{orbit finite polynomial
automata}, and more generally orbit finite systems dealing with polynomial
computations.

\paragraph{Organisation.} \AP The rest of the paper is organised as follows. In
\cref{sec:preliminaries}, we introduce formally the notions of \kl{Gröbner
bases}, \kl{effectively oligomorphic} actions, and \kl{well-quasi-orderings},
which are the main assumptions of our positive results. Then, we illustrate in
\cref{sec:act ex} how these assumptions can be satisfied in practice, providing
numerous examples of sets of indeterminates. After that, we introduce in
\cref{sec:weakgb} an adaptation of \kl{Buchberger's algorithm} to the
equivariant case, that computes a \kl{weak equivariant Gröbner basis} of an
\kl{equivariant ideal}. In \cref{sec:equivariant-grobner-basis}, we use
\kl{weak equivariant Gröbner bases} to prove our main positive
\cref{thm:compute-egb}, and we show that it provides a way to effectively
represent \kl{equivariant ideals} (\cref{cor:equivariant-ideals-computations}).
We continue by showing in \cref{sec:closure-properties} that the assumptions of our
\cref{thm:compute-egb} are closed under two natural operations
(\cref{lem:closure-properties-comp,lem:closure-properties-wqo}). The positive
results regarding the \kl{equivariant ideal membership problem} are then
leveraged to obtain several decision procedures for other problems in
\cref{sec:applications}. Finally, in \cref{sec:undecidability}, we show that
our assumptions are close to optimal by proving that the \kl{equivariant ideal
membership problem} is undecidable whenever one can find \kl(of){infinite
paths} in the set of indeterminates (\cref{thm:undecidable-paths}), which is
conjectured to be a complete characterisation of the undecidability of the
\kl{equivariant ideal membership problem} (\cref{rem:conj-wqo-infinite-path}).

% Include the bibliography
\bibliographystyle{plainurl}
\bibliography{papers.bib}

% If there are any appendices, we include them here.
\appendix
\section{intro}
By leveraging the same proof technique,
we can also show that the \kl{equivariant ideal membership problem} is
decidable under a weaker hypothesis, namely that the set of \kl{monomials}
$\mon[\om \ordplus 1]{\Indets}$ is a \kl{WQO}, which is also believed to be
equivalent to the first condition.

\begin{theorem}[name={Equivariant Ideal Membership},restate=thm:decide-equiv-ideal-mem]
  \label{thm:decide-equiv-ideal-mem}
  Let $\Indets$ be a totally ordered set of indeterminates
  equipped with a group action $\group \actson \Indets$, under our \kl{computability assumptions}.
  If $(\mon[\om \ordplus 1]{\Indets}, \gdivleq)$ is a \kl{WQO}, then one can decide the
  \kl{equivariant ideal membership problem}.
\end{theorem}


\section{Proofs of \cref{sec:examples}}

\AP A \intro{topological space} is a set $X$ equipped with a collection $\tau$
of subsets of $X$ that is stable under finite intersections and arbitrary
unions.\footnote{In particular, $\tau$ contains the empty set and $X$ itself.}
In a \kl{topological space}, elements of $\tau$ are called \intro{open
subsets}, while their complements (in $X$) are called \intro{closed subsets}. A
\kl{topological space} is \intro(space){Noetherian} when, for every sequence
$\seqof{U_i}[i \in \N]$ of \kl{open subsets}, there exists $n \in \N$ such that
$\bigcup_{i \in \N} U_i = \bigcup_{i \leq n} U_i$. We refer the readers to the
book \cite{JGL13} for a comprehensive introduction to \kl{Noetherian spaces}
and their usage in theoretical computer science. Let us briefly argue that
\kl{Noetherian spaces} generalize \kl{well-quasi-orders} in
\cref{ex:well-quasi-orders-are-noeth}, and encode the
\kl{Hilbert basis property} in \cref{ex:polynomials-noetherian}.

\begin{example}[ see \cite{JGL13}]
  \label{ex:well-quasi-orders-are-noeth}
  Let $(X, \leq)$ be a quasi-ordered set.
  Then, the set $X$ equipped with the \kl{topology} having 
  as \kl{open subsets} the upwards-closed subsets of $X$ is \kl(space){Noetherian}
  if and only if $(X, \leq)$ is \kl{well-quasi-ordered}.
\end{example}

\begin{example}[ see \cite{JGL13}]
  \label{ex:polynomials-noetherian}
  Let $\K$ be a field, and let $n \in \N$.
  The space $\K^n$ equipped with the \kl{Zariski topology}
  \kl(space){Noetherian}; where the \intro{Zariski topology}
  is the topology whose \kl{closed subsets} are finite unions of sets
  of the form $\setof{ \vec{x} \in \K^n}{ \forall p \in \idl, p(\vec{x}) = 0}$,
  where $\idl$ is an \kl{ideal} of $\poly{\K}{x_1, \dots, x_n}$.
\end{example}

\AP The advantage of \kl{Noetherian spaces} over \kl{well-quasi-orderings} and
\kl{Noetherian rings} is that they generalize both and can be \emph{combined}:
\kl{Noetherian spaces} are closed under finite sums, finite products,
considering finite words, considering finite trees, and many more \todo{cite}.
As a consequence, they provide a versatile tool to express the set of states of
a system, ensuring that a strong termination property holds.

\AP A \intro{topological well-structured transition system} with alphabet
$\Sigma$ is a \kl{topological space} $(X, \tau)$, equipped with a transition
function $\delta \colon X \times \Sigma \to X$, such that the following
properties hold: for every $U \in \tau$, $\mathrm{pre}^\exists(U)$, the set of
states $x \in X$ such that there exists $a \in \Sigma$ with $\delta(x, a) \in
U$, is an \kl{open subset}. Equivalently, the set $\mathrm{pre}^\forall(E)$ of
states $x \in X$ such that for every $a \in \Sigma$, $\delta(x, a) \in E$ is a
\kl{closed subset} of $X$ whenever $E$ is itself a \kl{closed subset} of $X$.
The natural decition problem for \kl{topological well-structured transition
systems} is the following \intro{open reachability problem} is decidable: given
an initial state $x_0 \in X$ and an \kl{open subset} $U \in \tau$, is it true that
there exists a word $w \in \Sigma^*$ such that $\delta^*(x_0, w) \in U$? The
prototypical algorithm to solve this problem is the following \intro{backward
algorithm}: start with $U_0 \defined U$, and iteratively compute $U_{i+1}
\defined U_i \cup \mathrm{pre}^\exists(U_i)$ until $U_i = U_{i+1}$, then check
whether $x_0 \in U_\text{last}$.
There are easy-to-state sufficient conditions  for such an algorithm to be computable and terminate:
\begin{enumerate}
  \item One is equipped with an effective representation of open subsets,
    where one is able to test equality of open subsets, compute unions of open subsets, and test 
    membership of a point in an open subset.
  \item The pre-image function $\mathrm{pre}^\exists$ is computable, i.e., one can
    compute the set $\mathrm{pre}^\exists(U)$ for every open subset $U$.
  \item The space $(X, \tau)$ is \kl{Noetherian}. 
\end{enumerate}

\AP Our \cref{cor:equivariant-ideals-computations} shows that
under some assumptions on $\Indets$, the set of finitely supported functions
$\Indets \to \K$ is a \kl{Noetherian space} with respect to the
\intro{equivariant Zariski topology}, i.e., the topology whose \kl{closed subsets}
are finite unions of sets of the form $E_{\idl} \defined \setof{f \in
\K^{(\Indets)}}{\forall p \in \idl, p(f) = 0}$, where $\idl$ is an
\kl{equivariant ideal} of $\poly{\K}{\Indets}$. Furthermore, we have an
effective representation of the \kl{closed subsets} in this topology, using
\kl{equivariant Gröbner bases} of \kl{equivariant ideals}. In particular, the
theory of \kl{topological well-structured transition systems} can be applied to
systems whose state space contains ``named registers'' that contain numbers and
are updated by polynomial functions.



\AP Let us fix a group $\group$ that acts on the set of indeterminates
$\Indets$, and on an alphabet $\Sigma$ in an \kl{effectively oligomorphic}
fashion. Let us now consider the case of \intro{orbit finite polynomial
automata}, that we define as follows: an \reintro{orbit finite polynomial
automaton} is a tuple $A \defined (Q, \Sigma, \delta, q_0, F)$, where $Q =
\K^{(\Indets)}$, $\Sigma$ is an \kl{orbit finite} alphabet, $\delta \colon
\Sigma \to (\Indets \to \poly{\K}{\Indets})$ is a \kl(func){finitely supported}
polynomial update function, and $F \in \poly{\K}{\Indets}$ is a polynomial
computing the result of the automaton. Given a letter $a \in \Sigma$ and a
state $q \in Q$, the update $\delta^*(q,a)$ is defined as the function from
$\Indets$ to $\K$ defined by $\delta^*(q,a) \colon x \mapsto \delta(a,x)[ q ]$,
which is well-defined because $\delta(a,x)$ is a \kl{finitely supported}
polynomial. The update function is naturally extended to words. Finally, the
output of an \kl{orbit finite polynomial automaton} on a word $w \in \Sigma^*$
is defined as $F(\delta^*(q_0, w))$.

While all of these reasosing could be done outside the realm of (effective)
\kl{topological well-structured transition systems}, we can use the modularity
of the theory to obtain more complex verification properties. Following the
lines of \cite[Theorem 6]{JGL10}, one can consider the case of communicating
orbit finite polynomial automata, where we have a collection processes that
communicate letters over a finite alphabet using lossy channels, and can
perform polynomial updates on their local state. Deciding whether such a system
can reach a state where one process fails to satisfy a given polynomial
invariant is a special case of the \kl{open reachability problem}, and is
decidable.


\begin{lemma}
  \label{lem:zeroness-problem-polynomial-automata}
  The \kl{zeroness problem for polynomial automata} is a special case of the
  \kl{open reachability problem} for \kl{topological well-structured transition systems}.
\end{lemma}
\begin{proof}
  Let $A = (Q, \Sigma, \delta, q_0, F)$ be a \kl{polynomial automaton}.
  We consider the topological space $(Q, \tau)$, where $\tau$ is the
  \kl{Zariski topology} on $\K^n$.
  Let $\idl$ be an \kl{ideal} of $\poly{\K}{x_1,\dots,x_n}$ generated by the polynomials
  $p_1, \dots, p_m$,
  and let $E \defined \setof{q \in Q}{\forall p \in \idl, p(q) = 0}$,
  a \kl{closed subset} of $Q$.
  Then,
  \begin{align*}
    q \in \mathrm{pre}^\forall(E) & \iff 
    \forall a \in \Sigma, \forall p \in \idl, p(\delta(q, a)) = 0 \\
                                  & \iff 
    \forall a \in \Sigma, \forall p \in \idl, p(\delta(q, a)) = 0 \\
                                  & \iff 
                                  \forall p \in \idl[J], p(q) = 0
  \end{align*}
  where $\idl[J] \defined \IdlGen{ \setof{ p_i[ x_i \mapsto \delta(\cdot, a)_i] }{ i \in \set{1, \dots, m}, a \in \Sigma } }$.
  In particular, one can represent \kl{closed subsets} of $Q$ as finite 
  lists of \kl{ideals} using their \kl{Gröbner bases}, and we showed that 
  one can effectively compute the pre-image of \kl{closed subsets} of $Q$
  via $\mathrm{pre}^\forall$ by substituting polynomials.
  In this representation, it is very easy to compute the union 
  of two \kl{closed subsets}, which is simply concatenating the two lists 
  of \kl{ideals} reperesenting them.
  To compute the intersection of two \kl{closed subsets} $E_1$ and $E_2$,
  one can assume without loss of generality that both are represented by a 
  single ideal (i.e., that they are irreducible closed subsets), respectively 
  $\idl_1$ and $\idl_2$.
  Then, an easy computation shows that 
  $E_1 \cap E_2 = \setof{q \in Q}{\forall p \in \idl_1 + \idl_2, p(q) = 0}$,
  where $\idl_1 + \idl_2$ is the sum of the two ideals.
  Whether a point $q \in Q$ is in a \kl{closed subset} $E$ is decidable
  because one can evaluate the generating polynomials on $q$ and check that 
  it is indeed $0$.
  The equality check is more complicated, and can be done by first 
  normalizing the list of ideals so that their intersection is trivial,
  which requires computing the intersection of ideals
  and performing equality checks on the resulting \kl{ideals}.

  As a consequence, it suffices to test the \kl{open reachability problem} for
  the \kl{topological well-structured transition system} $(Q, \tau)$ with the
  initial state $q_0$ and the \kl{open subset} $U = Q \setminus E_\text{final}$,
  where $E_\text{final} \defined \setof{q \in Q}{F(q) = 0}$ is the \kl{closed subset}
  of states where the automaton outputs zero.
\end{proof}


\section{Proofs of \cref{sec:undecidability}}

\begin{proofof}{lem:mon-rewrite-red-membership}
  Let $R$ be a monomial rewrite system, and let $\monelt_s, \monelt_t \in
  \mon{\Indets}$ be two monomials. We can encode the problem of deciding whether
  $\monelt_s$ can be rewritten into $\monelt_t$ using the rules of $R$ as an
  instance of the \kl{equivariant ideal membership problem} as follows:
  \begin{itemize}
    \item Let $H$ be the set of all polynomials of the form $\monelt - \monelt'$
      for all pairs
      $(\monelt, \monelt') \in R$.
    \item Then, we ask whether $\monelt_s - \monelt_t$ belongs to the ideal generated by $H$.
  \end{itemize}

  It is clear that if $\monelt_s$ can be rewritten into $\monelt_t$ using the
  rules of $R$, then $\monelt_s - \monelt_t$ belongs to the equivariant ideal generated by
  $H$. Conversely, if $\monelt_s - \monelt_t$ belongs to the ideal generated by
  $H$, then 
  \begin{equation}
    \label{eq:mon-rewrite-red-membership}
    \monelt_s - \monelt_t 
    = 
    \sum_{i=1}^n a_i \monelt[n]_i (\gelem_i \cdot \monelt_i - \gelem_i \cdot \monelt'_i)
    \quad .
  \end{equation}

  Let us write the (finite) graph $G$ whose vertices are the monomials
  $\monelt[n] (\gelem_i \cdot \monelt_i)$ and $\monelt[n] (\gelem_i \cdot
  \monelt'_i)$, and whose edges are the directed weighted edges labelled by
  $a_i$ (in a direction that makes the weight positive).

  Let us now analyse \cref{eq:mon-rewrite-red-membership}, and notice that
  identifying monomials in the left and right-hand sides of the equation allows
  us to show that $\monelt_s$ and $\monelt_t$ are vertices of $G$. Furthermore,
  we deduce that the sum of the weights of the edges having $\monelt_s$ as a
  source or target equals $1$, and that the sum of the weights of the edges
  having $\monelt_t$ as a source or target equals $-1$. Finally, for every
  vertex $v$ of $G$ that is not $\monelt_s$ or $\monelt_t$, the sum of the
  weights of the edges having $v$ as a source or target is $0$, again because
  of an analysis of the coefficient of the monomial $v$ in the sum of
  \cref{eq:mon-rewrite-red-membership}.

  Hence, the graph $G$ is a flow network, with a flow value of at least $1$
  from $\monelt_s$ to $\monelt_t$. As a consequence, there must exist a path
  from $\monelt_s$ to $\monelt_t$ in $G$, which is a witness
  of the fact that 
  one can rewrite $\monelt_s$ into $\monelt_t$ using the rules of $R$.
\end{proofof}



\end{document}
