%! TEX program = pdflatex
% WARNING: this is a generated file.
%
% Please do not edit this file directly. 
% - If you want to update the medatata of the paper (title, authors, abstract), please
%   edit the `paper-meta.yaml` file in the root of the repository.
% - If you want to update the content of the paper, please edit the latex files
%   in the `src` directory.
% - If you want to update the template itself (e.g., change the layout), please
%   edit the `templates/asmart/asmart-small.tex` file instead.
\documentclass[acmsmall,screen,review,]{acmart}

\usepackage[utf8]{inputenc}
\usepackage[T1]{fontenc}

\AtEndPreamble{%
\theoremstyle{acmdefinition}
\newtheorem{assumption}[theorem]{Assumption}
\newtheorem{remark}[theorem]{Remark}
}

\setcopyright{acmlicensed}
\copyrightyear{2018}
\acmYear{2018}
\acmDOI{XXXXXXX.XXXXXXX}

\acmJournal{JACM}
\acmVolume{37}
\acmNumber{4}
\acmArticle{111}
\acmMonth{8}


\usepackage{todonotes}

% babel for language settings
\usepackage[english]{babel}

% microtype for better typography
\usepackage{microtype}


% math packages
\usepackage{stmaryrd,thmtools,upgreek}


% graphics packages
\usepackage{graphicx}
\usepackage[obeyclassoptions,mode=tex]{standalone}
\usepackage{tikz}
\usetikzlibrary{backgrounds}
\usetikzlibrary{shapes.geometric}
\usetikzlibrary{positioning}
\usetikzlibrary{automata}
\usetikzlibrary{tikzmark}
\usetikzlibrary{patterns}
\usetikzlibrary{arrows}
\tikzset{every state/.style={minimum size=1pt}}
\usepackage{tikz-cd}


% links inside the document
\usepackage{hyperref}
\usepackage[capitalise,noabbrev,nameinlink]{cleveref}
\Crefname{assumption}{Assumption}{Assumptions}
\usepackage[composition,hyperref,xcolor,cleveref]{knowledge}
\knowledgeconfigure{notion}

% Tables 
\usepackage{booktabs}
\usepackage{varwidth}

% Packages for macro definitions
\usepackage{xparse}
\usepackage{xpatch}
\usepackage{tokcycle}
\usepackage{ifthen}

% Proof trees
\usepackage{bussproofs}

% Algorithms
\usepackage{algorithm2e}
\Crefname{algocfline}{Algorithm}{Algorithms}
\crefname{algocfline}{Algorithm}{Algorithms}
\crefname{algocf}{Algorithm}{Algorithms}
\Crefname{algocf}{Algorithm}{Algorithms}

% Colors 
\usepackage{ensps-colorscheme}
\knowledgestyle{intro notion}{color={ACMDarkBlue}, emphasize}
\knowledgestyle{notion}{color={B1}}




% we include whatever the user wants to include in the header

% we include libraries (tex files) usually written in the `lib` directory
% Upgreek letters
\makeatletter
\newcommand\mathgr[1]{\tokcycle
  {\addcytoks{##1}}
  {\processtoks{##1}}
  {\ifcsname up\expandafter\@gobble\string##1\endcsname
   \addcytoks[1]{\csname up\expandafter\@gobble\string##1\endcsname}%
    \else\addcytoks{##1}\fi}
  {\addcytoks{##1}}{#1}%
  \expandafter\mathrm\expandafter{\the\cytoks}%
}
\makeatother


% Create a new macro proofof
% taking as input a label of a theorem
% and creating a proof with a reference to that
% label
\NewDocumentEnvironment{proofof}{ m O{appendix} }{
    % if the command \#1 exists, then 
    % call \#1* to restate the theorem
    \ifcsname #1\endcsname
        \def\isInsideRestatedTheorem{1}
        \csname #1\endcsname*
    \fi
    \begin{proof}[Proof of {\cref{#1}} as stated on page {\pageref{#1}}]
        \phantomsection
        \label{#1:proof}
}{
        % if the optional argument is "appendix" 
        % then printout a "backlink"
        % and otherwise do nothing
        \ifthenelse{\equal{#2}{appendix}}{
        % Some link to go back to the theorem
        \marginpar{\vspace{-2em}\texttt{\small{\hyperref[#1]{$\triangleright$ Back to p.\pageref{#1}}}}}
        }{}
    \end{proof}
}

% Create a new macro proofref
% that takes as input a label of a theorem
% and creates a reference to its proof
\NewDocumentCommand{\proofref}{ m }{
    % checks if the label #1:proof exists, if yes
    % it creates a link to it, otherwise it writes nothing
    \IfRefUndefinedExpandable{#1:proof}{}{
        % Checks if we are inside a restated theorem
        % if yes, we do not print anything
        \ifdefined\isInsideRestatedTheorem
        \else
            \marginpar{\vspace{0.6em}\texttt{\small{\hyperref[#1:proof]{$\triangleright$ Proven p.\pageref{#1:proof}}}}}
        \fi
    }
}



\newcommand{\circled}[2]{%
\hypertarget{#2}{}%
\tikz[baseline=(char.base),color=A2,thick]{%
\node[shape=circle,draw,inner sep=1pt,font=\tiny] (char) {#1};%
}}
\newcommand{\circleref}[2]{%
\hyperlink{#2}{%
\tikz[baseline=(char.base),color=A2,thick]{%
\node[shape=circle,draw,inner sep=1pt,font=\tiny] (char) {#1};}%
}}

%!TEX root = ../atomic.sigconf.tex

\newcommand{\arka}[1]{\textcolor{red}{\textbf{Arka:} #1}}
\newcommand{\para}[1]{\paragraph{\textnormal{\textbf{#1}}}}

% Little math macros
\NewDocumentCommand{\set}{ m }{\{ #1 \}}
\NewDocumentCommand{\setof}{ m m }{\{ #1 \mid #2 \}}
\NewDocumentCommand{\card}{ m }{\left| #1 \right|}
\NewDocumentCommand{\seqof}{ m O{n \in \N} }{\left( #1 \right)_{#2}}

\NewDocumentCommand{\defined}{ }{\triangleq}
\newcommand{\defiff}{\overset{\mathrm{def}}{\iff}}
\newcommand{\defeq}{\overset{\mathrm{def}}{=}}

\newcommand{\subfin}{\subset_{\text{fin}}}

\NewDocumentCommand{\range}{ O{1} m }{[#1, #2]}

% functions of all sorts (injective, partial, surjective)
\newcommand{\topartial}{\rightharpoonup}
\newcommand{\toinj}{\hookrightarrow}
\newcommand{\tosurj}{\twoheadrightarrow}
\newcommand{\tobij}{\stackrel{\simeq}{\longrightarrow}}


% Automate the creation of new orderings
% based on a given symbol.
% For instance,
% \NewDocumentOrdering{\pref}{\preceq}{\prec}
% will create the following commands:
% \prefleq and \preflt
% that will respectively expand to
% \mathrel{\kl[\pref]{\preceq}} and \mathrel{\kl[\pref]{\prec}}
\NewDocumentCommand{\NewDocumentOrdering}{ m m m }{
    \expandafter\newcommand\csname #1leq\endcsname{
        \mathrel{\kl[#1]{#2}}
    }
    \expandafter\newcommand\csname #1lt\endcsname{
        \mathrel{\kl[#1]{#3}}
    }
    \knowledge{#1}{notion}
}

% Order macros
\NewDocumentCommand{\upset}{ O{} m }{{\uparrow_{#1} #2}}
\NewDocumentCommand{\dwset}{ O{} m }{{\downarrow_{#1} #2}}


% Number theory
\NewDocumentCommand{\factorial}{ O{} m }{
    \if\relax\detokenize{#1}\relax
        #2!
    \else
        (#2)!
    \fi
}

\newcommand{\A}{\mathcal{A}}
\newcommand{\R}{\mathbb{R}}
\newcommand{\C}{\mathbb{C}}
\newcommand{\F}{\mathcal{F}}
\newcommand{\Q}{\mathbb{Q}}
\newcommand{\N}{\mathbb{N}}
\newcommand{\K}{\mathbb{K}}
\newcommand{\X}{\mathcal{X}}
\newcommand{\Y}{\mathcal{Y}}

% group actions
\newcommand{\actson}{\curvearrowright}

% orders 
\NewDocumentCommand{\divleq}{}{
    \mathrel{\sqsubseteq^{\mathrm{div}}}
}
\NewDocumentCommand{\gdivleq}{ O{\group} }{
    \mathrel{\sqsubseteq^{\mathrm{div}}_{#1}}
}


\NewDocumentCommand{\Basis}{O{B}}{\mathcal{#1}}
\NewDocumentCommand{\LBasis}{O{B} m}{\mathcal{#1}_{#2}}

\newcommand{\poly}[2]{#1[#2]}
\newcommand{\aut}[2][]{\mathsf{Aut}_{#1}{(#2)}}
\newcommand{\mon}[2][]{\mathsf{Mon}_{#1}(#2)}
\newcommand{\perm}[1]{\mathsf{Perm}(#1)}
\newcommand{\otu}[2]{#1^{(#2)}}
\newcommand{\group}{\mathcal{G}}
\newcommand{\gen}[2]{\langle #1\rangle_{#2}}
\newcommand{\radoG}{\mathbb{G}_{\mathsf{Rado}}}
\newcommand{\radoV}{\mathbb{V}_{\mathsf{Rado}}}
\newcommand{\radoE}{\mathbb{E}_{\mathsf{Rado}}}
\newcommand{\cycleSet}[1][]{\mathsf{Cycles}_{#1}}

\newcommand{\ordinal}{\eta}


\newcommand{\hbp}{\text{Hilbert's basis property}}
\newcommand{\orbit}[2][]{\mathsf{orbit}_{#1}{(#2)}}

\newcommand{\order}[1][]{\prec_{#1}}
\newcommand{\ordereq}[1][]{\preceq_{#1}}

\NewDocumentCommand{\sOrderLt}{O{S}}{\prec_{#1}}
\NewDocumentCommand{\sOrderLeq}{O{S}}{\preceq_{#1}}

\newcommand{\revlex}[1][]{<_{\mathsf{RevLex}}^{#1}}
\newcommand{\revlexeq}[1][]{\leq_{\mathsf{RevLex}}^{#1}}
\newcommand{\gr}{Gr\"obner}
\newcommand{\dom}{\mathsf{dom}}
\newcommand{\spoly}[2]{\mathsf{D}(#1,#2)}
\newcommand{\spolyset}{\mathsf{DSet}}
\newcommand{\spolytext}{$\mathsf{S}$-polynomial}
\newcommand{\lcm}{\mathsf{LCM}}
\newcommand{\lc}[1][]{\mathsf{LC}_{#1}}
\newcommand{\lt}[1][]{\mathsf{LT}_{#1}}
\newcommand{\reducstep}[1]{\to_{#1}}
\newcommand{\reduc}[1]{\to^*_{#1}}
\newcommand{\rem}[2]{\mathsf{Rem}_{#1}(#2)}
\newcommand{\closure}[1]{\widehat{#1}}
\newcommand{\wforder}{\triangleleft}

\newcommand{\lm}[1][]{\mathop{\mathsf{LM}_{#1}}}
\newcommand{\cm}[1][]{\mathop{\mathsf{CM}_{#1}}}

\newcommand{\probBasic}[4]
{
\begin{flalign*}
\quad
\begin{tabular}{l  l}
  \multicolumn{2}{l}{\mathsf{#1}}\\
  \textbf{Input:}    & #2 \\
  \textbf{#4} & #3
\end{tabular}
&&
\end{flalign*}
}

\newcommand{\prob}[3]
{
\probBasic{#1}{#2}{#3}{Question:}
}

\input{lib/knowledges.kl}

% sample knowledge
\knowledge{notion}
 | kl-usage

\begin{document}

\title[Equivariant Gröbner bases]{Computability of Equivariant Gröbner bases \\
       2025-06-30 15:44:33 +0200
}


    \author{Arka Ghosh}
  \email{a.ghosh@uw.edu.pl}
  \orcid{0000-0003-3839-8459}
      \affiliation{
      \institution{Université de Bordeaux}
      \city{Bordeaux}
      \country{France}
    }
      \affiliation{
      \institution{University of Warsaw}
      \city{Warsaw}
      \country{Poland}
    }
      \author{Aliaume Lopez}
  \email{ad.lopez@uw.edu.pl}
  \orcid{0000-0002-4205-327X}
      \affiliation{
      \institution{University of Warsaw}
      \city{Warsaw}
      \country{Poland}
    }
      \renewcommand{\shortauthors}{%
    A. Ghosh, A. Lopez%
  }

\date{\today}
%
\begin{abstract}
    The ring of polynomials in infinitely many variables over a field is never Noetherian. However, it is often the case that the set of indeterminates comes equipped with an additional structure. This extra structure can be taken into account by the means of a group action on the indeterminates, and by considering polynomials up to this action. In this setting, one can recover Noetherianity by restricting the attention to equivariant ideals, under mild assumptions on the group action, as shown by Ghosh and Lasota in 2024. We extend this result by proving algorithmic counterparts to this theoretical result: we show that one can decide the equivariant membership problem, and that one can even compute equivariant Gröbner bases, under similar assumptions as the ones made by Ghosh and Lasota. In addition to these positive results, we also establish sufficient conditions for the equivariant membership problem to be undecidable. Finally, we discuss how these computational results can be applied to various mathematical and computer science decidability problems.
\end{abstract}



\begin{CCSXML}
<ccs2012>
  <concept>
  <concept_id>10002950.10003714.10003715.10003720</concept_id>
  <concept_desc>Mathematics of computing Computations on polynomials</concept_desc>
  <concept_significance></concept_significance>
 </concept>
  <concept>
  <concept_id>10002950.10003714.10003715.10003720.10003747</concept_id>
  <concept_desc>Mathematics of computing Gröbner bases and other special bases</concept_desc>
  <concept_significance></concept_significance>
 </concept>
  <concept>
  <concept_id>10003752.10003766.10003770</concept_id>
  <concept_desc>Theory of computation Automata over infinite objects</concept_desc>
  <concept_significance></concept_significance>
 </concept>
 </ccs2012>
\end{CCSXML}

\ccsdesc[500]{Mathematics of computing Computations on polynomials}
\ccsdesc[500]{Mathematics of computing Gröbner bases and other special bases}
\ccsdesc[500]{Theory of computation Automata over infinite objects}

%%
%% Keywords. The author(s) should pick words that accurately describe
%% the work being presented. Separate the keywords with commas.
\keywords{equivariant ideal, Hilbert basis, ideal membership problem, orbit finite, oligomorphic, well-quasi-ordering}

\received{20 February 2007}
\received[revised]{12 March 2009}
\received[accepted]{5 June 2009}

%%
%% This command processes the author and affiliation and title
%% information and builds the first part of the formatted document.
\maketitle

\klogo\ This document uses \href{https://ctan.org/pkg/knowledge}{knowledge}:
\kl[kl-usage]{notion} points to its \intro[kl-usage]{definition}. 

% Include the content of the paper
%!TEX root = ../atomic.ieee.tex
% LTeX: language=en
\section{Even newer introduction}
%
Consider the system $P$ of polynomial equations
\[
x^2 + yz = 2\ , \quad
x\neq y\neq z\in\Indets
\]
over a set of variables $\Indets$.
If $\Indets$ is finite then weak nullstellensatz \cite[\S 1 Theorem 1]{Cox2015Dict} says that $P$ has a complex solution if and only if the ideal $\idl[P]$ generated by the set of polynomials
\[
P' = \setof{x^2 + yz - 2}{x\neq y\neq z\in\Indets}
\]
does not contain the constant polynomial $1$.
The ideal $\idl[P]$ contains $1$ if and only if any(each) of its \Grbs{} \cite[\S 5 Definition 1]{Cox2015Grb} contains $1$.  
So to check if $P$ has a solution we need to compute a \Grb{} of $\idl[P]$.
This poses no problem since by Hilbert's basis theorem every ideal has a \Grb{} \cite[\S 5 Corollary 6]{Cox2015Grb},
and given a finite set of polynomials $B$ one can use the Buchberger's algorithm  to find a \Grb{} of the ideal generated by $B$ \cite[\S 7 Theorem 2]{Cox2015Grb}.

\Grbs{} are also useful for other computational purposes as well,
the most important of which is the decidability of the ideal membership problem:
given a polynomial $f$ and a finite set of polynomials $B$,
is $f$ in the ideal generated by $B$.

If $\Indets$ is countably infinite, one can still use the version of weak nullstellensatz give in \cite[Page 1]{LangNull}.
However, ideals in $\poly{\K}{\Indets}$ can have arbitrary behaviours which makes difficult to extend the notion of \Grbs{} in an useful way.
For example,
if $\Indets$ is the set of configurations of a deterministic Turing machine,
then the ideal generated by
\[
\setof{x - y}{x\to y}
\]
contains a polynomial $u - v$ if and only if $u$ is reachable by $v$.
Thus one must restrict to a specific class of ideals that are better behaved.
The class we are interested in this article is the class of ideals that are \kl{equivariant}, i.e.\ invariant under certain permutations of the variables.
For example, consider the ideal $\idl[P]$ we defined above,
that is generated by the sets
\[
P' = \setof{x^2 + yz - 2}{x\neq y\neq z\in\Indets} \ .
\]
For any $x\neq y\neq z\in\Indets$, a permutation $\pi$ of $\Indets$ takes the polynomial $x^2 + yz - 2$ to the polynomial $\pi(x)^2 + \pi(y)\pi(z) - 2$.
Let $\perm{\Indets}$ denote the set of all permutations of $\Indets$.
The set $P'$ is equivariant w.r.t. $\perm{\Indets}$, i.e.invariant under the permutations of $\Indets$.
Thus $\idl[P]$ is also equivariant (w.r.t. $\perm{\Indets}$).
With some difficulty we can prove:
%
\begin{lemma}
The ideal $\idl[P]$ is not finitely generated.
\end{lemma}
%
However, once we allow application of permutations while generating the ideal,
$\idl[P]$ is generated by any(every) polynomial.
Stated differently, for any $x\neq y\neq z$, $\idl[P]$ is the smallest ideal containing the polynomial $x^2 + yz - 2$.
Thus $\idl[P]$ is finitely generated as an equivariant ideal.
In fact \cite[Proposition 2]{COHEN67} shows that every equivariant ideal in $\poly{\K}{\Indets}$ is finitely generated.
This line of research continues in \cite{Emmott87} which extends the notions of a \Grbs{} and Buchberger's algorithm to this setting.
The above-mentioned results were rediscovered in \cite{AH07,AH08,HKL18}.
In \cite{HS12} these results were used to prove the Independent Set Conjecture in
algebraic statistics.
In \cite{HS12}, the authors also showed that one can even take a submonoid $\calM$ of $\inc{<}$ and prove existence and computability of finite Gr\"{o}bner basis assuming that $\gdivleq[\calM]$ is a well-partial-order.
These results were independently generalised in \cite{GHOLAS24} and \cite{LauSno23} which study ideals that are equivariant w.r.t. a subgroup $\group$ of permutations of $\Indets$.
In particular \cite[Theorem 11 and 12, Lemma 13]{GHOLAS24} and \cite[Proposition 3.2, Proposition 3.3]{LauSno23} give a necessary and a sufficient condition on the group $\group$ for the \kl{Equivariant Hilbert basis property}:
ideals that are equivariant w.r.t. $\group$ are finitely generated. 
The necessary and sufficient conditions are equivalent up to a well-known conjecture by Pouzet \cite[Problems 12]{POUZ24}.

\arka{write using homogeneous languages.
Can specify contributions more clearly. And also specify how much it differs from before.
Add a comment that every oligomorphic structure is homogeneous with a (possibly infinite) relational language}

This results imply the Noetherian property of vector subspaces that are equivariant with respect to the action of a group $\group$ satisfying the sufficient conditions (\cite[Page 21]{BFKM24} and \cite[Theorem 27]{GHOLAS24}),
and also the decidability of the zero-ness problem of weighted register automata
(\cite[Theorem 30 and Remark 33]{GHOLAS24}).
%
\subsection{Contribution}
%
In this paper, we bridge the gap between the
theoretical understanding of the \intro{equivariant Hilbert basis property} \cite[Property 4]{GHOLAS24}, and the computational aspects of \kl{equivariant
ideals}, by showing that a mild strengthening of the  sufficient condition for the \kl{Equivariant Hilbert basis property} guarantees computability of \kl{equivariant Gröbner bases} of an \kl{equivariant ideal} (\Cref{thm:compute-egb}).
We also show that


This result along also implies we can solve equations,
but does not tell us how the solution looks

we rectify the situation by 

this implies wqo implies extremely amenable

%
\section{Introduction}
\label{sec:intro}

\AP For a field $\K$ and a non-empty set $\Indets$ of indeterminates, we use
$\poly{\K}{\Indets}$ to denote the ring of polynomials with coefficients from $\K$
and indeterminates/variables from $\Indets$. A fundamental result in commutative
algebra is \intro{Hilbert's basis theorem}, stating that when $\Indets$ is finite,
every ideal in $\poly{\K}{\Indets}$ is finitely generated \cite{HILB1890}, where an
\kl{ideal} is a non-empty subset of $\poly{\K}{\Indets}$ that is closed under
addition and multiplication by elements of $\poly{\K}{\Indets}$. This property can
be rephrased as the fact that the set of polynomials $\poly{\K}{\Indets}$ is
\intro{Noetherian}. \kl{Hilbert's basis theorem} extends to the case where $\K$
is a ring that is itself \kl{Noetherian} \cite[Theorem 4.1]{Lang02}.

\AP A \Grb\ is a specific kind of generating set of a polynomial ideal
which allows easy checking of membership of a given polynomial in that ideal.
\kl{Gr\"{o}bner bases} were introduced by Buchberger who showed when $\Indets$ is
finite, every ideal in $\poly{\K}{\Indets}$ has a finite \kl{Gr\"{o}bner basis} and
that, for a given a set of polynomials in $\poly{\K}{\Indets}$, one can compute a
finite \kl{Gröbner basis} of the ideal generated by them via the so-called
\intro{Buchberger algorithm} \cite{BUCH76}. The
existence and computability of \Grbs\ implies the decidability of the
\kl{ideal membership problem}: given a polynomial $f$ and set of polynomial
$H$, decide whether $f$ is in the ideal generated by $H$. The theory of
\kl{Gr\"{o}bner bases} has applications in very diverse areas of computer
science, including integer programming \cite{Sturmfels96}, algebraic proof
systems \cite{algProof}, geometric reasoning \cite{Cox2015chGeom}, fixed
parameter tractability \cite{ACDM22}, program analysis \cite{SSM04} and
constraint satisfaction problems \cite{Mas21}.
In automata theory it has been used for deciding zeroness of polynomial
automata \cite{BEDUSHWO17}, reachability in symmetric Petri nets \cite{MAME82},
equivalence for string-to-string transducers \cite{HONKALA00} and equivalence
of polynomial differential equations \cite{CLEMENTE24}. 

\AP There has been a growing interest in the last few years for computational
models that are manipulating infinite data structures in a finite way, for
instance an automaton reading words on the infinite alphabet $\N$, while
maintaining a finite number of states. While this idea can be traced back to
the 90s with the notion of register automata \cite{KAFR94}, it has been revived
in with the development of the theory of \emph{orbit finite sets}. In this
setting, one would like to consider an infinite set of variables $\Indets$. As an
example, let us consider the set $\Indets$ of variables $x_i$ for $i \in \N$, and
the \kl{ideal} $\idlZ$ generated by the set $\setof{x_i}{i \in \N}$. It is
clear that $\idlZ$ is not finitely generated, and we conclude that the
\kl{Hilbert's basis theorem} (and a fortiori, the \kl{Gr\"{o}bner basis}
theory) does not extend to the case of infinite sets of indeterminates.

\AP However, in the applications mentionned above, the infinite set of
variables (data) comes with an extra structure: the behaviour of the considered
systems are invariant under the action of a group $\group$ on $\Indets$. The action
of this $\group$ on $\Indets$ naturally induces an action on $\poly{\K}{\Indets}$, by
renaming the variables. The typical example is the group of all permutations of
$\Indets$, which corresponds to seeing $\Indets$ as a set of \emph{indistinguishable}
names: one is not interested in the ideal $\idlZ$ generated by the set
$\setof{x_i}{i \in \N}$, but rather in the \kl{equivariant ideal} generated by
the set $\setof{x_i}{i \in \N}$, which is the smallest ideal that contains it
and is invariant under the action of $\group$. In this case, this ideal is
finitely generated by a single indeterminate, e.g. $x_1$. Please note that
equivariance does not imply finite generation in general: for instance, the
ideal $\idlZ$ is not finitely generated as an equivariant ideal with respect to
the trivial group.


\subsection{Related Research}
The above-mentioned results were rediscovered in \cite{AH07,AH08,HKL18}. In
\cite{HS12} these results were used to prove the Independent Set Conjecture in
algebraic statistics. In \cite{HS12}, the authors also showed that one can even
take a submonoid $\calM$ of $\inc{<}$ and prove existence and computability of
finite Gr\"{o}bner basis assuming that $\gdivleq[\calM]$ is a
well-partial-order. These results were significantly generalised in
\cite{GHOLAS24}, which gives a necessary and a sufficient condition on the
actions $\group\actson\Indets$ for the \kl{Equivariant Hilbert basis property}
to hold \cite[Theorems 11 and 12, Lemma 13]{GHOLAS24}. The necessary and
sufficient conditions are equivalent up to a well-known conjecture by Pouzet
\cite[Problems 12]{POUZ24}. But to obtain decision procedures, one still lacks
a generalisation of \kl{Buchberger's algorithm} to the equivariant case, except
under artificial extra assumptions \cite[Section 6]{GHOLAS24}. Overall, a
general understanding of the decidability of the \kl{equivariant ideal
membership problem} is still missing, and \emph{a fortiori}, a generalisation
of \kl{Buchberger's algorithm} to the equivariant case is still an open
problem.

Our results are part of a larger research direction that aims at establishing
an algorithmic theory of computation with orbit-finite sets. For instance,
\cite{BFKM24} studies equivariant subspaces of vector spaces generated by
orbit-finite sets, \cite{GHL22,GHL25} study solvability of orbit-finite systems
of linear equations and inequalities, and \cite{BFKM24,GHL22,Prz23} study duals
of vector spaces generated by orbit-finite sets.

\subsection{Contributions.}
\AP In this paper, we bridge the gap between the
theoretical understanding of the \intro{equivariant Hilbert basis property}
 \cite[Property 4]{GHOLAS24}, and the computational aspects of \kl{equivariant
ideals}, by showing that under mild assumptions on the group action, one can
compute an \kl{equivariant Gröbner basis} of an \kl{equivariant ideal}, hence,
that one can decide the \kl{equivariant ideal membership problem}. In order to
compute such sets, we will need to introduce some classical \kl{computability
assumptions} on the group action $\group \actson \Indets$, and on the set of
indeterminates $\Indets$. These will be defined in
\cref{sec:preliminaries}, but informally, we assume
that one can compute representatives of the orbits of elements under the action
of $\group$ (this is called \kl{effective oligomorphism}), and that one has
access to a total ordering on $\Indets$ that is computable, and
\kl(ord){compatible} with the action of $\group$. Please note that the ordering
on $\Indets$ is not required to be well-founded, and a typical example of our
computable assumptions would be the set $\Q$ of rationals, equipped with the
natural ordering $\leq$ and the group $\group$ would be the group of all
monotone bijections from $\Q$ to itself.

\AP Let us now focus on the mild semantic assumption that we will need to make
on the set of indeterminates $\Indets$ and the group $\group$, that will
guarantee the termination of our procedures. We refer to our preliminaries
(\cref{sec:preliminaries}) for a more detailed
discussion on these assumptions, but again informally, we ask that the set of
\kl{monomials} $\mon{\Indets}$ is well-behaved with respect to divisibility up
to the action of $\group$, which we write as the fact that $(\mon{\Indets},
\gdivleq)$ is a \kl{well-quasi-ordering} (\kl{WQO}). It is known from that this
is a necessary condition for the \kl{equivariant Hilbert basis property}
\cref{thm:equiv-hilbert-property}, and we will rely on a slightly stronger
condition, namely that $(\mon[Y]{\Indets}, \gdivleq)$ is a \kl{WQO}, whenever
$(Y, \leq)$ is one, which is conjectured to be equivalent to the first
condition. Beware that \cref{thm:equiv-hilbert-property,thm:compute-egb}
are
incomparable: the former does not talk about decidability, while the latter 
only considers \kl{equivariant ideals} that are already finitely presented, and we 
will show in
\cref{ex:non-wqo-undecidable} an example where \kl{equivariant
Gröbner bases} are computable, but the \kl{Hilbert basis property} fails.

\begin{theorem}[name={\cite[Theorem 11 and 12]{GHOLAS24}}]
  \label{thm:equiv-hilbert-property}
  Let $\Indets$ be a totally ordered set of indeterminates
  equipped with a group action $\group \actson \Indets$ that is 
  \kl(ord){compatible} with the ordering on $\Indets$.
  Then, $(\mon[\om]{\Indets}, \gdivleq)$ is a \kl{WQO}, if and only if 
  the \kl{equivariant Hilbert basis property} holds for $\poly{\K}{\Indets}$.
\end{theorem}

\begin{theorem}[name={Equivariant Gröbner Basis},restate=thm:compute-equiv-gb]
  \label{thm:compute-egb}
  Let $\Indets$ be a totally ordered set of indeterminates
  equipped with a group action $\group \actson \Indets$, under our \kl{computability assumptions}.
  If $(\mon[Y]{\Indets}, \gdivleq)$ is a \kl{WQO} for every 
  \kl{well-quasi-ordered} set $(Y,\leq)$, then one can
  compute an \kl{equivariant Gröbner bases} of \kl{equivariant ideals}.
\end{theorem}

\AP To prove our \cref{thm:compute-egb}, we will first introduce a weaker
notion of \kl{weak equivariant Gröbner basis}, which characterises the results
obtained by naïvely adapting \kl{Buchberger's algorithm} to the equivariant
case. Then, we will show that under our \kl{computability assumptions}, one can
start from a finite set of generators $H$ of an \kl{equivariant ideal}, and
compute a well-chosen \kl{weak equivariant Gröbner basis}, which happens to be
an \kl{equivariant Gröbner basis} of the ideal generated by $H$. As a
consequence, we obtain effective representations of \kl{equivariant ideals},
over which one can check membership, inclusion, and compute the sum and
intersection of \kl{equivariant ideals}
(\cref{cor:equivariant-ideals-computations}).

\AP We then focus on providing undecidability results for the \kl{equivariant
ideal membership problem} in the case where our effective assumptions are
satisfied, but the \kl{well-quasi-ordering} condition is not. This aims at
illustrating the fact that our assumptions are close to optimal. One classical
way for a set of structures to not be \kl{well-quasi-ordered} (when labelled
using integers) is to have the ability to represent an \emph{infinite path} (a
formal definition will be given in
\cref{sec:undecidability}). We prove that
whenever one can (effectively) represent an infinite path in the set of
\kl{monomials} $\mon{\Indets}$, then the \kl{equivariant ideal membership
problem} is undecidable.

\begin{theorem}[name={Undecidability of Equivariant Ideal Membership},restate=thm:undecidable-paths]
  \label{thm:undecidable-paths}
  Let $\Indets$ be a totally ordered set of indeterminates
  equipped with a group action $\group \actson \Indets$, under our \kl{computability assumptions}.
  If $\Indets$ contain an \kl(of){infinite path}
  then the \kl{equivariant ideal membership problem} is undecidable.
\end{theorem}

Finally, we illustrate how our positive results find applications in numerous
situations. This is done by providing families indeterminates that satisfy our
\kl{computability assumptions}, and for which we can compute \kl{equivariant
Gröbner bases}, and also by showing how our results can be used in the context
of \kl{topological well-structured transition systems} \cite{JGL10}, with
applications do the verification of infinite state systems such as orbit
finite weighted automata \cite{BOKLMO21}, \kl{orbit finite polynomial
automata}, and more generally orbit finite systems dealing with polynomial
computations.

\paragraph{Organisation.} \AP The rest of the paper is organised as follows. In
\cref{sec:preliminaries}, we introduce formally the notions of \kl{Gröbner
bases}, \kl{effectively oligomorphic} actions, and \kl{well-quasi-orderings},
which are the main assumptions of our positive results. Then, we illustrate in
\cref{sec:act ex} how these assumptions can be satisfied in practice, providing
numerous examples of sets of indeterminates. After that, we introduce in
\cref{sec:weakgb} an adaptation of \kl{Buchberger's algorithm} to the
equivariant case, that computes a \kl{weak equivariant Gröbner basis} of an
\kl{equivariant ideal}. In \cref{sec:equivariant-grobner-basis}, we use
\kl{weak equivariant Gröbner bases} to prove our main positive
\cref{thm:compute-egb}, and we show that it provides a way to effectively
represent \kl{equivariant ideals} (\cref{cor:equivariant-ideals-computations}).
We continue by showing in \cref{sec:closure-properties} that the assumptions of our
\cref{thm:compute-egb} are closed under two natural operations
(\cref{lem:closure-properties-comp,lem:closure-properties-wqo}). The positive
results regarding the \kl{equivariant ideal membership problem} are then
leveraged to obtain several decision procedures for other problems in
\cref{sec:applications}. Finally, in \cref{sec:undecidability}, we show that
our assumptions are close to optimal by proving that the \kl{equivariant ideal
membership problem} is undecidable whenever one can find \kl(of){infinite
paths} in the set of indeterminates (\cref{thm:undecidable-paths}), which is
conjectured to be a complete characterisation of the undecidability of the
\kl{equivariant ideal membership problem} (\cref{rem:conj-wqo-infinite-path}).
%!TEX root = ../atomic.asmart.tex
% LTeX: language=en
\section{Preliminaries}
\label{sec:preliminaries}

\paragraph{Partial orders, ordinals, and well-quasi-ordered sets.} \AP We
assume basic familiarity with partial orders and ordinals. In particular, we
will use the notation $\intro*\om$ for the first infinite ordinal (that is,
$(\N, \leq)$), and write $X \intro*\ordplus Y$ for the lexicographic sum of two
partial orders $X$ and $Y$. We will also use the usual notations for finite
ordinals, writing $\intro*\ordfin{n}$ for the finite ordinal of size $n$. For
instance, $\om \ordplus \ordfin{1}$ is the total order $\N \uplus
\set{+\infty}$, where $+\infty$ is the new largest element.

\AP In order to guarantee the termination of the algorithms presented in this
paper, a key ingredient will be the notion of \intro{well-quasi-ordering}
(WQO), that are sets $(X, \leq)$ such that every infinite sequence
$\seqof{x_i}[i \in \N]$ of elements of $X$ contains a pair $i < j$ such that
$x_i \leq x_j$.

\paragraph*{Polynomials.} \AP 
We assume basic familiarity with the theory of
commutative algebra, and polynomials. We will use the notation $\poly{\K}{\X}$
for the ring of polynomials with coefficients from a field $\K$ and
indeterminates/variables from a set $\X$, and $\mon{\X}$ for the set of
monomials in $\poly{\K}{\X}$. Letters $p,q,r$ are used to denote polynomials,
$\monelt,\monelt[n]$ are used to denote monomials, and $a,b,\alpha,\beta$ are
used to denote coefficients in $\K$.

A classical example of a \kl{WQO} is the set of monomials $\mon{\X}$, endowed
with the \kl{divisibility} relation $\divleq$ whenever $\X$ is finite. We
recall that a monomial $\monelt[m]$ \intro{divides} a monomial $\monelt[n]$ if
there exists a monomial $\monelt[l]$ such that $\monelt[m] \times \monelt[l] =
\monelt[n]$. In this case, we write $\monelt[m] \divleq \monelt[n]$. Note that
monomials can be seen as functions from $\X$ to $\N$ having a finite support,
and that the \kl{divisibility} relation can be extended to monomials that are
functions from $\X$ to $(X,\leq)$, where $X$ is any partially ordered set. In
this case, we write $\monelt[m] \divleq \monelt[n]$ if for every $x \in \X$, we
have $\monelt[m](x) \leq \monelt[n](x)$. We will write
$\intro*\mon[\omega+1]{\X}$ (resp. $\mon[\omega+\omega]{\X}$) for the set of
monomials that are functions from $\X$ to $\omega + 1$ (resp. $\omega +
\omega$).



Given an ordering $\varleq$ on the set of indeterminates $\Indets$, we define
the \intro{reverse lexicographic} (revlex) ordering on monomials as follows:
$\monelt[n] \revlexlt \monelt[m]$ if there exists an indeterminate $x \in \X$
such that $\monelt[n](x) < \monelt[m](x)$, and such that for every $y \in \X$,
if $x \varlt y$ then $\monelt[n](y) = \monelt[m](y)$. Remark that if
$\monelt[n] \revlexleq \monelt[m]$, then in particular $\monelt[n] \divleq
\monelt[m]$. 

\AP We can now use the \kl{revlex} ordering to identify particular elements in
a given polynomial. Namely, for a polynomial $p \in \poly{\K}{\X}$, we define
the \intro{leading monomial} $\lm(p)$ of $p$ as the largest monomial appearing
in $p$ with respect to the \kl{revlex} ordering, and the \intro{leading
coefficient} $\lc(p)$ of $p$ as the coefficient of $\lm(p)$ in $p$. We can then
define the \intro{leading term} $\lt(p)$ of $p$ as the product of its
\kl{leading monomial} and its \intro{leading coefficient}, and the
\intro{characteristic monomial} $\cm(p)$ of $p$ as the product of its
\kl{leading monomial} and all the indeterminates appearing in $p$. Because the
coefficients and monomial in question are highly dependent on the ordering
$\varleq$, we allow ourselves to write $\lm[\Indets](p)$ to highlight the
precise ordered set of variables that was used to compute the \kl{leading
monomial} of $p$.


\paragraph{Gröbner Bases.} \AP An \intro{ideal} $\idl$ of $\poly{\K}{\X}$ is a
non-empty subset of $\poly{\K}{\X}$ that is closed under addition and
multiplication by elements of $\poly{\K}{\X}$. Given a set $H \subseteq
\poly{\K}{\X}$, we denote by $\intro*\IdlGen{H}$ the ideal generated by $H$,
i.e. the smallest ideal that contains $H$. The \intro{ideal membership problem}
is the following decision problem: given a polynomial $p \in \poly{\K}{\X}$ and
a set of polynomials $H \subseteq \poly{\K}{\X}$, decide whether $p$ belongs to
the ideal $\IdlGen{H}$ generated by $H$. We know that this problem is decidable
when $\X$ is finite, and that it is even $\EXPTIME$-complete \cite{MAME82}.

\AP The classical approach to the \kl{ideal membership problem} is to use the
\kl{Gröbner basis} theory that was developed in the 70s by Buchberger
\cite{BUCH76}. Having fixed and ordering $\varleq$ on $\Indets$, a set $\Basis$
of polynomials is called a \intro{Gröbner basis} of an ideal $\idl$ if,
$\IdlGen{\Basis} = \idl$ and for every polynomial $p \in \idl$, there exists a
polynomial $q \in \Basis$ such that $\lm[\Indets](q) \divleq \lm[\Indets](p)$.



\paragraph{Group actions and equivariant ideals.}  A \intro{group} $\group$ is
a set equipped with a binary operation that is associative, has an identity
element and has inverses. In our setting, we are interested in infinite sets
$\X$ of indeterminates that is equipped with a \intro{group action} $\group
\actson \X$. This means that for each $\gelem \in \group$, we have a bijection
$\X \tobij \X$ that we denote by $x \mapsto \gelem  \cdot x$. A set $S
\subseteq \X$ is \intro{equivariant} under the action of $\group$ if for all
$\gelem \in \group$ and $x \in S$, we have $\gelem \cdot x \in S$. We give in
\cref{ex:idl-equiv} an example and a non-example of \kl{equivariant ideals}.

\begin{example}
    \label{ex:idl-equiv}
    Let $\X$ be any infinite set, and $\group$ be the 
    group of all bijections of $\X$. 
    Then the set $S_0 \subset \poly{\K}{\X}$ of all polynomials 
    whose set of coefficients sums to $0$ is an equivariant ideal.
    Conversely, the set of all polynomials that are multiple
    of $x \in X$ is an \kl{ideal} that is not \kl{equivariant}.
\end{example}
\begin{proof}
    Let $p,q\in S_0$, and $r \in \poly{\K}{\X}$.
    Then, $p \times r + q$ is in $S_0$. Remark that 
    $p,r$ and $q$ belong to a subset $\poly{\K}{\Y}$ of the 
    polynomials that uses only finitely many indeterminates.
    In this subset, the sum of all coefficients is obtained
    by applying the polynomials to the value $1$ for every indeterminate
    $y \in \Y$. We conclude that
    $(p \times r + q)(1,\dots, 1) 
    = p(1,\dots,1) \times r(1,\dots,1) + q(1,\dots,1)
    = 0 \times r(1, \dots, 1) + 0 = 0$, hence that
    $p \times r + q$ belongs to $S_0$. 
    Because $0$ is in $S_0$, we conclude that $S_0$ is an \kl{ideal}.
    Furthermore, if $\gelem \in \group$ and $p \in S_0$, then
    the sum of the coefficients $\gelem \cdot p$ is exactly
    the sum of the coefficients of $p$, hence is $0$ too.
    This shows that $S_0$ is \kl(ideal){equivariant}.

    It is clear that all multiples of a given polynomial $x \in \X$
    is an ideal of $\poly{\K}{\X}$. This is not an \kl{equivariant ideal}:
    take any bijection $\gelem \in \group$ that does not map $x$ to $x$ (it
    exists because $\X$ is infinite and $\group$ is all permutations),
    then $\gelem \cdot x$ is not a multiple of $x$, and therefore does 
    not belong to the ideal.
\end{proof}

\AP A group action $\group \actson \X$ is said to be \intro{compatible} with an
ordering $\leq$ on $\X$ if for all $\gelem \in \group$ and $x,y \in \X$, we
have $x \leq y$ if and only if $\gelem \cdot x \leq \gelem \cdot y$. Let us
point out that in this case, $\revlexleq$ is also \kl{compatible} with the
action of $\group$ on $\mon{\X}$, i.e. for all $\gelem \in \group$ and
monomials $\monelt[m], \monelt[n] \in \mon{\X}$, we have
$\monelt[m] \revlexleq \monelt[n]$ if and only if $\gelem \cdot \monelt[m]
\revlexleq \gelem \cdot \monelt[n]$.


\AP Given a monomial $\monelt[m] \in \mon{\X}$, we define the
\intro(monomial){domain} of $\monelt[m]$ as the set $\dom(\monelt[m])$ of
indeterminates $x \in \X$ such that $\monelt[m](x) \neq 0$. Note that the
domain of a monomial is equivariant: if $\gelem \in \group$, then $\gelem \cdot
\dom(\monelt[m]) = \dom(\gelem \cdot \monelt[m])$.



We say that the action is
\intro{effectively oligomorphic} if \todo{do it}.

We know from \cite{GHOLAS24} that a necessary condition for the \kl{equivariant
Hilbert basis property} to hold is that the set  $\mon{\X}$  of monomials is a
\kl{well-quasi-ordering} when endowed with the \intro{divisibility up-to
$\group$} relation ($\intro*\gdivleq$), which is defined as follows: for
$\monelt_1, \monelt_2 \in \mon{\X}$, we write $\monelt_1 \gdivleq \monelt_2$ if
there exists $\gelem \in \group$ such that $\monelt_1$ \kl{divides} $\gelem
\cdot \monelt_2$. Let us recall that a monomial $\monelt[m]$ \intro{divides} a
monomial $\monelt[n]$ if there exists a monomial $\monelt[l]$ such that
$\monelt[m] \times \monelt[l] = \monelt[n]$. This relation also extends to
monomials that are functions from $\X$ to $(X,\leq)$ with finite support, where
$X$ is any partially ordered set.

\AP Having fixed an ordering $\varleq$ on the set $\Indets$ of indeterminates,
we say that a set $\Basis \subseteq \poly{\K}{\X}$ is an \intro{equivariant
Gröbner basis} of an equivariant ideal $\idl$ if $\Basis$ is \kl{equivariant},
$\IdlGen{\Basis} = \idl$, and for every polynomial $p \in \idl$, there exists
$q \in \Basis$ such that $\lm[\varleq](q) \gdivleq \lm[\varleq](p)$ and
$\dom(q) \subseteq \dom(p)$.

Beware that even in the case of a finite set of variables, a \kl{Gröbner basis}
is not necessarily an \kl{equivariant Gröbner basis}, because of the
\kl{domain} condition. However, every \kl{equivariant Gröbner basis} is a
\kl{Gröbner basis}.

% LTeX: language=en
\section{Weak Equivariant Gröbner Bases}
\label{sec:weakgb}

\AP In this section we will prove that one can compute a good enough basis
using an adaptation of \kl{Buchberger's algorithm} to the equivariant case.

\begin{itemize}
  \item define the ordering on polynomials with the support
  \item explain that it is wqo
  \item explain that one can work with orbit finite sets
    without talking about representatives every time
  \item construct the buchberger algorithm using this ordering
  \item We obtain a set that is a \kl{weak equivariant Gröbner basis}:
    for every element of the ideal, we can find a polynomial in the
    \kl{weak equivariant Gröbner basis} that can be used to reduce it without
    respecting the support condition.
  \item Example of why it is not enough to obtain a real 
    decision procedure.
  (reducing and adding new indeterminates every time)
\end{itemize}


% LTeX: language=en
\section{Computing the Equivariant Gröbner Basis}
\label{sec:equivariant-grobner-basis}

The goal of this section is to strengthen the results of \cref{sec:algorithm},
and instead of being able to answer to the \kl{equivariant ideal membership
problem}, to compute an \kl{equivariant Gröbner basis} of an equivariant ideal.
Let us recall that a Gröbner basis is known to exist, but that it's
computability was an open question of \cite{GHOLAS24}.

The proof will closely follow the one of \cref{sec:algorithm}, where one starts
from a generating set $H$, and constructs a new set $H'$ together with a group
action, over which one computes a \kl{weak Gröbner basis} (\cref{sec:weakgb}).
The result is then used to derive an \kl{equivariant Gröbner basis} of the
\kl{equivariant ideal generated by} $H$. Informally, one wants to 
apply the technique of isolating finite sets of variables \emph{uniformly},
that is, compute \kl{weak Gröbner bases} for every possible fixed 
finite subset of variables. 


Let us fix a set $\Indets$ of indeterminates equipped with a total ordering
$\varleq$. We define $\IndetsCol \defined \Indets + \Indets$, that is, the
disjoint union of two copies of $\Indets$, ordered. It will be useful to refer
to the first copy (lower copy) and the second copy (upper copy), noting the
isomorphism between $\IndetsCol$ and $\set{\mathsf{first}, \mathsf{second}}
\times \Indets$, ordered lexicographically, where $\mathsf{first} <
\mathsf{second}$. We will also define $\forgetCol \colon \IndetsCol \to
\Indets$ that maps a colored variable to its underlying variable.
Beware that $\forgetCol$ is not an order preserving map.
We extend $\forgetCol$ as a morphism from polynomials in
$\poly{\K}{\IndetsCol}$ to polynomials in $\poly{\K}{\Indets}$.

Given a subset $V \subfin \Indets$, we build the injection $\colorWith{V}
\colon \Indets \to \IndetsCol$ that maps variables $x$ in $V$ to
$(\mathsf{fisrt}, x)$, and variables $x$ not in $V$ to $(\mathsf{second}, x)$.
Again, we extend these maps as morphisms from $\poly{\K}{\Indets}$ to
$\poly{\K}{\IndetsCol}$. We say that a polynomial $p \in \poly{\K}{\IndetsCol}$
is \intro{$V$-compatible} if $p \in \colorWith{V}(\poly{\K}{\Indets})$.

\begin{lemma}
  \label{lem:v-saturation-computable}
  Let $H$ be an \kl{orbit finite} subset of $\poly{\K}{\Indets}$.
  Then, $\freeColor(H) \defined \bigcup_{V \subfin \Indets} \colorWith{V}(H)$
  is a computable \kl{orbit finite} subset of $\poly{\K}{\IndetsCol}$.
\end{lemma}


We are now ready to write our algorithm to compute 
an \kl{equivariant Gröbner basis}.

\begin{algorithm}
    \caption{Computing \kl{equivariant Gröbner bases}}
    \label{alg:stronggb}
    \KwIn{An orbit finite set $H$ of polynomials}
    \KwOut{An orbit finite set $\Basis$ that is a \kl{equivariant Gröbner basis} of
      $\EqIdlGen{H}$}
    \Begin{
        $H_C \gets \freeColor(H)$\;
        $\Basis_C \gets \mathsf{weakgb}(H_C)$\;
        $\Basis \gets \forgetCol(\Basis_C)$\;
        \Return{$\Basis$}\;
    }
\end{algorithm}

To prove the correctness of our algorithm, let us first argue
that one can indeed compute the \kl{weak Gröbner basis} algorithm.

\begin{lemma}
  \label{lem:colored-hypothesis-sat}
  Assume that $(\Indets, \varleq, \group)$
  is \kl{effectively oligomorphic},
  and that $(\mon[\omega + \omega]{\Indets}, \gdivleq)$
  is a \kl{well-quasi-order}.
  Then,
  $\IndetsCol$ with its ordering and the 
  action of $\group$ acting on both components 
  simultaneously is \kl{effectively oligomorphic},
  and $(\mon{\IndetsCol}, \gdivleq)$ is a
  \kl{well-quasi-ordered} set.
\end{lemma}

Now, let us argue that the result of our algorithm
is a generating set of the desired ideal, which follows
from the fact that $\forgetCol$ and $\colorWith{\cdot}$
are morphisms that preserve variable names.

\begin{lemma}
  \label{lem:correct-gen-set}
  Let $H$ be an \kl{orbit finite} subset of $\poly{\K}{\Indets}$,
  then the result of \cref{alg:stronggb}
  is an \kl{orbit finite} generating set
  of $\EqIdlGen{H}$.
\end{lemma}
\begin{proof}
  Let us remark that $\forgetCol{\freeColor(H)} = H$.
  Because we know that $\mathsf{weakgb}{\freeColor(H)}$
  generates the same ideal as $\freeColor(H)$,
  and since $\forgetCol$ is a morphism,
  we conclude that 
  $\forgetCol(\mathsf{weakgb}{\freeColor(H)})$
  generates the same ideal as
  $\forgetCol(\freeColor(H)) = H$.
\end{proof}

Let us now prove that the resulting set in indeed
an \kl{equivariant Gröbner basis} of $\EqIdlGen{H}$.
To notice this, we will first prove one claim
regarding the computation of $\mathsf{weakgb}$.

\begin{lemma}
  \label{lem:weakgb-color-compatible}
  Let $H \subseteq \poly{\K}{\Indets}$
  and $V \subfin \Indets$.
  Let $H_\star \defined \freeColor(H)$
  and $H_V \defined \colorWith{V}(H)$.
  Then, 
  $\mathsf{weakgb}(H_V) \subseteq 
  \mathsf{weakgb}(H_\star)$,
  and $\mathsf{weakgb}(H_V)$ is 
  a \kl{$V$-compatible} subset.

  \begin{equation*}
   \mathsf{weakgb}(H_V)
   =
   \colorWith{V}(\forgetCol(\mathsf{weakgb}(H_\star)))
  \end{equation*}
\end{lemma}
\begin{proof}
  Let us first remark that 
  $\mathsf{weakgb}(H_V)$ is 
  a \kl{$V$-compatible} subset. To that end,
  notice that if $p$ and $q$ are \kl{$V$-compatible},
  then so is $\spoly{p}{q}$ by definition.
  Furthermore,
  if $p$ reduces to $q$, and $p$ is \kl{$V$-compatible},
  then $q$ is too, because its variables are included in
  those of $p$. Finally, it is clear that 
  the resulting set is
  \kl{$V$-compatible} by induction.

  For the inclusion, 
  we prove it by induction on the computation of the algorithm,
  the base case being trivial. 
  The set of \kl{$S$-polynomials} are clearly included,
  and the only thing to remark is that 
  if a \kl{$V$-compatible} polynomial is reducible
  in the freely colored setting, then it must be reducible
  by \kl{$V$-compatible} polynomials. 
\end{proof}

We now have all the ingredients to conclude our proof
\begin{lemma}
  \label{lem:strong-gb-correct}
  \Cref{alg:stronggb} is correct.
\end{lemma}
\begin{proof}
  Let $p \in \EqIdlGen{H}$,
  $H_\star = \freeColor(H)$,
  $V \defined \dom(p)$,
  $H_V \defined \colorWith{V}(H)$.
  We let $\Basis_\star = \mathsf{weakgb}(H_\star)$,
  and $\Basis_V = \mathsf{weakgb}(H_V)$.
  Finally, $\Basis = \forgetCol(\Basis_\star)$.

  Because of 
  \cref{lem:correct-gen-set}
  $p \in \EqIdlGen{\Basis}$,
  hence, one can write
  $p = \sum_{i \in I} a_i \monelt_i h_i$, 
  where $a_i \in \K$, $\monelt_i \in \mon{\Indets}$
  and $h_i \in \Basis$.

  Now, we obtain, thanks to 
  \cref{lem:weakgb-color-compatible}, the following
  equation:
  \begin{equation*}
    \colorWith{V}(p) \in 
    \IdlGen{\colorWith{V}(\Basis)}
    =
    \IdlGen{\colorWith{V}(\forget(\mathsf{weakgb}(H_\star)))}
    =
    \IdlGen{\mathsf{weakgb}(H_V)}
    = 
    \IdlGen{\Basis_V}
  \end{equation*}

  Because $\Basis_V$ is a 
  \kl{weak Gröbner basis},
  there exists an element $h \in \Basis_V$
  such that
  $\lm(h)$ divides $\lm(\colorWith{V}(p))$,
  and indeterminates of $h$ are smaller or 
  equal to the indeterminates of $\colorWith{V}(p)$.
  Since all the indeterminates of the latter
  are all colored with $\mathsf{first}$, this means
  that all variables in $h$ must be labelled with 
  $\mathsf{first}$. Because $h$ is a \kl{$V$-compatible}
  polynomial, this means that $h$ actually only uses
  variables from $\colorWith{V}(p)$.

  A fortiori, $\forget(h)$ only uses variables in $V$,
  and $\lm(h)$ divides $\lm(p)$. We conclude
  that $\Basis$ is an \kl{equivariant Gröbner basis}.
\end{proof}



% LTeX: language=en
%%!TEX root = ../atomic.asmart.tex
%
\section{Decision Procedure for the Equivariant Ideal Membership Problem}
\label{sec:refinements}

In the previous
\cref{sec:equivariant-grobner-basis},
we relied on a strong assumption on the set of indeterminates. Let us prove
that one can safely weaken this assumption to obtain a decision procedure for
the \kl{equivariant ideal membership problem}, leaving a very tight gap between
the hypothesis needed for the existence of \kl{equivariant Gröbner bases} and
the decidability of the \kl{equivariant ideal membership problem}.

The key idea of this section is to focus on a finite set $V \subfin \Indets$ of
indeterminates, and to consider what we call \kl{$V$-strong equivariant Gröbner
bases}, that are intuitively good enough to decide the \kl{equivariant ideal
membership problem} for all polynomials in $\poly{\K}{V}$, that is, for all
polynomials that only depend on the indeterminates in $V$.

\begin{definition}
  \label{def:strong-equiv-grob}
  Let $V \subseteq \X$ be a set of indeterminates.
  An orbit finite set $\Basis_V$
  of polynomials is called a \intro{$V$-strong equivariant Gröbner basis}
  of an \kl{equivariant ideal} $\idl$ 
  whenever $\Basis_V \cap \poly{\K}{V}$ is a
  \kl{Gröbner basis} of the ideal $\idl \cap \poly{\K}{V}$.
\end{definition}

Let us note that a set in as \kl{equivariant Gröbner basis}
if and only if it is a \kl{$V$-strong equivariant Gröbner basis} for
every finite set $V \subseteq \Indets$ of indeterminates.

\AP To compute a \kl{$V$-strong equivariant Gröbner basis} of an
\kl{equivariant ideal} $\idl$, we will compute a \kl{weak equivariant Gröbner
basis} of the ideal $\idl$ with respect to a new ordering of the indeterminates
and a new group action. Let us write $\IndetsV[V] \defined V \ordplus (\Indets
\setminus V)$, that is the totally ordered set of indeterminates having the
same elements as $\Indets$, but with the elements of $V$ ordered below all
other elements of $\Indets$. We will define a new group $\group_V$ that acts on
$\IndetsV[V]$ by restricting the group to elements $\gelem$ such that $\gelem
\cdot V = V$. In particular, the action of $\group_V$ on $\IndetsV[V]$ is
\kl(ord){compatible} with the ordering on $\IndetsV[V]$.

Let us call $\iota_V \colon \Indets \to \IndetsV[V]$ the identity map that maps
every element of $\Indets$ to itself. We extend this map as a morphism from
$\poly{\K}{\Indets}$ to $\poly{\K}{\IndetsV[V]}$. Beware that $\iota_V$ is not
an order-preserving map. Similarly, let us introduce $\upsilon_V \colon
\IndetsV[V] \to \Indets$ that maps every element of $\IndetsV[V]$ to itself,
and extends as a morphism from $\poly{\K}{\IndetsV[V]}$ to
$\poly{\K}{\Indets}$. Again, $\upsilon_V$ is not an order-preserving map. Using
these two constructions, we are ready to define the algorithm that computes a
\kl{$V$-strong equivariant Gröbner basis} of an \kl{equivariant ideal}:
$\mathsf{vsgb}(H, V) \defined \upsilon_V(\mathsf{weakgb}(\iota_V(H)))$.

Let us proceed similarly as in \cref{sec:equivariant-grobner-basis}
to prove that
$\mathsf{vsgb}$ is a computable \kl{equivariant function} that outputs a
\kl{$V$-strong equivariant Gröbner basis}, whenever $\Indets$ satisfies the
\kl{computability assumptions} and $(\mon[\om \ordplus \ordfin{1}]{\Indets},
\gdivleq[\group])$ is a \kl{well-quasi-ordering}.

\begin{lemma}
  \label{lem:strong-v-gb-algorithm}
  Let $H \subseteq \poly{\K}{\Indets}$,
  and $V \subfin \Indets$ be a finite set of indeterminates.
  Then, $\mathsf{vsgb}(H,V)$
  is a computable function that calls
  \kl{weakgb} on with valid inputs.
\end{lemma}
\begin{proof}
  For the algorithm to be computable, we will rely on the fact that 
  $\poly{\K}{\IndetsV[V]}$ is \kl{effectively oligomorphic} 
  and that the ordering on $\IndetsV[V]$ is also effective and 
  \kl(ord){compatible} with the action of $\group_V$.
  Then, following the same reasoning as in \cref{lem:colored-hypothesis-sat},
  we conclude that $\iota_V(H)$ is a computable orbit finite set
  of polynomials in $\poly{\K}{\IndetsV[V]}$, and that 
  $\upsilon_V(\kl{weakgb}(\iota_V(H)))$ is also computable and ouputs an 
  \kl{orbit finite set}.

  The only non-trivial check being that the set $(\mon{\IndetsV[V]},
  \gdivleq[\group_V])$ is a \kl{well-quasi-ordering}. To that end, let us
  consider an infinite sequence $\seqof{\monelt_i}[i \in \N]$ of elements of
  $\mon{\IndetsV[V]}$. Let us convert each $\monelt_i$ into a monomial
  $\monelt[n]_i$ in $\mon[\om \ordplus \ordfin{1}]{\Indets}$, by replacing the
  exponents of elements in $V$ by $\om$. Because $(\mon[\om \ordplus
  \ordfin{1}]{\Indets}, \gdivleq[\group])$ is a \kl{well-quasi-ordering}, there
  exists an infinite subsequence $\seqof{\monelt[n]_j}[j \in J]$ such that for
  all $i < j$ in $J$, we have a group element $\gelem_{i,j} \in \group$ such
  that $\gelem_{i,j}(\monelt[n]_i) \divleq \monelt[n]_j$. Let us now consider
  the action of $\gelem_{i,j}$ on the elements of $V$. Because $\gelem_{i,j}$
  must send an element of $V$ (having exponent $\om$ in $\monelt[n]_i$) to an
  element of $V$ (having exponent $\om$ in $\monelt[n]_j$), and since $V$ is a
  finite set, we can without loss of generality extract the subsequence $J$ so
  that $\gelem_{i,j}$ is the identity function on the variables of $V$ that
  appear in the monomials, for all $i < j$ in $J$.
  Now, because $\N^V$ with the pointwise ordering is a \kl{well-quasi-ordering}
  (as a finite product of $\N$ with the usual ordering), there exists $i < j$
  such that $\monelt[m]_i (x) \leq \monelt[m]_j (x)$ for all $x \in V$.
  As a consequence, 
  $\gelem_{i,j} (\monelt_i) \divleq \monelt_j$, and we have shown
  that $(\mon{\IndetsV[V]}, \gdivleq[\group_V])$ is a
  \kl{well-quasi-ordering}.
\end{proof}

\begin{lemma}
  \label{lem:correct-v-strong-gb}
  Let $H$ be an orbit finite set of polynomials in $\poly{\K}{\Indets}$,
  and let $V \subseteq \Indets$ be a finite set of indeterminates.
  Then, the result of $\mathsf{vsgb}(H,V)$ is a
  \kl{$V$-strong equivariant Gröbner basis} of $\EqIdlGen{H}$.
\end{lemma}
\begin{proof}
  Let $H_V = \iota_V(H)$, $\Basis_V
  = \mathsf{weakgb}(H_V)$, and $\Basis = \upsilon_V(\Basis_V)$.
  Let us first check that $\Basis$ is indeed a
  generating set of the ideal $\EqIdlGen{H}$. It is clear that
  $\upsilon_V(\IdlGen{\iota_V(H)}) = \IdlGen{H}$, because $\upsilon_V$ is a
  morphism of polynomial rings. Furthermore, because $\Basis_S$ is a \kl{weak
  equivariant Gröbner basis} of $\EqIdlGen[\group_V]{H_V}$, we have that
  $\IdlGen{\Basis_S} = \IdlGen{H_S}$, and therefore
  $\upsilon_V(\IdlGen{\Basis_S}) = \IdlGen{H}$.

  Let us prove that $\Basis$ is a \kl{$V$-strong equivariant Gröbner basis} of
  $\EqIdlGen{H}$. Let $p \in \poly{\K}{V} \cap \EqIdlGen{H}$, and let us
  consider a decomposition $\mathfrak{d} = \seqof{a_i \monelt_i h_i}[i \in I]$
  of $p$ such that $p = \sum_{i \in I} a_i \monelt_i h_i$, where $a_i \in \K$,
  $\monelt_i \in \mon{\Indets}$, and $h_i \in H$. Then, $\iota_V(p) = \sum_{i
  \in I} a_i \iota_V(\monelt_i) \iota_V(h_i)$, and $\iota_V(h_i) \in H_V$ for
  all $i \in I$, because $\iota_V$ is a morphism.
  Because $\Basis_S$ is a \kl{weak equivariant Gröbner basis} of
  $\IdlGen{H_V}$, there exists a decomposition
  $\mathfrak{d}'$ of $\iota_V(p)$ such that:
  \begin{equation*}
    \iota_V(p) = \sum_{j \in J} b_j \monelt'_j h'_j
    \quad ,
  \end{equation*}
  where $b_j \in \K$, $\monelt'_j \in \mon{\IndetsV[V]}$, and $h'_j \in
  \Basis_V$, and such that
  $\lm[\IndetsV](\iota_V(p)) = \lmdec[\IndetsV](\mathfrak{d}')$,
  and $\domdec(\mathfrak{d}') \subseteq \domdec(\iota_V(\mathfrak{d}))$.

  Remark that $\lm[\IndetsV](\iota_V(p)) = \iota_V(\lm[\Indets](p)) \in
  \mon{V}$, because all variables of $p$ are in $V$. As a consequence, we know
  that all the variables appearing in $\mathfrak{d}'$ are in $V$, since
  \kl{leading monomials} are computed using the \kl{reverse lexicographic
  ordering}, and variables in $V$ are ordered below all other variables in
  $\IndetsV[V]$. As a consequence, for the rest of the proof, we will assume
  that $\mathfrak{d}'$, $p$ and $\iota_V(p)$ are all in $\poly{\K}{V}$, where
  $V$ equipped with the ordering induced by $\Indets$ (which is the same as the
  one induced by $\IndetsV[V]$).

  Finally, let us consider a term $\monelt_i' h_i'$ appearing in
  $\mathfrak{d}'$, with maximal \kl{leading monomial}. We know that $\lm(p) =
  \lm(\monelt_i' h_i') = \monelt_i' \cdot \lm(h_i')$. In particular, $\lm(h_i')
  \divleq \lm(p)$, and $h_i' \in \poly{\K}{V}$.
  We have shown that $\Basis \cap \poly{\K}{V}$ is a \kl{Gröbner basis} of the
  ideal $\IdlGen{H} \cap \poly{\K}{V}$,
  i.e., that $\Basis$ is a \kl{$V$-strong equivariant Gröbner basis} of
  $\EqIdlGen{H}$.
\end{proof}


As a consequence, we have proven that \cref{thm:decide-equiv-ideal-mem}
holds
using the algorithm \kl{vsgb}. Furthermore, notice that the assumption of
\cref{thm:decide-equiv-ideal-mem}
is
that $(\mon[\om \ordplus \ordfin{1}]{\Indets}, \gdivleq[\group])$ is a
\kl{well-quasi-ordering}, which very close to the necessary condition that
$(\mon{\Indets}, \gdivleq[\group])$ being a \kl{well-quasi-ordering}, and is
weaker than the assumption used in \cref{thm:compute-egb}.

\todo[inline]{This looks false}
\begin{remark}
Let us briefly argue that using similar techniques, one can also improve the
computation of \kl{equivariant Gröbner bases} to obtain a decision procedure
under the assumption that $(\mon[\om \ordplus \om]{\Indets}, \gdivleq[\group])$
is a \kl{well-quasi-ordering}. 
\end{remark}
\begin{proof}
  The main idea is that one can adapt the proof of
  \cref{lem:strong-gb-correct} to only ever consider polynomials with
  colored variables that are \kl{$V$-compatible} to some finite set $V$ of
  indeterminates. This way, every variable in the polynomial must have a
  well-defined color. The algorithm of \kl{weakgb} can then be adapted to only
  consider such polynomials: the only operation that can create
  non-\kl{$V$-compatible} polynomials is the computation of \kl{$S$-polynomials},
  which can be changed to ignore all non-\kl{$V$-compatible} polynomials.
\end{proof}

%!TEX root = ../atomic.asmart.tex
% LTeX: language=en
\section{Undecidability Results}
\label{sec:undecidability}

\todo[inline]{Aliaume: update section intro}

In this section, we will show that under certain conditions, the
\kl{equivariant ideal membership problem} is undecidable. We aim to show that
it is the case when we assume the usual effectiveness conditions on the group
action, but we do not assume that $(\mon{\Indets}, \gdivleq)$ is a
well-quasi-ordering. Beware though that there are some pathological cases where
the \kl{equivariant ideal membership problem} is undecidable even when
$(\mon{\Indets}, \gdivleq)$ is not a well-quasi-ordering, as illustrated by the
following \cref{ex:non-wqo-undecidable}.

\begin{example}
  \label{ex:non-wqo-undecidable}
  Let $\Indets = \{x_1, x_2, \ldots\}$ be an infinite set of indeterminates,
  and let $\group$ be trivial group acting on $\Indets$.
  Then, the \kl{equivariant ideal membership problem} is decidable.
\end{example}
\begin{proof}
  Because the group is trivial, whenever one provides a finite set
  $H$ of generators of an \kl{equivariant ideal} $I$, one can
  in fact work in $\poly{\K}{V}$, where $V$ is the set of indeterminates
  that appear in $H$.
  Then, the \kl{equivariant ideal membership problem} is reduces to 
  the \kl{ideal membership problem} in $\poly{\K}{V}$, which is decidable.
\end{proof}

Avoiding pathological cases such as the one in \cref{ex:non-wqo-undecidable},
we can give one positive and one negative examples that are designed to
illustrate the results we will present in this section. On the positive side,
one can consider $\Indets_\Q$, the set of all rationals equipped with their
usual ordering, and the group of all bijections of $\Indets_\Q$ that preserve
the ordering (\cref{ex:q-is-super-wqo}). While on the negative side, one can
consider $\Indets_\Z$, the set of all integers equipped with their usual
ordering, and the group of all shifts of integers (\cref{ex:z-is-not-wqo}).

\paragraph{Monomial Reachability}
The undecidability results we will present in this section regarding the
\kl{equivariant ideal membership problem} will use the polynomials in a very
limited way: we will only need to consider \emph{monomials}, and there will
even be a bound on the maximal exponent used. Before going into the details of
our reductions, let us first introduce an intermediate problem that will be
easier to work with: the (equivariant) \kl{monomial reachability problem}. 

\begin{definition}
  \label{def:mon-rewrite-system}
  A \intro{monomial rewrite system} is a finite set of pairs of the form
  $\set{\monelt, \monelt'}$ where $\monelt, \monelt' \in \mon{\Indets}$.
  The \intro{monomial reachability problem} is the problem of deciding whether
  there exists a sequence of rewrites that transforms $\monelt_s$ into $\monelt_t$
  using the rules of a monomial rewrite system $R$, where
  a \intro(monrew){rewrite step} is a pair of the form
  \begin{equation*}
    \monelt[n] (\gelem \cdot \monelt)
    \leftrightarrow_R 
    \monelt[n] (\gelem \cdot \monelt')
    \text{ if } \set{\monelt, \monelt'} \in R
    \text{ and } \gelem \in \group
    \quad .
  \end{equation*}
\end{definition}

\begin{example}
  \label{ex:mon-rewrite-system}
  Let $\Indets = \N$ and $\group$ be the set of all bijections of $\Indets$.
  Then, the rewrite system $x_1^2 x_2^2 \leftrightarrow_R x_1^2$
  satisfies $\monelt \leftrightarrow_R^* x_1^2$ if and only if 
  $\monelt$ has all its exponents that are multiple of $2$.
\end{example}

The following \cref{lem:mon-rewrite-red-membership} shows that the \kl{monomial
reachability problem} can be reduced to the \kl{equivariant ideal membership
problem}, and follows the exact same reasoning as in the case of finitely many
indeterminates \cite{MAME82}. This reduction was also noticed in \cite[Theorem
64]{GHOLAS24}.


\begin{lemma}[label=lem:mon-rewrite-red-membership,ref=lem:mon-rewrite-red-membership]
  One can solve the \kl{monomial reachability problem}
  provided that one can solve the \kl{equivariant ideal membership problem}.
\end{lemma}

In order to show that the \kl{equivariant ideal membership problem} is
undecidable, it is therefore enough to show that the \kl{monomial reachability
problem} is undecidable. To that end, we will encode the Halting problem of a
Turing machine. There are two main obstacles to overcome: first, the
reversibility of the rewriting system, which can be (partially) solved by
considering \emph{reversible} Turing machines; and second, the fact that the
configurations of the Turing machine cannot staightforwardly be encoded as
monomials due to the commutativity of the multiplication.
To overcome the second issue, we will use the following notion of 
\kl{word encoding}.

\paragraph{Structures Containing Paths.} In this section, we will assume that
the set of indeterminates $\Indets$ contains an \intro(of){infinite path} $P
\defined \seqof{x_i}[i \in \N] \subseteq \Indets$, that is, a set $P$ of
indeterminates such that $\group$ acts on $P$ as a translation of the indices.
Formally, we require that, for all $\gelem \in \group$ and for all segments $J
\defined [k,l] \subfin \N$ such that $\gelem \cdot x_j \subseteq P$ for all $j
\in J$, there exists $n \in \N$ such that $\gelem \cdot x_j = x_{j + n}$ for
all $j \in J$ ; and there exists a $\gelem_{+1} \in \group$ such that
$\gelem_{+1} \cdot x_i = x_{i + 1}$ for all $i \in \N$. The prototypical
example being the set of all indeterminates $\Indets = \Z$ equipped with the
group $\group$ of all shifts.

\begin{remark}
  \label{rem:not-wqo}
  If there exists a \kl{word encoding} for a binary alphabet $\Sigma$, then
  $(\mon{\Indets}, \gdivleq)$ is not a \kl{well-quasi-ordering}.
  Indeed, the infix ordering relation on words over a binary alphabet is
  not a \kl{well-quasi-ordering}, as shown by the infinite antichain
  $\setof{ a b^n a}{n \in \N }$,
  and the \kl{word encoding} $\wenc{\cdot}$ is an order embedding of
  the infix ordering relation on words over $\Sigma$ into $(\mon{\Indets}, \gdivleq)$.
\end{remark}


\AP Let us fix a binary alphabet $\Sigma \defined \set{a,b}$, and let us define
a function $\intro*\wenc{ \cdot} \colon \Sigma^* \to \mon{\Indets}$, where
$\Sigma$ is a finite alphabet, that encodes a word $u \in \Sigma^*$ as a
monomial. Namely, we define inductively $\wenc{\varepsilon} \defined 1$,
$\wenc{a u} = x_0^3 x_1^2 x_2^1 x_3^3 (\gelem_{+4} \cdot \wenc{u})$ and
$\wenc{b u} = x_0^3 x_1^1 x_2^2 x_3^3 (\gelem_{+4} \cdot \wenc{u})$ for all $u
\in \Sigma^*$. Let us remark that \kl{monomial rewriting} applied on \kl{word
encodings} can simulate (reversible) string rewriting on words of a given size.

\begin{lemma}
  \label{lem:word-encoding-string-subst}
  Let $u,v,w \in \Sigma^*$ be three words, such that $|u| = |v|$,
  and let $\monelt[n] \in \mon{\Indets}$ be a monomial.
  The following are equivalent:
  \begin{enumerate}
    \item There exists $\gelem \in \group$
      such that $\wenc{w} = \monelt[m] (\gelem \cdot \wenc{u})$
      and $\monelt[n] = \monelt[m] (\gelem \cdot \wenc{v})$,
    \item There exists $x, y \in \Sigma^*$
      such that $x u y = w$ and $\wenc{x v y} = \monelt[n]$.
  \end{enumerate}
\end{lemma}
\begin{proof}
  \textbf{aliaume todo}
\end{proof}

Note that without loss of generality, thanks to
\cref{lem:word-encoding-string-subst}, we can assume that the alphabet is any
finite set, using a suitable unambiguous encoding of the alphabet into a binary
alphabet \cite{BERST09}. This bigger alphabet size will simplify the statement
and proof of the following \cref{lem:reversible-machine}, which explains how
to simulate a reversible Turing machine using \kl{monomial rewriting}. Given a
reversible Turing machine $M$ with a finite set $Q$ of states and tape alphabet
$\Sigma$, we will consider the following alphabet $\Gamma \defined \set{
\triangleleft, \triangleright } \times \set{ \text{pre}, \text{run},
\text{post} } \uplus Q \uplus \Sigma \uplus \set{ \square, \square_1, \square_2}$. The letter
$\square$ is a blank symbol, and the letters $\triangleleft$ and
$\triangleright$ are used to delimit the beginning and the end of the tape,
with some extra ``phase information''.

\begin{lemma}
  \label{lem:reversible-machine}
  There exists a
  \kl{monomial rewrite system} $R_M$ such that the following
  are equivalent for every $n \geq 1$:
  \begin{enumerate}
    \item $\wenc{ \triangleright^{\text{pre}} \square^n 
                  \triangleleft^{\text{pre}}
     } \leftrightarrow_{R_M}^* 
     \wenc{ \triangleright^{\text{post}} \square^n 
                  \triangleleft^{\text{post}} }$,
      \item $M$ halts on the empty word using a tape bounded by $n-1$ cells.
  \end{enumerate}
  Furthermore, every monomial that is 
  reachable from $\wenc{ \triangleright^{\text{pre}} \square^n \triangleleft^{\text{pre}} }$
  or $\wenc{ \triangleright^{\text{post}} \square^n \triangleleft^{\text{post}} }$
  is the image of a word of the form
  $\wenc{\triangleright^{\text{run}} u \triangleleft^{\text{run}}}$  
  where $u \in (Q \uplus \Sigma \uplus \square)^n$.
\end{lemma}
\begin{proof}
  The rewrite system simply acts on the tape of the reversible Turing machine 
  using blank symbols. Because transitions of the reversible Turing machine
  are substitutions of strings having the same size if one does not create new
  tape cells, the rewriting system can straigthforwardly simulate the 
  substitutions because of \cref{lem:word-encoding-string-subst}.
  To this monomial rewriting system, we add two rules,
  respectively of the form
  $\wenc{ \triangleright^{\text{pre}} \square^n \triangleleft^{\text{pre}} }
  \leftrightarrow_{R_M}^*
  \wenc{ \triangleright^{\text{run}} q_0 \square^{n-1} \triangleleft^{\text{run}} }$
  and 
  $\wenc{ \triangleright^{\text{post}} \square^n \triangleleft^{\text{post}} }
  \leftrightarrow_{R_M}^*
  \wenc{ \triangleright^{\text{run}} q_f \square^{n-1} \triangleleft^{\text{run}} }$,
  where $q_0$ is the initial state of the Turing machine $M$ and $q_f$ is its
  final state.
  This is not problematic because one can 
  simply write $\wenc{\triangleright^{\text{run}} q_0} (\gelem \cdot \wenc{\triangleleft^{\text{run}}})$
  for a suitable $\gelem \in \group$, in order
  to ignore the number of blank symbols in the tape,
  and because we can assume that the reversible Turing machine
  starts with a clean tape and ends with a clean tape.
\end{proof}

\Cref{lem:reversible-machine} shows that one can simulate the runs, provided we
know in advance the maximal size of the tape used by the reversible Turing
machine. The key ingredient that remains to be explained is how one can start
from a finite monomial $\monelt$ and create a tape of arbitrary size using a
\kl{monomial rewrite system}. Note that because the set of indeterminates can
very well contain multiple copies of an \kl(of){infinite path} $P$, we will not
be able to guarantee that we create exactly the tape $\wenc{
\triangleright^{\text{pre}} \square^n \triangleleft^{\text{pre}} }$ for some $n
\in \N$, but rather that we can find some equivalent monomial up to the action
of $\group$.

\begin{lemma}
  \label{lem:tape-creation}
  There exists a \kl{monomial rewrite system} $R_\text{pre}$
  such that for every monomial $\monelt \in \mon{\Indets}$, the following are
  equivalent:
  \begin{enumerate}
    \item $\wenc{ \triangleright^{\text{pre}} \square \triangleleft^{\text{pre}}} 
      \leftrightarrow_{R_\text{pre}}^* 
      \monelt$
    \item there exists $n \in \N$ and $\gelem \in \group$, such that
      $\gelem \cdot \monelt = \wenc{ \triangleright^{\text{pre}} \square^n 
                        \triangleleft^{\text{pre}} }$.
  \end{enumerate}
  Similarly, there exists a \kl{monomial rewrite system} $R_\text{post}$
  with analogue properties.
\end{lemma}
\begin{proof}
  The rewrite system is going to have two phases.
  In a first phase, we will consider the rules
  $\wenc{ \square \triangleleft^{\text{pre}} } \leftrightarrow_{R_\text{pre}}
   \wenc{ \square \square \triangleleft^{\text{pre}} }$.
  This is \emph{almost} leading to the desired \kl{word encoding}.
  The only thing left to check is that we do not reuse indeterminates when 
  creating new letters.
  To ensure that we did not, we will implement the 
  Floyd cycle-finding algorithm, which uses two pointers moving at different speeds.
  To that end, we will use two new blank symbols $\square_1$ and $\square_2$,
  and the rules 
  $\wenc{ \square_1 \square } ( \gelem \cdot \wenc{ \square_2 \square \square })
  \leftrightarrow_{R_\text{pre}}
  \wenc{ \square \square_1 } (\gelem \cdot \wenc{ \square \square \square_2 })$,
  where $\pi$ is a permutation of the indeterminates that ensures that the
  indetermitates of the two words are distinct.
  Finally, one adds the possibility to switch to the second phase,
  and end the second phase when pointer $\square_2$ reaches the end of the tape.

  \todo[inline]{Aliaume: complete the proof}
\end{proof}

\csname thm:undecidable-paths\endcsname
\begin{proof}
  It suffices to combine the rewriting systems $R_M$, $R_\text{pre}$ and 
  $R_\text{post}$ by taking their union.
\end{proof}


\begin{remark}
  \label{rem:more-generally}
  The undecidability result of \cref{cor:undecidability} can be generalized to
  any set of indeterminates in which one can encode words over a binary alphabet,
  and for which there is a \kl{monomial rewrite system} that can
  produce arbitrary long words.
  We strongly conjecture that this is the case for 
  the \kl{infinite dimensional vector space}
  \todo{cite ref}
\end{remark}

% LTeX: language=en
%%!TEX root = ../atomic.asmart.tex
%
\section{Relation to Existing Results and Examples}
\label{sec:examples}

In this section, we are interested in the consequences of our decidability
results. First, we will provide numerous examples of sets of indeterminates
that satisfy our \kl{computability assumptions} as well as our
\kl{well-quasi-ordering} conditions. Then, we will discuss how our results can
be applied to solve various decidability problems in theoretical computer
science.

\todo[inline]{for arka: integrate these examples}
\begin{example}
  \label{ex:q-is-super-wqo}
  The set $(\mon[Y]{\Indets_\Q}, \gdivleq)$ is a \kl{well-quasi-ordering} whenever $Y$ is
  one, and in particular $\Indets_\Q$ satisfies the computability assumptions and
  the termination assumptions of both \cref{thm:compute-egb}
  and
  \cref{thm:decide-equiv-ideal-mem}.
\end{example}
\begin{proof}
  Let $\seqof{\monelt_i}[i \in \N]$ be a sequence of monomials in
  $\mon[Y]{\Indets_\Q}$. Let us write each monomial $\monelt_i$ as
  a finite word $w_i$ over the alphabet $Y$, by writing all the exponents in the order 
  prescribed by the indeterminates.
  Because $Y$ is a \kl{well-quasi-ordering}, the set of all finite words over $Y$ is
  \kl{well-quasi-ordered} by the \emph{scattered subword} relation \cite{HIG52}.
  Now, if $w_i$ is a scattered subword of $w_j$, then
  $\monelt_i \gdivleq \monelt_j$, by choosing a suitable $\gelem \in \group$.
\end{proof}

\begin{example}
  \label{ex:z-is-not-wqo}
  The set $(\mon[Y]{\Indets_\Z}, \gdivleq)$ is not a \kl{well-quasi-ordering} whenever $Y$ is
  contains two distinct elements.
  In particular, $\Indets_\Z$ does not satisfy the termination assumptions of
  \cref{thm:compute-egb} and \cref{thm:decide-equiv-ideal-mem}.
\end{example}
\begin{proof}
  Assume that $Y$ has two distinct elements $a$ and $b$, and let us assume without loss of generality
  that $a \leq b$. The sequence of monomials 
  $\monelt_i \defined x_1^{b} x_2^{a} \cdots x_{i-1}^{a} x_i^{b}$
  forms an infinite antichain in $(\mon[Y]{\Indets_\Z}, \gdivleq)$.
  Indeed, if $\monelt_i \gdivleq \monelt_j$ for some $i < j$, then
  without loss of generality, $\gelem_i (x_1) = x_1$, and 
  $\gelem_i (x_i) = x_j$, because these are the only ones that can be 
  equipped with a large enough exponent.
  Therefore, $\gelem_i = \mathrm{id}$ since the group only contains translations.
  However, this implies that the exponent of $x_j$ in $\monelt_j$ is at most $a$,
  which contradicts the fact that it is $b$.
\end{proof}

\subsection{Crafting sets of indeterminates}
%
We give some interesting examples of group actions $\group \actson \Indets$ and discuss which of them satisfy the necessary condition of \Cref{thm:compute-egb}.
We also describe operations to build new group actions from old ones,
and discuss which of them preserve this condition.

\arka{Did we write somewhere that divisibility is same as labelled embedding}

\todo[inline]{for arka : add citations}

In all of our examples $\Indets$ is a set with some structure, described by some relations and functions on that set,
and $\group$ is the group $\aut{\Indets}$ of all automorphisms (i.e.\ bijections that preserve the structure) of $\Indets$.
To show that $(\mon[Y]{\Indets},\gdivleq)$ is a WQO we use the following strategy:
we define an equivariant one-to-one function $f_{\Indets} : \mon[Y]{\Indets} \to W_{\Indets}$ to some well-known well-quasi-ordered set $(W_{\Indets},\leq)$ such that $f_{\Indets}(\monelt[p]) \leq f_{\Indets}(\monelt[q])$ if and only if $\monelt[p] \gdivleq \monelt[q]$.
The reason why this strategy works is because in all of our examples, $\Indets$ will be a \intro{homogeneous} structure (\arka{cite wikipedia and macpherson survey}),
and $f_{\Indets}$ essentially maps an element $\monelt[p]$ of $\mon[Y]{\X}$,
thought as a finite induced substructure labelled by $Y$,
to its isomorphism class.
%
\begin{example}\label{ex:eq atoms}
Let $\A$ be an infinite set without any additional structure other than the equality relation.
Then $\aut{\A}$ is the set of all bijections of the set $\A$.
This action $\aut{\A} \actson \A$ does not preserve any linear order on $\A$.
However, $(\mon[Y]{\A}, \gdivleq)$ is a \kl{WQO} whenever $Y$ is a \kl{WQO}.
To see this define $f_{\A}$ to be the function which takes $\monelt[p]$ to the multiset of its coefficients.
For example, $f_{\A}(a^{y}a_2^{y'}c^{y} = \{y,y',y\}$ for every $a,b,c\in\A$ and $y,y'\in Y$.
We leave it to the reader to prove that for every $\monelt[p],\monelt[q]\in\mon[Y]{\A}$ we have $\monelt[p] \gdivleq[\aut{\A}] \monelt[q]$ if and only if $f(\monelt[p])$ is smaller than or equal to $f(\monelt[q])$ in the multiset ordering (\arka{cite MPRI lecture notes?}).
Note that the latter is a \kl{WQO}.
\end{example}
%
\begin{example}\label{ex:dlo}
Let $\D$ be a dense linear order without endpoints(\arka{cite wiki}).
Then $\aut{\D}$ is the set of all order preserving bijections of $\D$.
By definition, the action $\aut{\D} \actson \D$ preserves the linear order on $\D$.
Moreover, $(\mon[Y]{\A}, \gdivleq)$ is a \kl{WQO} whenever $Y$ is a \kl{WQO}.
Using the same strategy as \Cref{ex:eq atoms},
we define $f_{\D}$ to be the map which takes $\monelt[p]\in\mon[Y]{\D}$ to the corresponding word in $Y^*$ which we get by writing all the exponents in the order prescribed by the indeterminates.
For example, for every $a < b < c \in \D$ and $u,v\in Y$,
$f_{\D}(a^u b^v c^u) = uvu$.
We leave it to the reader to check that for every $\monelt[p],\monelt[q]\in\mon[Y]{\D}$ we have $\monelt[p] \gdivleq[\aut{\A}] \monelt[q]$ if and only if $f(\monelt[p])$ is smaller than or equal to $f(\monelt[q])$ in the scattered subword ordering, which is a \kl{WQO} due to Higman's lemma \arka{add citation}.
\end{example}
%
\begin{example}\label{ex:dlo}
Let $\calZ$ be the set of integers ordered by the usual ordering.
Then $\aut{\calZ}$ is the set of all order preserving bijections of $\D$.
Note that every order preserving bijection of the set $\calZ$ is a translation $n \mapsto n + c$ for some constant $c\in\calZ$.
By definition, the action $\aut{\calZ} \actson \calZ$ preserves the linear order on $\D$.
However, $(\mon[Y]{\calZ}, \gdivleq[\aut{\calZ}])$ is not a \kl{WQO} even when $Y$ is a singleton.
An example of an infinite antichain is the set $\setof{a b}{b\in\calZ\setminus\{a\}}$, for any fixed $a\in\calZ$.
\end{example}

\begin{example}\label{ex:rado}
Let $\G$ be the Rado graph (\arka{cite wiki}).
Then $\aut{\G}$ is the set of all automorphisms of the graph $\G$.
\end{example}
%
\begin{example}\label{ex:rado}
Let $\V$ be an infinite dimensional vector space over $\ftwo$.
Then $\aut{\V}$ is the set of all linear automorphisms.
\end{example}
%

\paragraph{Sets with atoms.}


\todo[inline]{Say that if one starts with atoms and equality, then we can only 
  have dimension 1, and that this is the case of the rationals.}

\paragraph{Relational structures.} Let $\mathbb{A}$ be an infinite relational
structure with finitely many relations. Then, one can consider the set of
polynomials $\poly{\K}{\mathbb{A}}$, where indeterminates are elements of the
universe of $\mathbb{A}$. The group of all automorphisms of $\mathbb{A}$ (i.e.,
bijections of the universe that preserve the relations) acts on
$\poly{\K}{\mathbb{A}}$ by permuting the indeterminates.

Natural examples are polynomials whose indeterminates are indexed by the
natural numbers (with inequality), or the rationals (with inequality). In this
setting, \kl{effective oligomorphcity} means that \todo{do it}. The fact that
$(\mon{\mathbb{A}}, \gdivleq)$ is a well-quasi-ordering corresponds to ordering
\emph{finite substructures} of $\mathbb{A}$ by the \emph{labelled induced
substructure} relation, and asking whether the class obtained is
well-quasi-ordered. This is a well-studied question in graph theory, where a
conjecture of Pouzet states that this holds with two labels if and only if it
holds for every ordinal. In particular, for such structures, it is therefore
conjectured that $(\mon{\Indets}, \gdivleq)$ is a well-quasi-ordering if and
only if $(\mon[\om \ordplus \ordfin{1}]{\Indets}, \gdivleq)$, $(\mon[\om
\ordplus \om]{\Indets}, \gdivleq)$, and $(\mon[\om^2]{\Indets}, \gdivleq)$ are
well-quasi-orderings too.
\arka{So we can use any well-ordered set of labels?} 
\todo[inline]{
  Cite \cite{POUZ72},
  \cite{DRT10} for the conjecture.
}


\paragraph{On reducts of structures.} \AP Let $\sigma$ be a finite relational
signature, and $\tau \subseteq \sigma$ be another finite relational signature.
Let $\mathbb{A}, \mathbb{B}$ be respectively a $\sigma$ and a $\tau$ structure. We say
that \intro{$\mathbb{B}$ is a reduct of $\mathbb{A}$} when $\mathbb{B}$ is
obtained from $\mathbb{A}$ by keeping the same universe, and relations. It was noted by \cite[Lemma 13]{GHOLAS24} that in
this case, the \kl{equivariant Hilbert basis property} transfers from
$\mathbb{A}$ to $\mathbb{B}$. Let us briefly argue that this transfer holds too
for our \cref{thm:decide-equiv-ideal-mem,thm:compute-egb}.

\begin{lemma}
  \label{lem:reducts-equiv-hilbert}
  Let $\mathbb{A}$ be a relational structure, let $\mathbb{B}$ be a 
  \kl(struct){reduct} of $\mathbb{A}$. Then, if $\mathbb{A}$ satisfies the
  hypotheses of \cref{thm:decide-equiv-ideal-mem},
  then one can decide the \kl{equivariant ideal membership problem} for
  $\poly{\K}{\mathbb{B}}$. Similarly, 
  if $\mathbb{A}$ satisfies the hypotheses of
  \cref{thm:compute-egb}, then one can compute an
  \kl{equivariant Gröbner basis} of an
  \kl{equivariant ideal} of $\poly{\K}{\mathbb{B}}$.
\end{lemma}
\begin{proof}
  \todo[inline]{Just write it, and it works.}
\end{proof}

\AP 
As a consequence, one can apply our results to structures that are not equipped 
with an ordering, because one can always consider the 

\todo[inline]{Talk about $\N$ I guess.}



\subsection{Topological Well-Structured Transition Systems}

The fact that (finite control) systems performing polynomial computations can
be verified folloms from the theory of \kl{Gröbner bases} on finitely many
indeterminates \cite{MULSEI02,BEDUSHWO17}. There were also numerous
applications to automata theory, suchz as deciding whether a weighted automaton
could be determinised (resp. desambiguated) \cite{BESM23,PUSM24}. We refer the
readers to a nice survey recapitulating the successes of the so-called
``Hilbert method'' automata theory \cite{BOJAN19}. In this section, we will
show how our results can be used in the framework of \kl{topological
well-structured transition systems}, that is an abstract setting to verify
infinite state systems that is more general than computing polynomial
invariants \cite{JGL07,JGL10}. As a consequence, we will show how our results
can be used to decide the zeroness of \kl{orbit finite polynomial automata}, a
new model of computation that generalizes \kl{polynomial automata}
\cite{BEDUSHWO17} and \kl{orbit finite weighted automata} \cite{BOKLMO21}.

\paragraph{Topological Well-Structured Transition Systems.} \AP The notion of
\kl{topological well-structured transition system} was introduced by
Goubault-Larrecq in \cite{JGL07}, noticing that the pre-existing notion of
\kl{Noetherian space} could serve as a topological generalisation of
\kl{well-quasi-orderings}, for which the celebrated decision procedures on
\kl{well-structured transition systems} can be applied. In particular,
Goubault-Larrecq used such systems to verify properties of \emph{polynomial
programs} computing over the rationals, that can communicate over lossy
channels using a finite alphabet \cite{JGL10}. 
\AP A \intro{topological space} is a set $X$ equipped with a collection $\tau$
of subsets of $X$ that is stable under finite intersections and arbitrary
unions.\footnote{In particular, $\tau$ contains the empty set and $X$ itself.}
In a \kl{topological space}, elements of $\tau$ are called \intro{open
subsets}, while their complements (in $X$) are called \intro{closed subsets}. A
\kl{topological space} is \intro(space){Noetherian} when, for every sequence
$\seqof{U_i}[i \in \N]$ of \kl{open subsets}, there exists $n \in \N$ such that
$\bigcup_{i \in \N} U_i = \bigcup_{i \leq n} U_i$. We refer the readers to the
book \cite{JGL13} for a comprehensive introduction to \kl{Noetherian spaces}
and their usage in theoretical computer science. Let us briefly argue that
\kl{Noetherian spaces} generalize \kl{well-quasi-orders} in
\cref{ex:well-quasi-orders-are-noeth}, and encode the
\kl{Hilbert basis property} in \cref{ex:polynomials-noetherian}.

\begin{example}[ see \cite{JGL13}]
  \label{ex:well-quasi-orders-are-noeth}
  Let $(X, \leq)$ be a quasi-ordered set.
  Then, the set $X$ equipped with the \kl{topology} having 
  as \kl{open subsets} the upwards-closed subsets of $X$ is \kl(space){Noetherian}
  if and only if $(X, \leq)$ is \kl{well-quasi-ordered}.
\end{example}

\begin{example}[ see \cite{JGL13}]
  \label{ex:polynomials-noetherian}
  Let $\K$ be a field, and let $n \in \N$.
  The space $\K^n$ equipped with the \kl{Zariski topology}
  \kl(space){Noetherian}; where the \intro{Zariski topology}
  is the topology whose \kl{closed subsets} are finite unions of sets
  of the form $\setof{ \vec{x} \in \K^n}{ \forall p \in \idl, p(\vec{x}) = 0}$,
  where $\idl$ is an \kl{ideal} of $\poly{\K}{x_1, \dots, x_n}$.
\end{example}

\AP The advantage of \kl{Noetherian spaces} over \kl{well-quasi-orderings} and
\kl{Noetherian rings} is that they generalize both and can be \emph{combined}:
\kl{Noetherian spaces} are closed under finite sums, finite products,
considering finite words, considering finite trees, and many more \todo{cite}.
As a consequence, they provide a versatile tool to express the set of states of
a system, ensuring that a strong termination property holds.

\AP A \intro{topological well-structured transition system} with alphabet
$\Sigma$ is a \kl{topological space} $(X, \tau)$, equipped with a transition
function $\delta \colon X \times \Sigma \to X$, such that the following
properties hold: for every $U \in \tau$, $\mathrm{pre}^\exists(U)$, the set of
states $x \in X$ such that there exists $a \in \Sigma$ with $\delta(x, a) \in
U$, is an \kl{open subset}. Equivalently, the set $\mathrm{pre}^\forall(E)$ of
states $x \in X$ such that for every $a \in \Sigma$, $\delta(x, a) \in E$ is a
\kl{closed subset} of $X$ whenever $E$ is itself a \kl{closed subset} of $X$.
The natural decition problem for \kl{topological well-structured transition
systems} is the following \intro{open reachability problem} is decidable: given
an initial state $x_0 \in X$ and an \kl{open subset} $U \in \tau$, is it true that
there exists a word $w \in \Sigma^*$ such that $\delta^*(x_0, w) \in U$? The
prototypical algorithm to solve this problem is the following \intro{backward
algorithm}: start with $U_0 \defined U$, and iteratively compute $U_{i+1}
\defined U_i \cup \mathrm{pre}^\exists(U_i)$ until $U_i = U_{i+1}$, then check
whether $x_0 \in U_\text{last}$.
There are easy-to-state sufficient conditions  for such an algorithm to be computable and terminate:
\begin{enumerate}
  \item One is equipped with an effective representation of open subsets,
    where one is able to test equality of open subsets, compute unions of open subsets, and test 
    membership of a point in an open subset.
  \item The pre-image function $\mathrm{pre}^\exists$ is computable, i.e., one can
    compute the set $\mathrm{pre}^\exists(U)$ for every open subset $U$.
  \item The space $(X, \tau)$ is \kl{Noetherian}. 
\end{enumerate}

\AP Our \cref{cor:equivariant-ideals-computations} shows that
under some assumptions on $\Indets$, the set of finitely supported functions
$\Indets \to \K$ is a \kl{Noetherian space} with respect to the
\intro{equivariant Zariski topology}, i.e., the topology whose \kl{closed subsets}
are finite unions of sets of the form $E_{\idl} \defined \setof{f \in
\K^{(\Indets)}}{\forall p \in \idl, p(f) = 0}$, where $\idl$ is an
\kl{equivariant ideal} of $\poly{\K}{\Indets}$. Furthermore, we have an
effective representation of the \kl{closed subsets} in this topology, using
\kl{equivariant Gröbner bases} of \kl{equivariant ideals}. In particular, the
theory of \kl{topological well-structured transition systems} can be applied to
systems whose state space contains ``named registers'' that contain numbers and
are updated by polynomial functions.


\paragraph{Consequences for orbit-finite polynomial automata.} Before
discussing the case of orbit finite polynomial automata, let us recall in
\cref{ex:polynomial-automata}
the
setting of \kl{polynomial automata} in the classical case, as studied by
\cite{BEDUSHWO17}, with techniques that dates back to \cite{MULSEI02}. We will
formally state in \cref{lem:zeroness-problem-polynomial-automata}
how the classical problem of deciding the \kl(pa){zeroness} of a
\kl{polynomial automaton} is a special case of the \kl{open reachability
problem} for \kl{topological well-structured transition systems}. Beware that
these are consequences of \cite[Section 6, Polynomial Games]{JGL10}.

\begin{definition}[Polynomial automata, as described in \cite{BEDUSHWO17}]
  \label{ex:polynomial-automata}
  A \intro{polynomial automaton} is a tuple $A \defined (Q, \Sigma, \delta, q_0, F)$,
  where $Q = \K^n$ for some finite $n \in \N$, $\Sigma$ is a finite alphabet,
  $\delta \colon Q \times \Sigma \to Q$ is a transition function such that 
  $\delta(\cdot,a)_i$ is a polynomial in the indeterminates $q_1, \dots, q_n$ for every
  $a \in \Sigma$ and every $i \in \set{1, \dots, n}$, $q_0 \in Q$ is the initial state,
  and $F \colon Q \to \K$ is a polynomial function describing the final result of the 
  automaton.
  The \intro{zeroness problem for polynomial automata} is the following decision problem:
  given a \kl{polynomial automaton} $A$, is it true that 
  for all words $w \in \Sigma^*$, the polynomial $F(\delta^*(q_0, w))$ is zero?
\end{definition}

\begin{lemma}
  \label{lem:zeroness-problem-polynomial-automata}
  The \kl{zeroness problem for polynomial automata} is a special case of the
  \kl{open reachability problem} for \kl{topological well-structured transition systems}.
\end{lemma}
\begin{proof}
  Let $A = (Q, \Sigma, \delta, q_0, F)$ be a \kl{polynomial automaton}.
  We consider the topological space $(Q, \tau)$, where $\tau$ is the
  \kl{Zariski topology} on $\K^n$.
  Let $\idl$ be an \kl{ideal} of $\poly{\K}{x_1,\dots,x_n}$ generated by the polynomials
  $p_1, \dots, p_m$,
  and let $E \defined \setof{q \in Q}{\forall p \in \idl, p(q) = 0}$,
  a \kl{closed subset} of $Q$.
  Then,
  \begin{align*}
    q \in \mathrm{pre}^\forall(E) & \iff 
    \forall a \in \Sigma, \forall p \in \idl, p(\delta(q, a)) = 0 \\
                                  & \iff 
    \forall a \in \Sigma, \forall p \in \idl, p(\delta(q, a)) = 0 \\
                                  & \iff 
                                  \forall p \in \idl[J], p(q) = 0
  \end{align*}
  where $\idl[J] \defined \IdlGen{ \setof{ p_i[ x_i \mapsto \delta(\cdot, a)_i] }{ i \in \set{1, \dots, m}, a \in \Sigma } }$.
  In particular, one can represent \kl{closed subsets} of $Q$ as finite 
  lists of \kl{ideals} using their \kl{Gröbner bases}, and we showed that 
  one can effectively compute the pre-image of \kl{closed subsets} of $Q$
  via $\mathrm{pre}^\forall$ by substituting polynomials.
  In this representation, it is very easy to compute the union 
  of two \kl{closed subsets}, which is simply concatenating the two lists 
  of \kl{ideals} reperesenting them.
  To compute the intersection of two \kl{closed subsets} $E_1$ and $E_2$,
  one can assume without loss of generality that both are represented by a 
  single ideal (i.e., that they are irreducible closed subsets), respectively 
  $\idl_1$ and $\idl_2$.
  Then, an easy computation shows that 
  $E_1 \cap E_2 = \setof{q \in Q}{\forall p \in \idl_1 + \idl_2, p(q) = 0}$,
  where $\idl_1 + \idl_2$ is the sum of the two ideals.
  Whether a point $q \in Q$ is in a \kl{closed subset} $E$ is decidable
  because one can evaluate the generating polynomials on $q$ and check that 
  it is indeed $0$.
  The equality check is more complicated, and can be done by first 
  normalizing the list of ideals so that their intersection is trivial,
  which requires computing the intersection of ideals
  and performing equality checks on the resulting \kl{ideals}.

  As a consequence, it suffices to test the \kl{open reachability problem} for
  the \kl{topological well-structured transition system} $(Q, \tau)$ with the
  initial state $q_0$ and the \kl{open subset} $U = Q \setminus E_\text{final}$,
  where $E_\text{final} \defined \setof{q \in Q}{F(q) = 0}$ is the \kl{closed subset}
  of states where the automaton outputs zero.
\end{proof}

\AP Let us fix a group $\group$ that acts on the set of indeterminates
$\Indets$, and on an alphabet $\Sigma$ in an \kl{effectively oligomorphic}
fashion. Let us now consider the case of \intro{orbit finite polynomial
automata}, that we define as follows: an \reintro{orbit finite polynomial
automaton} is a tuple $A \defined (Q, \Sigma, \delta, q_0, F)$, where $Q =
\K^{(\Indets)}$, $\Sigma$ is an \kl{orbit finite} alphabet, $\delta \colon
\Sigma \to (\Indets \to \poly{\K}{\Indets})$ is a \kl(func){finitely supported}
polynomial update function, and $F \in \poly{\K}{\Indets}$ is a polynomial
computing the result of the automaton. Given a letter $a \in \Sigma$ and a
state $q \in Q$, the update $\delta^*(q,a)$ is defined as the function from
$\Indets$ to $\K$ defined by $\delta^*(q,a) \colon x \mapsto \delta(a,x)[ q ]$,
which is well-defined because $\delta(a,x)$ is a \kl{finitely supported}
polynomial. The update function is naturally extended to words. Finally, the
output of an \kl{orbit finite polynomial automaton} on a word $w \in \Sigma^*$
is defined as $F(\delta^*(q_0, w))$.

\begin{example}
  \label{ex:orbit-finite-polynomial-automata}
  Let $\Indets = \Q$, and let $\group$ be the group of all
  order-preserving bijections of $\Q$.
  Let $\Sigma \defined \Q \times \Q$.
  Then, the following function are computable by 
  \kl{orbit finite polynomial automata}:
  the number $\mathrm{inc}(w)$ of letters $(a,b)$ such that $a < b$ in a word $w \in \Sigma^*$,
  the number $\mathrm{dec}(w)$ of letters $(a,b)$ such that $a > b$ in a word $w \in \Sigma^*$,
  and the number $(\mathrm{inc}(w) - \mathrm{dec}(w))^2$.
\end{example}

\AP As for \kl{polynomial automata}, the \intro(ofpa){zeroness problem} for
orbit finite polynomial automata is the following decision problem: decide if
for every input word $w$, the output $F(\delta^*(q_0, w))$ is zero. Solving the
\kl(ofpa){zeroness problem} for orbit finite polynomial automata allows us to
decide the equality of two such automata, by computing their substraction. 

It follows from the same reasoning as in
\cref{lem:zeroness-problem-polynomial-automata} that the
\kl(ofpa){zeroness problem} for orbit finite polynomial automata reduces to the
\kl{open reachability problem} for \kl{topological well-structured transition
systems} with the \kl{equivariant Zariski topology} on $\K^{(\Indets)}$.

\begin{corollary}
  \label{cor:orbit-finite-polynomial-automata-zeroness}
  Let $Y \defined (\N \ordplus \ordfin{1})^2$ be the set of pairs 
  of potentially infinite natural numbers ordered pointwise.
  Let $\Indets$ be a set of indeterminates that satisfies the
  \kl{computability assumptions} and such that $(\mon[Y]{\Indets}, \gdivleq)$ is a
  \kl{well-quasi-ordering}.
  Then, the \kl(ofpa){zeroness problem} is decidable for all \kl{orbit finite polynomial automata}
  with register names in $\Indets$, for every \kl{orbit finite} alphabet $\Sigma$,
  that is \kl{effectively oligomorphic} with respect to the action of $\group$.
\end{corollary}
\begin{proof}
  Let $A = (Q, \Sigma, \delta, q_0, F)$ be an \kl{orbit finite polynomial automaton}.
  We consider the topological space $(Q, \tau)$, where $\tau$ is the
  \kl{equivariant Zariski topology} on $\K^{(\Indets)}$.
  Let $E_\text{final} \defined \setof{q \in Q}{F(q) = 0}$ be the \kl{closed subset}
  of states where the automaton outputs zero.

  We proceed as in \cref{lem:zeroness-problem-polynomial-automata},
  representing \kl{closed subsets} of $Q$ as finite lists of \kl{equivariant
  ideals}, which is an effective representation of \kl{closed subsets} thanks
  to \cref{cor:equivariant-ideals-computations}. The only
  non-trivial check to apply the same reasoning as in
  \cref{lem:zeroness-problem-polynomial-automata} is that
  $\mathrm{pre}^\forall(E)$ is computable for every \kl{closed subset} $E$ of
  $Q$. To that end, let us fix an \kl{equivariant ideal} $\idl$ and an
  \kl{orbit finite} set $H$ of polynomials such that $\idl = \IdlGen{H}$. Then,
  we can compute the \kl{equivariant ideal} $\idl[J]$ generated by the
  polynomials $\mathrm{pullback}(p,a) \defined p [ x_i \mapsto \delta(a)(x_i)]$ for every pair $(p, a) \in H \times \Sigma$. Indeed, $H \times \Sigma$
  remains \kl{orbit finite} because the action of $\group$ on $\Indets$ is
  \kl{effectively oligomorphic}, and the function $\mathrm{pullback}$ is computable
  and \kl(func){equivariant}: indeed, given  $\gelem \in \group$, we can show that
  \begin{align*}
    \gelem \cdot \mathrm{pullback}(p, a) & = 
    \gelem \cdot (p [ x_i \mapsto \delta(a)(x_i)]) \\
    & = p [ x_i \mapsto (\gelem \cdot \delta(a, x_i))] \\
    & = p [ x_i \mapsto \delta(\gelem \cdot a, \gelem \cdot x_i))] \\
    & = (\gelem \cdot p) [ x_i \mapsto \delta(\gelem \cdot a, x_i)] \\
    & = \mathrm{pullback}(\gelem \cdot p, \gelem \cdot a).
  \end{align*}
  As a consequence, we can 
  compute an \kl{orbit finite set} of polynomials $H'$ that generate $\idl[J]$.
  Furthermore, it is clear that 
  $\mathrm{pre}^\forall(E)$ is exactly the set of states $q \in Q$ such that
  $\forall p \in \idl[J], p(q) = 0$.
  We have concluded.
\end{proof}

While all of these reasosing could be done outside the realm of (effective)
\kl{topological well-structured transition systems}, we can use the modularity
of the theory to obtain more complex verification properties. Following the
lines of \cite[Theorem 6]{JGL10}, one can consider the case of communicating
orbit finite polynomial automata, where we have a collection processes that
communicate letters over a finite alphabet using lossy channels, and can
perform polynomial updates on their local state. Deciding whether such a system
can reach a state where one process fails to satisfy a given polynomial
invariant is a special case of the \kl{open reachability problem}, and is
decidable.


\paragraph{Reachability problem of symmetric data Petri nets.}

\arka{To add : symmetric VAS equations}
\todo[inline]{for arka: define symmetric data Petri nets/ symmetric VAS equations}

\paragraph{Orbit-finite systems of equations}

\todo[inline]{for arka: do it}


%!TEX root = ../atomic.asmart.tex
%
\section{Concluding Remarks}
\label{sec:conclusion}

We have proven that the \kl{equivariant ideal membership problem} is decidable
under classical effective assumptions on the group action and the
representation of the set of indeterminates, and under a mild hypothesis on the
behavior of monomials. Furthermore, we have developed an algorithm that
computes an \kl{equivariant Gröbner basis} of an \kl{equivariant ideal}, and
shown that it provides an effective representation to work with ideals
(allowing us to compute unions, intersections, etc.) under a slightly stronger
hypothesis on the divisibility ordering. Finally, we have discussed how the
classical counter-examples to our assumptions on the behavior of monomials can
be turned into undecidability results for the \kl{equivariant ideal membership
problem}. Let us now discuss the main open questions raised by
the present paper.

\paragraph*{Total orderings.}
We assumed that the indeterminates $\Indets$ were equipped with a total ordering $\varleq$ that is preserved by the group action.
This assumption seems necessary,
as the notions of \kl{leading monomials} would cease to be well-defined without it.
However, we do not have a clear understanding of whether this assumption is vacuous or not.
Indeed, as noticed by \cite[Lemma 13]{GHOLAS24}, and \cref{lem:reducts-equiv-hilbert},
it often suffices to extend the structures of the indeterminates to account for a total ordering.
A conjecture of Pouzet \cite[Section 4.4]{POUZ24} states that such an ordering always exists,
and this was remarked by \cite[Remark 14]{GHOLAS24}.
Note that in this case, one would get a complete characterisation of the \kl{equivariant Hilbert basis property} \cite[Property 4]{GHOLAS24}.

\todo[inline]{complete this}
%
\arka{changed title}
\paragraph*{\kl{WQO} property under different sets of labels}
%
We conjecture that
$(\mon{\Indets}, \gdivleq)$ is a \kl{well-quasi-ordering} if and only if
$(\mon[Y]{\Indets}, \gdivleq)$ is a \kl{well-quasi-ordering} for every
\kl{well-quasi-ordered} set $Y$ of exponents. This is a form of Pouzet's
conjecture, which has been verified on some classes of structures.
\arka{some citation}
%
\paragraph*{Dichotomy.}
%
We conjecture that the ideal membership problem is undecidable for every action $\group\actson\X$ that is \kl{oligomorphic} but does not have the \kl{equivariant Hilbert basis property} (cf. \Cref{rem:conj-wqo-infinite-path}).
%
\paragraph*{Complexity.} We do not have complexity lower bounds, and there may
be better algorithms like adaptations of Faugère's algorithm that could be
better in practice.
%
\paragraph*{WQO dichotomy conjecture}
The condition that $(\mon[Y]{\Indets},\gdivleq)$ is a \kl{WQO} for every \kl{WQO} also guarantees coverability of Petri nets with data $\X$ is decidable \cite[Theorem 1]{Lasota16}.
In fact it is conjectured to be equivalent \cite[Conjecture 1]{Lasota16}.
%
\paragraph*{Duals of orbit-finitely generated vector spaces}
%
An interesting question posed by S\l{a}womir Lasota is that whether for group actions $\group\actson\Indets$ for which $(\mon[Y]{\Indets},\gdivleq)$ is a \kl{WQO},
duals of orbit-finitely generated vector spaces are also orbit-finitely generated.
This follows for the structures in \Cref{ex:eq atoms,ex:dlo} by the results of \cite{BFKM24,GHL22,Prz23}.
Proving it for the general case would be a big step towards generalising the results of \cite{GHL22}. 
%


% Include acknowledgements
\begin{acks}
Arka Ghosh was supported by the Polish National Science Centre (NCN) grant ``Linear algebra in orbit-finite dimension'' (2022/45/N/ST6/03242), the SAIF project, funded by the ``France 2030'' government investment plan managed by the French National Research Agency, under the reference ANR-23-PEIA-0006, and by the CNRS IRP Le Trójkąt project for collaboration between France and Poland.

Aliaume Lopez was supported by the Polish National Science Centre (NCN) grant ``Polynomial finite state computation'' (2022/46/A/ST6/00072).
\end{acks}

% Include the bibliography
\bibliographystyle{ACM-Reference-Format}
\bibliography{papers.bib}

% If there are any appendices, we include them here.
\appendix
\section{intro}
By leveraging the same proof technique,
we can also show that the \kl{equivariant ideal membership problem} is
decidable under a weaker hypothesis, namely that the set of \kl{monomials}
$\mon[\om \ordplus 1]{\Indets}$ is a \kl{WQO}, which is also believed to be
equivalent to the first condition.

\begin{theorem}[name={Equivariant Ideal Membership},restate=thm:decide-equiv-ideal-mem]
  \label{thm:decide-equiv-ideal-mem}
  Let $\Indets$ be a totally ordered set of indeterminates
  equipped with a group action $\group \actson \Indets$, under our \kl{computability assumptions}.
  If $(\mon[\om \ordplus 1]{\Indets}, \gdivleq)$ is a \kl{WQO}, then one can decide the
  \kl{equivariant ideal membership problem}.
\end{theorem}


\section{Proofs of \cref{sec:examples}}

\AP A \intro{topological space} is a set $X$ equipped with a collection $\tau$
of subsets of $X$ that is stable under finite intersections and arbitrary
unions.\footnote{In particular, $\tau$ contains the empty set and $X$ itself.}
In a \kl{topological space}, elements of $\tau$ are called \intro{open
subsets}, while their complements (in $X$) are called \intro{closed subsets}. A
\kl{topological space} is \intro(space){Noetherian} when, for every sequence
$\seqof{U_i}[i \in \N]$ of \kl{open subsets}, there exists $n \in \N$ such that
$\bigcup_{i \in \N} U_i = \bigcup_{i \leq n} U_i$. We refer the readers to the
book \cite{JGL13} for a comprehensive introduction to \kl{Noetherian spaces}
and their usage in theoretical computer science. Let us briefly argue that
\kl{Noetherian spaces} generalize \kl{well-quasi-orders} in
\cref{ex:well-quasi-orders-are-noeth}, and encode the
\kl{Hilbert basis property} in \cref{ex:polynomials-noetherian}.

\begin{example}[ see \cite{JGL13}]
  \label{ex:well-quasi-orders-are-noeth}
  Let $(X, \leq)$ be a quasi-ordered set.
  Then, the set $X$ equipped with the \kl{topology} having 
  as \kl{open subsets} the upwards-closed subsets of $X$ is \kl(space){Noetherian}
  if and only if $(X, \leq)$ is \kl{well-quasi-ordered}.
\end{example}

\begin{example}[ see \cite{JGL13}]
  \label{ex:polynomials-noetherian}
  Let $\K$ be a field, and let $n \in \N$.
  The space $\K^n$ equipped with the \kl{Zariski topology}
  \kl(space){Noetherian}; where the \intro{Zariski topology}
  is the topology whose \kl{closed subsets} are finite unions of sets
  of the form $\setof{ \vec{x} \in \K^n}{ \forall p \in \idl, p(\vec{x}) = 0}$,
  where $\idl$ is an \kl{ideal} of $\poly{\K}{x_1, \dots, x_n}$.
\end{example}

\AP The advantage of \kl{Noetherian spaces} over \kl{well-quasi-orderings} and
\kl{Noetherian rings} is that they generalize both and can be \emph{combined}:
\kl{Noetherian spaces} are closed under finite sums, finite products,
considering finite words, considering finite trees, and many more \todo{cite}.
As a consequence, they provide a versatile tool to express the set of states of
a system, ensuring that a strong termination property holds.

\AP A \intro{topological well-structured transition system} with alphabet
$\Sigma$ is a \kl{topological space} $(X, \tau)$, equipped with a transition
function $\delta \colon X \times \Sigma \to X$, such that the following
properties hold: for every $U \in \tau$, $\mathrm{pre}^\exists(U)$, the set of
states $x \in X$ such that there exists $a \in \Sigma$ with $\delta(x, a) \in
U$, is an \kl{open subset}. Equivalently, the set $\mathrm{pre}^\forall(E)$ of
states $x \in X$ such that for every $a \in \Sigma$, $\delta(x, a) \in E$ is a
\kl{closed subset} of $X$ whenever $E$ is itself a \kl{closed subset} of $X$.
The natural decition problem for \kl{topological well-structured transition
systems} is the following \intro{open reachability problem} is decidable: given
an initial state $x_0 \in X$ and an \kl{open subset} $U \in \tau$, is it true that
there exists a word $w \in \Sigma^*$ such that $\delta^*(x_0, w) \in U$? The
prototypical algorithm to solve this problem is the following \intro{backward
algorithm}: start with $U_0 \defined U$, and iteratively compute $U_{i+1}
\defined U_i \cup \mathrm{pre}^\exists(U_i)$ until $U_i = U_{i+1}$, then check
whether $x_0 \in U_\text{last}$.
There are easy-to-state sufficient conditions  for such an algorithm to be computable and terminate:
\begin{enumerate}
  \item One is equipped with an effective representation of open subsets,
    where one is able to test equality of open subsets, compute unions of open subsets, and test 
    membership of a point in an open subset.
  \item The pre-image function $\mathrm{pre}^\exists$ is computable, i.e., one can
    compute the set $\mathrm{pre}^\exists(U)$ for every open subset $U$.
  \item The space $(X, \tau)$ is \kl{Noetherian}. 
\end{enumerate}

\AP Our \cref{cor:equivariant-ideals-computations} shows that
under some assumptions on $\Indets$, the set of finitely supported functions
$\Indets \to \K$ is a \kl{Noetherian space} with respect to the
\intro{equivariant Zariski topology}, i.e., the topology whose \kl{closed subsets}
are finite unions of sets of the form $E_{\idl} \defined \setof{f \in
\K^{(\Indets)}}{\forall p \in \idl, p(f) = 0}$, where $\idl$ is an
\kl{equivariant ideal} of $\poly{\K}{\Indets}$. Furthermore, we have an
effective representation of the \kl{closed subsets} in this topology, using
\kl{equivariant Gröbner bases} of \kl{equivariant ideals}. In particular, the
theory of \kl{topological well-structured transition systems} can be applied to
systems whose state space contains ``named registers'' that contain numbers and
are updated by polynomial functions.



\AP Let us fix a group $\group$ that acts on the set of indeterminates
$\Indets$, and on an alphabet $\Sigma$ in an \kl{effectively oligomorphic}
fashion. Let us now consider the case of \intro{orbit finite polynomial
automata}, that we define as follows: an \reintro{orbit finite polynomial
automaton} is a tuple $A \defined (Q, \Sigma, \delta, q_0, F)$, where $Q =
\K^{(\Indets)}$, $\Sigma$ is an \kl{orbit finite} alphabet, $\delta \colon
\Sigma \to (\Indets \to \poly{\K}{\Indets})$ is a \kl(func){finitely supported}
polynomial update function, and $F \in \poly{\K}{\Indets}$ is a polynomial
computing the result of the automaton. Given a letter $a \in \Sigma$ and a
state $q \in Q$, the update $\delta^*(q,a)$ is defined as the function from
$\Indets$ to $\K$ defined by $\delta^*(q,a) \colon x \mapsto \delta(a,x)[ q ]$,
which is well-defined because $\delta(a,x)$ is a \kl{finitely supported}
polynomial. The update function is naturally extended to words. Finally, the
output of an \kl{orbit finite polynomial automaton} on a word $w \in \Sigma^*$
is defined as $F(\delta^*(q_0, w))$.

While all of these reasosing could be done outside the realm of (effective)
\kl{topological well-structured transition systems}, we can use the modularity
of the theory to obtain more complex verification properties. Following the
lines of \cite[Theorem 6]{JGL10}, one can consider the case of communicating
orbit finite polynomial automata, where we have a collection processes that
communicate letters over a finite alphabet using lossy channels, and can
perform polynomial updates on their local state. Deciding whether such a system
can reach a state where one process fails to satisfy a given polynomial
invariant is a special case of the \kl{open reachability problem}, and is
decidable.


\begin{lemma}
  \label{lem:zeroness-problem-polynomial-automata}
  The \kl{zeroness problem for polynomial automata} is a special case of the
  \kl{open reachability problem} for \kl{topological well-structured transition systems}.
\end{lemma}
\begin{proof}
  Let $A = (Q, \Sigma, \delta, q_0, F)$ be a \kl{polynomial automaton}.
  We consider the topological space $(Q, \tau)$, where $\tau$ is the
  \kl{Zariski topology} on $\K^n$.
  Let $\idl$ be an \kl{ideal} of $\poly{\K}{x_1,\dots,x_n}$ generated by the polynomials
  $p_1, \dots, p_m$,
  and let $E \defined \setof{q \in Q}{\forall p \in \idl, p(q) = 0}$,
  a \kl{closed subset} of $Q$.
  Then,
  \begin{align*}
    q \in \mathrm{pre}^\forall(E) & \iff 
    \forall a \in \Sigma, \forall p \in \idl, p(\delta(q, a)) = 0 \\
                                  & \iff 
    \forall a \in \Sigma, \forall p \in \idl, p(\delta(q, a)) = 0 \\
                                  & \iff 
                                  \forall p \in \idl[J], p(q) = 0
  \end{align*}
  where $\idl[J] \defined \IdlGen{ \setof{ p_i[ x_i \mapsto \delta(\cdot, a)_i] }{ i \in \set{1, \dots, m}, a \in \Sigma } }$.
  In particular, one can represent \kl{closed subsets} of $Q$ as finite 
  lists of \kl{ideals} using their \kl{Gröbner bases}, and we showed that 
  one can effectively compute the pre-image of \kl{closed subsets} of $Q$
  via $\mathrm{pre}^\forall$ by substituting polynomials.
  In this representation, it is very easy to compute the union 
  of two \kl{closed subsets}, which is simply concatenating the two lists 
  of \kl{ideals} reperesenting them.
  To compute the intersection of two \kl{closed subsets} $E_1$ and $E_2$,
  one can assume without loss of generality that both are represented by a 
  single ideal (i.e., that they are irreducible closed subsets), respectively 
  $\idl_1$ and $\idl_2$.
  Then, an easy computation shows that 
  $E_1 \cap E_2 = \setof{q \in Q}{\forall p \in \idl_1 + \idl_2, p(q) = 0}$,
  where $\idl_1 + \idl_2$ is the sum of the two ideals.
  Whether a point $q \in Q$ is in a \kl{closed subset} $E$ is decidable
  because one can evaluate the generating polynomials on $q$ and check that 
  it is indeed $0$.
  The equality check is more complicated, and can be done by first 
  normalizing the list of ideals so that their intersection is trivial,
  which requires computing the intersection of ideals
  and performing equality checks on the resulting \kl{ideals}.

  As a consequence, it suffices to test the \kl{open reachability problem} for
  the \kl{topological well-structured transition system} $(Q, \tau)$ with the
  initial state $q_0$ and the \kl{open subset} $U = Q \setminus E_\text{final}$,
  where $E_\text{final} \defined \setof{q \in Q}{F(q) = 0}$ is the \kl{closed subset}
  of states where the automaton outputs zero.
\end{proof}


\section{Proofs of \cref{sec:undecidability}}

\begin{proofof}{lem:mon-rewrite-red-membership}
  Let $R$ be a monomial rewrite system, and let $\monelt_s, \monelt_t \in
  \mon{\Indets}$ be two monomials. We can encode the problem of deciding whether
  $\monelt_s$ can be rewritten into $\monelt_t$ using the rules of $R$ as an
  instance of the \kl{equivariant ideal membership problem} as follows:
  \begin{itemize}
    \item Let $H$ be the set of all polynomials of the form $\monelt - \monelt'$
      for all pairs
      $(\monelt, \monelt') \in R$.
    \item Then, we ask whether $\monelt_s - \monelt_t$ belongs to the ideal generated by $H$.
  \end{itemize}

  It is clear that if $\monelt_s$ can be rewritten into $\monelt_t$ using the
  rules of $R$, then $\monelt_s - \monelt_t$ belongs to the equivariant ideal generated by
  $H$. Conversely, if $\monelt_s - \monelt_t$ belongs to the ideal generated by
  $H$, then 
  \begin{equation}
    \label{eq:mon-rewrite-red-membership}
    \monelt_s - \monelt_t 
    = 
    \sum_{i=1}^n a_i \monelt[n]_i (\gelem_i \cdot \monelt_i - \gelem_i \cdot \monelt'_i)
    \quad .
  \end{equation}

  Let us write the (finite) graph $G$ whose vertices are the monomials
  $\monelt[n] (\gelem_i \cdot \monelt_i)$ and $\monelt[n] (\gelem_i \cdot
  \monelt'_i)$, and whose edges are the directed weighted edges labelled by
  $a_i$ (in a direction that makes the weight positive).

  Let us now analyse \cref{eq:mon-rewrite-red-membership}, and notice that
  identifying monomials in the left and right-hand sides of the equation allows
  us to show that $\monelt_s$ and $\monelt_t$ are vertices of $G$. Furthermore,
  we deduce that the sum of the weights of the edges having $\monelt_s$ as a
  source or target equals $1$, and that the sum of the weights of the edges
  having $\monelt_t$ as a source or target equals $-1$. Finally, for every
  vertex $v$ of $G$ that is not $\monelt_s$ or $\monelt_t$, the sum of the
  weights of the edges having $v$ as a source or target is $0$, again because
  of an analysis of the coefficient of the monomial $v$ in the sum of
  \cref{eq:mon-rewrite-red-membership}.

  Hence, the graph $G$ is a flow network, with a flow value of at least $1$
  from $\monelt_s$ to $\monelt_t$. As a consequence, there must exist a path
  from $\monelt_s$ to $\monelt_t$ in $G$, which is a witness
  of the fact that 
  one can rewrite $\monelt_s$ into $\monelt_t$ using the rules of $R$.
\end{proofof}



\end{document}
