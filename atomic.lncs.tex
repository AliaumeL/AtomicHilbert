%! TEX program = pdflatex
% WARNING: this is a generated file.
%
% Please do not edit this file directly. 
% - If you want to update the medatata of the paper (title, authors, abstract), please
%   edit the `paper-meta.yaml` file in the root of the repository.
% - If you want to update the content of the paper, please edit the latex files
%   in the `src` directory.
% - If you want to update the template itself (e.g., change the layout), please
%   edit the `templates/lncs/lncs.tex` file instead.
\documentclass[runningheads]{llncs}

% create a new LNCS environment for assumptions
\newtheorem{assumption}[theorem]{Assumption}{\bfseries}{\itshape}

\usepackage[utf8]{inputenc}
\usepackage[T1]{fontenc}


\usepackage{lineno}
\linenumbers

\usepackage{todonotes}

% babel for language settings
\usepackage[english]{babel}

% microtype for better typography
\usepackage{microtype}


% math packages
\usepackage{amssymb,amsmath,stmaryrd,thmtools,upgreek}


% graphics packages
\usepackage{graphicx}
\usepackage[obeyclassoptions,mode=tex]{standalone}
\usepackage{tikz}
\usetikzlibrary{backgrounds}
\usetikzlibrary{shapes.geometric}
\usetikzlibrary{positioning}
\usetikzlibrary{automata}
\usetikzlibrary{tikzmark}
\usetikzlibrary{patterns}
\usetikzlibrary{arrows}
\tikzset{every state/.style={minimum size=1pt}}
\usepackage{tikz-cd}


% links inside the document
\usepackage{hyperref}
\usepackage[capitalise,noabbrev,nameinlink]{cleveref}
\Crefname{assumption}{Assumption}{Assumptions}
\usepackage[composition,hyperref,xcolor,cleveref]{knowledge}
\knowledgeconfigure{notion}


% Tables 
\usepackage{booktabs}
\usepackage{varwidth}

% Packages for macro definitions
\usepackage{xparse}
\usepackage{xpatch}
\usepackage{tokcycle}
\usepackage{ifthen}

% Proof trees
\usepackage{bussproofs}

% Algorithms
\usepackage{algorithm2e}
\Crefname{algocfline}{Algorithm}{Algorithms}
\crefname{algocfline}{Algorithm}{Algorithms}
\crefname{algocf}{Algorithm}{Algorithms}
\Crefname{algocf}{Algorithm}{Algorithms}

% Colors 
\usepackage{ensps-colorscheme}

% we include whatever the user wants to include in the header

% we include libraries (tex files) usually written in the `lib` directory

% Knowledge logo
\newcommand{\klogo}{%
\begin{tikzpicture}[scale=0.2,line/.style={draw, line width=0.2pt, line cap=round, line join=round}]
\coordinate (A00) at (0,0);
\coordinate (A01) at (0,1);
\coordinate (A10) at (1,0);
\coordinate (B10) at (1,0.2);
\coordinate (B01) at (0.2,1);

\coordinate (C01) at (0.4,0.7);
\coordinate (C10) at (0.7,0.4);
\coordinate (C12) at (0.4,1.2);
\coordinate (C21) at (1.2, 0.4);
\coordinate (C22) at (1.2, 1.2);

\coordinate (D00) at (C10);
\coordinate (D01) at (0.8,0.5);
\coordinate (D10) at (0.8,0.3);

\coordinate (E01) at (0.3,0.7);
\coordinate (E10) at (0.5,0.7);

\draw[line] (B01) -- (A01) -- (A00) -- (A10) -- (B10);
\draw[line] (C01) -- (C12) -- (C22) -- (C21) -- (C10);

\draw[line] (D01) -- (D00) -- (D10);
\draw[line] (E01) -- (E10);

\end{tikzpicture}%
}

% Upgreek letters
\makeatletter
\newcommand\mathgr[1]{\tokcycle
  {\addcytoks{##1}}
  {\processtoks{##1}}
  {\ifcsname up\expandafter\@gobble\string##1\endcsname
   \addcytoks[1]{\csname up\expandafter\@gobble\string##1\endcsname}%
    \else\addcytoks{##1}\fi}
  {\addcytoks{##1}}{#1}%
  \expandafter\mathrm\expandafter{\the\cytoks}%
}
\makeatother


% Create a new macro proofof
% taking as input a label of a theorem
% and creating a proof with a reference to that
% label
\NewDocumentEnvironment{proofof}{ m O{appendix} }{
    % if the command \#1 exists, then 
    % call \#1* to restate the theorem
    \ifcsname #1\endcsname
        \def\isInsideRestatedTheorem{1}
        \csname #1\endcsname*
    \fi
    \begin{proof}[Proof of {\cref{#1}} as stated on page {\pageref{#1}}]
        \phantomsection
        \label{#1:proof}
}{
        % if the optional argument is "appendix" 
        % then printout a "backlink"
        % and otherwise do nothing
        \ifthenelse{\equal{#2}{appendix}}{
        % Some link to go back to the theorem
        \marginpar{\vspace{-2em}\texttt{\small{\hyperref[#1]{$\triangleright$ Back to p.\pageref{#1}}}}}
        }{}
    \end{proof}
}

% Create a new macro proofref
% that takes as input a label of a theorem
% and creates a reference to its proof
\NewDocumentCommand{\proofref}{ m }{
    % checks if the label #1:proof exists, if yes
    % it creates a link to it, otherwise it writes nothing
    \IfRefUndefinedExpandable{#1:proof}{}{
        % Checks if we are inside a restated theorem
        % if yes, we do not print anything
        \ifdefined\isInsideRestatedTheorem
        \else
            \marginpar{\vspace{0.6em}\texttt{\small{\hyperref[#1:proof]{$\triangleright$ Proven p.\pageref{#1:proof}}}}}
        \fi
    }
}



\newcommand{\circled}[2]{%
\hypertarget{#2}{}%
\tikz[baseline=(char.base),color=A2,thick]{%
\node[shape=circle,draw,inner sep=1pt,font=\tiny] (char) {#1};%
}}
\newcommand{\circleref}[2]{%
\hyperlink{#2}{%
\tikz[baseline=(char.base),color=A2,thick]{%
\node[shape=circle,draw,inner sep=1pt,font=\tiny] (char) {#1};}%
}}

%!TEX root = ../atomic.tex

\newcommand{\arka}[1]{\textcolor{red}{\textbf{Arka:} #1}} 
% Little math macros
\NewDocumentCommand{\set}{ m }{\{ #1 \}}
\NewDocumentCommand{\setof}{ m m }{\{ #1 \mid #2 \}}
\NewDocumentCommand{\card}{ m }{\left| #1 \right|}
\NewDocumentCommand{\seqof}{ m O{n \in \N} }{\left( #1 \right)_{#2}}

\NewDocumentCommand{\defined}{ }{\triangleq}
\newcommand{\defiff}{\overset{\mathrm{def}}{\iff}}
\newcommand{\defeq}{\overset{\mathrm{def}}{=}}

\newcommand{\subfin}{\subset_{\text{fin}}}
\newcommand{\subseteqfin}{\subseteq_{\text{fin}}}

\NewDocumentCommand{\EXPTIME}{}{\ensuremath{\mathsf{EXPTIME}}}

\NewDocumentCommand{\range}{ O{1} m }{[#1, #2]}

% functions of all sorts (injective, partial, surjective)
\newcommand{\topartial}{\rightharpoonup}
\newcommand{\toinj}{\hookrightarrow}
\newcommand{\tosurj}{\twoheadrightarrow}
\newcommand{\tobij}{\stackrel{\simeq}{\longrightarrow}}


% Automate the creation of new orderings
% based on a given symbol.
% For instance,
% \NewDocumentOrdering{\pref}{\preceq}{\prec}
% will create the following commands:
% \prefleq and \preflt
% that will respectively expand to
% \mathrel{\kl[\pref]{\preceq}} and \mathrel{\kl[\pref]{\prec}}
\NewDocumentCommand{\NewDocumentOrdering}{ m m m }{
    \expandafter\newcommand\csname #1leq\endcsname{
        \mathrel{\kl[#1]{#2}}
    }
    \expandafter\newcommand\csname #1lt\endcsname{
        \mathrel{\kl[#1]{#3}}
    }
    \knowledge{#1}{notion}
}

% Order macros
\NewDocumentCommand{\upset}{ O{} m }{{\uparrow_{#1} #2}}
\NewDocumentCommand{\dwset}{ O{} m }{{\downarrow_{#1} #2}}


% Number theory
\NewDocumentCommand{\factorial}{ O{} m }{
    \if\relax\detokenize{#1}\relax
        #2!
    \else
        (#2)!
    \fi
}

\newcommand{\A}{\mathcal{A}}
\newcommand{\R}{\mathbb{R}}
\newcommand{\C}{\mathbb{C}}
\newcommand{\F}{\mathcal{F}}
\newcommand{\Q}{\mathbb{Q}}
\newcommand{\N}{\mathbb{N}}
\newcommand{\K}{\mathbb{K}}
\newcommand{\X}{\mathcal{X}}
\newcommand{\Y}{\mathcal{Y}}

% group actions
\newcommand{\actson}{\curvearrowright}

% orders 
\NewDocumentCommand{\divleq}{}{
    \mathrel{\sqsubseteq^{\mathrm{div}}}
}
\NewDocumentCommand{\gdivleq}{ O{\group} }{
  \mathrel{\kl[\gdivleq]{\sqsubseteq^{\mathrm{div}}_{#1}}}
}
\knowledge{\gdivleq}{notion}

\NewDocumentCommand{\monord}{}{\sqsubseteq}

\NewDocumentOrdering{revlex}{\sqsubseteq_{\mathsf{RevLex}}}{\sqsubset_{\mathsf{RevLex}}}
\NewDocumentOrdering{lex}{\sqsubseteq_{\mathsf{Lex}}}{\sqsubset_{\mathsf{Lex}}}

\NewDocumentCommand{\Basis}{O{B}}{\mathcal{#1}}
\NewDocumentCommand{\LBasis}{O{B} m}{\mathcal{#1}_{#2}}


\NewDocumentCommand{\Indets}{}{\mathcal{X}}

\NewDocumentCommand{\idl}{O{I}}{\mathcal{#1}}
\NewDocumentCommand{\IdlGen}{ m }{\withkl{\kl[\IdlGen]}{
  \mathopen{\cmdkl{\langle}}
  #1
\mathclose{\cmdkl{\rangle}}}}
\knowledge{\IdlGen}{notion}

\NewDocumentCommand{\EqIdlGen}{ O{\group} m }{\withkl{\kl[\EqIdlGen]}{
  \mathopen{\cmdkl{\langle}}
  #2 
  \mathclose{\cmdkl{\rangle}}_{#1}}}
\knowledge{\EqIdlGen}{notion}


\newcommand{\poly}[2]{#1[#2]}
\newcommand{\aut}[2][]{\mathsf{Aut}_{#1}{(#2)}}
\newcommand{\mon}[2][]{\mathsf{Mon}_{#1}(#2)}
\newcommand{\perm}[1]{\mathsf{Perm}(#1)}
\newcommand{\otu}[2]{#1^{(#2)}}
\newcommand{\group}{\mathcal{G}}
\newcommand{\gen}[2]{\langle #1\rangle_{#2}}
\newcommand{\radoG}{\mathbb{G}_{\mathsf{Rado}}}
\newcommand{\radoV}{\mathbb{V}_{\mathsf{Rado}}}
\newcommand{\radoE}{\mathbb{E}_{\mathsf{Rado}}}
\newcommand{\cycleSet}[1][]{\mathsf{Cycles}_{#1}}


\NewDocumentCommand{\FixG}{ O{\group} m }{{#1}^{\kl[\FixG]{\mathsf{fix}}}_{#2}}
\knowledge{\FixG}{notion}

\NewDocumentCommand{\monelt}{ O{m} }{\mathfrak{#1}}

\newcommand{\ordinal}{\eta}



\NewDocumentCommand{\gelem}{ O{\pi} }{\mathgr{#1}}

\newcommand{\hbp}{\text{Hilbert's basis property}}
\newcommand{\orbit}[2][]{\mathsf{orbit}_{#1}{(#2)}}

\newcommand{\order}[1][]{\prec_{#1}}
\newcommand{\ordereq}[1][]{\preceq_{#1}}

\NewDocumentCommand{\sOrderLt}{O{S}}{\prec_{#1}}
\NewDocumentCommand{\sOrderLeq}{O{S}}{\preceq_{#1}}

\newcommand{\revlex}[1][]{<_{\mathsf{RevLex}}^{#1}}
\newcommand{\revlexeq}[1][]{\leq_{\mathsf{RevLex}}^{#1}}
\newcommand{\gr}{Gr\"{o}bner}
\newcommand{\dom}{\mathsf{dom}}
\newcommand{\spoly}[2]{\mathsf{D}(#1,#2)}
\newcommand{\spolyset}{\mathsf{DSet}}
\newcommand{\spolytext}{$\mathsf{S}$-polynomial}
\newcommand{\lcm}{\mathsf{LCM}}
\newcommand{\lc}[1][]{\mathsf{LC}_{#1}}
\newcommand{\lt}[1][]{\mathsf{LT}_{#1}}
\newcommand{\reducstep}[1]{\to_{#1}}
\newcommand{\reduc}[1]{\to^*_{#1}}
\newcommand{\rem}[2]{\mathsf{Rem}_{#1}(#2)}
\newcommand{\closure}[1]{\widehat{#1}}
\newcommand{\wforder}{\triangleleft}

\newcommand{\lm}[1][]{\mathop{\mathsf{LM}_{#1}}}
\newcommand{\cm}[1][]{\mathop{\mathsf{CM}_{#1}}}

\newcommand{\probBasic}[4]
{
\begin{flalign*}
\quad
\begin{tabular}{l  l}
  \multicolumn{2}{l}{\mathsf{#1}}\\
  \textbf{Input:}    & #2 \\
  \textbf{#4} & #3
\end{tabular}
&&
\end{flalign*}
}

\newcommand{\prob}[3]
{
\probBasic{#1}{#2}{#3}{Question:}
}

\NewDocumentOrdering{pmon}{\preceq}{\prec}
\NewDocumentCommand{\pmoneq}{}{\mathrel{\kl[pmon]{\equiv}}}


%  ORDINALS, PARTIAL ORDERINGS, AND THEIR OPERATIONS
\newcommand{\ordfin}[1]{\kl[\ordfin]{#1}}
\newcommand{\om}{\kl[\om]{\omega}}
\newcommand{\ordplus}{\mathrel{\kl[\ordplus]{+}}}
\knowledge{\ordplus}{notion}
\knowledge{\ordfin}{notion}
\knowledge{\om}{notion}

\input{lib/knowledges.kl}

\newcommand{\repositoryUrl}{\url{https://github.com/AliaumeL/AtomicHilbert}}

\begin{document}
%
\title{Computability of Equivariant Gröbner bases \\
       main fe493af66696bfcf41d1b6c032f34e2167c36704 \\
       2025-07-02 15:12:28 +0200
}
\titlerunning{Equivariant Gröbner bases}

\author{
        Arka Ghosh\inst{1}\orcidID{0000-0003-3839-8459} \and
        Aliaume Lopez\inst{1}\orcidID{0000-0002-4205-327X}}
\authorrunning{A. Ghosh and A. Lopez}
\institute{University of Warsaw}

\date{\today}

%
\maketitle              % typeset the header of the contribution
%
\begin{abstract}
    The ring of polynomials in infinitely many variables over a field is never Noetherian. However, it is often the case that the set of indeterminates comes equipped with an additional structure. This extra structure can be taken into account by the means of a group action on the indeterminates, and by considering polynomials up to this action. In this setting, one can recover Noetherianity by restricting the attention to equivariant ideals, under mild assumptions on the group action, as shown by Ghosh and Lasota in 2024. We extend this result by proving algorithmic counterparts to this theoretical result: we show that one can decide the equivariant membership problem, and that one can even compute equivariant Gröbner bases, under similar assumptions as the ones made by Ghosh and Lasota. In addition to these positive results, we also establish sufficient conditions for the equivariant membership problem to be undecidable. Finally, we discuss how these computational results can be applied to various mathematical and computer science decidability problems.
\end{abstract}

% Include the content of the paper
%!TEX root = ../atomic.asmart.tex
% LTeX: language=en
\section{Introduction}
\label{sec:intro}

\arka{New abstract :
Let $\mathbb{K}$ be a field,
$\mathcal{X}$ be an infinite set (of indeterminates),
and $\mathcal{G}$ be a group acting on $\mathcal{X}$.
An ideal in the polynomial ring $\mathbb{K}[\mathcal{X}]$ is called equivariant if it is invariant under the action of $\mathcal{G}$.
In \cite{GHOLAS24} Ghosh and Lasota have given a necessary and a sufficient condition on the action of $\mathcal{G}$ on $\mathcal{X}$ for the Hilbert’s basis property : every equivariant ideal in $\mathbb{K}[\mathcal{X}]$ is finitely generated.
The necessary and sufficient conditions are equivalent up to a well-known conjecture of Pouzet.
We extend this result by showing that a mild strengthening of their sufficient condition ensures that one can decide the equivariant ideal
membership problem,
and that one can even compute equivariant Gröbner bases.
Moreover, we give a sufficient condition for the undecidability of the equivariant ideal membership problem.
This condition is satisfied by the most common examples not satisfying the Hilbert’s basis property.}

\todo[inline]{What about this one? I tried to (1) say what we do (2) 
not talk too much about what we do not do. Plus, I do not like to cite
in the abstract: you cannot look at the references if you just see
the abstract. If you really want to, we cite you and sławek by writing 
the full names}
The theory of Gröbner bases is a central tool in commutative algebra,
allowing to perform effective computations on polynomial ideals. Recently,
considering polynomial rings with infinitely many indeterminates up to the
action of a group of automorphisms has seen a renewed interest, both from a
mathematical and a computational point of view. In this paper, we provide
algorithms to effectively work with ideals that are invariant under the action
of a group on the set of indeterminates. Our algorithms rely on mild
computability assumptions, and a semantic termination assumption. We then show
that our computability assumptions are satisfied by many examples, and provide
undecidability results for classical examples of indeterminates that do not
satisfy our termination assumptions. We conjecture that our sufficient
termination assumption is actually a necessary one, in accordance to previous
conjectures from structural graph theory and well-quasi-orderings.

\AP For a field $\K$ and a non-empty set $\Indets$ of indeterminates, we use
$\poly{\K}{\Indets}$ to denote the ring of polynomials with coefficients from $\K$
and indeterminates/variables from $\Indets$. A fundamental result in commutative
algebra is \intro{Hilbert's basis theorem}, stating that when $\Indets$ is finite,
every ideal in $\poly{\K}{\Indets}$ is finitely generated \cite{HILB1890}, where an
\kl{ideal} is a non-empty subset of $\poly{\K}{\Indets}$ that is closed under
addition and multiplication by elements of $\poly{\K}{\Indets}$. This property can
be rephrased as the fact that the set of polynomials $\poly{\K}{\Indets}$ is
\intro{Noetherian}. \kl{Hilbert's basis theorem} extends to the case where $\K$
is a ring that is itself \kl{Noetherian} \cite[Theorem 4.1]{Lang02}.

\AP A \Grb\ is a specific kind of generating set of a polynomial ideal
which allows easy checking of membership of a given polynomial in that ideal.
\kl{Gr\"{o}bner bases} were introduced by Buchberger who showed when $\Indets$ is
finite, every ideal in $\poly{\K}{\Indets}$ has a finite \kl{Gr\"{o}bner basis} and
that, for a given a set of polynomials in $\poly{\K}{\Indets}$, one can compute a
finite \kl{Gröbner basis} of the ideal generated by them \cite{BUCH76}. The
existence and computability of \Grbs\ implies the decidability of the
\kl{ideal membership problem}: given a polynomial $f$ and set of polynomial
$H$, decide whether $f$ is in the ideal generated by $H$. The theory of
\kl{Gr\"{o}bner bases} has applications in very diverse areas of computer
science, including integer programming \cite{Sturmfels96}, algebraic proof
systems \cite{algProof}, geometric reasoning \cite{Cox2015chGeom}, fixed
parameter tractability \cite{ACDM22}, program analysis \cite{SSM04} and
constraint satisfaction problems \cite{Mas21}.
In automata theory it has been used for deciding zeroness of polynomial
automata \cite{BEDUSHWO17}, reachability in symmetric Petri nets \cite{MAME82},
equivalence for string-to-string transducers \cite{HONKALA00} and equivalence
of polynomial differential equations \cite{CLEMENTE24}. 

\AP There has been a growing interest in the last few years for computational
models that are manipulating infinite data structures in a finite way, for
instance an automaton reading words on the infinite alphabet $\N$, while
maintaining a finite number of states. While this idea can be traced back to
the 90s with the notion of register automata \cite{KAFR94}, it has been revived
in with the development of the theory of \emph{orbit finite sets}. In this
setting, one would like to consider an infinite set of variables $\Indets$. As an
example, let us consider the set $\Indets$ of variables $x_i$ for $i \in \N$, and
the \kl{ideal} $\idlZ$ generated by the set $\setof{x_i}{i \in \N}$. It is
clear that $\idlZ$ is not finitely generated, and we conclude that the
\kl{Hilbert's basis theorem} (and a fortiori, the \kl{Gr\"{o}bner basis}
theory) does not extend to the case of infinite sets of indeterminates.

\AP However, in the applications mentionned above, the infinite set of
variables (data) comes with an extra structure: the behaviour of the considered
systems are invariant under the action of a group $\group$ on $\Indets$. The action
of this $\group$ on $\Indets$ naturally induces an action on $\poly{\K}{\Indets}$, by
renaming the variables. The typical example is the group of all permutations of
$\Indets$, which corresponds to seeing $\Indets$ as a set of \emph{indistinguishable}
names: one is not interested in the ideal $\idlZ$ generated by the set
$\setof{x_i}{i \in \N}$, but rather in the \kl{equivariant ideal} generated by
the set $\setof{x_i}{i \in \N}$, which is the smallest ideal that contains it
and is invariant under the action of $\group$. In this case, this ideal is
finitely generated by a single indeterminate, e.g. $x_1$. Please note that
equivariance does not imply finite generation in general: for instance, the
ideal $\idlZ$ is not finitely generated as an equivariant ideal with respect to
the trivial group.
%
%\AP There has been a growing interest in understanding which groups $\group$
%and sets of variables $\Indets$ allow one to extend the \kl{Hilbert's basis theorem}
%to the equivariant case,
%and to adapt the theory of \kl{Gr\"{o}bner bases} to this setting \cite{BRDR11,HISU12,HIKRLE18,GHOLAS24,COHEN67},
%and there is an almost complete characterisation of the pairs $(\Indets,\group)$ for which the \kl{Equivariant Hilbert basis property} \cite[Theorems 11 and 12]{GHOLAS24}.
%But to obtain decision procedures, one still lacks a generalisation of \kl{Buchberger's algorithm} to the equivariant case, except under artificial extra assumptions \cite[Section 6]{GHOLAS24}.
%Overall, a general understanding of the decidability of the \kl{equivariant ideal membership problem} is still missing,
%and \emph{a fortiori}, a generalisation of \kl{Buchberger's algorithm} to the
%equivariant case is still an open problem.

\subsection{Related Research}
The above-mentioned results were rediscovered in \cite{AH07,AH08,HKL18}. In
\cite{HS12} these results were used to prove the Independent Set Conjecture in
algebraic statistics. In \cite{HS12}, the authors also showed that one can even
take a submonoid $\calM$ of $\inc{<}$ and prove existence and computability of
finite Gr\"{o}bner basis assuming that $\gdivleq[\calM]$ is a
well-partial-order. These results were significantly generalised in
\cite{GHOLAS24}, which gives a necessary and a sufficient condition on the
actions $\group\actson\Indets$ for the \kl{Equivariant Hilbert basis property}
to hold \cite[Theorems 11 and 12, Lemma 13]{GHOLAS24}. The necessary and
sufficient conditions are equivalent up to a well-known conjecture by Pouzet
\cite[Problems 12]{POUZ24}. But to obtain decision procedures, one still lacks
a generalisation of \kl{Buchberger's algorithm} to the equivariant case, except
under artificial extra assumptions \cite[Section 6]{GHOLAS24}. Overall, a
general understanding of the decidability of the \kl{equivariant ideal
membership problem} is still missing, and \emph{a fortiori}, a generalisation
of \kl{Buchberger's algorithm} to the equivariant case is still an open
problem.

\todo[inline]{imprecise and ``citation sludge''}
Last but not least,
our results are closely related to some recent results regarding computation with orbit-finite sets,
in particular on algebraic problems \cite{BFKM24,GHL22,GHL25,KKOT15,Prz23}.


\subsection{Contributions.}
\AP In this paper, we bridge the gap between the
theoretical understanding of \kl{Hilbert's basis property} in the equivariant
setting \cite{GHOLAS24}, and the computational aspects of \kl{equivariant
ideals}, by showing that under mild assumptions on the group action, one can
compute an \kl{equivariant Gröbner basis} of an \kl{equivariant ideal}, hence,
that one can decide the \kl{equivariant ideal membership problem}. In order to
compute such sets, we will need to introduce some classical \kl{computability
assumptions} on the group action $\group \actson \Indets$, and on the set of
indeterminates $\Indets$. These will be defined in
\cref{sec:preliminaries}, but informally, we assume
that one can compute representatives of the orbits of elements under the action
of $\group$ (this is called \kl{effective oligomorphism}), and that one has
access to a total ordering on $\Indets$ that is computable, and
\kl(ord){compatible} with the action of $\group$. Please note that the ordering
on $\Indets$ is not required to be well-founded, and a typical example of our
computable assumptions would be the set $\Q$ of rationals, equipped with the
natural ordering $\leq$ and the group $\group$ would be the group of all
monotone bijections from $\Q$ to itself.

\AP Let us now focus on the mild semantic assumption that we will need to make
on the set of indeterminates $\Indets$ and the group $\group$, that will
guarantee the termination of our procedures. We refer to our preliminaries
(\cref{sec:preliminaries}) for a more detailed
discussion on these assumptions, but again informally, we ask that the set of
\kl{monomials} $\mon{\Indets}$ is well-behaved with respect to divisibility up
to the action of $\group$, which we write as the fact that $(\mon{\Indets},
\gdivleq)$ is a \kl{well-quasi-ordering} (\kl{WQO}). It is known from that this
is a necessary condition for the \kl{equivariant Hilbert basis property}
\cref{thm:equiv-hilbert-property}, and we will rely on a slightly stronger
condition, namely that $(\mon[Y]{\Indets}, \gdivleq)$ is a \kl{WQO}, whenever
$(Y, \leq)$ is one, which is conjectured to be equivalent to the first
condition. Beware that \cref{thm:equiv-hilbert-property,thm:compute-egb}
are
incomparable: the former does not talk about decidability, while the latter 
only considers \kl{equivariant ideals} that are already finitely presented, and we 
will show in
\cref{ex:non-wqo-undecidable} an example where \kl{equivariant
Gröbner bases} are computable, but the \kl{Hilbert basis property} fails.

\begin{theorem}[name={\cite[Theorem 11]{GHOLAS24}}]
  \label{thm:equiv-hilbert-property}
  Let $\Indets$ be a totally ordered set of indeterminates
  equipped with a group action $\group \actson \Indets$ that is 
  \kl(ord){compatible} with the ordering on $\Indets$.
  Then, $(\mon[\om]{\Indets}, \gdivleq)$ is a \kl{WQO}, if and only if 
  the \kl{equivariant Hilbert basis property} holds for $\poly{\K}{\Indets}$.
\end{theorem}

\begin{theorem}[name={Equivariant Gröbner Basis},restate=thm:compute-equiv-gb]
  \label{thm:compute-egb}
  Let $\Indets$ be a totally ordered set of indeterminates
  equipped with a group action $\group \actson \Indets$, under our \kl{computability assumptions}.
  If $(\mon[Y]{\Indets}, \gdivleq)$ is a \kl{WQO} for every 
  \kl{well-quasi-ordered} set $(Y,\leq)$, then one can
  compute an \kl{equivariant Gröbner bases} of \kl{equivariant ideals}.
\end{theorem}

\AP To prove our \cref{thm:compute-egb}, we will first introduce a weaker
notion of \kl{weak equivariant Gröbner basis}, which characterises the results
obtained by naïvely adapting \kl{Buchberger's algorithm} to the equivariant
case. Then, we will show that under our \kl{computability assumptions}, one can
start from a finite set of generators $H$ of an \kl{equivariant ideal}, and
compute a well-chosen \kl{weak equivariant Gröbner basis}, which happens to be
an \kl{equivariant Gröbner basis} of the ideal generated by $H$. As a
consequence, we obtain effective representations of \kl{equivariant ideals},
over which one can check membership, inclusion, and compute the sum and
intersection of \kl{equivariant ideals}
(\cref{cor:equivariant-ideals-computations}).

\AP We then focus on providing undecidability results for the \kl{equivariant
ideal membership problem} in the case where our effective assumptions are
satisfied, but the \kl{well-quasi-ordering} condition is not. This aims at
illustrating the fact that our assumptions are close to optimal. One classical
way for a set of structures to not be \kl{well-quasi-ordered} (when labelled
using integers) is to have the ability to represent an \emph{infinite path} (a
formal definition will be given in
\cref{sec:undecidability}). We prove that
whenever one can (effectively) represent an infinite path in the set of
\kl{monomials} $\mon{\Indets}$, then the \kl{equivariant ideal membership
problem} is undecidable.

\begin{theorem}[name={Undecidability of Equivariant Ideal Membership},restate=thm:undecidable-paths]
  \label{thm:undecidable-paths}
  Let $\Indets$ be a totally ordered set of indeterminates
  equipped with a group action $\group \actson \Indets$, under our \kl{computability assumptions}.
  If $\Indets$ contain an \kl(of){infinite path}
  then the \kl{equivariant ideal membership problem} is undecidable.
\end{theorem}

Finally, we illustrate how our positive results find applications in numerous
situations. This is done by providing families indeterminates that satisfy our
\kl{computability assumptions}, and for which we can compute \kl{equivariant
Gröbner bases}, and also by showing how our results can be used in the context
of \kl{topological well-structured transition systems} \cite{JGL10}, with
applications do the verification of infinite state systems such as \kl{orbit
finite weighted automata} \cite{BOKLMO21}, \kl{orbit finite polynomial
automata}, and more generally orbit finite systems dealing with polynomial
computations.

\todo[inline]{aliaume: recompute the outline}
\paragraph{Organisation.} \AP The rest of the paper is organised as follows. In
\cref{sec:preliminaries}, we introduce formally the
notions of \kl{Gröbner bases}, \kl{effectively oligomorphic} sets, and
\kl{well-quasi-orderings}, which are the main assumptions of our positive
results. Then, we will present in \cref{sec:weakgb}
an adaptation of \kl{Buchberger's
algorithm} to the equivariant case, that computes a \kl{weak equivariant
Gröbner basis} of an \kl{equivariant ideal}. This is the central object of our
paper, and will be used to derive our two positive results. In
\cref{sec:equivariant-grobner-basis},
we use \kl{weak equivariant Gröbner bases} to prove our main
\cref{thm:compute-egb}.
Then, we refine our analysis in
\cref{sec:refinements},
we use \kl{weak equivariant Gröbner bases} to devise a decision procedure for
the \kl{equivariant ideal membership problem} under weaker assumptions
(\cref{thm:decide-equiv-ideal-mem}),
and discuss the potential other applications of \kl{weak equivariant Gröbner
bases}. Then, in
\cref{sec:undecidability}, we show that
our assumptions are close to optimal by proving that the \kl{equivariant ideal
membership problem} is undecidable whenever one can produce a \kl{word
encoding} function (\cref{thm:undecidable-paths}), and we
illustrate this result with a variety of examples. We provide a detailed
discussion on the applications of our positive results in \cref{sec:examples}.
Finally, in \cref{sec:conclusion}, we will
discuss on the need for a total ordering on the set of indeterminates, and the
possibility to relax the hypotheses of our results.


%!TEX root = ../atomic.asmart.tex
% LTeX: language=en
\section{Preliminaries}
\label{sec:preliminaries}

\paragraph{Partial orders, ordinals, well-founded sets, and well-quasi-ordered
sets.} \AP We assume basic familiarity with partial orders, well-founded sets,
and ordinals. We will use the notation $\intro*\om$ for the first infinite
ordinal (that is, $(\N, \leq)$), and write $X \intro*\ordplus Y$ for the
lexicographic sum of two partial orders $X$ and $Y$. We will also use the usual
notations for finite ordinals, writing $\intro*\ordfin{n}$ for the finite
ordinal of size $n$. For instance, $\om \ordplus \ordfin{1}$ is the total order
$\N \uplus \set{+\infty}$, where $+\infty$ is the new largest element.

\AP In order to guarantee the termination of the algorithms presented in this
paper, a key ingredient will be the notion of \intro{well-quasi-ordering}
(WQO), that are sets $(X, \leq)$ such that every infinite sequence
$\seqof{x_i}[i \in \N]$ of elements of $X$ contains a pair $i < j$ such that
$x_i \leq x_j$. Examples of \kl{well-quasi-orderings} include finite sets with
any ordering, or $\N \times \N$ with the product ordering. We refer the reader
to \cite{SCSC12} for a comprehensive introduction to \kl{well-quasi-orderings}
and their applications in computer science.

\paragraph*{Polynomials, monomials, divisibility.} \AP 
We assume basic familiarity with the theory of
commutative algebra, and polynomials. We will use the notation $\poly{\K}{\Indets}$
for the ring of polynomials with coefficients from a field $\K$ and
indeterminates/variables from a set $\Indets$, and $\mon{\Indets}$ for the set of
monomials in $\poly{\K}{\Indets}$. Letters $p,q,r$ are used to denote polynomials,
$\monelt,\monelt[n]$ are used to denote monomials, and $a,b,\alpha,\beta$ are
used to denote coefficients in $\K$.

A classical example of a \kl{WQO} is the set of monomials $\mon{\Indets}$,
endowed with the \kl{divisibility} relation $\intro*\divleq$ whenever $\Indets$
is finite. We recall that a monomial $\monelt[m]$ \intro{divides} a monomial
$\monelt[n]$ if there exists a monomial $\monelt[l]$ such that $\monelt[m]
\times \monelt[l] = \monelt[n]$. In this case, we write $\monelt[m]
\reintro*\divleq \monelt[n]$. Note that monomials can be seen as functions from
$\Indets$ to $\N$ having a finite support, and that the \kl{divisibility}
relation can be extended to monomials that are functions from $\Indets$ to
$(Y,\leq)$, where $Y$ is any partially ordered set. In this case, we write
$\monelt[m] \divleq \monelt[n]$ if for every $x \in \Indets$, we have
$\monelt[m](x) \leq \monelt[n](x)$. We will write $\intro*\mon[\om \ordplus
1]{\Indets}$ (resp. $\mon[\om^2]{\Indets}$) for the set of monomials that are
functions from $\Indets$ to $\om \ordplus \ordfin{1}$ (resp. $\om^2$).

\AP Unless otherwise specified, we will assume that the set of indeterminates
$\Indets$ comes equipped with a total ordering $\varleq$. Using this order, we
define the \intro{reverse lexicographic} (revlex) ordering on monomials as
follows: $\monelt[n] \intro*\revlexlt \monelt[m]$ if there exists an
indeterminate $x \in \Indets$ such that $\monelt[n](x) < \monelt[m](x)$, and such
that for every $y \in \Indets$, if $x \varlt y$ then $\monelt[n](y) =
\monelt[m](y)$. Remark that if $\monelt[n] \revlexleq \monelt[m]$, then in
particular $\monelt[n] \divleq \monelt[m]$. 

\AP We can now use the \kl{revlex} ordering to identify particular elements in
a given polynomial. Namely, for a polynomial $p \in \poly{\K}{\X}$, we define
the \intro{leading monomial} $\intro*\lm(p)$ of $p$ as the largest monomial
appearing in $p$ with respect to the \kl{revlex} ordering, and the
\intro{leading coefficient} $\intro*\lc(p)$ of $p$ as the coefficient of
$\lm(p)$ in $p$. We can then define the \intro{leading term} $\intro*\lt(p)$ of
$p$ as the product of its \kl{leading monomial} and its \kl{leading
coefficient}, and the \intro{characteristic monomial} $\intro*\cm(p)$ of $p$ as
the product of its \kl{leading monomial} and all the indeterminates appearing
in $p$. We also define the \intro(monomial){domain} of $\monelt[m]$ as the set
$\intro*\dom(\monelt[m])$ of indeterminates $x \in \X$ such that $\monelt[m](x) \neq
0$. Because the coefficients and monomial in question are highly dependent on
the ordering $\varleq$, we allow ourselves to write $\lm[\Indets](p)$ to
highlight the precise ordered set of variables that was used to compute the
\kl{leading monomial} of $p$.

Let us briefly argue in favor of the \kl{reverse lexicographic} ordering. In
the case of a finite set of indeterminates, one can choose any total ordering
on $\mon{\Indets}$, as long as it contains the \kl{divisibility}
quasi-ordering, and is compatible with the product of monomials.\footnote{This
is often called a \emph{monomial ordering}, see \cite{CLO15}.} In our case,
having an infinite number of indeterminates, we rely on a connection between
$\lm(p)$ and $\dom(p)$: $\dom(p) \subseteq \dwset{\dom(\lm(p))}$, where
$\dwset{S}$ is the downward closure of a set $S \subseteq \Indets$, i.e. the
set of all indeterminates $x \in \Indets$ such that $y \leq x$ for some $y \in
S$. This means that the \kl{leading monomial} encodes a \emph{global property}
of the polynomial, and it will be crucial in our termitation arguments. This is
already at the core of the \emph{elimination theorems} \cite[Chapter 3, Theorem
2]{CLO15}.


\paragraph{Ideals, and Gröbner Bases.} \AP An \intro{ideal} $\idl$ of
$\poly{\K}{\X}$ is a non-empty subset of $\poly{\K}{\X}$ that is closed under
addition and multiplication by elements of $\poly{\K}{\X}$. Given a set $H
\subseteq \poly{\K}{\X}$, we denote by $\intro*\IdlGen{H}$ the ideal generated
by $H$, i.e. the smallest ideal that contains $H$. The \intro{ideal membership
problem} is the following decision problem: given a polynomial $p \in
\poly{\K}{\X}$ and a set of polynomials $H \subseteq \poly{\K}{\X}$, decide
whether $p$ belongs to the ideal $\IdlGen{H}$ generated by $H$. We know that
this problem is decidable when $\X$ is finite, and that it is even
$\EXPTIME$-complete \cite{MAME82}. The classical approach to the \kl{ideal
membership problem} is to use the \kl{Gröbner basis} theory that was developed
in the 70s by Buchberger~\cite{BUCH76}. 
A set $\Basis$ of polynomials is called a \intro{Gröbner basis} of
an ideal $\idl$ if, $\IdlGen{\Basis} = \idl$ and for every polynomial $p \in
\idl$, there exists a polynomial $q \in \Basis$ such that $\lm[\Indets](q)
\divleq \lm[\Indets](p)$.

Given a \kl{Gröbner basis} $\Basis$ of an ideal $\idl$, and a polynomial $p$,
it suffices to iteratively reduce the \kl{leading monomial} of $p$ by
subtracting multiples of elements in $\Basis$, until one cannot apply any
reductions. If the result is $0$, then $p$ belongs to $\idl$, and otherwise it
does not. 



\paragraph{Group actions, equivariance, and orbit finite sets.}  \AP A
\intro{group} $\group$ is a set equipped with a binary operation that is
associative, has an identity element and has inverses. In our setting, we are
interested in infinite sets $\X$ of indeterminates that is equipped with a
\intro{group action} $\group \actson \X$. This means that for each $\gelem \in
\group$, we have a bijection $\X \tobij \X$ that we denote by $x \mapsto \gelem
\cdot x$. A set $S \subseteq \X$ is \intro{equivariant} under the action of
$\group$ if for all $\gelem \in \group$ and $x \in S$, we have $\gelem \cdot x
\in S$. We give in \cref{ex:idl-equiv} an example and a non-example
of \kl{equivariant ideals}.

\arka{Maybe we can simply assume $\group$ is a subgroup of bijections of
$\Indets$. Then we can also use the notation $\pi(x)$ which is less confusing
than $\pi\cdot x$} \todo[inline]{In my opinion, the new notation is more
  confusing because now $p(x)$ is very diffrent from $\pi(x)$: one is
polynomial evaluation/substitution, the other is a group action.}

\begin{example}
    \label{ex:idl-equiv}
    Let $\Indets$ be any infinite set, and $\group$ be the 
    group of all bijections of $\Indets$. 
    Then the set $S_0 \subset \poly{\K}{\Indets}$ of all polynomials 
    whose set of coefficients sums to $0$ is an equivariant ideal.
    Conversely, the set of all polynomials that are multiple
    of $x \in X$ is an \kl{ideal} that is not \kl{equivariant}.
\end{example}
\begin{proof}
    Let $p,q\in S_0$, and $r \in \poly{\K}{\Indets}$.
    Then, $p \times r + q$ is in $S_0$. Remark that 
    $p,r$ and $q$ belong to a subset $\poly{\K}{\Indets}$ of the 
    polynomials that uses only finitely many indeterminates.
    In this subset, the sum of all coefficients is obtained
    by applying the polynomials to the value $1$ for every indeterminate
    $y \in \Indets$. We conclude that
    $(p \times r + q)(1,\dots, 1) 
    = p(1,\dots,1) \times r(1,\dots,1) + q(1,\dots,1)
    = 0 \times r(1, \dots, 1) + 0 = 0$, hence that
    $p \times r + q$ belongs to $S_0$. 
    Because $0$ is in $S_0$, we conclude that $S_0$ is an \kl{ideal}.
    Furthermore, if $\gelem \in \group$ and $p \in S_0$, then
    the sum of the coefficients $\gelem \cdot p$ is exactly
    the sum of the coefficients of $p$, hence is $0$ too.
    This shows that $S_0$ is \kl(ideal){equivariant}.

    It is clear that all multiples of a given polynomial $x \in \Indets$
    is an ideal of $\poly{\K}{\Indets}$. This is not an \kl{equivariant ideal}:
    take any bijection $\gelem \in \group$ that does not map $x$ to $x$ (it
    exists because $\Indets$ is infinite and $\group$ is all permutations),
    then $\gelem \cdot x$ is not a multiple of $x$, and therefore does 
    not belong to the ideal.
\end{proof}

\AP An \kl{equivariant set} is said to be \intro{orbit finite} if it is the
union of finitely many \intro{orbits} under the action of $\group$. We denote
$\intro*\orbit[\group]{E}$ for the set of all elements $\gelem \cdot x$ for
$\gelem \in \group$ and $x \in E$. Equivalently, an \reintro{orbit finite set}
is a set of the form $\orbit[\group]{E}$ for some finite set $E$. Not every
\kl{equivariant subset} is \kl{orbit finite}, as shown in
\cref{ex:orbit-finite}. However, \kl{orbit finite sets} are
robust in the sense that \kl{equivariant subsets} of \kl{orbit finite sets} are
also \kl{orbit finite}, and similarly, an \kl{equivariant subset} of $E^n$ is
\kl{orbit finite} whenever $E$ is \kl{orbit finite} and $n \in \N$ is finite.
For algorithmic purposes, \kl{orbit finite sets} are the ones that can be taken
as input as a finite set of representatives (one for each orbit). The notions
of \kl{equivariance} and \kl{orbit finite sets} from a computational
perspective are discussed in \cite{BOJAN16inf}, and we refer the reader to this
book for a more comprehensive introduction to the topic.

\begin{example}
  \label{ex:orbit-finite}
  Let $\Indets = \N$, and $\group$ be all permutations 
  that fixes prime numbers. The
  set of all polynomials whose coefficients sum to $0$ is an 
  \kl{equivariant ideal}, but it is not \kl{orbit finite},
  since all the polynomials $x_p - x_q$ for $p \neq q$ primes
  are in distinct orbits under the action of $\group$.
\end{example}

\AP A function $f \colon X \to Y$ between two sets $X$ and $Y$ equipped with
actions $\group \actson X$ and $\group \actson Y$ is said to be
\intro(func){equivariant} if for all $\gelem \in \group$ and $x \in X$, we have
$f(\gelem \cdot x) = \gelem \cdot f(x)$. For instance, the
\kl(monomial){domain} of a monomial is an \kl{equivariant function} if $\gelem
\in \group$, then $\gelem \cdot \dom(\monelt[m]) = \dom(\gelem \cdot
\monelt[m])$. Let us point out that the image of an \kl{orbit finite set} under
an \kl{equivariant function} is \kl{orbit finite}, and that the algorithms that
we will develop in this paper will all be \kl(func){equivariant}.

\paragraph*{Computability assumptions.} \AP We say that the action is
\intro{effectively oligomorphic} if :
%
\begin{enumerate}
\item It is \intro{oligomorphic}, i.e.\ for every $n \in \N$ and every \kl{orbit
finite set} $E \subseteq \Indets$,
the set $E^n$ is \kl{orbit finite} under the action of $\group$ on $\Indets^n$.
\item There exists an algorithm that decides whether two elements $\vec{x},
\vec{y} \in \Indets^*$ are in the same orbit under the action of $\group$ on $\Indets^*$.
\item There exists an algorithm which on input $n\in\N$ outputs a set $A\subseteqfin\Indets^n$ such that $|A\cap U| = 1$ for every orbit $U\in\Indets^n$.
\end{enumerate}
%


A group action $\group \actson \X$ is said to be \intro(ord){compatible}
with an ordering $\leq$ on $\X$ if for all $\gelem \in \group$ and $x,y \in
\X$, we have $x \leq y$ if and only if $\gelem \cdot x \leq \gelem \cdot y$.
Let us point out that in this case, $\revlexleq$ is also \kl(ord){compatible} with
the action of $\group$ on $\mon{\X}$, i.e. for all $\gelem \in \group$ and
monomials $\monelt[m], \monelt[n] \in \mon{\X}$, we have $\monelt[m] \revlexleq
\monelt[n]$ if and only if $\gelem \cdot \monelt[m] \revlexleq \gelem \cdot
\monelt[n]$.
Our \intro{computability assumptions} on the tuple $(\Indets, \group,
\leq)$ will therefore be that $\group$ acts \kl{effectively oligomorphic} on
$\Indets$, and that its action is \kl(ord){compatible} with the ordering $\leq$
on $\Indets$.

\begin{example}
  \label{ex:computability-assumptions}
  Let $\Indets \defined \Q$ and $\group$ be the group of all
  order preserving bijections of $\Q$.
  Then, $\group$ acts \kl{effectively oligomorphically} on $\Indets$,
  and its action is \kl(ord){compatible} with the ordering of $\Q$ by definition.
\end{example}

Note that under our \kl{computability assumptions}, the set of polynomials
$\poly{\K}{\Indets}$ is also \kl{effectively oligomorphic} under the action of
$\group$ on $\Indets$. This is because a polynomial $p \in \poly{\K}{\Indets}$
can be seen as an element of $(\K \times \Indets^{\leq d})^n$ where $n$ is the
number of monomials in $p$, and $d$ is the maximal degree of a monomial
appearing in $p$. Beware that the set of polynomials $\poly{\K}{\Indets}$ is
not \kl{orbit finite}, precisely because the orbit of a polynomial $p$ under 
the action of $\group$ cannot change the degree of $p$, and that there are 
polynomials of arbitrarily large degree.

\paragraph{Equivariant Gröbner bases.} \AP We know from \cite{GHOLAS24} that a
necessary condition for the \kl{equivariant Hilbert basis property} to hold is
that the set  $\mon{\X}$  of monomials is a \kl{well-quasi-ordering} when
endowed with the \intro{divisibility up-to $\group$} relation
($\intro*\gdivleq$), which is defined as follows: for $\monelt_1, \monelt_2 \in
\mon{\X}$, we write $\monelt_1 \gdivleq \monelt_2$ if there exists $\gelem \in
\group$ such that $\monelt_1$ \kl{divides} $\gelem \cdot \monelt_2$. This
relation also extends to monomials that are functions from $\Indets$ to
$(Y,\leq)$ with finite support, where $Y$ is any partially ordered set. We say
that a set $\Basis \subseteq \poly{\K}{\X}$ is an \intro{equivariant Gr\"{o}bner
basis} of an equivariant ideal $\idl$ if $\Basis$ is \kl{equivariant},
$\IdlGen{\Basis} = \idl$, and for every polynomial $p \in \idl$, there exists
$q \in \Basis$ such that $\lm[\Indets](q) \gdivleq \lm[\Indets](p)$ and
$\dom(q) \subseteq \dom(p)$, following the definition of \cite{GHOLAS24}.

Beware that even in the case of a finite set of variables, a \kl{Gröbner basis}
is not necessarily an \kl{equivariant Gröbner basis}, because of the
\kl(polynomial){domain} condition. However, every \kl{equivariant Gröbner
basis} is a \kl{Gröbner basis}.

\begin{example}
  \label{ex:equivariant-gb}
  Let $\Indets \defined \set{ x_1, x_2 }$,
  with $x_1 \varleq x_2$,
  and $\group$ be the trivial group.
  Let us furthermore consider the ideal $\idl$ \kl(idl){generated by}
  $\set{ x_1, x_2 }$.
  Then, the set $\Basis \defined \set{ x_2 - x_1, x_1 }$ is a
  \kl{Gröbner basis} of $\idl$, but not an \kl{equivariant Gröbner basis}.
  Indeed, $x_2 \in \idl$, but there is no polynomial $q \in \Basis$
  such that $\lm(q) \divleq x_2$ and $\dom(q) \subseteq \dom(x_2)$.
\end{example}

In the finite case, one can always compute an \kl{equivariant Gröbner basis} by
computing \kl{Gröbner bases} for every possible ordering of the indeterminates,
and taking their union.\footnote{This algorithm is correct because we are
  considering the \kl{reverse lexicographic} ordering.}

\section{Related Research}
\todo[inline]{Check new version. Maybe merge?}
%
The existence of \kl{equivariant Gr\"{o}bner basis} for the case when $\group$ is equal to the group $\symgr{\Indets}$ of all bijections from $\Indets$ to itself,
is proven in \cite[Proposition 2]{COHEN67},
and the computability can be found in \cite[Theorem 10]{COHEN87} and \cite{Emmott87}.
Their definition of Gr\"{o}bner basis is different than ours.
The crucial observation made by them is that ideal invariant under the action of $\symgr{\Indets}$ is also invariant under the action of monoid $\inc{<}$ of one-to-one functions from $\Indets$ to itself which preserve a fixed well-order $<$.
Well-ordering the set of variables allows one to extend intuition from the classical setting of polynomials rings with finitely many variables to the infinitely many variable case.

%Let $<$ be a well-order on $\Indets$.
%Let $\inc{<}$ be the set of all strictly increasing one-to-one function from $\Indets$ to itself.
%An obvious but crucial observation is that,
%an ideal that is invariant under the action of $\symgr{\Indets}$ is also invariant under the action of $\inc{<}$.
%Hence we can consider ideals invariant under the action of $\inc{<}$.
%Using the well-order we define the leading monomial of a polynomial by the short-lex order : \arka{todo:define $\shortlexlt$}.
%Since $<$ is a well-order, $\shortlexlt$ is also a well-order.
%Hence any sequence of reductions that decreases the leading monomial with respect to $\shortlexlt$ terminates.
%Again, due to the fact that $<$ is a well-order,
%the divisibility relation $\gdivleq[\inc{<}]$ upto $\inc{<}$ is a \kl{well-partial-order}.
%As a consequence, for every $\inc{<}$-equivariant ideal $\idl[I]$,
%we can find a finite set of polynomials $B\subseteq \idl[I]$ which is a Gr\"{o}bner basis of $\idl{I}$ in the sense that,
%for every polynomial $f$ in $\idl[I]$,
%there exists a polynomial $b$ in $B$ such that $\lm(b) \gdivleq[\inc{<}] \lm(f)$.
%It can be proven that $B$ is a set of generators of $\idl[I]$.
%This proves that every $\inc{<}$-equivariant ideal, and hence every $\symgr{\Indets}$-equivariant ideal is finitely generated.
%Now for the computation of a Gr\"{o}bner basis,
%one can extend the Buchberger's algorithm to this setting,
%which terminates due to the fact that $\gdivleq[\inc{<}]$ is a \kl{well-partial-order}.

The above mentioned results were rediscovered in \cite{AH07,AH08,HKL18}.
In \cite{HS12} these results were used to prove the Independent Set Conjecture in algebraic statistics.
In \cite{HS12}, the authors also showed that one can even take a submonoid $\calM$ of $\inc{<}$ and prove existence and computability of finite Gr\"{o}bner basis assuming that $\gdivleq[\calM]$ is a well-partial-order.

\arka{Either add \cite{GHOLAS24} here or merge this part into the intro}

\cite[Corollary 59]{GHOLAS24} shows that one can use a similar strategy for the action $\aut{\calQ}\actson\calQ$ (\Cref{ex:dlo}).
However, it becomes difficult to do the same for a general action $\group\actson\Indets$ satisfying the conditions in \Cref{thm:compute-egb}.


Last but not least,
our result fits nicely into the paradigm of computation with orbit-finite sets, in particular to the study of orbit-finitely generated vector spaces,
orbit-finite systems of equations and inequalities.
\arka{Add citations}



% LTeX: language=en
\section{Weak Equivariant Gröbner Bases}
\label{sec:weakgb}

\AP In this section we prove that a natural adaptation of \kl{Buchberger's
algorithm} to the equivariant setting computes a \kl{weak equivariant Gröbner
basis} of an \kl{equivariant ideal}. This can be seen as an analysis of the
classical algorithm in the equivariant setting. We will assume for the rest of
the section that $\Indets$ is a set of indeterminates equipped with a group
$\group$ acting \kl{effectively oligomorphically} on $\X$, and that $\X$ is
equipped with a total ordering $\varleq$ that is \kl{compatible} with the
action of $\group$.

\begin{definition}
  \label{def:decomposition}
  Let $H$ be a set of polynomials. A \intro{decomposition} of $p$
  with respect to $H$ is given by a finite sequence 
  $\mathfrak{d} \defined \seqof{(a_i, \monelt_i, h_i)}[i \in I]$ such that
   $ p = \sum_{i \in I} a_i \monelt_i h_i$,
  where $a_i \in \K$, $\monelt_i \in \mon{\X}$, and $h_i \in H$ for all $i \in I$.
  The \intro{domain of the decomposition}
  that we write $\intro*\domdec(\mathfrak{d})$ is defined as the union
  of the domains of the polynomials $\monelt_i h_i$ for all $i \in I$.
  The \intro{leading monomial of the decomposition} is defined as
  $
    \intro*\lmdec(\mathfrak{d}) \defined \max(\seqof{\lm(\monelt_i h_i)}[i \in I])
  $.
\end{definition}

Leveraging the notion of decomposition, we can define a weakening of the notion
of \kl{equivariant Gröbner basis} that we will use in this section.

\begin{definition}
  An \kl{equivariant set} $\Basis$ of polynomials is 
  a \intro{weak equivariant Gröbner basis} of an \kl{equivariant ideal}
  $\idl$ if $\IdlGen{\Basis} = \idl$, and if for every polynomial $p \in \idl$,
  and decomposition $\mathfrak{d}$ of $p$ with respect to $\Basis$, there
  exists a decomposition $\mathfrak{d}'$ of $p$ with respect to $\Basis$ such that
  $\domdec(\mathfrak{d}') \subseteq \domdec(\mathfrak{d})$,
  and 
  such that $\lmdec(\mathfrak{d}') = \lm(p)$.
\end{definition}

\AP To compute \kl{weak equivariant Gröbner bases}, we will use a rewriting
relation. Given $p,r \in \poly{\K}{\X}$, we write $p \intro*\toeucl{H}
r$ if and only if there exists $q \in H$, $a \in \K$, and $\monelt \in
\mon{\X}$ such that $p = a \monelt q + r$, $\dom(r) \subseteq \dom(p)$, and
$\lm[\Indets](r) \revlexlt \lm[\Indets](p)$. In order to simplify the
notations, we will write $p \intro*\pmonleq r$ to denote $\dom(r) \subseteq
\dom(p)$, and $\lm[\Indets](r) \revlexlt \lm[\Indets](p)$, leaving the
ordered set of indeterminates $\Indets$ implicit.

\begin{lemma}
  \label{lem:chm}
  The quasi-ordering $\pmonleq$ is \kl{compatible} with the action of $\group$,
  and is well-founded.
\end{lemma}
\begin{proof}
  The first property is immediate because $\dom$, $\lm$, and $\revlexleq$ are
  compatible with the group action $\group$. 
  The second property follows from the fact that $\revlexlt$ is a total
  well-founded ordering whenever one has fixed finitely many possible 
  indeterminates. In a decreasing sequence, the support of the leading 
  monomials is also decreasing, so that sequence only contains finitely many 
  indeterminates, hence we conclude.
\end{proof}

As a consequence of \cref{lem:chm}, we know that the rewriting relation
$\toeucl{H}$ is \kl{terminating} for every set $H$.

\begin{definition}
  \label{def:normalisation}
  Let $H$ be a set of polynomials, and let $p \in \poly{\K}{\X}$ be a
  polynomial. We say that $p$ is \intro{normalised} with respect to $H$ if
  there are no transitions $p \toeucl{H} r$. 
  The set of \intro{remainders} of $p$ with respect to $H$ is 
  denoted $\rem{H}{p}$, and is defined as the set of all polynomials $r$ such that
  $p \toeucl{H}^* r$ and $r$ is normalised with respect to $H$.
\end{definition}

\begin{lemma}
  \label{lem:normalisation}
  Let $H$ be an \kl{orbit finite set} of polynomials, and let $p \in \poly{\K}{\X}$ be a
  polynomial. Then $\rem{H}{p}$ is finite.
  Furthermore, this computation
  is equivariant is: for all $\gelem \in \group$, we have
  $\rem{H}{\gelem \cdot p} = \gelem \cdot \rem{H}{p}$.
\end{lemma}
\begin{proof}
  Let us write $H = \orbit[\group]{H'}$, where $H'$ is a finite set of
  polynomials.
  Because the relation $\toeucl{H}$ is \kl{terminating}, it suffices to 
  show that for every polynomial $p$, there are finitely many polynomials $r$ 
  such that $p \toeucl{H} r$. This is because 
  $p \toeucl{H} r$ implies that 
  $p = \alpha \monelt[n] (\gelem\cdot q) + r$ for some $q \in H'$, 
  $\alpha \in \K$, $\monelt[n] \in \mon{\X}$, and $\gelem \in \group$.
  Because, $\lm(r) \revlexlt \lm(q)$, we  
  conclude that $\lm(p) = \lm(\alpha \monelt[n] (\gelem\cdot q))$, and 
  therefore $r$ is uniquely determined by the choice of $q \in H'$ and the
  choice of $\gelem \in \group$ that maps the \kl{domain} of $q$ to the \kl{domain} of
  $p$. There are finitely elements in $H'$ and finitely many such functions
  $\gelem \in \group$ because both domains are finite.
\end{proof}

\AP Now that we have a quasi-ordering on polynomials, we will prove that given
an \kl{orbit finite} set $H$ of generators, we can compute a \kl{weak
equivariant Gröbner basis}. The computation will closely follow the classical
\kl{Buchberger's algorithm}. The main idea being to saturate the set of
generators $H$ to remove some \emph{critical pairs} of the rewriting relation
$\toeucl{H}$. Namely, given two polynomials $p$ and $q$ in $H$, we compute the
set $\intro*\CancelPoly{p}{q}$ of cancellations between $p$ and $q$ as the set of
polynomials of the form $r = \alpha \monelt[n] p + \beta \monelt[m] q$ such
that $\lm(r) < \max(\monelt[n] \lm(p), \monelt[m]\lm(q))$, where $\alpha,\beta
\in \K$, and where $\monelt[n], \monelt[m] \in \mon{\X}$. Let us recall that
given two monomials $\monelt[n], \monelt[m] \in \mon{\X}$, one can compute
$\intro*\lcm(\monelt[n], \monelt[m])$ as the least common multiple of the two
monomials, and that this in an \kl{equivariant operation}.

\begin{lemma}
  \label{lem:spoly}
  Let $p$ and $q$ be two polynomials in $\poly{\K}{\X}$.
  All the polynomials in $\CancelPoly{p}{q}$ are obtained by multiplying a monomial
  with
  the \kl{S-polynomial} $\spoly{p}{q}$ of $p$ and $q$, defined as
  \begin{equation}
    \label{eq:spoly}
    \intro*\spoly{p}{q} \defined
    \frac{\lcm(\lm(p), \lm(q))}{\lt(p)} \times p
    - \frac{\lcm(\lm(p), \lm(q))}{\lt(q)} \times q
    \quad .
  \end{equation}
\end{lemma}
\begin{proof}
  Let $p,q \in \poly{\K}{\X}$, and let $r \in \CancelPoly{p}{q}$.
  By definition, there exists $\alpha,\beta \in \K$ and $\monelt[n], \monelt[m]
  \in \mon{\X}$ such that $r = \alpha \monelt[n] p + \beta \monelt[m] q$ and
  $\lm(r) < \max(\monelt[n] \lm(p), \monelt[m] \lm(q))$.
  In particular,
  we conclude that $\lm(\monelt[n] p) = \lm(\monelt[m] q)$, and that 
  $\alpha \lc(\monelt[n] p) + \beta \lc(\monelt[m] q) = 0$.

  Let us write $\Delta = \lcm(\lm(p), \lm(q))$.
  Because $\lm(\monelt[n] p) = \lm(\monelt[m] q)$, there exists a monomial 
  $\monelt[l] \in \mon{\X}$ such that 
  $\lm(\monelt[n] p) = \monelt[l] \Delta = \lm(\monelt[m] q)$.
  Furthermore,
  we know that $\lc(p) \beta = - \lc(q) \alpha$.
  As a consequence, one can rewrite $r$ as follows:
  \begin{equation*}
    r = 
    \monelt[l] \alpha \lc(p) 
    \left[
      \frac{\Delta}{\lt(p)} \times p
      - \frac{\Delta}{\lt(q)} \times q
    \right]
    = 
    \monelt[l] \alpha \lc(p) \times \spoly{p}{q} \ .
  \end{equation*}
  We have concluded.
\end{proof}

Remark that the \kl{S-polynomial} is equivariant: if $\gelem \in \group$, then
$\spoly{\gelem \cdot p}{\gelem \cdot q} = \gelem \cdot \spoly{p}{q}$. We are
now ready to define our saturation algorithm. Whenever $p,q \in H$, these
polynomials clearly belong to $\IdlGen{H}$. Given a set $H$, we write
$\intro*\spolyset(H) \defined \bigcup_{p,q \in H} \rem{H}{\spoly{p}{q}}$.

\begin{lemma}
  \label{lem:spoly-orbit-finite}
  Let $H$ be an orbit finite set of polynomials. Then $\spolyset(H)$ is also an
  orbit finite set of polynomials.
\end{lemma}
\begin{proof}
  \todo[inline]{write it...}
\end{proof}

We are now ready to define the saturation algorithm that will compute 
\kl{weak equivariant Gröbner bases}, described in \cref{alg:weakgb}.

\begin{algorithm}
    \caption{Computing \kl{weak equivariant Gröbner bases}}
    \label{alg:weakgb}
    \KwIn{An orbit finite set $H$ of polynomials}
    \KwOut{An orbit finite set $\Basis$ that is a \kl{weak equivariant Gröbner basis} of
      $\EqIdlGen{H}$}
    \Begin{
        $\Basis \gets H$\;
        \Repeat{$\Basis$ stabilizes}{
            $\Basis \gets \Basis \cup \spolyset(\Basis)$\;
        }
        \Return{$\Basis$}\;
    }
\end{algorithm}

In order for \cref{alg:weakgb} to be an actual algorithm, we need several 
effectivity assumptions. 

\begin{lemma}
  \label{lem:weakgb-computable}
  Assume that the order $\varleq$ is effectively computable, and 
  that the action of $\group$ is \kl{effectively oligomorphic}. Then, the
  algorithm \cref{alg:weakgb} is an actual algorithm,
  that can be run on a finite representation of the input set $H$.
\end{lemma}
\begin{proof}
  \todo[inline]{
    use the effectivity assumptions: one can compute the orbits of
    pairs of elements
    }
\end{proof}

\begin{lemma}
  \label{lem:weakgb-termination}
  Assume that $(\mon[\omega]{\X}, \gdivleq)$ is a \kl{WQO}. Then, 
  \cref{alg:weakgb} terminates.
\end{lemma}
\begin{proof}
  Let $\seqof{H_n}[n \in \N]$ be the sequence of sets of polynomials
  computed by \cref{alg:weakgb}. 
  We associate to each set $H_n$ the set $L_n$ of \kl{characteristic monomials} of the
  polynomials in $H_n$. Because the set of monomials is a \kl{WQO}, and because 
  the sequences are non-decreasing for inclusion, there exists an 
  $n \in \N$ such that, for every $\monelt \in L_{n+1}$, there exists
  $\monelt[n] \in L_n$, such that $\monelt[n] \gdivleq \monelt$.

  We will prove that $H_{n+1} = H_n$ by contradiction. Assume towards this
  contradiction that there exists some $r \in H_{n+1} \setminus H_n$. By
  definition of $H_{n+1}$, there exists $p,q \in H_n$ such that $r \in
  \rem{H_n}{\spoly{p}{q}}$. In particular, $r$ is \kl{normalised} with respect
  to $H_n$. However, because $r \in H_{n+1}$, $\cm(r) \in L_{n+1}$, and
  therefore there exists $\monelt[n] \in L_n$ such that $\monelt[n] \gdivleq
  \cm(r)$. This provides us with a polynomial $t \in H_n$ such that $\cm(t)$
  and an element $\gelem \in \group$ such that $\cm(t) \divleq \gelem \cdot
  \cm(r)$. Because $H_n$ is \kl{equivariant}, we can assume that $\gelem$ is
  the identity. In particular, one concludes that there exists $\monelt[n] \in
  \mon{\X}$ such that $\lm(t) \times \monelt[n] = \lm(r)$, and therefore
  $\lm(t) \gdivleq \cm(r)$. Therefore, one can find some $\alpha \in \K$ such
  that the polynomial $r' \defined r - \alpha \monelt[n] t$ satisfies $r'
  \pmonlt r$, and in particular, $r \toeucl{\pmonleq}{H_n} r'$.
  This contradicts the fact that $r$ is \kl{normalised} with respect to $H_n$.
\end{proof}

\begin{lemma}
  \label{lem:weakgb-correctness}
  Assume that $(\mon[\omega]{\X}, \gdivleq)$ is a \kl{WQO}. Then, the set $\Basis$ computed
  by \cref{alg:weakgb} 
  is a \kl{weak equivariant Gröbner basis} of the ideal
  $\EqIdlGen{H}$.
\end{lemma}
\begin{proof}
  It is clear that $\Basis$ is a generating set of $\EqIdlGen{H}$, because
  one only add polynomials that are in the ideal generated by $H$ at every step.

  Let $p \in \EqIdlGen{H}$ be a polynomial,
  and let $\mathfrak{d}$ be a decomposition of $p$ with respect to
  $\Basis$, that is, a decomposition of the form
  \begin{equation}
    p = \sum_{i \in I} \alpha_i \monelt_i p_i
    \quad .
  \end{equation}
  Where $\alpha_i \in \K$, $p_i \in \Basis$, and $\monelt_i \in \mon{\X}$,
  for all $i \in I$.

  Let us extend the ordering $\pmonleq$ to decompositions as follows:
  \begin{equation}
    \mathfrak{d}_1 \pmonleq \mathfrak{d}_2
    \quad \text{if and only if} \quad
    \dom(\mathfrak{d}_1) \subseteq \dom(\mathfrak{d}_2)
    \text{ and } \lm(\mathfrak{d}_1) \revlexleq \lm(\mathfrak{d}_2)
    \quad .
  \end{equation}

  Because of \cref{lem:chm}, we know that the ordering $\pmonleq$ is
  \kl{well-founded}. As a consequence, one can find a minimal decomposition
  $\mathfrak{d}'$ of $p$ with respect to $\Basis$ such that $\mathfrak{d}'
  \pmonleq \mathfrak{d}$. We now distinguish two cases, depending on whether
  the leading monomial $\monelt[l]$ of the decomposition $\mathfrak{d}'$ is
  equal to the leading monomial of $p$ or not.

  \begin{description}
    \item[Case 1:] $\monelt[l] = \lm(p)$.
      In this case, we conclude immediately, because $\lm(\mathfrak{d}') =
      \lm(p)$, and by assumption $\dom(\mathfrak{d}') \subseteq \dom(\mathfrak{d})$.

    \item[Case 2:] $\monelt[l] \neq \lm(p)$.
      In this case, it must be that the set $J$ the set of indices such that
      $\lm(\monelt_i p_i) = \monelt[l]$.
      Let us remark that 
      the sum of leading coefficients 
      of the polynomials in $J$ must vanish: $\sum_{i \in J} \alpha_i \lc(p_i) = 0$.
      As a concequence, the set $J$ has size at least $2$.
      Let us distinguish one element $\star \in J$, and 
      write $J_\star = J \setminus \set{\star}$.
      We conclude that 
      $\alpha_\star = - \sum_{i \in J_\star} \alpha_i \lc(p_i) / \lc(p_\star)$.
      Let us now rewrite $p$ as follows:
      \begin{equation}
        p = \sum_{i \in J_\star} \alpha_i 
        \left(\monelt_i p_i - \frac{\lc(p_i)}{\lc(p_\star)} \monelt_\star p_\star\right)
        + \sum_{i \in I \setminus J} \alpha_i \monelt_i p_i
        \quad .
      \end{equation}
      Now, by definition,
      polynomials $\alpha_i \monelt_i p_i$ for $i \in J \setminus J$ have 
      leading monomials
      strictly smaller than $\monelt[l]$.
      Furthermore,
      the polynomials
      $\monelt_i p_i - \frac{\lc(p_i)}{\lc(p_\star)} \monelt_\star p_\star$ for $i \in J_\star$
      cancel their leading monomials, hence they belong
      to the set $\CancelPoly{p_i}{p_\star}$.
      By \cref{lem:spoly}, we know that these polynomials are obtained by
      multiplying the \kl{S-polynomial} $\spoly{p_i}{p_\star}$ by some monomial.
      Because \cref{alg:weakgb} terminated, we know that 
      $\spoly{p_i}{p_\star} \toeucl{\Basis}^* 0$ by contruction.

      By definition of the rewriting relation, we conclude that one can rewrite
      $\spoly{p_i}{p_\star}$ as combination of polynomials in $\Basis$ that
      have smaller or equal leading monomials, and do not introduce new
      indeterminates.

      We conclude that
      the whole sum is composed of polynomials with leading monomials 
      strictly smaller than $\monelt[l]$, and using a subset of the indeterminates
      used in $\mathfrak{d}'$, leading to a contradiction
      because of the minimality of the latter. 
  \end{description}
\end{proof}

As a consequence of the above lemmas, we can now conclude that the 
\cref{alg:weakgb} computes a \kl{weak equivariant Gröbner basis} of the
ideal $\EqIdlGen{H}$, as stated in \cref{thm:weakgb-comput}.

\begin{theorem}
  \label{thm:weakgb-comput}
  Assume that $(\mon[\omega]{\X}, \gdivleq)$ is a \kl{WQO}, and that the order
  $\varleq$ is effectively computable, and that the action of $\group$ is
  \kl{effectively oligomorphic}. 
  Then, the algorithm $\mathsf{weakgb}$ that takes as input a finite set $H$ of generators of an
  \kl{equivariant ideal} $\idl$ and computes a \kl{weak equivariant Gröbner
  basis} $\Basis$ of $\idl$.
\end{theorem}

% LTeX: language=en
\section{Computing the Equivariant Gröbner Basis}
\label{sec:equivariant-grobner-basis}

The goal of this section is to strengthen the results of \cref{sec:algorithm},
and instead of being able to answer to the \kl{equivariant ideal membership
problem}, to compute an \kl{equivariant Gröbner basis} of an equivariant ideal.
Let us recall that a Gröbner basis is known to exist, but that it's
computability was an open question of \cite{GHOLAS24}.

The proof will closely follow the one of \cref{sec:algorithm}, where one starts
from a generating set $H$, and constructs a new set $H'$ together with a group
action, over which one computes a \kl{weak Gröbner basis} (\cref{sec:weakgb}).
The result is then used to derive an \kl{equivariant Gröbner basis} of the
\kl{equivariant ideal generated by} $H$. Informally, one wants to 
apply the technique of isolating finite sets of variables \emph{uniformly},
that is, compute \kl{weak Gröbner bases} for every possible fixed 
finite subset of variables. 


Let us fix a set $\Indets$ of indeterminates equipped with a total ordering
$\varleq$. We define $\IndetsCol \defined \Indets + \Indets$, that is, the
disjoint union of two copies of $\Indets$, ordered. It will be useful to refer
to the first copy (lower copy) and the second copy (upper copy), noting the
isomorphism between $\IndetsCol$ and $\set{\mathsf{first}, \mathsf{second}}
\times \Indets$, ordered lexicographically, where $\mathsf{first} <
\mathsf{second}$. We will also define $\forgetCol \colon \IndetsCol \to
\Indets$ that maps a colored variable to its underlying variable.
Beware that $\forgetCol$ is not an order preserving map.
We extend $\forgetCol$ as a morphism from polynomials in
$\poly{\K}{\IndetsCol}$ to polynomials in $\poly{\K}{\Indets}$.

Given a subset $V \subfin \Indets$, we build the injection $\colorWith{V}
\colon \Indets \to \IndetsCol$ that maps variables $x$ in $V$ to
$(\mathsf{fisrt}, x)$, and variables $x$ not in $V$ to $(\mathsf{second}, x)$.
Again, we extend these maps as morphisms from $\poly{\K}{\Indets}$ to
$\poly{\K}{\IndetsCol}$. We say that a polynomial $p \in \poly{\K}{\IndetsCol}$
is \intro{$V$-compatible} if $p \in \colorWith{V}(\poly{\K}{\Indets})$.

\begin{lemma}
  \label{lem:v-saturation-computable}
  Let $H$ be an \kl{orbit finite} subset of $\poly{\K}{\Indets}$.
  Then, $\freeColor(H) \defined \bigcup_{V \subfin \Indets} \colorWith{V}(H)$
  is a computable \kl{orbit finite} subset of $\poly{\K}{\IndetsCol}$.
\end{lemma}


We are now ready to write our algorithm to compute 
an \kl{equivariant Gröbner basis}.

\begin{algorithm}
    \caption{Computing \kl{equivariant Gröbner bases}}
    \label{alg:stronggb}
    \KwIn{An orbit finite set $H$ of polynomials}
    \KwOut{An orbit finite set $\Basis$ that is a \kl{equivariant Gröbner basis} of
      $\EqIdlGen{H}$}
    \Begin{
        $H_C \gets \freeColor(H)$\;
        $\Basis_C \gets \mathsf{weakgb}(H_C)$\;
        $\Basis \gets \forgetCol(\Basis_C)$\;
        \Return{$\Basis$}\;
    }
\end{algorithm}

To prove the correctness of our algorithm, let us first argue
that one can indeed compute the \kl{weak Gröbner basis} algorithm.

\begin{lemma}
  \label{lem:colored-hypothesis-sat}
  Assume that $(\Indets, \varleq, \group)$
  is \kl{effectively oligomorphic},
  and that $(\mon[\omega + \omega]{\Indets}, \gdivleq)$
  is a \kl{well-quasi-order}.
  Then,
  $\IndetsCol$ with its ordering and the 
  action of $\group$ acting on both components 
  simultaneously is \kl{effectively oligomorphic},
  and $(\mon{\IndetsCol}, \gdivleq)$ is a
  \kl{well-quasi-ordered} set.
\end{lemma}

Now, let us argue that the result of our algorithm
is a generating set of the desired ideal, which follows
from the fact that $\forgetCol$ and $\colorWith{\cdot}$
are morphisms that preserve variable names.

\begin{lemma}
  \label{lem:correct-gen-set}
  Let $H$ be an \kl{orbit finite} subset of $\poly{\K}{\Indets}$,
  then the result of \cref{alg:stronggb}
  is an \kl{orbit finite} generating set
  of $\EqIdlGen{H}$.
\end{lemma}
\begin{proof}
  Let us remark that $\forgetCol(\freeColor(H)) = H$.
  Because we know that $\mathsf{weakgb}(\freeColor(H))$
  generates the same ideal as $\freeColor(H)$,
  and since $\forgetCol$ is a morphism,
  we conclude that 
  the set of polynomial
  $\forgetCol(\mathsf{weakgb}(\freeColor(H)))$
  generates the same ideal as
  $\forgetCol(\freeColor(H)) = H$.
\end{proof}

Let us now prove that the resulting set in indeed
an \kl{equivariant Gröbner basis} of $\EqIdlGen{H}$.

\begin{lemma}
  \label{lem:strong-gb-correct}
  \Cref{alg:stronggb} is correct.
\end{lemma}
\begin{proof}
  Let $p \in \IdlGen{H}$,
  $H_\star = \freeColor(H)$,
  $V \defined \dom(p)$,
  $H_V \defined \colorWith{V}(H)$.
  We let $\Basis_\star = \mathsf{weakgb}(H_\star)$.
  Finally, $\Basis = \forgetCol(\Basis_\star)$.

  It is clear that $\colorWith{V}(p)$
  belongs to $\IdlGen{H_V}$.
  
  Let us write 
  \begin{equation*}
    \colorWith{V}(p) = \sum_{i=1}^n a_i \monelt_i h_i
  \end{equation*}
  Where $a_i \in \K$, $\monelt_i \in \mon{\IndetsCol}$,
  and $h_i \in \Basis_\star$ is \kl{$V$-compatible}.
  Such a decomposition exists
  because $H_V \subseteq H_\star \subseteq \Basis_\star$.
  We order decompositions based on their maximal leading 
  monomial and the set of indeterminates present in the
  decomposition. This is a well-founded ordering, and 
  we can assume that our decomposition was minimal.

  Assume by contradiction that $\lm(\colorWith{V}(p))$ is not the maximal
  monomial of the decomposition. Then one can use \kl{$S$-polynomials} to
  reduce the leading monomial of the decomposition, keeping the same set of
  indeterminates, and therefore remaining $V$-compatible.

  This means that the leading monomial of $\colorWith{V}(p)$ is exactly
  the leading monomial of some polynomial $\monelt_i h_i \in \Basis_\star$
  appearing in the decomposition, and therefore 
  $\colorWith{V}(p) - b \monelt_i h_i$ is a valid 
  reduction for some $b \in \K$.
\end{proof}



% LTeX: language=en
%%!TEX root = ../atomic.asmart.tex
%
\section{Decision Procedure for the Equivariant Ideal Membership Problem}
\label{sec:algorithm}

In this section we provide a decision procedure for the \kl{equivariant ideal
membership problem}, we will leverage the results of the previous section
to compute what we call \kl{$S$-strong equivariant Gröbner bases} for a finite set
$S \subfin \X$ of indeterminates.

\begin{definition}
  \label{def:strong-equiv-grob}
  Let $S \subseteq \X$ be a set of indeterminates.
  An orbit finite set $\Basis_S$
  of polynomials is called an \intro{$S$-strong equivariant Gröbner basis}
  of an \kl{equivariant ideal} $\idl$ with respect to $\pmonleq$
  whenever $\Basis_S \cap \poly{\K}{S}$ is a
  \kl{Gröbner basis} of the ideal $\idl \cap \poly{\K}{S}$ with respect to
  $\pmonleq$.
\end{definition}

\AP To compute an \kl{$S$-strong equivariant Gröbner basis} of an
\kl{equivariant ideal} $\idl$, we will compute a \kl{weak equivariant Gröbner
basis} of the ideal $\idl$ but with respect to a new ordering $\varleq_S$ and a
new group action $\group_S \actson \X$ that is the restriction of $\group$ to
the operations that fix the indeterminates in $S$, i.e., such that for all
$\gelem \in \group_S$ and $x \in S$, we have $\gelem(x) = x$. The ordering
$\varleq_S$ is defined as $\varleq$ on $S$ and $\X \setminus S$, and orders all
elements of $S$ below all elements of $\X \setminus S$.
\begin{equation}
  \label{eq:order-S}
  x \varleq_S y \iff
  \begin{cases}
    x \varleq y & \text{if } x,y \in S \lor x,y \not\in S  \\
    \top        & \text{if } x \in S \land y \not\in S \\
    \bot        & \text{if } x \not\in S \land y \in S
  \end{cases}
\end{equation}

\AP Let us first show that if one can compute \kl{weak equivariant Gröbner
bases} in $(\X, \group_S, \varleq_S)$, then one can compute \kl{$S$-strong
equivariant Gröbner bases} in $(\X, \group, \varleq)$. This will crucially rely
on the fact that whenever $q$ has indeterminates smaller (for $\varleq_S$) than
$p \in \poly{\K}{S}$, then $q$ is in $\poly{\K}{S}$ too, because of the choice
to put all elements of $S$ below all elements not in $S$.

\begin{lemma}
  \label{lem:local-to-supported}
  Let $\X$ be a set of indeterminates, $\group$ a group acting
  on $\X$ \kl{effectively oligomorphically}, and $\varleq$ a total ordering
  on $\X$ that is \kl{respected by the group action}.
  Let $S$ be a finite set of indeterminates, and 
  assume that one can compute \kl{weak equivariant Gröbner bases}
  in $(\X, \group_S, \varleq_S)$.
  Then, one can compute an \kl{$S$-strong equivariant Gröbner basis}
  for any finitely presented \kl{equivariant ideal} $\idl$ in $(\X, \group, \varleq)$.
\end{lemma}
\begin{proof}
  Let $H$ be a finite set of polynomials
  that generates the \kl{equivariant ideal} $\idl$.
  We will compute a new finite set $H_S$ of polynomials
  such that $\gen{H}{\group} = \gen{H_S}{\group_S}$.
  To do so, we will use the fact that $\X$ is \kl{effectively oligomorphic}:
  for every polynomial $p \in H$, we can compute a finite set
  $p_S$ of representatives of the \kl{$\group_S$-orbits} of $p$.
  We define $H_S$ to be the union of all the $p_S$ for $p \in H$.

  It is clear that $\EqIdlGen[\group_S]{H_S} \subseteq \EqIdlGen{H}$.
  Conversely, let $p \in \EqIdlGen{H}$, then $p$ is in the \kl{$\group$-orbit}
  of some $p' \in H$. By definition, there exists a $\pi \in \group$ such that
  $p = \pi(p')$. We can compute a representative $p_S$ of the
  \kl{$\group_S$-orbit} of $p'$, and we can compute a $\pi_S \in \group_S$
  such that $\pi_S(p_S) = p$. We conclude that $p$ is in the \kl{$\group_S$-orbit}
  of $p_S$, hence $p \in \EqIdlGen[\group_S]{H_S}$.
  Hence, $\EqIdlGen{H} = \EqIdlGen[\group_S]{H_S}$.


  
  Let us now compute  \kl{weak equivariant Gröbner basis} of
  $\EqIdlGen[\group_S]{H_S}$, that we call $\Basis_S$. Remark that for every
  polynomial $p \in \poly{\K}{S} \cap \EqIdlGen[\group_S]{H_S}$, there exists
  some $q \in \Basis_S$ such that $\lm[S](q) \divleq \lm[S](p)$, and such that
  the variables appearing in $\lm[S](q)$ are all smaller than some variable of
  $\lm[S](p)$. In particular, one concludes that $q$ is in $\poly{\K}{S}$.
  Furthermore, because of the way one defined $\varleq_S$, we have that
  $\lm[S](p) = \lm(p)$ and $\lm[S](q) = \lm(q)$.

  Hence, we conclude that $\Basis_S$ is a \kl{Gröbner basis} of the ideal
  $\EqIdlGen{H} \cap \poly{\K}{S}$, as expected.
  \todo[inline]{this is actually false, we have a problem because 
  we still care about the domains in our order $\pmonleq$, but I think
  we should just handwave this.}
\end{proof}

We will now focus on proving that one can compute \kl{weak equivariant
Gröbner bases} in $(\X, \group_S, \varleq_S)$, under mild assumptions 
on the original group action $\group$.
The first step is to remark that decidability condictions are preserved
by the construction of $\varleq_S$ and $\group_S$.

\begin{lemma}
todo
\end{lemma}

Then, we will prove that a mild assumption on the monomials allows us to
conclude that $(\mon[\omega]{\X}, \gdivleq[\group_S])$ is a
\kl{well-quasi-ordering}: it suffices to assume that $(\mon[\omega + 1]{\X},
\gdivleq[\group])$ is a \kl{well-quasi-ordering} itself.

\begin{lemma}
  \label{lem:wqo-mon-S}
  Let $\X$ be a set of indeterminates, $\group$ a group acting
  on $\X$ \kl{effectively oligomorphically}, and $\varleq$ a total ordering
  on $\X$ that is \kl{respected by the group action}.
  Assume that $(\mon[\omega + 1]{\X}, \gdivleq[\group])$ is a \kl{well-quasi-ordering}.
  Then, $(\mon[\omega]{\X}, \gdivleq[\group_S])$ is a \kl{well-quasi-ordering}.
\end{lemma}


We can now conclude with our first main result, which is the
decidability of the \kl{equivariant ideal membership problem}.

\todo[inline]{restate}


%!TEX root = ../atomic.asmart.tex
% LTeX: language=en
\section{Undecidability Results}
\label{sec:undecidability}

In this section, we aim to show that the \kl{equivariant ideal membership
problem} is undecidable under the usual \kl{computability assumptions} on the
group action, when we do not assume that $(\mon{\Indets}, \gdivleq)$ is a
\kl{well-quasi-ordering}. In particular, this would show that computing
\kl{equivariant Gröbner bases} is not possible in these settings, proving the
optimality of our decidability
\cref{thm:compute-egb}.
Beware that there are some pathological cases where the \kl{equivariant ideal
membership problem} is easily decidable, even when $(\mon{\Indets}, \gdivleq)$
is not a well-quasi-ordering, as illustrated by the following
\cref{ex:non-wqo-undecidable}, and it is not possible to obtain
such a dichotomy result.

\begin{example}
  \label{ex:non-wqo-undecidable}
  Let $\Indets = \{x_1, x_2, \ldots\}$ be an infinite set of indeterminates,
  and let $\group$ be trivial group acting on $\Indets$.
  Then, the \kl{equivariant ideal membership problem} is decidable.
  Indeed, since the group is trivial, whenever one provides a finite set
  $H$ of generators of an \kl{equivariant ideal} $I$, one can
  in fact work in $\poly{\K}{V}$, where $V$ is the set of indeterminates
  that appear in $H$.
  Then, the \kl{equivariant ideal membership problem} is reduces to 
  the \kl{ideal membership problem} in $\poly{\K}{V}$, which is decidable.
\end{example}


\AP However, one we are able to prove the undecidability of the \kl{equivariant
ideal membership problem} under the assumption that the set of indeterminates
$\Indets$  contains an \intro(of){infinite path} $P \defined \seqof{x_i}[i \in
\N] \subseteq \Indets$, that is, a set of indeterminates such that $(x_i,x_j)
\in P^2$ is in the same orbit as $(x_0, x_1)$ if and only if $|i - j| = 1$, for
all $i,j \in \N$. We similarly define \reintro(of){finite paths} by considering
finitely many elements. The prototypical example of a set of indeterminates
containing an \kl(of){infinite path} is $\Indets = \Z$ equipped with the group
$\group$ of all shifts. The presence of an \kl(of){infinite path} clearly
prevents $(\mon{\Indets}, \gdivleq)$ from being a \kl{well-quasi-ordering}, as
shown by the following \cref{rem:not-wqo}. Furthermore, for
indeterminates obtained by considering \kl{homogeneous structures} and their
automorphism groups
(\cref{sec:act ex}),
the presence of an \kl(of){infinite path} has been conjectured to be a
necessary and sufficient condition for $(\mon{\Indets}, \gdivleq)$ to be a
\kl{well-quasi-ordering}: this follows from a conjecture of Schmitz restated in
\cref{conj:wqo-infinite-path}, that generalises one
of Pouzet (\cref{rem:conj-wqo-pouzet}), as explained in
\cref{rem:conj-wqo-infinite-path}.


\begin{remark}
  \label{rem:not-wqo}
  Assume that $\Indets$ contains an \kl(of){infinite path}
  $P \defined \seqof{x_i}[i \in \N]$.
  Then, the set of monomials $\setof{x_0^3 x_1^1 \cdots x_{n-1}^1 x_n^2}{n \in \N}$
  is an infinite antichain in $(\mon{\Indets}, \gdivleq)$.
  Indeed, assume that there exists $n < m$, and a group element $\gelem \in \group$ such that
  $\gelem \cdot \monelt_n \divleq \monelt_m$.
  Then, $\gelem \cdot x_0 = x_0$, because it is the only indeterminate with 
  exponent $3$ in $\monelt_m$. Furthermore, 
  $\gelem \cdot (x_0,x_1) = (x_i,x_j)$ implies that 
  $|i - j| = 1$, and since $\gelem \cdot x_0 = x_0$, we conclude
  $\gelem \cdot x_1 = x_1$. By an immediate induction, we 
  conclude that $\gelem \cdot x_i = x_i$ for all $0 \leq i \leq n$,
  but then we also have that the degree of $\gelem \cdot x_n$ is less than $2$
  in $\monelt_m$, which contradicts the fact that $\gelem \cdot \monelt_n \divleq \monelt_m$.
\end{remark}

\begin{conjecture}[Schmitz]
  \label{conj:wqo-infinite-path}
  Let $\mathcal{C}$ be a class of finite structures. Then, the following are
  equivalent:
  \begin{enumerate}
    \item The class of structures of $\mathcal{C}$ labelled with 
      any \kl{well-quasi-ordered} set $(Y, \leq)$ is
      itself \kl{well-quasi-ordered} under the
      labelled-induced-substructure relation.
    \item For every existential formula $\varphi(x,y)$,
      there exists $N_\varphi \in \N$, such 
      that $\varphi$ does not \kl(efo){define paths} of length greater than $N_\varphi$
      in the structures of $\mathcal{C}$.
  \end{enumerate}
  Where a formula \intro(efo){defines a path} of length $n$ in a structure
  if there exists $n$ distinct elements $a_0, \ldots, a_{n-1}$ in the structure
  such that $\varphi(a_i, a_j)$ holds if and only if $|i - j| = 1$.
\end{conjecture}

\begin{remark}
  \label{rem:conj-wqo-pouzet}
  The conjecture of Schmitz is a generalization of Pouzet's conjecture
  \cite{POUZ72} that states that a class $\mathcal{C}$  of finite structures is
  \kl{well-quasi-ordered} under the labelled induced-substructure relation for
  every \kl{well-quasi-ordered} set of labels, 
  if and only if it is the case for the set of two incomparable labels.
\end{remark}

\begin{remark}
  \label{rem:conj-wqo-infinite-path}
  Let $\Indets$ be an infinite \kl{homogeneous structure},
  such that $(\mon{\Indets}, \gdivleq)$ is not a \kl{well-quasi-ordering}.
  Then, the collection of finite substructures of $\Indets$
  labelled by $(\N,\leq)$ is not \kl{well-quasi-ordered} under the
  labelled-induced-substructure relation.
  Hence, if one believes that \cref{conj:wqo-infinite-path} holds,
  there exists an existential formula $\varphi(x,y)$ such that
  $\varphi$ defines arbitrarily long paths in $\Indets$.
  Because $\Indets$ is \kl{homogeneous},
  this means that $\varphi$ defines an infinite path in $\Indets$,
  and in particular, 
  $\Indets$ contains an \kl(of){infinite path} $P$, as introduced
  for generic sets of indeterminates.
\end{remark}

\paragraph{Monomial Reachability}
The undecidability results we will present in this section regarding the
\kl{equivariant ideal membership problem} will use the polynomials in a very
limited way: we will only need to consider \emph{monomials}, and there will
even be a bound on the maximal exponent used. Before going into the details of
our reductions, let us first introduce an intermediate problem that will be
easier to work with: the (equivariant) \kl{monomial reachability problem}. 

\begin{definition}
  \label{def:mon-rewrite-system}
  A \intro{monomial rewrite system} is a finite set of pairs of the form
  $\set{\monelt, \monelt'}$ where $\monelt, \monelt' \in \mon{\Indets}$.
  The \intro{monomial reachability problem} is the problem of deciding whether
  there exists a sequence of rewrites that transforms $\monelt_s$ into $\monelt_t$
  using the rules of a monomial rewrite system $R$, where
  a \intro(monrew){rewrite step} is a pair of the form
  \begin{equation*}
    \monelt[n] (\gelem \cdot \monelt)
    \leftrightarrow_R 
    \monelt[n] (\gelem \cdot \monelt')
    \text{ if } \set{\monelt, \monelt'} \in R
    \text{ and } \gelem \in \group
    \quad .
  \end{equation*}
\end{definition}

\begin{example}
  \label{ex:mon-rewrite-system}
  Let $\Indets = \N$ and $\group$ be the set of all bijections of $\Indets$.
  Then, the rewrite system $x_1^2 x_2^2 \leftrightarrow_R x_1^2$
  satisfies $\monelt \leftrightarrow_R^* x_1^2$ if and only if 
  $\monelt$ has all its exponents that are multiple of $2$.
\end{example}

The following \cref{lem:mon-rewrite-red-membership} shows that the \kl{monomial
reachability problem} can be reduced to the \kl{equivariant ideal membership
problem}, and follows the exact same reasoning as in the case of finitely many
indeterminates \cite{MAME82}. This reduction was also noticed in \cite[Theorem
64]{GHOLAS24}.


\begin{lemma}[label=lem:mon-rewrite-red-membership,ref=lem:mon-rewrite-red-membership]
  One can solve the \kl{monomial reachability problem}
  provided that one can solve the \kl{equivariant ideal membership problem}.
\end{lemma}

In order to show that the \kl{equivariant ideal membership problem} is
undecidable, it is therefore enough to show that the \kl{monomial reachability
problem} is undecidable. To that end, we will encode the Halting problem of a
Turing machine. There are two main obstacles to overcome: first, the
reversibility of the rewriting system, which can be (partially) solved by
considering \emph{deterministic} and \emph{reversible} Turing machines; and
second, the fact that the configurations of the Turing machine cannot
staightforwardly be encoded as monomials due to the commutativity of the
multiplication.


\paragraph{Structures Containing Paths.} \AP Let us assume for the rest of this
section that $\Indets$ is a set of indeterminates that contains an
\kl(of){infinite path}, let us fix a binary alphabet $\Sigma \defined
\set{a,b}$. Given a \kl(of){finite path} $P \defined \seqof{x_i}[0 \leq i <
4n]$, we define a function $\intro*\wenc{ \cdot}_P \colon \Sigma^{\leq n} \to
\mon{\Indets}$, where $\Sigma$ is a finite alphabet, that encodes a word $u \in
\Sigma^{\leq n}$ as a monomial. Namely, we define inductively
$\wenc{\varepsilon} \defined 1$, $\wenc{a u}_P = x_0^4 x_1^2 x_2^1 x_3^3
(\mathsf{shift}_{+4} \cdot \wenc{u}_P)$ and $\wenc{b u}_P = x_0^4 x_1^1 x_2^2
x_3^3 (\mathsf{shift}_{+4} \cdot \wenc{u}_P)$ for all $u \in \Sigma^*$, where
$\mathsf{shift}_{+k}$ acts on $P$ by shifting the indices by
$k$.\footnote{There may be no element $\gelem \in \group$ that acts like
$\mathsf{shift}_{+1}$, we only use it as a function.} Let us remark that
\kl{monomial rewriting} applied on \kl{word encodings} can simulate
(reversible) string rewriting on words of a given size.

\begin{lemma}
  \label{lem:word-encoding-string-subst}
  Let $P,Q$ be two \kl(of){finite paths} in $\Indets$,
  such that $(p_0,p_1)$ is in the same orbit as 
  $(q_0,q_1)$.
  Let $u,v,w \in \Sigma^*$ be three words, such that $|u| = |v| \leq |w|$,
  and let $\monelt[n] \in \mon{\Indets}$ be a monomial.
  Assume that there exists $\gelem \in \group$
      such that $\wenc{w}_P = \monelt[m] (\gelem \cdot \wenc{u}_Q)$,
       $\monelt[n] = \monelt[m] (\gelem \cdot \wenc{v}_Q)$,
  and that $\wenc{w}_P$, $\wenc{u}_Q$ and $\wenc{v}_Q$
  are well-defined.
  Then,
      there exists $x, y \in \Sigma^*$
      such that $x u y = w$ and $\wenc{x v y}_P = \monelt[n]$.
\end{lemma}
\begin{proof}
  Let us write $\gelem \cdot q_0 = p_k$ for some $k \in \N$.
  Because the only indeterminates with degree $4$ in $\wenc{w}_P$ are
  the ones of the form $p_{4i}$, we have that $k$ is a multiple of $4$
  (i.e. at the start of a letter block).
  Since $(q_0, q_1)$ is in the same orbit as $(p_0, p_1)$,
  and both $P$ and $Q$ are \kl(of){finite paths},
  we conclude that $\gelem \cdot (q_0, q_1) = (p_{4i}, p_{4i+1})$
  or $\gelem \cdot (q_0, q_1) = (p_{4i+1}, p_{4i-1})$.
  Applying the same reasoning, thrice, 
  we have either $\gelem \cdot (q_0, q_1, q_2, q_3) = (p_{4i}, p_{4i+1}, p_{4i+2}, p_{4i+3})$
  or $\gelem \cdot (q_0, q_1, q_2, q_3) = (p_{4i}, p_{4i-1}, p_{4i-2}, p_{4i-3})$.
  However, in the second case, the exponent of $p_{4i-3}$ in $\wenc{w}_P$ is at most $2$,
  which is incompatible with the fact that the one of $q_3$ in $\wenc{u}_Q$ is $3$.
  By induction on the length of $u$, we immediately obtain that 
  $\gelem \cdot \wenc{u}_Q = \mathsf{shift}_{+4i} \cdot \wenc{u}_P$ and
  therefore that 
  $w = x u y$ for some $x,y \in \Sigma^*$.
  Finally, because $\wenc{v}_Q$ uses exactly the same indeterminates as 
  $\wenc{u}_Q$, we can also conclude that
  $\wenc{xvy}_P = \monelt[n]$.
\end{proof}

\Cref{lem:word-encoding-string-subst} shows that all encodings
using \kl(of){finite paths} with the same initial orbit are compatible with
each other for the purpose of \kl{monomial rewriting}. Let us now assume that
the alphabet is any finite set of letters, using a suitable unambiguous
encoding of the alphabet in binary \cite{BERST09}. This bigger alphabet size
will simplify the statement and proof of the following
\cref{lem:reversible-machine}, which explains how to simulate a
reversible Turing machine using \kl{monomial rewriting}. Given a reversible
Turing machine $M$ with a finite set $Q$ of states and tape alphabet $\Sigma$,
we will consider the following alphabet $\Gamma \defined \set{ \triangleleft,
  \triangleright } \times \set{ \text{pre}, \text{run}, \text{post} } \uplus Q
  \uplus \Sigma \uplus \set{ \square, \square_1, \square_2}$. The letter
  $\square$ is a blank symbol, and the letters $\triangleleft$ and
  $\triangleright$ are used to delimit the beginning and the end of the tape,
  with some extra ``phase information''. In a first \kl{monomial rewrite
  system}, we will encode a run of a reversible Turing machine $M$ on a fixed
  size input tape (\cref{lem:reversible-machine}), and in a second
  \kl{monomial rewrite system}, we will create a tape of arbitrary size
  (\cref{lem:tape-creation}). The union of these two \kl{monomial
  rewrite systems} will then be used to prove the undecidability of the
  \kl{equivariant ideal membership problem} in \cref{thm:undecidable-paths}.

\begin{lemma}
  \label{lem:reversible-machine}
  Let us fix $(x_0, x_1)$ a pair of indeterminates.
  There exists a
  \kl{monomial rewrite system} $R_M$ such that the following
  are equivalent for every $n \geq 1$,
  and for any \kl(of){finite path} $P$ of length $4(n+2)$ 
  such that $(p_0, p_1)$ is in the same orbit as $(x_0, x_1)$:
  \begin{enumerate}
    \item $\wenc{ \triangleright^{\text{run}} q_0 \square^{n-1}
                  \triangleleft^{\text{run}}
     }_P \leftrightarrow_{R_M}^* 
     \wenc{ \triangleright^{\text{run}} q_f \square^{n-1}
                  \triangleleft^{\text{run}} }_P$,
      \item $M$ halts on the empty word using a tape bounded by $n-1$ cells.
  \end{enumerate}
  Furthermore, every monomial that is 
  reachable from $\wenc{ \triangleright^{\text{run}} q_0 \square^{n-1} \triangleleft^{\text{pre}} }_P$
  or $\wenc{ \triangleright^{\text{run}} q_f \square^{n-1} \triangleleft^{\text{run}} }_P$
  is the image of a word of the form
  $\wenc{\triangleright^{\text{run}} u \triangleleft^{\text{run}}}_P$  
  where $u \in (Q \uplus \Sigma \uplus \square)^n$.
\end{lemma}
\begin{proof}
  Transitions of the deterministic reversible Turing machine using bounded tape size can be 
  modelled as a reversible string rewriting system using finitely many rules 
  of the form $u \leftrightarrow v$, where $u$ and $v$ are words
  over $(Q \uplus \Sigma \uplus \square)$ having the same length $\ell$
  For each rule $u \leftrightarrow v$, we create rules 
  $\wenc{u}_P \leftrightarrow_{R_M} \wenc{v}_P$ 
  for every \kl(of){finite path} $P$ of length $4l$.
  Note that there are only orbit finitely many such \kl(of){finite paths} $P$,
  and one can effectively list some representatives,
  because $\Indets$ is \kl{effectively oligomorphic}.
  This system is clearly complete, in the sense that one can perform a substitution
  by applying a monomial rewriting rule, but \cref{lem:word-encoding-string-subst}
  also tells us it is correct, in the sense that it cannot perform anything else
  than string substitutions.
  Furthermore, we can assume 
  that the reversible Turing machine
  starts with a clean tape and ends with a clean tape.
\end{proof}

\Cref{lem:reversible-machine} shows that one can simulate the
runs, provided we know in advance the maximal size of the tape used by the
reversible Turing machine. The key ingredient that remains to be explained is
how one can start from a finite monomial $\monelt$ and create a tape of
arbitrary size using a \kl{monomial rewrite system}. The difficulty is that we
will not be able to ensure that we follow one specific \kl(of){finite path}
when creating the tape.

\begin{lemma}
  \label{lem:tape-creation}
  Let $(x_0, x_1)$ be a pair of indeterminates, $P$ be a \kl(of){finite path}
  such that $(p_0, p_1)$ is in the same orbit as $(x_0, x_1)$.
  There exists a \kl{monomial rewrite system} $R_\text{pre}$
  such that for every monomial $\monelt \in \mon{\Indets}$, the following are
  equivalent:
  \begin{enumerate}
    \item $\wenc{ \triangleright^{\text{pre}} \square \square_1 \square_2 \triangleleft^{\text{pre}}}_P
      \leftrightarrow_{R_\text{pre}}^* 
      \monelt$
      and $\wenc{\triangleright^{\text{run}}}_{P'}
      \gdivleq \monelt$ for some \kl(of){finite path} $P'$ such that
      $(p_0', p_1')$ is in the same orbit as $(x_0, x_1)$.
    \item There exists $n \geq 2$ and a \kl(of){finite path} $P'$ such that 
      $(p_0', p_1')$ is in the same orbit as $(x_0, x_1)$,
      and 
      $\monelt = \wenc{ \triangleright^{\text{run}} q_0 \square^{n}
      \triangleleft^{\text{run}} }_{P'}$.
  \end{enumerate}
  Similarly, there exists a \kl{monomial rewrite system} $R_\text{post}$
  with analogue properties using $q_f$ instead of $q_0$.
\end{lemma}
\begin{proof}
  We create the following rules,
  where $P_1$ and $P_2$ range over \kl(of){finite paths} such that
  their first two elements are in the same orbit as $(x_0, x_1)$,
  and assuming that the indeterminates of $P_1$ and $P_2$ are disjoint:
  \begin{enumerate}
    \item Cell creation: 
      $\wenc{\triangleright^{\text{pre}} \square}_{P_1}
        \wenc{ \square_1 \square_2 \triangleleft^{\text{pre}}}_{P_2}
      \leftrightarrow_{R_\text{pre}}
      \wenc{\triangleright^{\text{pre}} \square_1}_{P_1}
      \wenc{ \square \square \square_2 \triangleleft^{\text{pre}}}_{P_2}$
    \item Linearity checking:
      $\wenc{\square_1 \square}_{P_1} \wenc{\square_2 \triangleleft^{\text{pre}}}_{P_2}
      \leftrightarrow_{R_\text{pre}}
      \wenc{\square \square_1}_{P_1} \wenc{\square_2 \triangleleft^{\text{pre}}}_{P_2}$
    \item Phase transition:
      $\wenc{\triangleright^{\text{pre}} \square}_{P_1}
       \wenc{\square_1 \square_2 \triangleleft^{\text{pre}}}_{P_2}
      \leftrightarrow_{R_\text{pre}}
       \wenc{\triangleright^{\text{run}} q_0}_{P_1}
       \wenc{\square \square \triangleleft^{\text{run}}}_{P_2}$
  \end{enumerate}
  Note that there are only orbit finitely many such pairs of monomials,
  and that we can enumerate representative of these orbits because 
  $\Indets$ is \kl{effectively oligomorphic}.

  Let us first argue that this system is complete. Because there exists an
  infinite path $P_{\infty}$, it is indeed possible to reach
  $\wenc{\triangleright^{\text{run}} q_0 \square^n
  \triangleleft^{\text{run}}}_{P_\infty}$ by repeatedly applying the first
  rule, and then the second rule until $\square_1$ reaches the end of the tape,
  and continuing so until one decides to apply the third rule to reach the
  desired tape configuration.

  We now claim that the system is correct, in the sense that it can only reach
  valid tape encodings. First, let us observe that in a rewrite sequence, one
  can always assume that the rewriting takes the form of applying the first
  rule, then the second rule until one cannot apply it anymore, and repeating
  this process until one applies the third rule. Because rule (2) ensures that
  when we add new indeterminates using rule (1), they were not already present
  in the monomial, and because rule (1) ensures that locally the structure of
  the indeterminates remains a \kl(of){finite path}, we can conclude that the
  whole set of indeterminates used come from a \kl(of){finite path} $P'$. As a
  consequence, if one can reach a state where (1) or (3) are applicable, then
  the tape is of the form $\wenc{ \triangleright^{\text{pre}} \square^n
  \square_1 \square_2 \triangleleft^{\text{pre}} }_{P'}$, with $n \geq 1$. It
  follows that when one can apply rule (3), the monomial obtained is of the
  form $\wenc{ \triangleright^{\text{run}} q_0 \square^n
  \triangleleft^{\text{run}} }_{P'}$, where $P'$ is a \kl(of){finite path} such
  that $(p_0', p_1')$ is in the same orbit as $(x_0, x_1)$. 
\end{proof}

\csname thm:undecidable-paths\endcsname*
\begin{proof}
  It suffices to combine the rewriting systems $R_M$, $R_\text{pre}$ and 
  $R_\text{post}$ by taking their union.
\end{proof}


\begin{remark}
  \label{rem:more-generally}
  The undecidability result of \cref{thm:undecidable-paths} can be generalised to
  any set of indeterminates in which one can encode words over a binary alphabet,
  and for which there is a \kl{monomial rewrite system} that can
  produce arbitrary long words.
  We strongly conjecture that this is the case for 
  the \kl{infinite dimensional vector space}, as defined in 
  \cref{ex:bit vector}.
\end{remark}

% LTeX: language=en
%%!TEX root = ../atomic.asmart.tex
%
\section{Relation to Existing Results and Examples}
\label{sec:examples}

In this section, we are interested in the consequences of our decidability
results. First, we will provide numerous examples of sets of indeterminates
that satisfy our \kl{computability assumptions} as well as our
\kl{well-quasi-ordering} conditions. Then, we will discuss how our results can
be applied to solve various decidability problems in theoretical computer
science.

\todo[inline]{for arka: integrate these examples}
\begin{example}
  \label{ex:q-is-super-wqo}
  The set $(\mon[Y]{\Indets_\calQ}, \gdivleq)$ is a \kl{well-quasi-ordering} whenever $Y$ is
  one, and in particular $\Indets_\calQ$ satisfies the computability assumptions and
  the termination assumptions of both \cref{thm:compute-egb}
  and
  \cref{thm:decide-equiv-ideal-mem}.
\end{example}
\begin{proof}
  Let $\seqof{\monelt_i}[i \in \N]$ be a sequence of monomials in
  $\mon[Y]{\Indets_\calQ}$. Let us write each monomial $\monelt_i$ as
  a finite word $w_i$ over the alphabet $Y$, by writing all the exponents in the order 
  prescribed by the indeterminates.
  Because $Y$ is a \kl{well-quasi-ordering}, the set of all finite words over $Y$ is
  \kl{well-quasi-ordered} by the \emph{scattered subword} relation \cite{HIG52}.
  Now, if $w_i$ is a scattered subword of $w_j$, then
  $\monelt_i \gdivleq \monelt_j$, by choosing a suitable $\gelem \in \group$.
\end{proof}

\begin{example}
  \label{ex:z-is-not-wqo}
  The set $(\mon[Y]{\Indets_\Z}, \gdivleq)$ is not a \kl{well-quasi-ordering} whenever $Y$ is
  contains two distinct elements.
  In particular, $\Indets_\Z$ does not satisfy the termination assumptions of
  \cref{thm:compute-egb} and \cref{thm:decide-equiv-ideal-mem}.
\end{example}
\begin{proof}
  Assume that $Y$ has two distinct elements $a$ and $b$, and let us assume without loss of generality
  that $a \leq b$. The sequence of monomials 
  $\monelt_i \defined x_1^{b} x_2^{a} \cdots x_{i-1}^{a} x_i^{b}$
  forms an infinite antichain in $(\mon[Y]{\Indets_\Z}, \gdivleq)$.
  Indeed, if $\monelt_i \gdivleq \monelt_j$ for some $i < j$, then
  without loss of generality, $\gelem_i (x_1) = x_1$, and 
  $\gelem_i (x_i) = x_j$, because these are the only ones that can be 
  equipped with a large enough exponent.
  Therefore, $\gelem_i = \mathrm{id}$ since the group only contains translations.
  However, this implies that the exponent of $x_j$ in $\monelt_j$ is at most $a$,
  which contradicts the fact that it is $b$.
\end{proof}

\subsection{Crafting sets of indeterminates}
%
We give some interesting examples of group actions $\group \actson \Indets$ and discuss which of them satisfy the necessary condition of \Cref{thm:compute-egb}.
We also describe operations to build new group actions from old ones,
and discuss which of them preserve this condition.

\arka{Did we write somewhere that divisibility is same as labelled embedding}

\todo[inline]{for arka : add citations}

In all of our examples $\Indets$ is a set with some structure, described by some relations and functions on that set,
and $\group$ is the group $\aut{\Indets}$ of all automorphisms (i.e.\ bijections that preserve and reflect the structure) of $\Indets$.
Moreover, all the structures in our examples are \intro{homogenous} :
isomorphism between finite induced substructures extends to an automorphism of the whole structure.
We refer the reader to \cite[Chapter 7]{BOJAN16inf} and \cite{homsurvey} for more details on homogeneous structures.
%To show that $(\mon[Y]{\Indets},\gdivleq)$ is a WQO we use the following strategy:
%we define an one-to-one function $f_{\Indets} : \mon[Y]{\Indets} \to W_{\Indets}$ to some well-known well-quasi-ordered set $(W_{\Indets},\leq)$ such that $f_{\Indets}(\monelt[p]) \leq f_{\Indets}(\monelt[q])$ if and only if $\monelt[p] \gdivleq \monelt[q]$.
%The reason why this strategy works is because in all of our examples, $\Indets$ will be a \intro{homogeneous} structure (\arka{cite wikipedia and macpherson survey}),
%and $f_{\Indets}$ essentially maps an element $\monelt[p]$ of $\mon[Y]{\X}$,
%thought as a finite induced substructure labelled by $Y$,
%to its isomorphism class.
%
\begin{example}\label{ex:eq atoms}
Let $\A$ be an infinite set without any additional structure other than the equality relation.
Then $\aut{\A}$ is the set of all bijections of the set $\A$.
This action $\aut{\A} \actson \A$ does not preserve any linear order on $\A$.
However, $(\mon[Y]{\A}, \gdivleq)$ is a \kl{WQO} whenever $Y$ is a \kl{WQO}.
Observe that up to isomorphism, a monomial $\monelt[p]$ in $\mon[Y]{\A}$ can be though as a multiset of its coefficients.
For example, the monomials $a^{y}b^{z}c^{y}$ can be though as the multiset $\{y,z,y\}$ for every $a,b,c\in\A$ and $y,z\in Y$.
We leave it to the reader to prove that for every $\monelt[p],\monelt[q]\in\mon[Y]{\A}$ we have $\monelt[p] \gdivleq[\aut{\A}] \monelt[q]$ if and only if the multiset of coefficients of $\monelt[p]$ is smaller than or equal to the multiset of coefficients of $\monelt[q]$ in the multiset ordering \cite[Section 1.5]{SCSC17}.
Note that the latter is a \kl{WQO} \cite[Corollary 1.21]{SCSC17}.
\end{example}
%
\begin{example}\label{ex:dlo}
Let $\calQ$ be the set of rational numbers ordered by the usual ordering.
Note that under this ordering, $\calQ$ is a dense linear order without endpoints.
We write $\calQ$ instead of $\Q$ to emphasise that we use its elements as indeterminates and not as coefficients of polynomials. 
$\aut{\calQ}$ is the set of all order preserving bijections of $\calQ$.
By definition, the action $\aut{\calQ} \actson \calQ$ preserves the linear order on $\calQ$.
Moreover, $(\mon[Y]{\calQ}, \gdivleq)$ is a \kl{WQO} whenever $Y$ is a \kl{WQO}.
From a monomial $\monelt[p]$ in $\mon[Y]{\calQ}$ we get a word in $Y^*$ by writing all the exponents in the order prescribed by the indeterminates.
For example, for every $a < b < c \in \calQ$ and $u,v\in Y$,
from the monomial $(a^u b^v c^u)$ we get the word $uvu$.
We claim that for every $\monelt[p],\monelt[q]\in\mon[Y]{\D}$ we have $\monelt[p] \gdivleq[\aut{\A}] \monelt[q]$ if and only if the word corresponding word to $(\monelt[p])$ is smaller than or equal to the word corresponding word to $(\monelt[q])$ in the scattered subword ordering, which is a \kl{WQO} due to Higman's lemma \cite{HIG52}.
\end{example}
%
\begin{example}\label{ex:int}
Let $\calZ$ be the set of integers ordered by the usual ordering.
Then $\aut{\calZ}$ is the set of all order preserving bijections of $\D$.
Note that every order preserving bijection of the set $\calZ$ is a translation $n \mapsto n + c$ for some constant $c\in\calZ$.
By definition, the action $\aut{\calZ} \actson \calZ$ preserves the linear order on $\Z$.
However, $(\mon[Y]{\calZ}, \gdivleq[\aut{\calZ}])$ is not a \kl{WQO} even when $Y$ is a singleton.
An example of an infinite antichain is the set $\setof{a b}{b\in\calZ\setminus\{a\}}$, for any fixed $a\in\calZ$.
\end{example}

\begin{example}\label{ex:rado}
Let $\G$ be the Rado graph (\cite[Section 7.3.1]{BOJAN16inf},\cite[Example 2.2.1]{homsurvey}),
i.e.\ the unique (up to isomorphism) homogeneous infinite graph such that every finite graph is isomorphic to some finite induced subgraph of $\G$.
Then $\aut{\G}$ is the set of all automorphisms of the graph $\G$.
The action $\aut{\G}\actson\G$ does not preserve any linear order of $\G$,
since due to homogeneity,
for every pair of vertices $a,b\in\G$, there exists $\pi\in\aut{\G}$ such that $\pi(a) = b$ and $\pi(b) = a$.
Moreover, $(\mon[Y]{\G}, \gdivleq[\aut{\G}])$ is not a \kl{WQO} even when $Y$ is singleton.
For instance, any subset $\setof{\monelt[c]_{n}}{n = 3,4,5,\dots}$ of $\mon[Y]{\G}$ such that the subgraph of $\G$ induced by $\dom(\monelt[c]_{n})$ is a cycle of length $n$, is an infinite antichain.
\end{example}
%
\begin{example}\label{ex:bit vector}
Let $\V$ be an infinite dimensional vector space over $\ftwo$.
Then $\aut{\V}$ is the set of all linear automorphisms.
The action $\aut{\V}\actson\V$ does not preserve any linear order of $\V$,
since for every pair of linearly independent vectors $u,v\in\V$,
there exists $\pi\in\aut{\V}$ such that $\pi(u) = v$ and $\pi(v) = u$.
Moreover, $(\mon[Y]{\V}, \gdivleq[\aut{\V}])$ is not a WQO even when $Y$ is a singleton.
For instance, given an infinite set of linearly independent vectors $\{v_1,v_2,\dots\}$ in $\V$,
the subset $\setof{\monelt[s]_{n}}{n = 3,4,5,\dots}$ of $\mon[Y]{\V}$ such that \[
\dom(\monelt[s]_{n}) = \{v_1,\dots,v_n,(v_1 + v_2),(v_2 + v_3),\dots,(v_{n-1}+v_n),(v_n+v_1)\}
\]
is an infinite antichain.
\arka{Trying to think of an easy argument}
\end{example}
%
\begin{example}\label{ex:dense tree}
Let $\T$ denote the dense-tree structure
\cite[Section 7.3.3]{BOJAN16inf},
which is the unique (up to isomorphism) homogeneous infinite tree such that every finite tree is isomorphic to a topological minor of $\T$.
The tree structure is given by the \intro{closed common ancestor} function,
which takes as input a pair of nodes of $\T$ and returns their closest common ancestor.
In this case, $\aut{\T}$ is the set of bijections $\pi$ of $\T$ that preserves and reflects ancestry:
$\pi(a)$ is an ancestor of $\pi(b)$ if and only if $a$ is an ancestor of $b$.
The action $\aut{\T}\actson\T$ does not preserve any linear order,
since for every pair of nodes $a$ and $b$ such that their closest common ancestor $c$ is neither $a$ or $b$,
there exists $\pi\in\aut{\T}$ such that $\pi(a) = b$ and $\pi(b) = a$.
However, $(\mon[Y]{\T},\gdivleq[\aut{\T}])$ is a \kl{WQO} whenever $Y$ is a \kl{WQO}.
This follows from Kruskal's theorem \cite[Page 212]{Kruskal60} in the following way.
Extend $Y$ to another \kl{WQO} $Y'$ by adding an element $\bot$ which is smaller than every element in $Y$.
The domain of an element $\monelt[p]\in\mon[Y]{\T}$ induces a finite subtree of $\T$ which is the smallest subtree of $\T$ containing $\dom(\monelt[p])$ and closed under the closest common ancestor relation.
The coefficients of $\monelt[p]$ defines a partial labelling of nodes of this subtree using $Y$.
By labelling the unlabelled nodes by $\bot$ we get a finite subtree of $\T$ labelled by $Y'$.
Due to homogeneity,
$\monelt[p] \gdivleq \monelt[q]$ if and only if the labelled subtree corresponding to $\monelt[p]$ is isomorphic to a topological minor of the labelled subtree corresponding to $\monelt[q]$.
Now it follows from Kruskal's lemma that $(\mon[Y]{\T},\gdivleq[\aut{\T}])$ is a \kl{WQO}.
\end{example}
%
For the remainder of this section fix a pair of group actions
$\G\actson\X$ and $\calH\actson\Y$
%
\begin{example}\label{ex:union}
The group actions $\G\actson\X$ and $\calH\actson\Y$ can be extended to a group action $(\G\times\calH)\actson(\X\uplus\Y)$ by defining
\[
(\pi,\sigma)(z) =
\begin{cases}
\pi(z) & \text{if }z\in\X,\\
\sigma(z) & \text{if }z\in \Y.
\end{cases}
\]
If the actions $\G\actson\X$ and $\calH\actson\Y$ preserve linear orders then so does the actions $(\G\times\calH)\actson(\X\uplus\Y)$.
Fix a \kl{WQO} $Q$.
We have
\[
(\mon[Q]{\X\uplus\Y}, \gdivleq[\G\times\calH])
\cong
(\mon[Q]{\X}\times\mon[Y]{\Y}, \gdivleq[\G]\times\gdivleq[\calH])
\]
where $\gdivleq[\G]\times\gdivleq[\calH]$ is the product of the orders $\gdivleq[\G]$ and $\gdivleq[\calH]$ defined as
\[
(x,y) \gdivleq[\G]\times\gdivleq[\calH] (x',y')
\quad
\text{if}
\quad
x \gdivleq[\G] x' \text{ and } y \gdivleq[\calH] y' \ .
\]
This implies if $(\mon[Q]{\X},\gdivleq{\G}$ and $(\mon[Q]{\Y},\gdivleq{\calH}$ are \kl{WQOs},
so is $(\mon[Q]{\X\uplus\Y}, \gdivleq[\G\times\calH])$ \cite[Lemma 1.5]{SCSC17}.
\end{example}
%
\begin{example}[\protect{\textnormal{\cite[Example 10]{GHOLAS24}}}]\label{ex:product}
We extend the group actions $\G\actson\X$ and $\calH\actson\Y$ to a group action $(\G \times \calH)\actson(\X\times\Y)$, as 
\[
(\pi,\sigma)((x,y)) = (\pi(x),\sigma(y)) \ .
\]
This group action corresponds to taking products of structures (\arka{citation}).
$(\mon[Q]{\X\times\Y}, \gdivleq[\G \times \calH])$ is not a \kl{WQO} even when $Q$ is a singleton.
For every element $q\in Q$ and two infinite sets $\{x_1,x_2,\dots\}\subseteq\X$ and $\{y_1,y_2,\dots\}\subseteq\Y$,
the set
\[
\setof{(x_1,y_1)^q(x_1,y_2)^q(x_2,y_2)\dots(x_{n-1},y_n)^q(x_n,y_n)^q(x_n,y_0)^q}{
n = 3,4,5,\dots} \subseteq \mon[Q]{\X\times\Y}
\]
is an infinite antichain.
\end{example}
%
The above example shows that products of structures do not preserve our \kl{WQO} property.
Now we show that the lexicographic product of structures
(\cite[Section 2]{GHOLAS24}) preserve this property.
\arka{any other citation?}
%
\begin{example}\label{ex:nested product}
Let $\G\otimes\calH$ be the group whose elements are of the form $(\pi,(\sigma^{x})_{x\in\X})$, where $\sigma_{x}\in\calH$ for every $x\in\X$.
Multiplication is defined as
\[
(\pi_1,(\sigma^{x}_1)_{x\in\X})\cdot(\pi_2,(\sigma^{x}_2)_{x\in\X})
=
(\pi_1\cdot\pi_2, (\sigma_1^{\pi_2(x)}\cdot\sigma_2^x)_{x\in\X}) \ .
\]
The action $\G\otimes\calH \actson \X\times\Y$ is given as
\[
(\pi,(\sigma^{x})_{x\in\X})(x',y') =
(\pi(x'),\sigma^{x'}(y'))\ , \text{ for }(x',y')\in\X\times\Y \ .
\]
Essentially,
each element $x\in\X$ carries its own copy $\{x\}\times\Y$ of the structure $\Y$,
and different copies of the structure $\Y$ can be permuted independently.
If $\G\actson\X$ and $\calH\actson\Y$ preserves the linear orders $<_{\X}$ and $<_{\Y}$, respectively,
then $\G\otimes\calH \actson \X\times\Y$ preserves the lexicographic linear order on $\X\times\Y$ defined as
\[
(x,y) <_{\ell ex} (x',y')\quad\text{if}\quad
\text{$x <_{\X} x'$, or $x = x'$ and $y <_{\Y} y'$.}
\]
Moreover, for every \kl{WQO} $Q$,
if $(\mon[Q]{\X},\gdivleq{\G})$ and $(\mon[Q]{\Y},\gdivleq{\calH})$ are \kl{WQOs},
so is $(\mon[Q]{\X\uplus\Y}, \gdivleq[\G\otimes\calH])$ \cite[Lemma 9]{GHOLAS24}.
\end{example}
%
%\paragraph{Sets with atoms.}
%
%
%\todo[inline]{Say that if one starts with atoms and equality, then we can only 
%  have dimension 1, and that this is the case of the rationals.}
%
%\paragraph{Relational structures.} Let $\mathbb{A}$ be an infinite relational
%structure with finitely many relations. Then, one can consider the set of
%polynomials $\poly{\K}{\mathbb{A}}$, where indeterminates are elements of the
%universe of $\mathbb{A}$. The group of all automorphisms of $\mathbb{A}$ (i.e.,
%bijections of the universe that preserve the relations) acts on
%$\poly{\K}{\mathbb{A}}$ by permuting the indeterminates.
%
%Natural examples are polynomials whose indeterminates are indexed by the
%natural numbers (with inequality), or the rationals (with inequality). In this
%setting, \kl{effective oligomorphcity} means that \todo{do it}. The fact that
%$(\mon{\mathbb{A}}, \gdivleq)$ is a well-quasi-ordering corresponds to ordering
%\emph{finite substructures} of $\mathbb{A}$ by the \emph{labelled induced
%substructure} relation, and asking whether the class obtained is
%well-quasi-ordered. This is a well-studied question in graph theory, where a
%conjecture of Pouzet states that this holds with two labels if and only if it
%holds for every ordinal. In particular, for such structures, it is therefore
%conjectured that $(\mon{\Indets}, \gdivleq)$ is a well-quasi-ordering if and
%only if $(\mon[\om \ordplus \ordfin{1}]{\Indets}, \gdivleq)$, $(\mon[\om
%\ordplus \om]{\Indets}, \gdivleq)$, and $(\mon[\om^2]{\Indets}, \gdivleq)$ are
%well-quasi-orderings too.
%\arka{So we can use any well-ordered set of labels?} 
%\todo[inline]{
%  Cite \cite{POUZ72},
%  \cite{DRT10} for the conjecture.
%}


\paragraph{On reducts of structures.} \AP Let $\sigma$ be a finite relational
signature, and $\tau \subseteq \sigma$ be another finite relational signature.
Let $\mathbb{A}, \mathbb{B}$ be respectively a $\sigma$ and a $\tau$ structure. We say
that \intro{$\mathbb{B}$ is a reduct of $\mathbb{A}$} when $\mathbb{B}$ is
obtained from $\mathbb{A}$ by keeping the same universe, and relations. It was noted by \cite[Lemma 13]{GHOLAS24} that in
this case, the \kl{equivariant Hilbert basis property} transfers from
$\mathbb{A}$ to $\mathbb{B}$. Let us briefly argue that this transfer holds too
for our \cref{thm:decide-equiv-ideal-mem,thm:compute-egb}.

\arka{The next lemma should be rewritten, right?}

\begin{lemma}
  \label{lem:reducts-equiv-hilbert}
  Let $\mathbb{A}$ be a relational structure, let $\mathbb{B}$ be a 
  \kl(struct){reduct} of $\mathbb{A}$. Then, if $\mathbb{A}$ satisfies the
  hypotheses of \cref{thm:decide-equiv-ideal-mem},
  then one can decide the \kl{equivariant ideal membership problem} for
  $\poly{\K}{\mathbb{B}}$. Similarly, 
  if $\mathbb{A}$ satisfies the hypotheses of
  \cref{thm:compute-egb}, then one can compute an
  \kl{equivariant Gröbner basis} of an
  \kl{equivariant ideal} of $\poly{\K}{\mathbb{B}}$.
\end{lemma}
\begin{proof}
  \todo[inline]{Just write it, and it works.}
\end{proof}

\AP 
As a consequence, one can apply our results to structures that are not equipped 
with an ordering, because one can always consider the 

\todo[inline]{Talk about $\N$ I guess.}

\paragraph{Computable oligomorphicity}
\arka{for me}


\subsection{Applications}


\paragraph{Polynomial computations.} \AP The fact that (finite control) systems
performing polynomial computations can be verified follows from the theory of
\kl{Gröbner bases} on finitely many indeterminates \cite{MULSEI02,BEDUSHWO17}.
There were also numerous applications to automata theory, such as deciding
whether a weighted automaton could be determinised (resp. desambiguated)
\cite{BESM23,PUSM24}. We refer the readers to a nice survey recapitulating the
successes of the so-called ``Hilbert method'' automata theory \cite{BOJAN19}. A
natural consequence of the effective computations of \kl{equivariant Gröbner
bases} is that one can apply the same decision techniques to \emph{orbit finite
polynomial computations}. For simplicity and clarity, we will focus on
\kl{polynomial automata} without states or zero-tests \cite{BEDUSHWO17}, but
the same reasoning would apply to more general systems as we will discuss in
\cref{rem:topological-wsts}.


\AP Before discussing the case of orbit finite polynomial automata, let us
recall in \cref{ex:polynomial-automata} the setting of \kl{polynomial automata}
in the classical case, as studied by \cite{BEDUSHWO17}, with techniques that
dates back to \cite{MULSEI02}. A \intro{polynomial automaton} is a tuple $A
\defined (Q, \Sigma, \delta, q_0, F)$, where $Q = \K^n$ for some finite $n \in
\N$, $\Sigma$ is a finite alphabet, $\delta \colon Q \times \Sigma \to Q$ is a
transition function such that $\delta(\cdot,a)_i$ is a polynomial in the
indeterminates $q_1, \dots, q_n$ for every $a \in \Sigma$ and every $i \in
\set{1, \dots, n}$, $q_0 \in Q$ is the initial state, and $F \colon Q \to \K$
is a polynomial function describing the final result of the automaton. The
\intro{zeroness problem for polynomial automata} is the following decision
problem: given a \kl{polynomial automaton} $A$, is it true that for all words
$w \in \Sigma^*$, the polynomial $F(\delta^*(q_0, w))$ is zero? It is known
that the \kl{zeroness problem for polynomial automata} is decidable
\cite{BEDUSHWO17}, using the theory of \kl{Gröbner bases} on finitely many
indeterminates. Let us now propose a new model of computation called \kl{orbit
finite polynomial automata}, and prove an analogue decidability result.

\AP Let us fix a group $\group$ that acts on the set of indeterminates
$\Indets$, and on an alphabet $\Sigma$ in an \kl{effectively oligomorphic}
fashion. We write $\K^{(\Indets)}$ for the set of finitely supported functions
from $\Indets$ to $\K$, i.e., the set of functions $f \colon \Indets \to \K$
such that there exists a finite set $S \subseteq \Indets$ such that $f(x) = 0$
for every $x \notin S$. Given an element $f \in \K^{(\Indets)}$, and given a
polynomial $p \in \poly{\K}{\Indets}$, we write $p(f)$ for the evaluation of
$p$ on $f$.

\todo[inline]{for arka: we never defined finitely supported functions.}

\begin{definition}
  \label{def:orbit-finite-polynomial-automaton}
  An \reintro{orbit finite polynomial
  automaton} is a tuple $A \defined (Q, \Sigma, \delta, q_0, F)$, where $Q =
  \K^{(\Indets)}$, $\Sigma$ is an \kl{orbit finite} alphabet, $\delta \colon
  \Sigma \to (\Indets \to \poly{\K}{\Indets})$ is a \kl(func){finitely supported}
  polynomial update function, and $F \in \poly{\K}{\Indets}$ is a polynomial
  computing the result of the automaton. 

  Given a letter $a \in \Sigma$ and a
  state $q \in Q$, the update $\delta^*(q,a)$ is defined as the function from
  $\Indets$ to $\K$ defined by $\delta^*(q,a) \colon x \mapsto \delta(a,x)[ q ]$,
  which is well-defined because $\delta(a,x)$ is a \kl{finitely supported}
  polynomial. The update function is naturally extended to words. Finally, the
  output of an \kl{orbit finite polynomial automaton} on a word $w \in \Sigma^*$
  is defined as $F(\delta^*(q_0, w))$.
\end{definition}


\begin{example}
  \label{ex:orbit-finite-polynomial-automata}
  Let $\Indets = \calQ$, and let $\group$ be the group of all
  order-preserving bijections of $\calQ$.
  Let $\Sigma \defined \calQ \times \calQ$.
  Then, the following function are computable by 
  \kl{orbit finite polynomial automata}:
  the number $\mathrm{inc}(w)$ of letters $(a,b)$ such that $a < b$ in a word $w \in \Sigma^*$,
  the number $\mathrm{dec}(w)$ of letters $(a,b)$ such that $a > b$ in a word $w \in \Sigma^*$,
  and the number $(\mathrm{inc}(w) - \mathrm{dec}(w))^2$.
\end{example}

\AP As for \kl{polynomial automata}, the \intro(ofpa){zeroness problem} for
orbit finite polynomial automata is the following decision problem: decide if
for every input word $w$, the output $F(\delta^*(q_0, w))$ is zero. Solving the
\kl(ofpa){zeroness problem} for orbit finite polynomial automata allows us to
decide the equality of two such automata, by computing their difference. Let us
prove that the \kl(ofpa){zeroness problem} is decidable for \kl{orbit finite
polynomial automata}.

\todo[inline]{the proof of the following is incorrect and I do not know
  how to fix it easily}

\begin{theorem}
  \label{cor:orbit-finite-polynomial-automata-zeroness}
  Let $\Indets$ be a set of indeterminates that satisfies the
  \kl{computability assumptions} and such that $(\mon[Y]{\Indets}, \gdivleq)$ is a
  \kl{well-quasi-ordering}, for every \kl{well-quasi-ordered} set $(Y, \leq)$.
  Then, the \kl(ofpa){zeroness problem} is decidable for all \kl{orbit finite polynomial automata}
  with register names in $\Indets$, for every \kl{orbit finite} alphabet $\Sigma$,
  that is \kl{effectively oligomorphic} with respect to the action of $\group$.
\end{theorem}
\begin{proof}
  Let $A = (Q, \Sigma, \delta, q_0, F)$ be an \kl{orbit finite polynomial
  automaton}. Following the classical \emph{backward procedure} for such
  systems, we will compute a sequence of sets $E_0 \defined \setof{ q \in Q }{
  F(q) = 0 }$, and $E_{i+1} \defined \mathrm{pre}^\forall(E_i) \cap E_i$, where
  $\mathrm{pre}^\forall(E)$ is the set of states $q \in Q$ such that for every
  $a \in \Sigma$, $\delta^*(q,a) \in E$. We will prove that the sequence of
  sets $E_i$ stabilises, and that it is computable. As an immediate
  consequence, it suffices to check that $q_0 \in E_{\infty}$, where $E_\infty$
  is the limit of the sequence $(E_i)_{i \in \N}$, to decide the
  \kl(ofpa){zeroness problem}.

  The only idea of the proof is to notice that all the sets $E_i$ are
  representable as zero-sets of \kl{equivariant ideals} in
  $\poly{\K}{\Indets}$, allowing us to leverage the effective computations of
  \cref{cor:equivariant-ideals-computations}. Given a set $H$ of polynomials,
  we write $\mathcal{V}(H)$ the collections of states $q \in Q$ such that $p(q)
  = 0$ for all $p \in H$.
  It is easy to see that $E_0 = \mathcal{V}(\set{F}) = \mathcal{V}(\idl_0)$,
  where $\idl_0$ is the \kl{equivariant ideal} generated by $F$. 
  Furthermore, assuming that $E_i = \mathcal{V}(\idl_i)$, we can
  see that 
  \begin{align*}
    \mathrm{pre}^\forall(E_i) 
    & = \setof{ q \in Q }{ \forall a \in \Sigma, \delta^*(q,a) \in E_i } \\
    & = \setof{ q \in Q }{ \forall a \in \Sigma, \forall p \in \idl_i, p(\delta^*(q,a)) = 0 } \\
    & = \setof{ q \in Q }{ \forall p' \in \idl[J], p'(q) = 0 }
  \end{align*}
  Where, the \kl{equivariant ideal} $\idl[J]$ is generated by the
  polynomials $\mathrm{pullback}(p,a) \defined p [ x \mapsto \delta(a)(x)]$
  for every pair $(p, a) \in \idl_i \times \Sigma$. 
  As a consequence, we have $E_{i+1} = \mathcal{V}(\idl_{i+1})$, where
  $\idl_{i+1} = \idl_i + \idl[J]$.

  Because the sequence $\seqof{ \idl_i }[ i \in \N]$ is increasing, and thanks
  to the \kl{equivariant Hilbert basis property} of $\poly{\K}{\Indets}$, there
  exists an $n_0 \in \N$ such that $\idl_{n_0} = \idl_{n_0 + 1} = \idl_{n_0 +
  2} = \cdots$. In particular, we do have $E_{n_0} = E_{n_0 + 1} = E_{n_0 + 2}
  = \cdots$.

  Let us argue that we can compute the sequence $\idl_i$ effectively.
  First,  $\idl_0 = \EqIdlGen{F}$ is finitely represented.
  Now, 
  given an \kl{equivariant ideal} $\idl$, represented by an \kl{orbit finite}
  set of generators $H$,
  we can compute the \kl{equivariant ideal} $\idl[J]$ generated by the
  polynomials $\mathrm{pullback}(p,a) \defined p [ x_i \mapsto \delta(a)(x_i)]$
  for every pair $(p, a) \in H \times \Sigma$. Indeed, $H \times \Sigma$ is
  \kl{orbit finite} because the action of $\group$ on $\Indets$ is
  \kl{effectively oligomorphic}, and the function $\mathrm{pullback}$ is
  computable and \kl(func){equivariant}: indeed, given $\gelem \in \group$, we can
  show that
  \begin{align*}
    \gelem \cdot \mathrm{pullback}(p, a) & = 
    \gelem \cdot (p [ x_i \mapsto \delta(a)(x_i)]) \\
    & = p [ x_i \mapsto (\gelem \cdot \delta(a, x_i))] \\
    & = p [ x_i \mapsto \delta(\gelem \cdot a, \gelem \cdot x_i))] \\
    & = (\gelem \cdot p) [ x_i \mapsto \delta(\gelem \cdot a, x_i)] \\
    & = \mathrm{pullback}(\gelem \cdot p, \gelem \cdot a).
  \end{align*}
  
  Finally, one can detect when the sequence stabilises, by checking whether
  $\idl_i = \idl_{i+1}$, which is decidable because the
  \kl{equivariant ideal membership problem} is decidable 
  by \cref{thm:compute-egb}.

  To conclude, it remains to check whether $q_0 \in E_\infty$,
  which amounts to check that $q_0 \in \mathcal{V}(\idl_\infty)$.
  This is equivalent to checking whether for every element $p \in \Basis$
  where $\Basis$ is an \kl{equivariant Gröbner basis} of $\idl_\infty$, we have
  $p(q_0) = 0$, which can be done by enumerating relevant orbits.
\end{proof}


\begin{remark}
  \label{rem:topological-wsts}
  The notion of
  \intro{topological well-structured transition system} was introduced by
  Goubault-Larrecq in \cite{JGL07}, noticing that the pre-existing notion of
  \kl{Noetherian space} could serve as a topological generalisation of
  \kl{Noetherian rings} (where ideal-based method can be applied),
  and 
  \kl{well-quasi-orderings}, for which the celebrated decision procedures on
  \kl{well-structured transition systems} can be applied \cite{ABDU96}. In particular,
  Goubault-Larrecq used such systems to verify properties of \emph{polynomial
  programs} computing over the complex numbers, that can communicate over lossy
  channels using a finite alphabet \cite{JGL10}. 
  Because of \cref{cor:equivariant-ideals-computations}, we do have an 
  effective way to compute on the topological spaces at hand, 
  and therefore we can apply the theory of
  \kl{topological well-structured transition systems} to verify systems
  such as \emph{orbit finite polynomial automata communicating using a finite alphabet
  over lossy channels}.
  We refer to \cite[Chapter 9]{JGL13} for a survey on the theory of 
  Noetherian spaces.
\end{remark}

\paragraph{Reachability problem of symmetric data Petri nets.}

\arka{To add : symmetric VAS equations}
\todo[inline]{for arka: define symmetric data Petri nets/ symmetric VAS equations}

\paragraph{Orbit-finite systems of equations}

\todo[inline]{for arka: do it}


%!TEX root = ../atomic.asmart.tex
%
\section{Concluding Remarks}
\label{sec:conclusion}

\todo[inline]{Write it}

\paragraph*{Total orderings.} In all of our paper, we assumed that the set of
indeterminates $\Indets$ is equipped with a total ordering $\varleq$. This
assumption seems necessary, as the notions of leading monomials would cease to
be well-defined without it. However, we do not have a clear understanding of
whether this assumption is vacuous or not. It was conjectured by Pouzet and
restated by Ghosh and Lasota that every \kl{effectively oligomorphic} group
action on a countable set of indeterminates that satisfies the
\kl{well-quasi-ordering} condition on its monomials admits a total ordering
compatible with the  group action.

\paragraph*{All our hypotheses and Pouzet in the middle.} We conjecture that
$(\mon{\Indets}, \gdivleq)$ is a \kl{well-quasi-ordering} if and only if
$(\mon[Y]{\Indets}, \gdivleq)$ is a \kl{well-quasi-ordering} for every
\kl{well-quasi-ordered} set $Y$ of exponents. This is a form of Pouzet's
conjecture, which has been verified on some classes of structures.

\paragraph*{Undecidability.} It would be nice to obtain a dichotomy result
for the decidability, but it seems beyond reach for now.
talk about VASS here  for coverability.

\paragraph*{Complexity.} We do not have complexity lower bounds, and there may
be better algorithms like adaptations of Faugère's algorithm that could be
better in practice.


% Include acknowledgements

% Include the bibliography
\bibliographystyle{splncs04}
\bibliography{papers.bib}

% If there are any appendices, we include them here.
\appendix
\section{intro}
By leveraging the same proof technique,
we can also show that the \kl{equivariant ideal membership problem} is
decidable under a weaker hypothesis, namely that the set of \kl{monomials}
$\mon[\om \ordplus 1]{\Indets}$ is a \kl{WQO}, which is also believed to be
equivalent to the first condition.

\begin{theorem}[name={Equivariant Ideal Membership},restate=thm:decide-equiv-ideal-mem]
  \label{thm:decide-equiv-ideal-mem}
  Let $\Indets$ be a totally ordered set of indeterminates
  equipped with a group action $\group \actson \Indets$, under our \kl{computability assumptions}.
  If $(\mon[\om \ordplus 1]{\Indets}, \gdivleq)$ is a \kl{WQO}, then one can decide the
  \kl{equivariant ideal membership problem}.
\end{theorem}


\section{Proofs of \cref{sec:examples}}

\AP A \intro{topological space} is a set $X$ equipped with a collection $\tau$
of subsets of $X$ that is stable under finite intersections and arbitrary
unions.\footnote{In particular, $\tau$ contains the empty set and $X$ itself.}
In a \kl{topological space}, elements of $\tau$ are called \intro{open
subsets}, while their complements (in $X$) are called \intro{closed subsets}. A
\kl{topological space} is \intro(space){Noetherian} when, for every sequence
$\seqof{U_i}[i \in \N]$ of \kl{open subsets}, there exists $n \in \N$ such that
$\bigcup_{i \in \N} U_i = \bigcup_{i \leq n} U_i$. We refer the readers to the
book \cite{JGL13} for a comprehensive introduction to \kl{Noetherian spaces}
and their usage in theoretical computer science. Let us briefly argue that
\kl{Noetherian spaces} generalize \kl{well-quasi-orders} in
\cref{ex:well-quasi-orders-are-noeth}, and encode the
\kl{Hilbert basis property} in \cref{ex:polynomials-noetherian}.

\begin{example}[ see \cite{JGL13}]
  \label{ex:well-quasi-orders-are-noeth}
  Let $(X, \leq)$ be a quasi-ordered set.
  Then, the set $X$ equipped with the \kl{topology} having 
  as \kl{open subsets} the upwards-closed subsets of $X$ is \kl(space){Noetherian}
  if and only if $(X, \leq)$ is \kl{well-quasi-ordered}.
\end{example}

\begin{example}[ see \cite{JGL13}]
  \label{ex:polynomials-noetherian}
  Let $\K$ be a field, and let $n \in \N$.
  The space $\K^n$ equipped with the \kl{Zariski topology}
  \kl(space){Noetherian}; where the \intro{Zariski topology}
  is the topology whose \kl{closed subsets} are finite unions of sets
  of the form $\setof{ \vec{x} \in \K^n}{ \forall p \in \idl, p(\vec{x}) = 0}$,
  where $\idl$ is an \kl{ideal} of $\poly{\K}{x_1, \dots, x_n}$.
\end{example}

\AP The advantage of \kl{Noetherian spaces} over \kl{well-quasi-orderings} and
\kl{Noetherian rings} is that they generalize both and can be \emph{combined}:
\kl{Noetherian spaces} are closed under finite sums, finite products,
considering finite words, considering finite trees, and many more \todo{cite}.
As a consequence, they provide a versatile tool to express the set of states of
a system, ensuring that a strong termination property holds.

\AP A \intro{topological well-structured transition system} with alphabet
$\Sigma$ is a \kl{topological space} $(X, \tau)$, equipped with a transition
function $\delta \colon X \times \Sigma \to X$, such that the following
properties hold: for every $U \in \tau$, $\mathrm{pre}^\exists(U)$, the set of
states $x \in X$ such that there exists $a \in \Sigma$ with $\delta(x, a) \in
U$, is an \kl{open subset}. Equivalently, the set $\mathrm{pre}^\forall(E)$ of
states $x \in X$ such that for every $a \in \Sigma$, $\delta(x, a) \in E$ is a
\kl{closed subset} of $X$ whenever $E$ is itself a \kl{closed subset} of $X$.
The natural decition problem for \kl{topological well-structured transition
systems} is the following \intro{open reachability problem} is decidable: given
an initial state $x_0 \in X$ and an \kl{open subset} $U \in \tau$, is it true that
there exists a word $w \in \Sigma^*$ such that $\delta^*(x_0, w) \in U$? The
prototypical algorithm to solve this problem is the following \intro{backward
algorithm}: start with $U_0 \defined U$, and iteratively compute $U_{i+1}
\defined U_i \cup \mathrm{pre}^\exists(U_i)$ until $U_i = U_{i+1}$, then check
whether $x_0 \in U_\text{last}$.
There are easy-to-state sufficient conditions  for such an algorithm to be computable and terminate:
\begin{enumerate}
  \item One is equipped with an effective representation of open subsets,
    where one is able to test equality of open subsets, compute unions of open subsets, and test 
    membership of a point in an open subset.
  \item The pre-image function $\mathrm{pre}^\exists$ is computable, i.e., one can
    compute the set $\mathrm{pre}^\exists(U)$ for every open subset $U$.
  \item The space $(X, \tau)$ is \kl{Noetherian}. 
\end{enumerate}

\AP Our \cref{cor:equivariant-ideals-computations} shows that
under some assumptions on $\Indets$, the set of finitely supported functions
$\Indets \to \K$ is a \kl{Noetherian space} with respect to the
\intro{equivariant Zariski topology}, i.e., the topology whose \kl{closed subsets}
are finite unions of sets of the form $E_{\idl} \defined \setof{f \in
\K^{(\Indets)}}{\forall p \in \idl, p(f) = 0}$, where $\idl$ is an
\kl{equivariant ideal} of $\poly{\K}{\Indets}$. Furthermore, we have an
effective representation of the \kl{closed subsets} in this topology, using
\kl{equivariant Gröbner bases} of \kl{equivariant ideals}. In particular, the
theory of \kl{topological well-structured transition systems} can be applied to
systems whose state space contains ``named registers'' that contain numbers and
are updated by polynomial functions.



\AP Let us fix a group $\group$ that acts on the set of indeterminates
$\Indets$, and on an alphabet $\Sigma$ in an \kl{effectively oligomorphic}
fashion. Let us now consider the case of \intro{orbit finite polynomial
automata}, that we define as follows: an \reintro{orbit finite polynomial
automaton} is a tuple $A \defined (Q, \Sigma, \delta, q_0, F)$, where $Q =
\K^{(\Indets)}$, $\Sigma$ is an \kl{orbit finite} alphabet, $\delta \colon
\Sigma \to (\Indets \to \poly{\K}{\Indets})$ is a \kl(func){finitely supported}
polynomial update function, and $F \in \poly{\K}{\Indets}$ is a polynomial
computing the result of the automaton. Given a letter $a \in \Sigma$ and a
state $q \in Q$, the update $\delta^*(q,a)$ is defined as the function from
$\Indets$ to $\K$ defined by $\delta^*(q,a) \colon x \mapsto \delta(a,x)[ q ]$,
which is well-defined because $\delta(a,x)$ is a \kl{finitely supported}
polynomial. The update function is naturally extended to words. Finally, the
output of an \kl{orbit finite polynomial automaton} on a word $w \in \Sigma^*$
is defined as $F(\delta^*(q_0, w))$.

While all of these reasosing could be done outside the realm of (effective)
\kl{topological well-structured transition systems}, we can use the modularity
of the theory to obtain more complex verification properties. Following the
lines of \cite[Theorem 6]{JGL10}, one can consider the case of communicating
orbit finite polynomial automata, where we have a collection processes that
communicate letters over a finite alphabet using lossy channels, and can
perform polynomial updates on their local state. Deciding whether such a system
can reach a state where one process fails to satisfy a given polynomial
invariant is a special case of the \kl{open reachability problem}, and is
decidable.


\begin{lemma}
  \label{lem:zeroness-problem-polynomial-automata}
  The \kl{zeroness problem for polynomial automata} is a special case of the
  \kl{open reachability problem} for \kl{topological well-structured transition systems}.
\end{lemma}
\begin{proof}
  Let $A = (Q, \Sigma, \delta, q_0, F)$ be a \kl{polynomial automaton}.
  We consider the topological space $(Q, \tau)$, where $\tau$ is the
  \kl{Zariski topology} on $\K^n$.
  Let $\idl$ be an \kl{ideal} of $\poly{\K}{x_1,\dots,x_n}$ generated by the polynomials
  $p_1, \dots, p_m$,
  and let $E \defined \setof{q \in Q}{\forall p \in \idl, p(q) = 0}$,
  a \kl{closed subset} of $Q$.
  Then,
  \begin{align*}
    q \in \mathrm{pre}^\forall(E) & \iff 
    \forall a \in \Sigma, \forall p \in \idl, p(\delta(q, a)) = 0 \\
                                  & \iff 
    \forall a \in \Sigma, \forall p \in \idl, p(\delta(q, a)) = 0 \\
                                  & \iff 
                                  \forall p \in \idl[J], p(q) = 0
  \end{align*}
  where $\idl[J] \defined \IdlGen{ \setof{ p_i[ x_i \mapsto \delta(\cdot, a)_i] }{ i \in \set{1, \dots, m}, a \in \Sigma } }$.
  In particular, one can represent \kl{closed subsets} of $Q$ as finite 
  lists of \kl{ideals} using their \kl{Gröbner bases}, and we showed that 
  one can effectively compute the pre-image of \kl{closed subsets} of $Q$
  via $\mathrm{pre}^\forall$ by substituting polynomials.
  In this representation, it is very easy to compute the union 
  of two \kl{closed subsets}, which is simply concatenating the two lists 
  of \kl{ideals} reperesenting them.
  To compute the intersection of two \kl{closed subsets} $E_1$ and $E_2$,
  one can assume without loss of generality that both are represented by a 
  single ideal (i.e., that they are irreducible closed subsets), respectively 
  $\idl_1$ and $\idl_2$.
  Then, an easy computation shows that 
  $E_1 \cap E_2 = \setof{q \in Q}{\forall p \in \idl_1 + \idl_2, p(q) = 0}$,
  where $\idl_1 + \idl_2$ is the sum of the two ideals.
  Whether a point $q \in Q$ is in a \kl{closed subset} $E$ is decidable
  because one can evaluate the generating polynomials on $q$ and check that 
  it is indeed $0$.
  The equality check is more complicated, and can be done by first 
  normalizing the list of ideals so that their intersection is trivial,
  which requires computing the intersection of ideals
  and performing equality checks on the resulting \kl{ideals}.

  As a consequence, it suffices to test the \kl{open reachability problem} for
  the \kl{topological well-structured transition system} $(Q, \tau)$ with the
  initial state $q_0$ and the \kl{open subset} $U = Q \setminus E_\text{final}$,
  where $E_\text{final} \defined \setof{q \in Q}{F(q) = 0}$ is the \kl{closed subset}
  of states where the automaton outputs zero.
\end{proof}


\section{Proofs of \cref{sec:undecidability}}

\begin{proofof}{lem:mon-rewrite-red-membership}
  Let $R$ be a monomial rewrite system, and let $\monelt_s, \monelt_t \in
  \mon{\Indets}$ be two monomials. We can encode the problem of deciding whether
  $\monelt_s$ can be rewritten into $\monelt_t$ using the rules of $R$ as an
  instance of the \kl{equivariant ideal membership problem} as follows:
  \begin{itemize}
    \item Let $H$ be the set of all polynomials of the form $\monelt - \monelt'$
      for all pairs
      $(\monelt, \monelt') \in R$.
    \item Then, we ask whether $\monelt_s - \monelt_t$ belongs to the ideal generated by $H$.
  \end{itemize}

  It is clear that if $\monelt_s$ can be rewritten into $\monelt_t$ using the
  rules of $R$, then $\monelt_s - \monelt_t$ belongs to the equivariant ideal generated by
  $H$. Conversely, if $\monelt_s - \monelt_t$ belongs to the ideal generated by
  $H$, then 
  \begin{equation}
    \label{eq:mon-rewrite-red-membership}
    \monelt_s - \monelt_t 
    = 
    \sum_{i=1}^n a_i \monelt[n]_i (\gelem_i \cdot \monelt_i - \gelem_i \cdot \monelt'_i)
    \quad .
  \end{equation}

  Let us write the (finite) graph $G$ whose vertices are the monomials
  $\monelt[n] (\gelem_i \cdot \monelt_i)$ and $\monelt[n] (\gelem_i \cdot
  \monelt'_i)$, and whose edges are the directed weighted edges labelled by
  $a_i$ (in a direction that makes the weight positive).

  Let us now analyse \cref{eq:mon-rewrite-red-membership}, and notice that
  identifying monomials in the left and right-hand sides of the equation allows
  us to show that $\monelt_s$ and $\monelt_t$ are vertices of $G$. Furthermore,
  we deduce that the sum of the weights of the edges having $\monelt_s$ as a
  source or target equals $1$, and that the sum of the weights of the edges
  having $\monelt_t$ as a source or target equals $-1$. Finally, for every
  vertex $v$ of $G$ that is not $\monelt_s$ or $\monelt_t$, the sum of the
  weights of the edges having $v$ as a source or target is $0$, again because
  of an analysis of the coefficient of the monomial $v$ in the sum of
  \cref{eq:mon-rewrite-red-membership}.

  Hence, the graph $G$ is a flow network, with a flow value of at least $1$
  from $\monelt_s$ to $\monelt_t$. As a consequence, there must exist a path
  from $\monelt_s$ to $\monelt_t$ in $G$, which is a witness
  of the fact that 
  one can rewrite $\monelt_s$ into $\monelt_t$ using the rules of $R$.
\end{proofof}



\end{document}
