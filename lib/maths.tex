% Little math macros
\NewDocumentCommand{\set}{ m }{\{ #1 \}}
\NewDocumentCommand{\setof}{ m m }{\{ #1 \mid #2 \}}
\NewDocumentCommand{\card}{ m }{\left| #1 \right|}
\NewDocumentCommand{\seqof}{ m O{n \in \Nat} }{\left( #1 \right)_{#2}}
\NewDocumentCommand{\defined}{ }{\triangleq}

\NewDocumentCommand{\range}{ O{1} m }{[#1, #2]}

% functions of all sorts (injective, partial, surjective)
\newcommand{\topartial}{\rightharpoonup}
\newcommand{\toinj}{\hookrightarrow}
\newcommand{\tosurj}{\twoheadrightarrow}
\newcommand{\tobij}{\stackrel{\simeq}{\longrightarrow}}


% Automate the creation of new orderings
% based on a given symbol.
% For instance,
% \NewDocumentOrdering{\pref}{\preceq}{\prec}
% will create the following commands:
% \prefleq and \preflt
% that will respectively expand to
% \mathrel{\kl[\pref]{\preceq}} and \mathrel{\kl[\pref]{\prec}}
\NewDocumentCommand{\NewDocumentOrdering}{ m m m }{
    \expandafter\newcommand\csname #1leq\endcsname{
        \mathrel{\kl[#1]{#2}}
    }
    \expandafter\newcommand\csname #1lt\endcsname{
        \mathrel{\kl[#1]{#3}}
    }
    \knowledge{#1}{notion}
}

% Order macros
\NewDocumentCommand{\upset}{ O{} m }{{\uparrow_{#1} #2}}
\NewDocumentCommand{\dwset}{ O{} m }{{\downarrow_{#1} #2}}


% Number theory
\NewDocumentCommand{\factorial}{ O{} m }{
    \if\relax\detokenize{#1}\relax
        #2!
    \else
        (#2)!
    \fi
}

\newcommand{\A}{\mathcal{A}}
\newcommand{\R}{\mathbb{R}}
\newcommand{\C}{\mathbb{C}}
\newcommand{\F}{\mathcal{F}}
\newcommand{\Q}{\mathbb{Q}}
\newcommand{\N}{\mathbb{N}}
\newcommand{\K}{\mathbb{K}}
\newcommand{\X}{\mathcal{X}}
\newcommand{\Y}{\mathcal{Y}}

\newcommand{\poly}[2]{#1[#2]}
\newcommand{\aut}[2][]{\mathsf{Aut}_{#1}{(#2)}}
\newcommand{\mon}[2][]{\mathsf{Mon}_{#1}(#2)}
\newcommand{\perm}[1]{\mathsf{Perm}(#1)}
\newcommand{\otu}[2]{#1^{(#2)}}
\newcommand{\group}{\mathcal{G}}
\newcommand{\gen}[2]{\langle #1\rangle_{#2}}
\newcommand{\radoG}{\mathbb{G}_{\mathsf{Rado}}}
\newcommand{\radoV}{\mathbb{V}_{\mathsf{Rado}}}
\newcommand{\radoE}{\mathbb{E}_{\mathsf{Rado}}}
\newcommand{\cycleSet}[1][]{\mathsf{Cycles}_{#1}}
\newcommand{\ordinal}{\eta}
\newcommand{\defiff}{\overset{\mathrm{def}}{\iff}}
\newcommand{\defeq}{\overset{\mathrm{def}}{=}}
\newcommand{\hbp}{\text{Hilbert's basis property}}
\newcommand{\orbit}[2][]{\mathsf{orbit}_{#1}{(#2)}}
\newcommand{\order}[1][]{\prec_{#1}}
\newcommand{\ordereq}[1][]{\preceq_{#1}}
\newcommand{\revlex}[1][]{<_{\mathsf{RevLex}}^{#1}}
\newcommand{\revlexeq}[1][]{\leq_{\mathsf{RevLex}}^{#1}}
\newcommand{\gr}{Gr\"obner}
\newcommand{\dom}{\mathsf{dom}}
\newcommand{\spoly}[2]{\mathsf{S}(#1,#2)}
\newcommand{\spolyset}[2]{\mathsf{Sset}_{#1}(#2)}
\newcommand{\spolytext}{$\mathsf{S}$-polynomial}
\newcommand{\lcm}{\mathsf{LCM}}
\newcommand{\lc}[1][]{\mathsf{LC}_{#1}}
\newcommand{\lt}[1][]{\mathsf{LT}_{#1}}
\newcommand{\reducstep}[1]{\to_{#1}}
\newcommand{\reduc}[1]{\to^*_{#1}}
\newcommand{\rem}[2]{\mathsf{Rem}_{#1}(#2)}
\newcommand{\closure}[1]{\widehat{#1}}
\newcommand{\wforder}{\triangleleft}

\newcommand{\probBasic}[4]
{
\begin{flalign*}
\quad
\begin{tabular}{l  l}
  \multicolumn{2}{l}{\mathsf{#1}}\\
  \textbf{Input:}    & #2 \\
  \textbf{#4} & #3
\end{tabular}
&&
\end{flalign*}
}

\newcommand{\prob}[3]
{
\probBasic{#1}{#2}{#3}{Question:}
}
