%!TEX root = ../atomic.asmart.tex
% LTeX: language=en
%
\section{Concluding Remarks}
\label{sec:conclusion}

We have given a sufficient condition for \kl{equivariant Gröbner bases} to be
computable, under natural \kl{computability assumptions}, and we have shown
that our sufficient condition is close to being optimal since the
undecidability of the \kl{equivariant ideal membership problem} can be derived
for a large class of group actions that do not satisfy our condition.
Let us now discuss some open questions and conjectures that arise from our work.

\paragraph{Total orderings on the set of indeterminates} We assumed that the
indeterminates $\Indets$ were equipped with a total ordering $\varleq$ that is
preserved by the group action. This assumption seems necessary, as the notions
of \kl{leading monomials} would cease to be well-defined without it. However,
we do not have a clear understanding of whether this assumption is vacuous or
not. Indeed, as noticed by \cite[Lemma 13]{GHOLAS24}, and
\cref{lem:reducts-equiv-hilbert}, it often suffices to extend the structures of
the indeterminates to account for a total ordering. A conjecture of Pouzet
\cite[Problems 12]{POUZ24} states that such an ordering always exists, and this
was remarked by \cite[Remark 14]{GHOLAS24}. Note that in this case, one would
get a complete characterisation of the group actions for which the
\kl{equivariant Hilbert basis property} holds \cite[Property 4]{GHOLAS24}.

\paragraph{Labelled well-quasi-orderings and dichotomy conjectures} As noted in
\Cref{sec:undecidability}, there are many conjectures relating the fact that
$(\mon[Y]{\Indets}, \gdivleq)$ is a \kl{well-quasi-ordering} (for every
\kl{well-quasi-ordered} set $Y$) and the presence of long
paths of some king (\cref{conj:wqo-infinite-path,rem:conj-wqo-infinite-path}).
In particular, Pouzet's conjecture \cite{POUZ72} would imply that for actions
arising from \kl{homogeneous structures} (as in the examples given in
\cref{sec:act ex}), \cref{thm:compute-egb} and \cref{thm:undecidable-paths} are
two sides of a dichotomy theorem: either the \kl{equivariant ideal membership
problem} is undecidable and there are \kl{equivariant ideals} that are not
\kl{orbit-finitely generated}, or every \kl{equivariant ideal} is
\kl{orbit-finitely generated} and one can compute \kl{equivariant Gröbner
bases}. Let us note that for some classes of graphs having bounded clique
width, Pouzet's conjecture is known to hold \cite{DRT10,LOPEZ24}.
This leads us to the following conjecture:

\begin{conjecture}
  For every action $\group\actson\Indets$ of a group $\group$ on a set
  of indeterminates that is \kl{effectively oligomorphic}, exactly
  one of the following holds:
  \begin{enumerate}
    \item The \kl{equivariant ideal membership problem} is decidable. 
    \item There exists an \kl{equivariant ideal} that is not
      \kl{orbit-finitely generated}.
  \end{enumerate}
\end{conjecture}

Let us point out that a similar conjecture was already stated in the context of
Petri nets with data. Indeed, the condition that $(\mon[Y]{\Indets},\gdivleq)$
is a \kl{WQO} for every \kl{WQO} $Y$ also guarantees coverability of Petri nets
with data $\X$ is decidable \cite[Theorem 1]{Lasota16}, and it was actually
conjectured to be a necessary condition \cite[Conjecture 1]{Lasota16}. 

\paragraph{Complexity} In the present paper, we have focused on the
decidability of the \kl{equivariant ideal membership problem} and the
computability of \kl{equivariant Gröbner bases}. However, we have not addressed
the complexity of such problems, and have only adapted the most basic
algorithms for computing \kl{Gröbner bases}. It would be interesting to know,
on the theoretical side, if one can obtain complexity lower bounds for such
problems, but also on the more practical side if advanced algorithms like
Faugère's algorithm \cite{FAUGERE02} can be adapted to the equivariant setting
and yield better performance in practice.


\paragraph*{Duals of orbit-finitely generated vector spaces} An interesting
question posed by S\l{a}womir Lasota is that whether for group actions
$\group\actson\Indets$ for which $(\mon[Y]{\Indets},\gdivleq)$ is a \kl{WQO}
for every \kl{WQO} $Y$, the duals of orbit-finitely generated vector spaces are
also orbit-finitely generated. This is a well-known fact that finitely
generated vector spaces have finitely generated duals, and the extension to the
orbit-finite setting would find applications in solving orbit-finite linear
systems of equations and extend the results of \cite{GHL22}. It is already
known to hold for actions originating from the \kl{homogeneous structures}
considered in \Cref{ex:eq atoms,ex:dlo}, via the work of
\cite{BFKM24,GHL22,Prz23}. 

