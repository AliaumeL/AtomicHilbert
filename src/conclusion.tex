%!TEX root = ../atomic.asmart.tex
%
\section{Concluding Remarks}
\label{sec:conclusion}

We have proven that the \kl{equivariant ideal membership problem} is decidable
under classical effective assumptions on the group action and the
representation of the set of indeterminates, and under a mild hypothesis on the
behavior of monomials. Furthermore, we have developed an algorithm that
computes an \kl{equivariant Gröbner basis} of an \kl{equivariant ideal}, and
shown that it provides an effective representation to work with ideals
(allowing us to compute unions, intersections, etc.) under a slightly stronger
hypothesis on the divisibility ordering. Finally, we have discussed how the
classical counter-examples to our assumptions on the behavior of monomials can
be turned into undecidability results for the \kl{equivariant ideal membership
problem}. Let us now discuss the main open questions raised by
the present paper.

\paragraph*{Total orderings.}
We assumed that the indeterminates $\Indets$ were equipped with a total ordering $\varleq$ that is preserved by the group action.
This assumption seems necessary,
as the notions of \kl{leading monomials} would cease to be well-defined without it.
However, we do not have a clear understanding of whether this assumption is vacuous or not.
Indeed, as noticed by \cite[Lemma 13]{GHOLAS24}, and \cref{lem:reducts-equiv-hilbert},
it often suffices to extend the structures of the indeterminates to account for a total ordering.
A conjecture of Pouzet \cite[Section 4.4]{POUZ24} states that such an ordering always exists,
and this was remarked by \cite[Remark 14]{GHOLAS24}.
Note that in this case, one would get a complete characterisation of the \kl{equivariant Hilbert basis property} \cite[Property 4]{GHOLAS24}.

\todo[inline]{complete this}
%
\arka{changed title}
\paragraph*{\kl{WQO} property under different sets of labels}
%
We conjecture that
$(\mon{\Indets}, \gdivleq)$ is a \kl{well-quasi-ordering} if and only if
$(\mon[Y]{\Indets}, \gdivleq)$ is a \kl{well-quasi-ordering} for every
\kl{well-quasi-ordered} set $Y$ of exponents. This is a form of Pouzet's
conjecture, which has been verified on some classes of structures.
\arka{some citation}
%
\paragraph*{Dichotomy.}
%
We conjecture that the ideal membership problem is undecidable for every action $\group\actson\X$ that is \kl{oligomorphic} but does not have the \kl{equivariant Hilbert basis property} (cf. \Cref{rem:conj-wqo-infinite-path}).
%
\paragraph*{Complexity.} We do not have complexity lower bounds, and there may
be better algorithms like adaptations of Faugère's algorithm that could be
better in practice.
%
\paragraph*{WQO dichotomy conjecture}
The condition that $(\mon[Y]{\Indets},\gdivleq)$ is a \kl{WQO} for every \kl{WQO} also guarantees coverability of Petri nets with data $\X$ is decidable \cite[Theorem 1]{Lasota16}.
In fact it is conjectured to be equivalent \cite[Conjecture 1]{Lasota16}.
%
\paragraph*{Duals of orbit-finitely generated vector spaces}
%
An interesting question posed by S\l{a}womir Lasota is that whether for group actions $\group\actson\Indets$ for which $(\mon[Y]{\Indets},\gdivleq)$ is a \kl{WQO},
duals of orbit-finitely generated vector spaces are also orbit-finitely generated.
This follows for the structures in \Cref{ex:eq atoms,ex:dlo} by the results of \cite{BFKM24,GHL22,Prz23}.
Proving it for the general case would be a big step towards generalising the results of \cite{GHL22}. 
%
