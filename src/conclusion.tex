%!TEX root = ../atomic.asmart.tex
%
\section{Concluding Remarks}
\label{sec:conclusion}

\todo[inline]{Write it}

\paragraph*{Total orderings.} In all of our paper, we assumed that the set of
indeterminates $\Indets$ is equipped with a total ordering $\varleq$. This
assumption seems necessary, as the notions of leading monomials would cease to
be well-defined without it. However, we do not have a clear understanding of
whether this assumption is vacuous or not. It was conjectured by Pouzet and
restated by Ghosh and Lasota that every \kl{effectively oligomorphic} group
action on a countable set of indeterminates that satisfies the
\kl{well-quasi-ordering} condition on its monomials admits a total ordering
compatible with the  group action.

\paragraph*{All our hypotheses and Pouzet in the middle.} We conjecture that
$(\mon{\Indets}, \gdivleq)$ is a \kl{well-quasi-ordering} if and only if
$(\mon[Y]{\Indets}, \gdivleq)$ is a \kl{well-quasi-ordering} for every
\kl{well-quasi-ordered} set $Y$ of exponents. This is a form of Pouzet's
conjecture, which has been verified on some classes of structures.

\paragraph*{Undecidability.} It would be nice to obtain a dichotomy result
for the decidability, but it seems beyond reach for now.
talk about VASS here  for coverability.

\paragraph*{Complexity.} We do not have complexity lower bounds, and there may
be better algorithms like adaptations of Faugère's algorithm that could be
better in practice.
