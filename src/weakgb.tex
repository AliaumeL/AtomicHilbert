% LTeX: language=en
\section{Weak Equivariant Gröbner Bases}
\label{sec:weakgb}

\AP In this section we prove that a natural adaptation of \kl{Buchberger's
algorithm} to the equivariant setting computes a \kl{weak equivariant Gröbner
basis} of an \kl{equivariant ideal}. This can be seen as a characterisation
result.

\AP
Given a set of indeterminates $\X$ equipped with a group $\group$ acting
effectively oligomorphically on $\X$, and such that 
$\X$ is equipped with a total ordering $\leq$ that is compatible with the
action of $\group$, we will define a total ordering on the set of monomials

\AP A \intro{weak equivariant Gröbner basis} is a finite set $\Basis$ of
polynomials such that for every polynomial 
\begin{enumerate}
  \item $\Basis$ is a generating set of an equivariant ideal $\idl$,
  \item $p \in \EqIdlGen{\Basis}$, there
exists a polynomial $q \in \Basis$ such that $p \toeucl{\pmonleq}{q}^* p'$, and
$p'$ is a polynomial that only contains indeterminates that are smaller or
equal than the indeterminates of $p$ with respect to the ordering $\leq$.
\end{enumerate}

\AP
\begin{enumerate}
  \item First, we define the revlex ordering, show that it has some 
    basic properties.
  \item Then, we define leading monomials, leading terms, etc.
  \item Then, we define S-polynomials, and show that it is equivariant
  \item Then, we explain the saturation algorithm, show 
    that it is equivariant.
  \item Then, we show that the saturation algorithm terminates.
  \item We conclude that we compute a \kl{weak equivariant Gröbner basis}.
\end{enumerate}

\AP Having fixed a total ordering $\monord$ on monomials, we can define the
\intro{leading monomial} of a polynomial $p$ as the largest monomial appearing
in $p$ with respect to $\monord$. We can then define the \intro{leading term}
of $p$ as the product of its \kl{leading monomial} and its \intro{leading
coefficient}, and the \intro{characteristic monomial} of $p$ as the product of
its \kl{leading monomial} and all the indeterminates appearing in $p$.

