% LTeX: language=en
\section{Weak Equivariant Gröbner Bases}
\label{sec:weakgb}

\AP In this section we will prove that one can compute a good enough basis
using an adaptation of \kl{Buchberger's algorithm} to the equivariant case.

\begin{itemize}
  \item define the ordering on polynomials with the support
  \item explain that it is wqo
  \item explain that one can work with orbit finite sets
    without talking about representatives every time
  \item construct the buchberger algorithm using this ordering
  \item We obtain a set that is a \kl{weak equivariant Gröbner basis}:
    for every element of the ideal, we can find a polynomial in the
    \kl{weak equivariant Gröbner basis} that can be used to reduce it without
    respecting the support condition.
  \item Example of why it is not enough to obtain a real 
    decision procedure.
  (reducing and adding new indeterminates every time)
\end{itemize}

