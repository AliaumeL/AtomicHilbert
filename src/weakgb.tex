% LTeX: language=en
\section{Weak Equivariant Gröbner Bases}
\label{sec:weakgb}

\AP In this section we prove that a natural adaptation of \kl{Buchberger's
algorithm} to the equivariant setting computes a \kl{weak equivariant Gröbner
basis} of an \kl{equivariant ideal}. This can be seen as an analysis of the
classical algorithm in the equivariant setting. We will assume for the rest of
the section that $\Indets$ is a set of indeterminates equipped with a group
$\group$ acting \kl{effectively oligomorphically} on $\X$, and that $\X$ is
equipped with a total ordering $\varleq$ that is \kl(ord){compatible} with the
action of $\group$. The crucial object of this section is the notion of
\kl{decomposition} of a polynomial with respect to a set $H$
(\cref{def:decomposition}).

\begin{definition}
  \label{def:decomposition}
  Let $H$ be a set of polynomials. A \intro{decomposition} of $p$
  with respect to $H$ is given by a finite sequence 
  $\mathfrak{d} \defined \seqof{(a_i, \monelt_i, h_i)}[i \in I]$ such that
   $ p = \sum_{i \in I} a_i \monelt_i h_i$,
  where $a_i \in \K$, $\monelt_i \in \mon{\X}$, and $h_i \in H$ for all $i \in I$.
  The \intro{domain of the decomposition}
  that we write $\intro*\domdec(\mathfrak{d})$ is defined as the union
  of the domains of the polynomials $\monelt_i h_i$ for all $i \in I$.
  The \intro{leading monomial of the decomposition} is defined as
  $
    \intro*\lmdec(\mathfrak{d}) \defined \max(\seqof{\lm(\monelt_i h_i)}[i \in I])
  $.
\end{definition}

Leveraging the notion of decomposition, we can define a weakening of the notion
of \kl{equivariant Gröbner basis}, that essentially mimics the classical notion
of \kl{equivariant Gröbner basis} at the level of \kl{decompositions} instead
o,f polynomials.

\begin{definition}
  An \kl{equivariant set} $\Basis$ of polynomials is 
  a \intro{weak equivariant Gröbner basis} of an \kl{equivariant ideal}
  $\idl$ if $\IdlGen{\Basis} = \idl$, and if for every polynomial $p \in \idl$,
  and decomposition $\mathfrak{d}$ of $p$ with respect to $\Basis$, there
  exists a decomposition $\mathfrak{d}'$ of $p$ with respect to $\Basis$ such that
  $\domdec(\mathfrak{d}') \subseteq \domdec(\mathfrak{d})$,
  and 
  such that $\lmdec(\mathfrak{d}') = \lm(p)$.
\end{definition}

\AP To compute \kl{weak equivariant Gröbner bases}, we will use a rewriting
relation. Given $p,r \in \poly{\K}{\X}$, we write $p \intro*\toeucl{H}
r$ if and only if there exists $q \in H$, $a \in \K$, and $\monelt \in
\mon{\X}$ such that $p = a \monelt q + r$, $\dom(r) \subseteq \dom(p)$, and
$\lm[\Indets](r) \revlexlt \lm[\Indets](p)$. In order to simplify the
notations, we will write $p \intro*\pmonlt r$ to denote $\dom(r) \subseteq
\dom(p)$, and $\lm[\Indets](r) \revlexlt \lm[\Indets](p)$, leaving the
ordered set of indeterminates $\Indets$ implicit.
The relation $\pmonleq$ is extended to decompositions by using 
the analogues of $\dom$ and $\lm$ for decompositions.

\begin{lemma}
  \label{lem:chm}
  The quasi-ordering $\pmonleq$ is \kl(ord){compatible} with the action of $\group$,
  and is well-founded on polynomials, and on \kl{decompositions} of polynomials.
\end{lemma}
\begin{proof}
  The first property is immediate because $\dom$, $\lm$, and $\revlexleq$ are
  compatible with the group action $\group$. 
  The second property follows from the fact that $\revlexlt$ is a total
  well-founded ordering whenever one has fixed finitely many possible 
  indeterminates. In a decreasing sequence, the support of the leading 
  monomials is also decreasing, so that sequence only contains finitely many 
  indeterminates, hence we conclude.
  The same proof works for decompositions.
\end{proof}

\AP As a consequence of \cref{lem:chm}, we know that the rewriting
relation $\toeucl{H}$ is \kl{terminating} for every set $H$. Given a $H$ of
polynomials, and given a polynomial $p \in \poly{\K}{\X}$, we say that $p$ is
\intro{normalised} with respect to $H$ if there are no transitions $p
\toeucl{H} r$. The set of \intro{remainders} of $p$ with respect to $H$ is
denoted $\intro*\rem{H}{p}$, and is defined as the set of all polynomials $r$
such that $p \toeucl{H}^* r$ and $r$ is \kl{normalised} with respect to $H$.
The following lemma states that $\rem{H}{\cdot}$ is a computable function from
in our setting.

\begin{lemma}
  \label{lem:normalisation}
  Let $H$ be an \kl{orbit finite set} of polynomials, and let $p \in \poly{\K}{\X}$ be a
  polynomial. Then $\rem{H}{p}$ is finite.
  Furthermore, this computation
  is \kl(func){equivariant}.
\end{lemma}
\begin{proof}
  Let us write $H = \orbit[\group]{H'}$, where $H'$ is a finite set of
  polynomials.
  Because the relation $\toeucl{H}$ is \kl{terminating}, it suffices to 
  show that for every polynomial $p$, there are finitely many polynomials $r$ 
  such that $p \toeucl{H} r$. This is because 
  $p \toeucl{H} r$ implies that 
  $p = \alpha \monelt[n] (\gelem\cdot q) + r$ for some $q \in H'$, 
  $\alpha \in \K$, $\monelt[n] \in \mon{\X}$, and $\gelem \in \group$.
  Because, $\lm(r) \revlexlt \lm(q)$, we  
  conclude that $\lm(p) = \lm(\alpha \monelt[n] (\gelem\cdot q))$, and 
  therefore $r$ is uniquely determined by the choice of $q \in H'$ and the
  choice of $\gelem \in \group$ that maps the \kl(poly){domain} of $q$ to the \kl(poly){domain} of
  $p$. There are finitely elements in $H'$ and finitely many such functions
  $\gelem \in \group$ because both domains are finite.
\end{proof}

\AP Now that we have a quasi-ordering on polynomials, we will prove that given
an \kl{orbit finite} set $H$ of generators, we can compute a \kl{weak
equivariant Gröbner basis}. The computation will closely follow the classical
\kl{Buchberger's algorithm}. The main idea being to saturate the set of
generators $H$ to remove some \emph{critical pairs} of the rewriting relation
$\toeucl{H}$. Namely, given two polynomials $p$ and $q$ in $H$, we compute the
set $\intro*\CancelPoly{p}{q}$ of cancellations between $p$ and $q$ as the set of
polynomials of the form $r = \alpha \monelt[n] p + \beta \monelt[m] q$ such
that $\lm(r) < \max(\monelt[n] \lm(p), \monelt[m]\lm(q))$, where $\alpha,\beta
\in \K$, and where $\monelt[n], \monelt[m] \in \mon{\X}$. Let us recall that
given two monomials $\monelt[n], \monelt[m] \in \mon{\X}$, one can compute
$\intro*\lcm(\monelt[n], \monelt[m])$ as the least common multiple of the two
monomials, and that this in an \kl{equivariant operation}.
Using this, we can introduce the \intro{S-polynomial} of two polynomials $p$ and $q$
as in \cref{eq:spoly}.
 \begin{equation}
    \label{eq:spoly}
    \intro*\spoly{p}{q} \defined
    \frac{\lcm(\lm(p), \lm(q))}{\lt(p)} \times p
    - \frac{\lcm(\lm(p), \lm(q))}{\lt(q)} \times q
    \quad .
  \end{equation}



\begin{lemma}
  \label{lem:spoly}
  Let $p$ and $q$ be two polynomials in $\poly{\K}{\X}$.
  All the polynomials in $\CancelPoly{p}{q}$ are obtained by multiplying a monomial
  with
  their \kl{S-polynomial} $\spoly{p}{q}$.
\end{lemma}
\begin{proof}
  Let $p,q \in \poly{\K}{\X}$, and let $r \in \CancelPoly{p}{q}$.
  By definition, there exists $\alpha,\beta \in \K$ and $\monelt[n], \monelt[m]
  \in \mon{\X}$ such that $r = \alpha \monelt[n] p + \beta \monelt[m] q$ and
  $\lm(r) < \max(\monelt[n] \lm(p), \monelt[m] \lm(q))$.
  In particular,
  we conclude that $\lm(\monelt[n] p) = \lm(\monelt[m] q)$, and that 
  $\alpha \lc(\monelt[n] p) + \beta \lc(\monelt[m] q) = 0$.

  Let us write $\Delta = \lcm(\lm(p), \lm(q))$.
  Because $\lm(\monelt[n] p) = \lm(\monelt[m] q)$, there exists a monomial 
  $\monelt[l] \in \mon{\X}$ such that 
  $\lm(\monelt[n] p) = \monelt[l] \Delta = \lm(\monelt[m] q)$.
  Furthermore,
  we know that $\lc(p) \beta = - \lc(q) \alpha$.
  As a consequence, one can rewrite $r$ as follows:
  \begin{equation*}
    r = 
    \monelt[l] \alpha \lc(p) 
    \left[
      \frac{\Delta}{\lt(p)} \times p
      - \frac{\Delta}{\lt(q)} \times q
    \right]
    = 
    \monelt[l] \alpha \lc(p) \times \spoly{p}{q} \ .
  \end{equation*}
  We have concluded.
\end{proof}

Remark that the \kl{S-polynomial} is equivariant: if $\gelem \in \group$, then
$\spoly{\gelem \cdot p}{\gelem \cdot q} = \gelem \cdot \spoly{p}{q}$. Given a
set $H$, we write $\intro*\spolyset(H) \defined \bigcup_{p,q \in H}
\rem{H}{\spoly{p}{q}}$. We are now ready to define the saturation algorithm
that will compute \kl{weak equivariant Gröbner bases}, described in
\cref{alg:weakgb}.
Let us remark that \cref{alg:weakgb}
is
an actual algorithm (\cref{lem:weakgb-computable}) that is
\kl(func){equivariant}.

\begin{algorithm}
  \caption{Computing \kl{weak equivariant Gröbner bases} using the algorithm \intro{weakgb}.}
    \label{alg:weakgb}
    \KwIn{An orbit finite set $H$ of polynomials}
    \KwOut{An orbit finite set $\Basis$ that is a \kl{weak equivariant Gröbner basis} of
      $\EqIdlGen{H}$}
    \Begin{
        $\Basis \gets H$\;
        \Repeat{$\Basis$ stabilizes}{
            $\Basis \gets \Basis \cup \spolyset(\Basis)$\;
        }
        \Return{$\Basis$}\;
    }
\end{algorithm}

\begin{lemma}
  \label{lem:weakgb-computable}
  The \cref{alg:weakgb} is a computable function that is itself
  \kl(func){equivariant}.
\end{lemma}
\begin{proof}
  Let us prove that $\spolyset(H)$ is a computable function
  that is \kl(func){equivariant}. First,
  one can  compute $H^2$, which is an \kl{orbit finite} set of pairs,
  because $H$ is \kl{orbit finite} and $\Indets$ is 
  \kl{effectively oligomorphic}.
  Then, one can remark that $\spoly{p}{q}$ is computable and \kl(func){equivariant},
  hence one can compute $\spolyset(H)$ as the set of all
  $\spoly{p}{q}$ for $(p,q) \in H^2$.
  Furthermore, one can decide whether the set $\Basis$ stabilizes because the 
  membership of a polynomial $p$ in $\Basis$ is decidable, since 
  $\poly{\K}{\Indets}$ is \kl{effectively oligomorphic} and $\Basis$ is \kl{orbit finite}.
\end{proof}


Let us now use the semantic assumptions to prove the termination of
\cref{alg:weakgb}
(\cref{lem:weakgb-termination})
and the correctness of the resulting orbit finite set
(\cref{lem:weakgb-correctness}).

\begin{lemma}
  \label{lem:weakgb-termination}
  Assume that $(\mon{\X}, \gdivleq)$ is a \kl{WQO}. Then, 
  \cref{alg:weakgb} terminates
  on every \kl{orbit finite} set $H$ of polynomials.
\end{lemma}
\begin{proof}
  Let $\seqof{H_n}[n \in \N]$ be the sequence of (orbit finite) sets of polynomials
  computed by \cref{alg:weakgb}. 
  We associate to each set $H_n$ the set $L_n$ of \kl{characteristic monomials} of the
  polynomials in $H_n$. Because the set of monomials is a \kl{WQO}, and because 
  the sequences are non-decreasing for inclusion, there exists an 
  $n \in \N$ such that, for every $\monelt \in L_{n+1}$, there exists
  $\monelt[n] \in L_n$, such that $\monelt[n] \gdivleq \monelt$.

  We will prove that $H_{n+1} = H_n$ by contradiction. Assume towards this
  contradiction that there exists some $r \in H_{n+1} \setminus H_n$. By
  definition of $H_{n+1}$, there exists $p,q \in H_n$ such that $r \in
  \rem{H_n}{\spoly{p}{q}}$. In particular, $r$ is \kl{normalised} with respect
  to $H_n$. However, because $r \in H_{n+1}$, $\cm(r) \in L_{n+1}$, and
  therefore there exists $\monelt[n] \in L_n$ such that $\monelt[n] \gdivleq
  \cm(r)$. This provides us with a polynomial $t \in H_n$ such that $\cm(t)$
  and an element $\gelem \in \group$ such that $\cm(t) \divleq \gelem \cdot
  \cm(r)$. Because $H_n$ is \kl{equivariant}, we can assume that $\gelem$ is
  the identity. In particular, one concludes that there exists $\monelt[n] \in
  \mon{\X}$ such that $\lm(t) \times \monelt[n] = \lm(r)$, and therefore
  $\lm(t) \gdivleq \cm(r)$. Therefore, one can find some $\alpha \in \K$ such
  that the polynomial $r' \defined r - \alpha \monelt[n] t$ satisfies $r'
  \pmonlt r$, and in particular, $r \toeucl{H_n} r'$.
  This contradicts the fact that $r$ is \kl{normalised} with respect to $H_n$.
\end{proof}

\begin{lemma}
  \label{lem:weakgb-correctness}
  Assume that $\Basis$ is the output of \cref{alg:weakgb}. Then, it 
  is a \kl{weak equivariant Gröbner basis} of the ideal
  $\EqIdlGen{H}$.
\end{lemma}
\begin{proof}
  It is clear that $\Basis$ is a generating set of $\EqIdlGen{H}$, because
  one only add polynomials that are in the ideal generated by $H$ at every step.

  Let $p \in \EqIdlGen{H}$ be a polynomial,
  and let $\mathfrak{d}$ be a \kl{decomposition} of $p$ with respect to
  $\Basis$, that is, a \kl{decomposition} of the form
  \begin{equation}
    p = \sum_{i \in I} \alpha_i \monelt_i p_i
    \quad .
  \end{equation}
  Where $\alpha_i \in \K$, $p_i \in \Basis$, and $\monelt_i \in \mon{\X}$,
  for all $i \in I$.

  Leveriging \cref{lem:chm}, we know that the ordering $\pmonleq$ is
  \kl{well-founded}. As a consequence, we can consider a minimal 
  decomposition $\mathfrak{d}'$ of $p$ with respect to $\Basis$ such that $\mathfrak{d}'
  \pmonleq \mathfrak{d}$. We now distinguish two cases, depending on whether
  the leading monomial $\lmdec(\mathfrak{d}')$ of the decomposition $\mathfrak{d}'$ is
  equal to the leading monomial of $p$ or not.

  \begin{description}
    \item[Case 1:] $\lmdec(\mathfrak{d}') = \lm(p)$.
      In this case, we conclude immediately,
      as we also have by assumption $\dom(\mathfrak{d}') \subseteq \dom(\mathfrak{d})$.

    \item[Case 2:] $\lmdec(\mathfrak{d}') \neq \lm(p)$.
      In this case, it must be that the set $J$ the set of indices such that
      $\lm(\monelt_i p_i) = \lmdec(\mathfrak{d}')$.
      Let us remark that 
      the sum of leading coefficients 
      of the polynomials in $J$ must vanish: $\sum_{i \in J} \alpha_i \lc(p_i) = 0$.
      As a concequence, the set $J$ has size at least $2$.
      Let us distinguish one element $\star \in J$, and 
      write $J_\star = J \setminus \set{\star}$.
      We conclude that 
      $\alpha_\star = - \sum_{i \in J_\star} \alpha_i \lc(p_i) / \lc(p_\star)$.
      Let us now rewrite $p$ as follows:
      \begin{equation}
        p = \sum_{i \in J_\star} \alpha_i 
        \left(\monelt_i p_i - \frac{\lc(p_i)}{\lc(p_\star)} \monelt_\star p_\star\right)
        + \sum_{i \in I \setminus J} \alpha_i \monelt_i p_i
        \quad .
      \end{equation}
      Now, by definition,
      polynomials $\alpha_i \monelt_i p_i$ for $i \in J \setminus J$ have 
      leading monomials
      strictly smaller than $\monelt[l]$.
      Furthermore,
      the polynomials
      $\monelt_i p_i - \frac{\lc(p_i)}{\lc(p_\star)} \monelt_\star p_\star$ for $i \in J_\star$
      cancel their leading monomials, hence they belong
      to the set $\CancelPoly{p_i}{p_\star}$.
      By \cref{lem:spoly}, we know that these polynomials are obtained by
      multiplying the \kl{S-polynomial} $\spoly{p_i}{p_\star}$ by some monomial.
      Because \cref{alg:weakgb} terminated, we know that 
      $\spoly{p_i}{p_\star} \toeucl{\Basis}^* 0$ by contruction.

      By definition of the rewriting relation, we conclude that one can rewrite
      $\spoly{p_i}{p_\star}$ as combination of polynomials in $\Basis$ that
      have smaller or equal leading monomials, and do not introduce new
      indeterminates.

      We conclude that
      the whole sum is composed of polynomials with leading monomials 
      strictly smaller than $\monelt[l]$, and using a subset of the indeterminates
      used in $\mathfrak{d}'$, leading to a contradiction
      because of the minimality of the latter. 
  \end{description}
\end{proof}

As a consequence of the above lemmas, we can now conclude that the 
\cref{alg:weakgb} computes a \kl{weak equivariant Gröbner basis} of the
ideal $\EqIdlGen{H}$, as stated in \cref{thm:weakgb-comput}.

\begin{theorem}
  \label{thm:weakgb-comput}
  Assume that $(\mon[\omega]{\X}, \gdivleq)$ is a \kl{WQO}, and that the order
  $\varleq$ is effectively computable, and that the action of $\group$ is
  \kl{effectively oligomorphic}. 
  Then, the algorithm $\mathsf{weakgb}$ that takes as input a finite set $H$ of generators of an
  \kl{equivariant ideal} $\idl$ and computes a \kl{weak equivariant Gröbner
  basis} $\Basis$ of $\idl$.
\end{theorem}
