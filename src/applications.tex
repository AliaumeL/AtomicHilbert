% LTeX: language=en
%!TEX root = ../atomic.sigconf.tex
%
\subsection{Closure properties}
\label{sec:closure-properties}

\AP In this section, we are interested in listing the operations on sets of
indeterminates equipped with a group action that preserve our \kl{computability
assumptions} and the \kl{well-quasi-ordering} property ensuring that our
\cref{thm:compute-egb} can be applied.  Indeed, it is often tedious to prove
that a given group action $\grp[G] \actson \X$ satisfies the \kl{computability
assumptions} and the \kl{well-quasi-ordering} property, and we aim to provide a
list of operations that preserve these properties, so that simpler examples
(\cref{ex:dlo,ex:eq atoms,ex:dense tree}) can serve as building blocks to model
complex systems. 

\paragraph{Structural operations.} Let us first focus on three standard
operations on sets of indeterminates: the \kl{disjoint sum} (that was already
at play in \cref{sec:equivariant-grobner-basis}), the \kl{direct product} (that
will fail to preserve our assumptions), and its variant, the \kl{lexicographic
product}. For the remainder of this section, we fix a pair of group actions
$\grp[H] \actson \X$ and $\grp[G] \actson \Y$, where $\X$ is equipped with a
total order $<_{\X}$ and $\Y$ is equipped with a total order $<_{\Y}$. 

\AP The \intro{disjoint sum} $\X \ordplus \Y$ is the disjoint union of $\X$ and
$\Y$, equipped with the total order obtained by stating that all elements of
$\X$ are smaller than all elements of $\Y$, and preserving the original
orderings inside $\X$ and $\Y$. The group $\grp[G] \times \grp[H]$ acts on $\X
\ordplus \Y$ by acting as $\grp[H]$ on $\X$ and as $\grp[G]$ on $\Y$.

\begin{lemma}
  \label{lem:disjoint-sum}
  If $\grp[G] \actson \X$ and $\grp[H] \actson \Y$ are
  \kl{well-structured} (resp. \kl{effectively oligomorphic}),
  then so is $\grp[G] \times \grp[H] \actson \X \ordplus \Y$.
\end{lemma}
\begin{proof}
  The \kl{divisibility up to $\grp[G] \times \grp[H]$} order is essentially the
  disjoint sum of the orders $\gdivleq[\grp[G]]$ and
  $\gdivleq[\grp[H]]$, hence is a \kl{WQO} if both orders are \kl{WQOs}
  \cite[Lemma 1.5]{SCSC17}. Furthermore, it is folklore that 
  the disjoint sum of two \kl{oligomorphic} actions is itself \kl{oligomorphic}.

  Let us now check that the action is \kl{effectively oligomorphic} when both
  actions are. It is an easy check that the action defined is
  \kl(ord){compatible} with the total ordering on the set of indeterminates.
  To list representatives of the orbits in $(\X \ordplus \Y)^n$ for a fixed $n
  \in \N$, we can list representatives $u_\X$ of the orbits in $\X^{\leq n}$,
  representatives $u_\Y$  of the orbits in $\Y^{\leq n}$, and words $u_\text{tag}
  \in \set{0,1}^n$, and consider triples $(u_\X, u_\Y, u_\text{tag})$ such that
  $|u_\X| + |u_\Y| = n$, $|u_\text{tag}|_0 = |u_\X|$, and $|u_\text{tag}|_1 =
  |u_\Y|$. It is an easy check that one can effectively decide whether two such
  triples are in the same orbit.
\end{proof}

\AP
The \intro{direct product} $\X \ordtimes \Y$ is the Cartesian product
$\X \times \Y$, equipped with the lexicographic ordering defined as
\[
  (x_1,y_1) <_{\X \ordtimes \Y} (x_2,y_2) \text{ if } x_1 <_{\X} x_2
  \text{ or } (x_1 = x_2 \text{ and } y_1 <_{\Y} y_2) \ .
\]
The group $\grp[G] \times \grp[H]$ acts on $\X \ordtimes \Y$ by acting as
$\grp[H]$ on the first component and as $\grp[G]$ on the second component.

\begin{lemma}
  \label{lem:direct-product}

  When $\X$ and $\Y$ are infinite,
  $(\mon[Q]{\X \ordtimes \Y}, \gdivleq[\grp[G] \times \grp[H]])$
  is not a \kl{WQO}, even with $Q = \set{0,1}$.
\end{lemma}
\begin{proof}
We restate the antichain given in \cite[Example 10]{GHOLAS24},
that will also be used is \cref{rem:multiple-orbits} of \cref{sec:undecidability}
when discussing the undecidability of the \kl{equivariant ideal membership problem}.
Let $\{x_1,x_2,\dots\}$ and $\{y_1,y_2,\dots\}$ be infinite subsets of $\X$ and $\Y$ respectively.
For $n = 3,4,\dots$, let $\monelt[c]_n$ be the monomial
\[
\monelt[c]_n = (x_1,y_1)(x_1,y_2)(x_2,y_2)(x_2,y_3)\cdots(x_n,y_n)(x_n,y_1) \ .
\]
Then $\setof{\monelt[c]_n}{n = 3,4,\dots}$ is an infinite antichain.
\end{proof}

\AP
The failure to consider \kl{direct products} is somewhat unfortunate,
and motivates the introduction of the \intro{lexicographic product} $\X
\otimes \Y$,
whose underlying set is also $\X \times \Y$,
with the same lexicographic ordering as the \kl{direct product},
but where the group $\grp[G] \lexGroupAction \grp[H]$
is defined as pairs $(\gelem,(\gelem[\sigma]^{x})_{x\in\X})$,
where $\gelem \in \grp[G]$ and $\gelem[\sigma]^{x} \in \grp[H]$ for every $x\in\X$,
and where the multiplication
is defined as
\begin{equation}
  (\gelem_1,(\gelem[\sigma]_1^{x})_{x\in\X})
  (\gelem_2,(\gelem[\sigma]_2^{x})_{x\in\X})
= (\gelem_1\gelem_2, (\gelem[\sigma]_1^{\gelem_2(x)}\gelem[\sigma]_2^x)_{x\in\X})
\quad .
\end{equation}
This group is sometimes called the \emph{wreath product} or the
 \emph{semidirect product}
of $\grp[G]$ and $\grp[H]$. It acts on $\X \otimes \Y$ as
\begin{equation}
  (\gelem,(\gelem[\sigma]^{x})_{x\in\X})\cdot(x',y') = 
  (\gelem\cdot x', \gelem[\sigma]^{x'}\cdot y') \quad ,
\end{equation}
for every $(x',y')\in\X \otimes \Y$.
Essentially, it means that every element $x\in\X$ carries its own copy $\{x\}\times\Y$ of the structure $\Y$,
and one can act independently on different copies of the structure $\Y$.

\begin{lemma}[{\cite[Lemmas 9 and 39]{GHOLAS24}}]
  \label{lem:lexicographic-product}
  If $\grp[G] \actson \X$ and $\grp[H] \actson \Y$ are
  \kl{well-structured} (resp. \kl{effectively oligomorphic}),
  then so is $(\grp[G] \otimes \grp[H]) \actson (\X \times \Y)$.
\end{lemma}

\begin{corollary}[name={}, restate={cor:closure-properties}]
  \label{cor:closure-properties}
  The class of group actions satisfying our \kl{computability assumptions} and
  \kl{well-quasi-ordering} property is closed under
  \kl{disjoint sums} and \kl{lexicographic products},
  but not under \kl{direct products}.
\end{corollary}


\paragraph{Reducts and nicely orderable actions.} \AP Another important operation
on group actions is the notion of \kl{reduct}, which allows one to encode
actions that do not preserve a linear order into actions that do.
We say that $\group\actson\Indets$ is a \intro{reduct} of
another group action $\calH\actson\Y$ if there exists a bijection 
$f \colon \Indets\to\Y$ such that, for every 
$\gelem[\theta] \in \calH$, we have some $\gelem \in \group$ such that
$f^{-1} \circ \gelem[\theta] \circ f$ acts like $\gelem$ on $\Indets$.
This is called an \intro{effective reduct} if $f$ is computable.

\begin{theorem}[name={}, restate={thm:reducts-computable}]
  \label{thm:reducts-computable}
  Let $\calH\actson\Y$ be an action satisfying the requirements of 
  \cref{cor:equivariant-ideals-computations}, and let
  $\group\actson\Indets$ be an \kl{effective reduct} of $\calH\actson\Y$.
  Then one has an \emph{effective representation} of
  the \kl{equivariant ideals} of $\poly{\K}{\Indets}$
  satisfying the properties of \cref{cor:equivariant-ideals-computations}.
\end{theorem}


\Cref{thm:reducts-computable} implies that one can apply our results to an
action $\group\actson\Indets$ that does not preserve a linear order, as soon as
it is  a \kl{reduct} of some another action $\calH\actson\Indets$ which does
preserves a linear order. For example,
$\aut{\EqualityAtoms}\actson\EqualityAtoms$ is a \kl{reduct} of
$\aut{\OrderAtoms}\actson\OrderAtoms$ assuming $\EqualityAtoms$ is countable.
Similarly, let $\T_<$ be the countable dense-meet tree with a lexicographic
ordering, as defined in \cite[Remark 6.14]{KRS21}.\footnote{The remark says
  that finite meet-trees expanded with a lexicographic ordering is a
  Fra\"{i}sse class, from which it follows that there exists a Fra\"{i}sse
  limit $\T_<$ for that class.} Let $\group$ be the group of bijections of
  $\T_<$ which do not necessarily preserve the lexicographic ordering. Then
  $\group\actson\T_<$ is isomorphic to $\aut{\TreeAtoms}\actson\TreeAtoms$, and
  hence $\aut{\TreeAtoms}\actson\TreeAtoms$ is a \kl{reduct} of
  $\aut{\T_<}\actson\T_<$.

\AP We say that an action $\group\actson\Indets$ is \intro{nicely orderable} if
there exists another action $\calH\actson\Y$ such that $\group\actson\Indets$
is a \kl{reduct} of $\calH\actson\Y$, $\calH\actson\Y$ preserves a linear order
on $\Y$, and $\calH \actson \Y$ satisfies our \kl{computability assumptions}. In
the case of actions originating from \kl{homogeneous} structures, it is
conjectured that being \kl{well-structured} is equivalent to being \kl{nicely
orderable}~\cite[Problems 12]{POUZ24}.

\subsection{Applications}
\label{sec:applications}

\paragraph{Polynomial computations.} \AP The fact that (finite control) systems
performing polynomial computations can be verified follows from the theory of
\kl{Gröbner bases} on finitely many indeterminates \cite{MULSEI02,BEDUSHWO17}.
There were also numerous applications to automata theory, such as deciding
whether a weighted automaton could be determinised (resp. desambiguated)
\cite{BESM23,PUSM24}. We refer the readers to a nice survey recapitulating the
successes of the ``Hilbert method'' automata theory \cite{BOJAN19}. A natural
consequence of the effective computations of \kl{equivariant Gröbner bases} is
that one can apply the same decision techniques to \emph{orbit finite
polynomial computations}. For simplicity and clarity, we will focus on
\kl{polynomial automata} without states or zero-tests \cite{BEDUSHWO17}, but
the same reasoning would apply to more general systems.


\AP Before discussing the case of orbit finite polynomial automata, let us
recall the setting of \kl{polynomial automata} in the classical case, as
studied by \cite{BEDUSHWO17}, with techniques that dates back to
\cite{MULSEI02}. A \intro{polynomial automaton} is a tuple $A \defined (Q,
\Sigma, \delta, q_0, F)$, where $Q = \K^n$ for some finite $n \in \N$, $\Sigma$
is a finite alphabet, $\delta \colon Q \times \Sigma \to Q$ is a transition
function such that $\delta(\cdot,a)_i$ is a polynomial in the indeterminates
$q_1, \dots, q_n$ for every $a \in \Sigma$ and every $i \in \set{1, \dots, n}$,
$q_0 \in Q$ is the initial state, and $F \colon Q \to \K$ is a polynomial
function describing the final result of the automaton. The \intro{zeroness
problem for polynomial automata} is the following decision problem: given a
\kl{polynomial automaton} $A$, is it true that for all words $w \in \Sigma^*$,
the polynomial $F(\delta^*(q_0, w))$ is zero? It is known that the \kl{zeroness
problem for polynomial automata} is decidable \cite{BEDUSHWO17}, using the
theory of \kl{Gröbner bases} on finitely many indeterminates. 


\newcommand{\toequiv}{\stackrel{\text{\tiny{eq}}}{\to}}

\AP Let us now propose a new model of computation called \kl{orbit finite
polynomial automata}, and prove an analogue decidability result. Let us fix an
\kl{effectively oligomorphic} action $\group \actson \Indets$, such that there
exists finitely many indeterminates $V \subfin \Indets$ such that $\group$ acts
as the identity on $V$. Given such a function $f \colon \Indets \to \K$, and
given a polynomial $p \in \poly{\K}{\Indets}$, we write $p(f)$ for the
evaluation of $p$ on $f$, that belongs to $\K$. Let us emphasis that the model
is purposely designed to be simple and illustrate the usage of \kl{equivariant
Gröbner bases}, and not meant to be a fully-fledged model of computation.

\begin{definition}
  \label{def:orbit-finite-polynomial-automaton}
  An \intro{orbit finite polynomial
  automaton} over $\K$ and $\Indets$ 
  is a tuple $A \defined (Q, \delta, q_0, F)$, where $Q =
  \Indets \to \K$, $q_0 \in Q$ is a function that is non-zero for finitely
  many indeterminates, $\delta \colon
  \Indets \times \Indets \toequiv \poly{\K}{\Indets}$ 
  is a
  polynomial update function, and $F \in \poly{\K}{V}$ is a polynomial 
  computing the result of the automaton. 

  Given a letter $a \in \Indets$ and a
  state $q \in Q$, the updated state $\delta^*(a,q)  \in Q$ is defined as the function from
  $\Indets$ to $\K$ defined by $\delta^*(a,q) \colon x \mapsto \delta(a,x)( q )$.
  The update function is naturally extended to words. Finally, the
  output of an \kl{orbit finite polynomial automaton} on a word $w \in \Indets^*$
  is defined as $F(\delta^*(w,q_0))$.
\end{definition}

\AP \kl{Orbit finite polynomial automata} can be used to model programs that
read a string $w \in \Indets^*$ from left to right, having as internal state a
dictionary of type \texttt{dict[indet, number]}, which is updated using
polynomial computations. As for \kl{polynomial automata}, the
\intro(ofpa){zeroness problem} for orbit finite polynomial automata is the
following decision problem: decide if for every input word $w$, the output
$F(\delta^*(w, q_0))$ is zero.



The \kl{orbit finite polynomial automata} model could be extended to allow for
inputs of the form $\Indets^k$ for some $k \in \N$, or even be recast in the
theory of nominal sets \cite{BOJAN16inf}. Furthermore, leveraging the closure
properties of \cref{cor:closure-properties},
one can also reduce the equivalence problem for \kl{orbit finite polynomial
automata} to the \kl(ofpa){zeroness problem}, by considering the sum action 
on the registers to compute the difference of the two results. We leave
a more detailed investigation of the generalisation of \kl{polynomial automata}
to the orbit finite setting for future work.

\begin{theorem}[name={orbit finite polynomial automata},
  restate=thm:orbit-finite-polynomial-automata-zeroness]
  \label{cor:orbit-finite-polynomial-automata-zeroness}
  Let $\Indets$ be a set of indeterminates that satisfies the
  \kl{computability assumptions} and such that $(\mon[Y]{\Indets}, \gdivleq)$ is a
  \kl{well-quasi-ordering}, for every \kl{well-quasi-ordered} set $(Y, \leq)$.
  Then, the \kl(ofpa){zeroness problem} is decidable for
  \kl{orbit finite polynomial automata} over $\K$ and $\Indets$.
\end{theorem}


%\begin{remark}
%  \label{rem:topological-wsts}
%  The proof of \cref{cor:orbit-finite-polynomial-automata-zeroness} can be
%  recast in the more general setting of 
%  \intro{topological well-structured transition system}, that were introduced by
%  Goubault-Larrecq in \cite{JGL07}, who noticed that the pre-existing notion of
%  \emph{Noetherian space} could serve as a topological generalisation of
%  \kl{Noetherian rings} (where ideal-based method can be applied),
%  and 
%  \kl{well-quasi-orderings}, for which the celebrated decision procedures on
%  \emph{well-structured transition systems} can be applied \cite{ABDU96}. In particular,
%  Goubault-Larrecq used such systems to verify properties of \emph{polynomial
%  programs} computing over the complex numbers, that can communicate over lossy
%  channels using a finite alphabet \cite{JGL10}. 
%  Because of \cref{cor:equivariant-ideals-computations}, we do have an 
%  effective way to compute on the topological spaces at hand, 
%  and therefore we can apply the theory of
%  \kl{topological well-structured transition systems} to verify systems
%  such as \emph{orbit finite polynomial automata communicating using a finite alphabet
%  over lossy channels}.
%  We refer to \cite[Chapter 9]{JGL13} for a survey on the theory of 
%  Noetherian spaces.
%\end{remark}

\paragraph{Reachability problem of symmetric data Petri nets.} The classical
model of Petri nets was extended to account for arbitrary data attached to
tokens to form what is called data Petri nets. We will not discuss the precise
definitions of these models, but point out  that a reversible data Petri net is
exactly what is called a \kl{monomial rewriting system} \cite[Section
8]{GHOLAS24}. Because reachability in such rewriting systems can be decided
using \kl{equivariant ideal membership} queries \cite[Theorem 64]{GHOLAS24}, we
can use \cref{thm:reducts-computable} to show
\cref{cor:rev data VAS}.
Note that \kl{monomial rewrite systems} will be at the
center of our undecidability results in \cref{sec:undecidability}.

\begin{theorem}[name={Reachability in Reversible Data Petri Nets},
  restate=cor:rev-data-VAS]
  \label{cor:rev data VAS}
  For every \kl{nicely orderable} group action $\group\actson\Indets$,
  the \intro(rpnd){reachability problem} for \intro{reversible Petri nets with data} in $\Indets$
  is decidable.
\end{theorem}



\paragraph{Orbit-finite systems of equations} \AP The classical theory of
solving finite systems of linear equations has been generalised to the infinite
setting by \cite{GHL22}, \cite[Section 9]{GHOLAS24}. In this setting, one
considers an \kl{effectively oligomorphic} group action $\group\actson\Indets$,
and the vector space $\lin(\Indets^n)$ generated by the indeterminates
$\Indets^n$ over $\K$. An \intro{orbit-finite linear system of equations} asks
whether a given vector $u \in \lin(\Indets^n)$ is in the vector space generated
by an \kl{orbit-finite} set of vectors $V$ in $\lin(\Indets^n)$ \cite[Section
9]{GHOLAS24}. It has been shown that the \intro(oflns){solvability} of these
systems of equations reduces to the \kl{equivariant ideal membership problem}
\cite[Theorem 68]{GHOLAS24}, and as a consequence of this reduction and
\Cref{thm:reducts-computable} we obtain the following theorem.


\begin{theorem}[name={Solvability of Orbit-Finite Systems of Equations},
  restate=cor:lin-solv]
  \label{cor:lin solv}
  For every \kl{nicely orderable} group action $\group\actson\Indets$,
  the solvability problem for orbit-finite systems of equations
  is decidable.
\end{theorem}

Note that the above corollary is an extension of
\cite[Theorem 6.1]{GHL22} to all nicely orderable group actions.



