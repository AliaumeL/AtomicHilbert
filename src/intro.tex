\section{Introduction}
\label{sec:intro}

Hilbert’s bases theorem and Gröbner basis computations
are central tools. But people want to use infinite sets
of indeterminates.


\cite{GHOLAS24} provides a relatively tight understanding of sets $\X$ of
indeterminates equipped with a \kl{group action} $\group \curvearrowright \X$
that have the Hilbert basis property, based on a property of the
\kl{divisibility order} of \kl{monomials}.

\begin{itemize}
    \item The set of \kl{monomials} is well-quasi-ordered
        by the divisibility up-to $\group$ relation,
        and the \kl{group action} \kl{respects a linear order}.
    \item The Hilbert basis property holds.
    \item The set of \kl{monomials} is well-quasi-ordered
        by the divisibility up-to $\group$ relation.
\end{itemize}

However, even in the strong case where the \kl{group action} was supposed to
\kl{respect a linear order}, the decidability of the corresponding ideal
membership problem was left open.

\paragraph{Contributions.}

\begin{theorem}
    \label{thm:decid-equiv-idl}
    Let $(\X, \group)$ be an \kl{effectively oligomorphic} set of indeterminates, such that the 
    action of $\group$ on $\X$ \kl{respects a computable linear order}.
    Then, the following implication holds:
    \begin{enumerate}
        \item \label{item:mono-wqo-om1}
            The set of \kl{$(\omega+1)$-monomials}
            is \kl{well-quasi-ordered} by 
            the \kl{divisibility up-to $\group$} relation.
        \item \label{item:decid-equiv-idl}
            The \kl{equivariant ideal membership problem} is decidable in $\poly{\K}{\X}$.
    \end{enumerate}
\end{theorem}

\begin{conjecture}
    \label{lem:undecid-equiv-idl}
    Let $(\X, \group)$ be any 
    set of indeterminates equipped with a \kl{group action} $\group \curvearrowright \X$. If 
    $\mon{\X}$ is not a \kl{well-quasi-order}
    for the \kl{divisibility up-to $\group$} relation, then the \kl{equivariant ideal membership problem} is undecidable in $\poly{\K}{\X}$.
\end{conjecture}


\paragraph{Organisation of the paper.}
\begin{enumerate}
    \item We introduce an abstract formulation of the problem 
        in terms of indeterminates equipped with a group action, and
        provide several examples and non-examples.
    \item We present the algorithmic assumptions and the main decidability
        result.
    \item We prove that mild changes in the algorithmic assumptions
        lead to undecidability.
    \item We conclude by discussing the relationship with existing
        results: in particular we provide examples of relational
        structures and groups that satisfy the algorithmic assumptions.
\end{enumerate}
