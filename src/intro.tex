\section{Introduction}
\label{sec:intro}

A fundamental result in commutative algebra is Hilbert’s basis theorem, which
states that every ideal in a polynomial ring over a field has a finite
generating set. This can equivalently be formulated as a termination result:
the ring of polynomials over a field is Noetherian, i.e., every ascending chain
of ideals stabilises.
However, this result alone in not sufficient to effectively
manipulate ideals in polynomial rings. For instance, deciding 
whether a given polynomial belongs to a given (finitely presented) ideal
is non-trivial ; in fact, it has been shown to be \EXPTIME-complete
\cite{MAME82}.

In this context, Gröbner bases are there to provide a finite representation of
ideals that can effectively be used to solve membership problems, and more
generally to effectively manipulate ideals \cite{CLO15}. Given a linearisation
of the divisibility relation on monomials, one can turn any finite set $H$ of
polynomials into a rewriting system $\to_H$ that generalises the Euclidean
division of integers: $p \to_H r$ if there exists $q \in H, b \in
\poly{\K}{\X}$ such that $p = bq + r$, and the maximal monomial of $r$ is
strictly smaller than the one of $p$. This rewriting system clearly terminates,
and a Gröbner basis for $H$ is a set of polynomials $\Basis$ such that for
every polynomial $p$ in the ideal generated by $H$, one has $p {\to_\Basis}^*
0$. In particular, it can be shown that deciding whether a polynomial belongs
to an ideal generated by a Gröbner basis is polynomial-time solvable. One of
the most well-known algorithms to compute Gröbner bases is Buchberger's
algorithm \cite{BUCH76}.

A crucial requirement for the validity of this algebraic machinery is that the
set of indeterminates is finite. It is easy to see that the Hilbert basis
property does not hold for polynomial rings over infinite sets of
indeterminates: for instance, the ideal generated by the infinite set of all
indeterminates is not finitely generated.

However, in many cases, one faces infinite sets of indeterminates that behave
similarly to finite sets. As an example, consider the set are $\X =
\setof{x_i}{i \in \N}$, where one is only interested in the fact that
indeterminates are equal or not, i.e., indeterminates are considered up to the
action of the group $\group$ of permutations of $\X$. In this case, the ideal
generated by the set of all indeterminates can be finitely represented ``up to
the action of $\group$'' as the ideal generated by the single monomial $x_1$.
Focusing on ideals that are invariant under the action of $\group$ is a
natural concept that has been studied in the literature under the name of
\emph{equivariant ideals}, for instance in \cite{BRDR11,HIKRLE18,GHOLAS24}.

This follows a more general research program that aims at understanding 
algorithmic properties of infinite structures that have symmetries
(data vass, register automata, orbit finite weighted automata, etc.)

\cite{GHOLAS24} provides a relatively tight understanding of sets $\X$ of
indeterminates equipped with a \kl{group action} $\group \curvearrowright \X$
that have the Hilbert basis property, based on a property of the
\kl{divisibility order} of \kl{monomials}.

\begin{itemize}
    \item The set of \kl{monomials} is well-quasi-ordered
        by the divisibility up-to $\group$ relation,
        and the \kl{group action} \kl{respects a linear order}.
    \item The Hilbert basis property holds.
    \item The set of \kl{monomials} is well-quasi-ordered
        by the divisibility up-to $\group$ relation.
\end{itemize}

\begin{tikzpicture}[
    prop/.style={draw, 
                 rectangle,
                 text width=0.2\textwidth,
                 minimum height=2cm,
                 rounded corners, 
                 fill=D5hint,
                 align=center},
    impl/.style={->, >=stealth, thick},
    ]
    % one node per property of the itemize above
    \node[prop] (prop1) at (0,0) {WQO + $\exists$ linear ordering};
    \node[prop] (prop2) at (4,0) {The Hilbert basis property holds};
    \node[prop] (prop3) at (8,0) {WQO};

    % arrows between the nodes
    \draw[impl] (prop1) -- (prop2);
    \draw[impl] (prop2) -- (prop3);
\end{tikzpicture}

However, the decidability of the corresponding ideal membership problem was
left open. And in general, the computation of a basis. Indeed, the interaction
between the group action and the ideal yields non-trivial results.

\begin{example}
    Let $\X = \setof{ x_i }{ i \in \N}$ and
    let $\group$ be the group of permutations of $\X$.
    The \kl{equivariant ideal}
    generated by the polynomial $p = x_1 + x_2$ is
    the set of all polynomials with constant term $0$.

    Note that 
    $\gen{\orbit[\group]{p}}{} \neq \gen{p}{\group}$,
    and
    $\orbit[\group]{\gen{p}{}} \neq \gen{p}{\group}$.
\end{example}

It was shown that an \kl{equivariant Gröbner basis}
exists, where the definition in carefully adapted 

\paragraph{Contributions.}

\begin{theorem}
    \label{thm:decid-equiv-idl}
    Let $(\X, \group)$ be an \kl{effectively oligomorphic} set of indeterminates, such that the 
    action of $\group$ on $\X$ \kl{respects a computable linear order}.
    Then, the following implication holds:
    \begin{enumerate}
        \item \label{item:mono-wqo-om1}
            The set of \kl{$(\omega+1)$-monomials}
            is \kl{well-quasi-ordered} by 
            the \kl{divisibility up-to $\group$} relation.
        \item \label{item:decid-equiv-idl}
            The \kl{equivariant ideal membership problem} is decidable in $\poly{\K}{\X}$.
    \end{enumerate}
\end{theorem}

\begin{conjecture}
    \label{lem:undecid-equiv-idl}
    Let $(\X, \group)$ be any 
    set of indeterminates equipped with a \kl{group action} $\group \curvearrowright \X$. If 
    $\mon{\X}$ is not a \kl{well-quasi-order}
    for the \kl{divisibility up-to $\group$} relation, then the \kl{equivariant ideal membership problem} is undecidable in $\poly{\K}{\X}$.
\end{conjecture}


\paragraph{Organisation of the paper.}
\begin{enumerate}
    \item We introduce an abstract formulation of the problem 
        in terms of indeterminates equipped with a group action, and
        provide several examples and non-examples.
    \item We present the algorithmic assumptions and the main decidability
        result.
    \item We prove that mild changes in the algorithmic assumptions
        lead to undecidability.
    \item We conclude by discussing the relationship with existing
        results: in particular we provide examples of relational
        structures and groups that satisfy the algorithmic assumptions.
\end{enumerate}
