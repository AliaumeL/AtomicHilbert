%!TEX root = ../atomic.asmart.tex
% LTeX: language=en
\section{Introduction}
\label{sec:intro}

\AP Given a field $\K$ and a set $\X$ of indeterminates, one can consider the
set $\poly{\K}{\X}$ of polynomials over $\K$ in the indeterminates of $\X$.
\footnote{}
In this setting, an \intro{ideal} is a non-empty subset of $\poly{\K}{\X}$ that is
closed under addition and multiplication by elements of $\poly{\K}{\X}$. A
fundamental result in algebra is \intro{Hilbert's basis theorem}, which states
whenever $\X$ is finite, the set of polynomials $\poly{\K}{\X}$ is
\intro{Noetherian} \cite{HILB1890}: there are no infinite increasing sequences
$\idl_0 \subsetneq \idl_1 \subsetneq \cdots$ of \kl{ideals} in $\poly{\K}{\X}$.
More generally, one can extend this result to the case where $\K$ is a ring
that is itself \kl{Noetherian}, but we will focus on the easier case of fields.

\AP There are numerous applications of \kl{Hilbert's basis theorem} in algebra,
geometry and computer science. For the latter, the use of the so-called
``Hilbert method'' \cite{BOJAN19,SCHMUDE2021}, which consists of reducing a
combinatorial problem to an ideal membership problem has found numerous
non-trivial applications: from the decidability of the equivalence for
string-to-string transducer models \cite{HONKALA00,BEDUSHWO17}, to the
equivalence of polynomial differential equations \cite{CLEMENTE24}.

\AP There has been a growing interest in the last few years for computational
models that are manipulating infinite data structures in a finite way, for
instance an automaton reading words on the infinite alphabet $\N$, while
maintaining a finite number of states. While this idea can be traced back to
the 90s with the notion of register automata \cite{KAFR94}, it has been revived
in with the development of the theory of \emph{orbit finite sets}. In this
setting, the decidability of the equivalence for \emph{weighted register
automata} has been shown to be decidable \cite{BOKLMO24,BOKLMO21} by leveraging
an analogue of \kl{Hilbert's basis theorem} for vector spaces (that is, a dimension argument).

\AP We should mention that the \kl{Hilbert's basis theorem} does not hold for
infinite sets of variables. Let us consider in the rest of this introduction
the (infinite) set $\X$ of variables $\{x_1,x_2,\ldots\}$. The sequence of
ideals $\idl_n = \IdlGen{\setof{x_i}{i \leq n}}$ is both increasing and
infinite, where we used the notation $\intro*\IdlGen{H}$ for the \intro{ideal
generated by} the set $H$, that is the smallest (for inclusion) \kl{ideal}
containing $H$. Yet, one can still consider infinite sets of variables if one
considers them up to the \kl{action of a group} $\group$.
To continue with our example, we can consider the group of
all monotone injections of $\N$ into itself, which acts on $\X$ by applying the
function to the variable's indices. For instance, $(+1) \in \group$ and $(+1)
\cdot x_1 = x_2$. The action of $\group$ on $\X$ naturally induces an action on
$\poly{\K}{\X}$, by renaming the variables. For instance, the polynomial $5
x_1^2 + x_2$ is sent to $5 x_2^2 + x_3$ by the action of $(+1)$. When
considering subsets $H$ of $\poly{\K}{\X}$, we will be interested in those that
are \intro{equivariant}, i.e., invariant under the action of $\group$: $\group
\cdot H = H$. For instance, Note that the \intro{orbit} of a set $H$, denoted
$\intro*\orbit{H} \defined \group \cdot H$ is always \kl{equivariant}. In this
setting, we can state an analogue of \kl{Hilbert's basis theorem}: \intro{the
Equivariant Hilbert basis property}, stating that there exists no infinite
increasing sequence of \kl{equivariant} \kl{ideals} in $\poly{\K}{\X}$. Note
that in our example, the \intro{equivariant ideal generated by} $\set{x_1}$,
written $\intro*\EqIdlGen{\set{x_1}}$, already equals $\EqIdlGen{x_1, x_2}$,
since the action of $(+1)$ sends $x_1$ to $x_2$.

\AP The \kl{Equivariant Hilbert basis property} does not hold in general, for
instance if the group $\group$ is the single element group, then
\kl{equivariant} \kl{ideals} are just \kl{ideals}. However, it has been shown
to hold in the case of $\X = \setof{x_i}{i \in \N}$ and $\group$ is the group
of all monotone injections of $\N$ into itself \cite{HIKRLE18}, or in the case
of an infinite set $\X$ and $\group$ is all permutations of $\X$
\cite{BRDR11,HISU12,HIKRLE18}. A generalisation of these results was recently
obtained by \cite{GHOLAS24}, where they almost characterise the pairs
$(\X,\group)$ for which the \kl{Equivariant Hilbert basis property} holds. 

\AP A natural question remains open in this setting: is there a general
decision procedure for the \intro{equivariant ideal membership problem}, that
is, given a finite set $H \in \poly{\K}{\X}$ and a polynomial $p \in
\poly{\K}{\X}$, does $p$ belong to the \kl{equivariant ideal} generated by $H$?
Note that the decidability of the \kl{equivariant ideal membership problem} is
a crucial element in the ``Hilbert method'', and that even in the case of
finite sets of variables, it is already \EXPTIME-complete \cite{MAME82}. In the
finite variables case, a well-known decision procedure is based on the notion
of \kl{Gröbner bases}, which are special generating sets of \kl{ideals} for
which solving the \kl{ideal membership problem} is straightforward. The main
technical part of such procedures is to compute a \kl{Gröbner basis} from a
finite set $H$ of generators, which can be done by using \kl{Buchberger's
algorithm} \cite{BUCH76}, that is closely related to the Knuth-Bendix
completion algorithm \cite{KNBEND70}.

\AP All current decision procedures for the \kl{equivariant ideal membership
problem} are based on the \kl{Gröbner basis} approach, and a suitable
adaptation of \kl{Buchberger's algorithm} to the equivariant case. A crucial
requirement for the success of these techniques is the fact that indeterminates
$\X$ are equipped with a \emph{well-founded} \emph{total} ordering $\leq$ that
is \intro{compatible} with the action of $\group$ on $\X$. This means that if
$x \leq y$, then $\gelem \cdot x \leq \gelem \cdot y$ for all $\gelem \in
\group$. This is a very strong requirement, that is not satisfied in many
cases, for instance when $\X = \setof{x_i}{i \in \Q}$ and $\group$ is the group
of all monotone bijections of $\Q$ into itself, which is a natural example in
many applications \cite{BOKLMO24}.

\paragraph{Contributions.} \AP In this paper, we continue the work of
\cite{GHOLAS24} and show that the \kl{equivariant ideal membership problem} is
decidable under mild assumptions. In order to explain these requirements and
how they relate to previous works, let us first recall that a set $(X, \leq)$ is a
\intro{well-quasi-ordering} (WQO) if every infinite sequence $\seqof{x_i}[i \in
\N]$ of elements of $X$ contains a pair $i < j$ such that $x_i \leq x_j$.
We know from \cite{GHOLAS24} that a
necessary condition for the \kl{equivariant Hilbert basis property} to hold is
that the set  $\mon{\X}$  of monomials is a
\kl{well-quasi-ordering} when endowed with the \intro{divisibility up-to
$\group$} relation ($\intro*\gdivleq$), which is defined as follows: for
$\monelt_1, \monelt_2 \in \mon{\X}$, we write $\monelt_1 \gdivleq
\monelt_2$ if there exists $\gelem \in \group$ such that $\monelt_1$ \kl{divides}
$\gelem \cdot \monelt_2$.
Let us recall that a monomial $\monelt[m]$ \intro{divides} a monomial $\monelt[n]$ if
there exists a monomial $\monelt[k]$ such that $\monelt[m] \times \monelt[k] = \monelt[n]$.


\AP Notice that one can extend the
notion of monomials, that are functions from $\X$ to $\N$ having a finite
support, to the case of $\mon[\alpha]{\X}$, where $\alpha$ is any partially
ordered set. With this convention, $\mon[\omega]{\X} = \mon{\X}$. Note that
\kl{divisibility} extends to $\mon[\alpha]{\X}$ naturally, and that we can therefore
ask whether $(\mon[\alpha]{\X}, \gdivleq)$ is a \kl{WQO}. A well-known
result from Dickson states that in the case of finite sets of variables,
$(\mon[\alpha]{\X}, \gdivleq)$ is a \kl{WQO} if and only if $\alpha$ is
itself a \kl{WQO}, hence our approach will not lose any generality in the finite case \cite{SCSC12}.


\AP In order to devise a decision procedure, we also will assume that basic
operations on the set of indeterminates and on the group action are decidable.
Namely, we will assume that the action of $\group$ on $\X$ is
\intro{effectively oligomorphic}, which means that: for every two tuples
$(x_1,\dots,x_n), (y_1,\dots,y_n)\in \X^n$, one can decide whether they are in
the same $\group$-orbit, i.e., whether there exists $\gelem \in \group$ such
that $\gelem \cdot x_i = y_i$ for all $i \leq n$ ; And that for every finite
subset $S \subfin \X$ and every $n \in \N$, one can compute a finite set of
representatives for the $\FixG{S}$-orbits of $\X^n$, where $\intro*\FixG{S}$ is the
subgroup of $\group$ that fixes the elements of $S$.

\AP We will also assume that there exists a total ordering $\leq$ on $\X$ that
is \kl{compatible} with the action of $\group$, i.e., if $x \leq y$, but
crucially, we do not assume that $\leq$ is well-founded. Again, to devise our algorithm
we will assume that this ordering is \intro(ordering){effective}, i.e. To compute with this
we will assume that one can decide whether $x \leq y$ for all $x,y
\in \X$. The first and main result of this paper is the following:

\begin{theorem}
  \label{thm:decide-equiv-ideal-mem}
  Let $(\X, \group, \leq)$ be a set of indeterminates, a group acting \kl{effectively
  oligomorphically}
  on $\X$ and an
  \kl{effective total ordering} $\leq$ on $\X$ that is compatible with the action of
  $\group$. 
  If $(\mon[\omega + 1]{\X}, \gdivleq)$ is a \kl{WQO}, then one can decide the
  \kl{equivariant ideal membership problem}.
\end{theorem}

\AP
We then build on this result to compute so-called 
\kl{equivariant Gröbner bases} for equivariant ideals.
Note that already in the case of finite sets of variables, there are 
multiple (equivalent) definitions of \kl{Gröbner bases}, and will defer 
their presentation to our preliminaries section.
In the case of infinite sets of variables, we define an
\intro{equivariant Gröbner basis} as a finite set of generators $\Basis$ of
an \kl{equivariant ideal} $\idl$ such that, for every finite set $S \subfin \X$,
the set $\orbit{\Basis} \cap \poly{\K}{S}$ is a \kl{Gröbner basis} of the \kl{ideal} $\idl \cap
\poly{\K}{S}$.
In particular, if $\Basis$ is an \kl{equivariant Gröbner basis}, then
the following holds:
\begin{equation}
  \forall S \subfin \X,
  \EqIdlGen{H} \cap \poly{\K}{S} = \IdlGen{\orbit{\Basis} \cap \poly{\K}{S}}
  \quad .
\end{equation}
In particular, having an \kl{equivariant Gröbner basis} allows one to efficiently compute membership
queries, as they reduce to the case of finite sets of variables, and that
a \kl{Gröbner basis} of the corresponding (non-equivariant) ideal can be computed 
easily from $\Basis$, as a finite set of representatives of
$\orbit{\Basis} \cap \poly{\K}{S}$.


\begin{theorem}
  \label{thm:compute-egb}
  Let $(\X, \group, \leq)$ be a set of indeterminates, a group acting \kl{effectively
  oligomorphically}
  on $\X$ and an
  \kl{effective total ordering} $\leq$ on $\X$ that is compatible with the action of
  $\group$. 
  If $(\mon[\omega + \omega]{\X}, \gdivleq)$ is a \kl{WQO}, then one can
  compute an \kl{equivariant Gröbner basis} from a finite set $H$ of generators.
\end{theorem}

Note that the requirement that $(\mon[\omega + \omega]{\X}, \gdivleq)$ is a
\kl{WQO} is a generalisation of the requirement that $(\mon[\omega+1]{\X},
\gdivleq)$ is a \kl{WQO} in the previous \cref{thm:decide-equiv-ideal-mem}.


\paragraph{Organisation.} \AP
The rest of the paper is organised as follows.
In \cref{sec:preliminaries}, we introduce formally what a
\kl{Gröbner basis} is, and present the background on 
how to compute them using \kl{Buchberger's algorithm}.
Then, we will present in \cref{sec:weakgb} an adaptation of 
\kl{Buchberger's algorithm} to the \kl{equivariant case}, that computes
a \kl{weak equivariant Gröbner basis} of an \kl{equivariant ideal}.
In \cref{sec:algorithm}, we use \kl{weak equivariant Gröbner bases} to devise a decision
procedure for the \kl{equivariant ideal membership problem}, and prove
our \cref{thm:decide-equiv-ideal-mem}.
The procedure will be based on the following idea:
given a finite set $S \subfin \X$ of indeterminates, and a finite set $H$ of generators, we will compute a \kl{weak equivariant Gröbner basis} 
$\Basis_S$ that has the property that 
$\EqIdlGen{H} \cap \poly{\K}{S} = \IdlGen{\Basis_S \cap \poly{\K}{S}}$, for this 
particular set $S$.
We will call such sets \kl{$S$-strong equivariant Gröbner bases} for 
$\EqIdlGen{H}$.
Then, in \cref{sec:equivariant-grobner-basis},
we will show how to transform our algorithm that computes 
\kl{$S$-strong equivariant Gröbner bases} into an algorithm that computes
\kl{equivariant Gröbner bases} for \kl{equivariant ideals}.
In \cref{sec:examples}
we will show how to use our algorithm to solve some non-trivial examples. This will also be a way
to show that our restrictions are very mild, and that one can really apply our theoretical
results.
Finally, in \cref{sec:conclusion}, we will discuss on the need 
for a total ordering on the set of indeterminates, and the possibility
to relax the hypotheses of our results.
%

\arka{Add grants}
%
\section{Slightly different intro}
%
\arka{Alternate title : Computability of Gr\"{o}bner bases of equivariant polynomial ideals}

For a field $\K$ and a non-empty set $\X$ we use $\poly{\K}{\X}$ to denote the ring of polynomials with coefficients from $\K$ and indeterminates/variables from $\X$.
Hilbert's basis theorem says that when $\X$ is finite, every ideal in $\poly{\K}{\X}$ is finitely generated.
\footnote{The theorem is slightly more general \cite[Theorem 4.1]{Lang02}. 
However we will mostly be interested in this version.}
A \kl{Gr\"{o}bner basis} is a specific kind of generating set of a polynomial ideal which allows easy checking of membership of a given polynomial in that ideal.
\kl{Gr\"{o}bner bases} were introduced by Buchberger who showed when $\X$ is finite,
every ideal in $\poly{\K}{\X}$ has a finite \kl{Gr\"{o}bner basis} and that,
for a given a set of polynomials in $\poly{\K}{\X}$,
one can compute a finite \kl{Gr\"{o}bner basis} of the ideal generated by them.
This implies the decidability of ideal membership problem:
given a polynomial $f$ and set of polynomial $H$, decide whether $f$ is in the ideal generated by $H$.
The theory of \kl{Gr\"{o}bner bases} has applications in several areas,
including
integer programming \cite{Sturmfels96},
algebraic proof systems \cite{algProof},
geometric reasoning \cite{Cox2015chGeom},
fixed parameter tractability \cite{ACDM22},
program analysis \cite{SSM04}
and constraint satisfaction problems \cite{Mas21}.
%and automata theory \cite{BOJAN19,SCHMUDE2021,HONKALA00,BEDUSHWO17,CLEMENTE24,MAME82}.
In automata theory it has been used for deciding zero-ness of polynomial automata \cite{BEDUSHWO17},
reachability in symmetric Petri nets \cite{MAME82},
equivalence for string-to-string transducers \cite{HONKALA00}
and equivalence of polynomial differential equations \cite{CLEMENTE24}.
There has been a growing interest in extending results on classical models of computation (such as finite state automata and Petri nets) to models of computation with infinite alphabet (such as register automata and Petri nets with data).
To extend the aforementioned results,
it is necessary to extend the notion of \kl{Gr\"{o}bner basis} to ideals in polynomial rings with \emph{infinitely} many variables.

Here it is important to note that Hilbert's basis theorem does not automatically extend to this setting.
For example, when $\X$ is infinite,
the ideal $\idlZ$ generated by the set $\setof{x}{x\in\X}$ can not be finitely generated.
Which is why we consider ideals that are \kl{equivariant} w.r.t.\ a fixed group $\group$,
i.e.\ invariant under the \emph{variable-wise} action of $\group$ on $\poly{\K}{\X}$.
For example, although the set $\idlZ$ is not finitely generated as an ideal,
it is finitely generated as an equivariant ideal w.r.t.\ the group $\symgr{\X}$ of permutations of $\X$.
Since for some(every) $x\in\X$,
$\idlZ$ is the smallest ideal which contains $\set{x}$ as a subset and is also invariant under $\symgr{\X}$.

However, equivariance doesn't always guarantee finite generation.
For example, $\idlZ$ is not finitely generated as an equivariant ideal w.r.t. the trivial group.
Hence we are interested in the \kl{Hilbert's basis property}:
every ideal in $\poly{\K}{\X}$ that is equivariant w.r.t.\ $\group$,
is also finitely generated.

The authors of \cite{GHOLAS24} have given a necessary and sufficient condition  
on the action of $\group$ on $\X$ for the Hilbert's basis property to hold.
In fact, they have extended the notion of \kl{Gr\"{o}bner bases} to this setting,
and shown that under the aforementioned condition,
every equivariant ideal has a finite \kl{Gr\"{o}bner basis}.
As our main result (\Cref{thm:compute-egb}) we show that the same condition also ensures that given a finite set of polynomials,
we can compute a finite \kl{Gr\"{o}bner basis} of the equivariant ideal generated by them.
We also complement this result by showing that a common reason why this condition fails to hold,
also guarantees undecidability of the equivariant ideal membership problem
\arka{add reference}.

\arka{Organisation:}

