%!TEX root = ../atomic.sigconf.tex
% LTeX: language=en
%
\section{Introduction}
\label{sec:intro}
\textbf{TODO: use well-structured and $\omega$-well-structured throughout the paper,
to not repeat \kl{well-quasi-ordering} all the time.}

\AP For a field $\K$ and a non-empty set $\Indets$ of indeterminates, we use
$\poly{\K}{\Indets}$ to denote the ring of polynomials with coefficients from $\K$
and indeterminates/variables from $\Indets$. A fundamental result in commutative
algebra is \intro{Hilbert's basis theorem}, stating that when $\Indets$ is finite,
every ideal in $\poly{\K}{\Indets}$ is finitely generated~\cite{HILB1890}, where an
\kl{ideal} is a non-empty subset of $\poly{\K}{\Indets}$ that is closed under
addition and multiplication by elements of $\poly{\K}{\Indets}$.
This property follows from 
\kl{Hilbert's basis theorem}, stating that for every ring 
$\A$ that is \kl{Noetherian}, 
the polynomial ring $\poly{\A}{x}$ in one variable over $\A$ is also
\kl{Noetherian}~\cite[Theorem 4.1]{Lang02}.

\AP In this paper, we will 
assume that elements of $\K$ can be effectively represented 
and that basic operations on $\K$ are computable 
($+$, $-$, $\times$, $/$, and equality test).
In this setting, a \Grb\ is a specific kind of generating set of a polynomial ideal
which allows easy checking of membership of a given polynomial in that ideal.
\kl{Gr\"{o}bner bases} were introduced by Buchberger who showed when $\Indets$ is
finite, every ideal in $\poly{\K}{\Indets}$ has a finite \kl{Gr\"{o}bner basis} and
that, for a given a set of polynomials in $\poly{\K}{\Indets}$, one can compute a
finite \kl{Gröbner basis} of the ideal generated by them via the so-called
\intro{Buchberger algorithm}~\cite{BUCH76}. The
existence and computability of \Grbs\ implies the decidability of the
\kl{ideal membership problem}: given a polynomial $f$ and set of polynomial
$H$, decide whether $f$ is in the ideal generated by $H$. 
More generally, \kl{Gr\"{o}bner bases} provide effective representations of ideals,
over which one can decide inclusion, equality, and compute sums or intersections
of ideals~\cite{CLO15}.

%\arka{remove the para below}
%
%The theory of
%\kl{Gr\"{o}bner bases} has applications in very diverse areas of computer
%science, including integer programming \cite{Sturmfels96}, algebraic proof
%systems \cite{algProof}, geometric reasoning \cite{Cox2015chGeom}, fixed
%parameter tractability \cite{ACDM22}, program analysis \cite{SSM04} and
%constraint satisfaction problems \cite{Mas21}.
%In automata theory it has been used for deciding zeroness of polynomial
%automata \cite{BEDUSHWO17}, reachability in symmetric Petri nets \cite{MAME82},
%equivalence for string-to-string transducers \cite{HONKALA00} and equivalence
%of polynomial differential equations \cite{CLEMENTE24}. 

%\AP There has been a growing interest in the last few years for computational
%models that are manipulating infinite data structures in a finite way, for
%instance an automaton reading words on the infinite alphabet $\N$, while
%maintaining a finite number of states. While this idea can be traced back to
%the 90s with the notion of register automata \cite{KAFR94}, it has been revived
%in with the development of the theory of \emph{orbit finite sets}. In this
%setting, one would like to consider an infinite set of variables $\Indets$.
%
%\arka{break in flow}
%As an example, let us consider the set $\Indets$ of variables $x_i$ for $i \in \N$, and
%the \kl{ideal} $\idlZ$ generated by the set $\setof{x_i}{i \in \N}$. It is
%clear that $\idlZ$ is not finitely generated, and we conclude that the
%\kl{Hilbert's basis theorem} (and a fortiori, the \kl{Gr\"{o}bner basis}
%theory) does not extend to the case of infinite sets of indeterminates.

\AP
In addition to their interest in commutative algebra,
these decidability results have important applications in 
other areas of computer science. For instance, the so-called 
``Hilbert Method'' that reduces verifications 
of certain problems on automata and transducers to 
computations on polynomial ideals has been successfully applied
to polynomial automata, and equivalence of string-to-string transducers
of linear growth, and we refer to~\cite{BOJAN19}
for a survey on these applications.

\AP
In this paper, we are interested in 
extending the theory of \kl{Gr\"{o}bner bases} to the case where
the set $\Indets$ of indeterminates is infinite.
As an example, let us consider $\Indets$ to be the 
set of variables $x_i$ for $i \in \N$, and
the \kl{ideal} $\idlZ$ generated by the set $\setof{x}{x \in \Indets}$.
It is clear that $\idlZ$ is not finitely generated. As a consequence,
\kl{Hilbert's basis theorem}, and a fortiori the theory of 
\kl{Gr\"{o}bner bases},
does not extend to the case of infinite sets of indeterminates.

\AP
Thankfully, the infinite set $\Indets$ of
variables (data) often comes with an extra structure,
usually given by relations and functions defined on $\Indets$,
and one is often interested in systems that are invariant
under the action of the group $\group$ of structure preserving
bijections of $\Indets$.
For instance, in the above example,
one may not be interested in the ideal $\idlZ$ generated by the set
$\setof{x}{x \in \Indets}$,
but rather in the \kl{equivariant ideal} generated by
the set $\setof{x}{x \in \Indets}$,
which is the smallest ideal that contains it and is invariant
under the action of $\group$.
In this case, this ideal is finitely generated by any single
indeterminate $x \in \Indets$.
This motivates the study of \kl{equivariant ideals},
that is highly dependent on the specific choice of group action 
$\group \actson \Indets$:
for instance, the ideal $\idlZ$ is not finitely generated as an equivariant
ideal with respect to the trivial group. 
A general analysis of the \kl{equivariant Hilbert basis property}
stating that
``every \kl{equivariant ideal} is orbit finitely generated''
has been recently given in \cite{GHOLAS24},
and this paper aims at providing a computational counterpart.

\textbf{Strict extensions of the results of \cite{GHOLAS24} 
for orbit-finite linear systems of equations
and data Petri nets should be mentioned here or later.}


\subsection{Contributions.}

\arka{Short. Strengthening is mild in the sense it is conjectured(?) to be equivalent}

\arka{add applications}

\AP In this paper, we bridge the gap between the
theoretical understanding of the \intro{equivariant Hilbert basis property}
\cite[Property 4]{GHOLAS24}, and the computational aspects of \kl{equivariant
ideals}, by showing that under mild assumptions on the group action, one can
compute an \kl{equivariant Gröbner basis} of an \kl{equivariant ideal}, hence,
that one can decide the \kl{equivariant ideal membership problem}. 
In order to
compute such sets, we will need to introduce some classical \kl{computability
assumptions} on the group action $\group \actson \Indets$, and on the set of
indeterminates $\Indets$. These will be defined in
\cref{sec:preliminaries}, but informally, we assume
that one can compute representatives of the orbits of elements under the action
of $\group$ (this is called \kl{effective oligomorphism}), and that one has
access to a total ordering on $\Indets$ that is computable, and
\kl(ord){compatible} with the action of $\group$. 

A typical example satisfying these \kl{computability assumptions} is the
set $\Q$ of rationals, equipped with the natural ordering $\leq$, and the
group $\group$ of all order-preserving bijections from $\Q$ to itself.

\AP Let us now focus on the semantic assumption that we will need to make
on the set of indeterminates $\Indets$ and the group $\group$, that will
guarantee the termination of our procedures. We refer to our preliminaries
(\cref{sec:preliminaries}) for a more detailed
discussion on these assumptions, but again informally, we ask that the set of
\kl{monomials} $\mon{\Indets}$ is well-behaved with respect to divisibility up
to the action of $\group$. 
A monomial $\monelt$ can be seen as a function 
from $\Indets$ to $\N$ with finite support, 
and divisibility amounts to the pointwise comparison of these functions.
By allowing to first relabel the variables of a monomial
using the action of $\group$, we obtain a
generalised divisibility relation $\gdivleq$ on $\mon{\Indets}$.
Our semantic assumption is that \emph{generalised monomials},
that is monomials whose variables are labelled by elements of a
\kl{well-quasi-ordered} set $(Y, \leq)$,
or equivalently functions from $\Indets$ to $Y$ with finite support,
which we write as the fact that $(\mon[Y]{\Indets}, \gdivleq)$
is a \kl{well-quasi-ordering} (\kl{WQO}).

\AP For instance, when $\Indets$ is the set $\Q$ of rationals,
an example of a generalised monomial
could be $x_{1/2}^{(2,\bullet)} x_{3/4}^{(1, \circ)}$,
where $Y = \N \times \set{\circ, \bullet}$.
To a monomial $\monelt$, one can associate 
the word obtained by listing the labels of the variables
of $\monelt$ in increasing order. It turns out that 
$\monelt \gdivleq \monelt[n]$ if and only if 
the word associated to $\monelt$ is a subsequence of the word
associated to $\monelt[n]$. Since words over a \kl{well-quasi-ordered}
alphabet are \kl{well-quasi-ordered} under the subsequence relation
\cite{HIG52}, we conclude that 
that $(\mon[Y]{\Indets}, \gdivleq)$ is a \kl{WQO}.

\AP Our main positive result states that under these assumptions,
one can compute an \kl{equivariant Gröbner basis} of an \kl{equivariant ideal}.

\begin{theorem}[name={Equivariant Gröbner Basis},restate=thm:compute-equiv-gb]
  \label{thm:compute-egb}
  Let $\Indets$ be a totally ordered set of indeterminates
  equipped with a group action $\group \actson \Indets$, under our \kl{computability assumptions}.
  If $(\mon[Y]{\Indets}, \gdivleq)$ is a \kl{WQO} for every 
  \kl{well-quasi-ordered} set $(Y,\leq)$, then one can
  compute an \kl{equivariant Gröbner bases} of \kl{equivariant ideals}.
\end{theorem}

\AP We then focus on providing undecidability results for the \kl{equivariant
ideal membership problem} in the case where our effective assumptions are
satisfied, but the \kl{well-quasi-ordering} condition is not. This aims at
illustrating the fact that our assumptions are close to optimal. One classical
way for a set of structures to not be \kl{well-quasi-ordered} (when labelled
using integers) is to have the ability to represent an \emph{infinite path} (a
formal definition will be given in
\cref{sec:undecidability}). We prove that
whenever one can (effectively) represent an infinite path in the set of
\kl{monomials} $\mon{\Indets}$, then the \kl{equivariant ideal membership
problem} is undecidable.

\begin{theorem}[name={Undecidability of Equivariant Ideal Membership},restate=thm:undecidable-paths]
  \label{thm:undecidable-paths}
  Let $\Indets$ be a totally ordered set of indeterminates
  equipped with a group action $\group \actson \Indets$, under our \kl{computability assumptions}.
  If $\Indets$ contains an \kl(of){infinite path}
  then the \kl{equivariant ideal membership problem} is undecidable.
\end{theorem}

\AP
Finally, we illustrate how our positive results find applications in numerous
situations. This is done by providing families of indeterminates
 that satisfy our
\kl{computability assumptions}, and for which we can compute \kl{equivariant
Gröbner bases}, and also by showing how our results can be used in the context
of \kl{topological well-structured transition systems} \cite{JGL10}, with
applications to the verification of infinite state systems such as orbit
finite weighted automata \cite{BOKLMO21}, \kl{orbit finite polynomial
automata}, and more generally orbit finite systems dealing with polynomial
computations.

\begin{corollary}[name={}, restate={cor:closure-properties}]
  \label{cor:closure-properties}
  The class of group actions satisfying our \kl{computability assumptions} and
  \kl{well-quasi-ordering} property is closed under
  \kl{disjoint sums} and \kl{lexicographic products},
  but not under \kl{direct products}.
\end{corollary}

\begin{theorem}[name={}, restate={thm:reducts-computable}]
  \label{thm:reducts-computable}
  Let $\calH\actson\Y$ be an action satisfying the requirements of 
  \cref{cor:equivariant-ideals-computations}, and let
  $\group\actson\Indets$ be an \kl{effective reduct} of $\calH\actson\Y$.
  Then one has an \emph{effective representation} of
  the \kl{equivariant ideals} of $\poly{\K}{\Indets}$
  satisfying the properties of \cref{cor:equivariant-ideals-computations}.
\end{theorem}

\begin{theorem}[name={Orbit Finite Polynomial Automata},
  restate=thm:orbit-finite-polynomial-automata-zeroness]
  \label{cor:orbit-finite-polynomial-automata-zeroness}
  Let $\Indets$ be a set of indeterminates that satisfies the
  \kl{computability assumptions} and such that $(\mon[Y]{\Indets}, \gdivleq)$ is a
  \kl{well-quasi-ordering}, for every \kl{well-quasi-ordered} set $(Y, \leq)$.
  Then, the \kl(ofpa){zeroness problem} is decidable for
  \kl{orbit finite polynomial automata} over $\K$ and $\Indets$.
\end{theorem}


\begin{corollary}[name={Reachability in Reversible Data Petri Nets},
  restate=cor:rev-data-VAS]
  \label{cor:rev data VAS}
  For every \kl{nicely orderable} group action $\group\actson\Indets$,
  the reachability problem for reversible Petri nets with data in $\Indets$
  is decidable.
\end{corollary}

\begin{corollary}[name={Solvability of Orbit-Finite Systems of Equations},
  restate=cor:lin-solv]
  \label{cor:lin solv}
  For every \kl{nicely orderable} group action $\group\actson\Indets$,
  the solvability problem for orbit-finite systems of equations
  is decidable.
\end{corollary}

\subsection{Related Research}
\arka{needs rewrite}


\textbf{todo: talk about the reduction game}
of \cite[Section 7]{GHOLAS24} that handles the dense linear order.

Say that before: total ordering that is well-founded, OR some games. New
examples: dense meet trees. For decidability, the case of weighted automata
followed from hilbert's basis property, but for polynomial ones, we need
equivariant grobner bases. Rewrite the condition of \cite{GHOLAS24} 
to make the three conditions apparent.

\textbf{todo: what to do about this paragraph}
It is known that this
is a necessary condition for the \kl{equivariant Hilbert basis property}
\cref{thm:equiv-hilbert-property}, and we will rely on a slightly stronger
condition, namely that $(\mon[Y]{\Indets}, \gdivleq)$ is a \kl{WQO}, whenever
$(Y, \leq)$ is one, which is conjectured to be equivalent to the first
condition. Beware that \cref{thm:equiv-hilbert-property,thm:compute-egb}
are
incomparable: the former does not talk about decidability, while the latter 
only considers \kl{equivariant ideals} that are already finitely presented, and we 
will show in
\cref{ex:non-wqo-undecidable} an example where \kl{equivariant
Gröbner bases} are computable, but the \kl{equivariant Hilbert basis property} fails.



\begin{theorem}[name={\cite[Theorem 11 and 12]{GHOLAS24}}]
  \label{thm:equiv-hilbert-property}
  Let $\Indets$ be a totally ordered set of indeterminates
  equipped with a group action $\group \actson \Indets$ that is 
  \kl(ord){compatible} with the ordering on $\Indets$.
  Then, $(\mon{\Indets}, \gdivleq)$ is a \kl{WQO}, if and only if 
  the \kl{equivariant Hilbert basis property} holds for $\poly{\K}{\Indets}$.
\end{theorem}

\textbf{which above mentioned results?}
The above-mentioned results were rediscovered in \cite{AH07,AH08,HKL18}. In
\cite{HS12} these results were used to prove the Independent Set Conjecture in
algebraic statistics. The necessary and
sufficient conditions are equivalent up to a well-known conjecture by Pouzet
\cite[Problems 12]{POUZ24}. But to obtain decision procedures, one still lacks
a generalisation of \kl{Buchberger's algorithm} to the equivariant case, except
under artificial extra assumptions \cite[Section 6]{GHOLAS24}. Overall, a
general understanding of the decidability of the \kl{equivariant ideal
membership problem} is still missing, and \emph{a fortiori}, a generalisation
of \kl{Buchberger's algorithm} to the equivariant case is still an open
problem.

Our results are part of a larger research direction that aims at establishing
an algorithmic theory of computation with orbit-finite sets. For instance,
\cite{BFKM24} studies equivariant subspaces of vector spaces generated by
orbit-finite sets, \cite{GHL22,GHL25} study solvability of orbit-finite systems
of linear equations and inequalities, and \cite{BFKM24,GHL22,Prz23} study duals
of vector spaces generated by orbit-finite sets.


\paragraph{Organisation.} \AP The rest of the paper is organised as follows. In
\cref{sec:preliminaries}, we introduce formally the notions of \kl{Gröbner
bases}, \kl{effectively oligomorphic} actions, and \kl{well-quasi-orderings},
which are the main assumptions of our positive results. Then, we illustrate in
\cref{sec:act ex} how these assumptions can be satisfied in practice, providing
numerous examples of sets of indeterminates. After that, we introduce in
\cref{sec:weakgb} an adaptation of \kl{Buchberger's algorithm} to the
equivariant case, that computes a \kl{weak equivariant Gröbner basis} of an
\kl{equivariant ideal}. In \cref{sec:equivariant-grobner-basis}, we use
\kl{weak equivariant Gröbner bases} to prove our main positive
\cref{thm:compute-egb}, and we show that it provides a way to effectively
represent \kl{equivariant ideals} (\cref{cor:equivariant-ideals-computations}).
We continue by showing in \cref{sec:closure-properties} that the assumptions of our
\cref{thm:compute-egb} are closed under two natural operations
(\cref{lem:closure-properties-comp,lem:closure-properties-wqo}). The positive
results regarding the \kl{equivariant ideal membership problem} are then
leveraged to obtain several decision procedures for other problems in
\cref{sec:applications}. Finally, in \cref{sec:undecidability}, we show that
our assumptions are close to optimal by proving that the \kl{equivariant ideal
membership problem} is undecidable whenever one can find \kl(of){infinite
paths} in the set of indeterminates (\cref{thm:undecidable-paths}), which is
conjectured to be a complete characterisation of the undecidability of the
\kl{equivariant ideal membership problem} (\cref{rem:conj-wqo-infinite-path}).




\AP To prove our \cref{thm:compute-egb}, we will first introduce a weaker
notion of \kl{weak equivariant Gröbner basis}, which characterises the results
obtained by naïvely adapting \kl{Buchberger's algorithm} to the equivariant
case. Then, we will show that under our \kl{computability assumptions}, one can
start from a finite set of generators $H$ of an \kl{equivariant ideal}, and
compute a well-chosen \kl{weak equivariant Gröbner basis}, which happens to be
an \kl{equivariant Gröbner basis} of the ideal generated by $H$. As a
consequence, we obtain effective representations of \kl{equivariant ideals},
over which one can check membership, inclusion, and compute the sum and
intersection of \kl{equivariant ideals}
(\cref{cor:equivariant-ideals-computations}).


