%!TEX root = ../atomic.sigconf.tex
% LTeX: language=en
%
\section{Introduction}
\label{sec:intro}
\AP For a field $\K$ and a non-empty set $\Indets$ of indeterminates, we use
$\poly{\K}{\Indets}$ to denote the ring of polynomials with coefficients from $\K$
and indeterminates/variables from $\Indets$. A fundamental result in commutative
algebra is \intro{Hilbert's basis theorem}, stating that when $\Indets$ is finite,
every ideal in $\poly{\K}{\Indets}$ is finitely generated~\cite{HILB1890}, where an
\kl{ideal} is a non-empty subset of $\poly{\K}{\Indets}$ that is closed under
addition and multiplication by elements of $\poly{\K}{\Indets}$.
This property follows from 
\kl{Hilbert's basis theorem}, stating that for every ring 
$\A$ that is \kl{Noetherian}, 
the polynomial ring $\poly{\A}{x}$ in one variable over $\A$ is also
\kl{Noetherian}~\cite[Theorem 4.1]{Lang02}.

\AP In this paper, we will 
assume that elements of $\K$ can be effectively represented 
and that basic operations on $\K$ are computable 
($+$, $-$, $\times$, $/$, and equality test).
In this setting, a \Grb\ is a specific kind of generating set of a polynomial ideal
which allows easy checking of membership of a given polynomial in that ideal.
\kl{Gr\"{o}bner bases} were introduced by Buchberger who showed when $\Indets$ is
finite, every ideal in $\poly{\K}{\Indets}$ has a finite \kl{Gr\"{o}bner basis} and
that, for a given a set of polynomials in $\poly{\K}{\Indets}$, one can compute a
finite \kl{Gröbner basis} of the ideal generated by them via the so-called
\intro{Buchberger algorithm}~\cite{BUCH76}. The
existence and computability of \Grbs\ implies the decidability of the
\kl{ideal membership problem}: given a polynomial $f$ and set of polynomial
$H$, decide whether $f$ is in the ideal generated by $H$. 
More generally, \kl{Gr\"{o}bner bases} provide effective representations of ideals,
over which one can decide inclusion, equality, and compute sums or intersections
of ideals~\cite{CLO15}.

%\arka{remove the para below}
%
%The theory of
%\kl{Gr\"{o}bner bases} has applications in very diverse areas of computer
%science, including integer programming \cite{Sturmfels96}, algebraic proof
%systems \cite{algProof}, geometric reasoning \cite{Cox2015chGeom}, fixed
%parameter tractability \cite{ACDM22}, program analysis \cite{SSM04} and
%constraint satisfaction problems \cite{Mas21}.
%In automata theory it has been used for deciding zeroness of polynomial
%automata \cite{BEDUSHWO17}, reachability in symmetric Petri nets \cite{MAME82},
%equivalence for string-to-string transducers \cite{HONKALA00} and equivalence
%of polynomial differential equations \cite{CLEMENTE24}. 

%\AP There has been a growing interest in the last few years for computational
%models that are manipulating infinite data structures in a finite way, for
%instance an automaton reading words on the infinite alphabet $\N$, while
%maintaining a finite number of states. While this idea can be traced back to
%the 90s with the notion of register automata \cite{KAFR94}, it has been revived
%in with the development of the theory of \emph{orbit finite sets}. In this
%setting, one would like to consider an infinite set of variables $\Indets$.
%
%\arka{break in flow}
%As an example, let us consider the set $\Indets$ of variables $x_i$ for $i \in \N$, and
%the \kl{ideal} $\idlZ$ generated by the set $\setof{x_i}{i \in \N}$. It is
%clear that $\idlZ$ is not finitely generated, and we conclude that the
%\kl{Hilbert's basis theorem} (and a fortiori, the \kl{Gr\"{o}bner basis}
%theory) does not extend to the case of infinite sets of indeterminates.

\AP
In addition to their interest in commutative algebra,
these decidability results have important applications in 
other areas of computer science. For instance, the so-called 
``Hilbert Method'' that reduces verifications 
of certain problems on automata and transducers to 
computations on polynomial ideals has been successfully applied
to polynomial automata, and equivalence of string-to-string transducers
of linear growth, and we refer to~\cite{BOJAN19}
for a survey on these applications.

\AP
In this paper, we are interested in 
extending the theory of \kl{Gr\"{o}bner bases} to the case where
the set $\Indets$ of indeterminates is infinite.
As an example, let us consider $\Indets$ to be the 
set of variables $x_i$ for $i \in \N$, and
the \kl{ideal} $\idlZ$ generated by the set $\setof{x}{x \in \Indets}$.
It is clear that $\idlZ$ is not finitely generated. As a consequence,
\kl{Hilbert's basis theorem}, and a fortiori the theory of 
\kl{Gr\"{o}bner bases},
does not extend to the case of infinite sets of indeterminates.

\AP
Thankfully, the infinite set $\Indets$ of
variables (data) often comes with an extra structure,
usually given by relations and functions defined on $\Indets$,
and one is often interested in systems that are invariant
under the action of the group $\group$ of structure preserving
bijections of $\Indets$.
For instance, in the above example,
one may not be interested in the ideal $\idlZ$ generated by the set
$\setof{x}{x \in \Indets}$,
but rather in the \kl{equivariant ideal} generated by
the set $\setof{x}{x \in \Indets}$,
which is the smallest ideal that contains it and is invariant
under the action of $\group$.
In this case, this ideal is finitely generated by any single
indeterminate $x \in \Indets$.
This motivates the study of \kl{equivariant ideals},
that is highly dependent on the specific choice of group action 
$\group \actson \Indets$:
for instance, the ideal $\idlZ$ is not finitely generated as an equivariant
ideal with respect to the trivial group. 
A general analysis of the \kl{equivariant Hilbert basis property}
stating that
``every \kl{equivariant ideal} is orbit finitely generated''
has been recently given in \cite{GHOLAS24},
and this paper aims at providing a computational counterpart.




\subsection{Contributions.} \AP In this paper, we bridge the gap between the
theoretical understanding of the \intro{equivariant Hilbert basis property}
\cite[Property 4]{GHOLAS24}, and the computational aspects of \kl{equivariant
ideals}, by showing that under mild assumptions on the group action, one can
compute an \kl{equivariant Gröbner basis} of an \kl{equivariant ideal}, hence,
that one can decide the \kl{equivariant ideal membership problem}. 

\AP We divide our hypotheses in two parts. First, we will require some
\kl{computability assumptions} to be satisfied by the group action that are
fairly standard in the literature on computation with infinite data. Then, we
will require a semantic assumption on the set of indeterminates that will
guarantee the termination of our procedures, that we call being
\kl{well-structured}, and implies that the set of monomials is
\kl{well-quasi-ordered} with respect to divisibility. Both of these will be
formally introduced in \cref{sec:preliminaries}. Our main positive result
states that under these assumptions, one can compute an \kl{equivariant Gröbner
basis} of an \kl{equivariant ideal}.

\begin{theorem}[name={Equivariant Gröbner Basis},restate=thm:compute-equiv-gb]
  \label{thm:compute-egb}
  Let $\Indets$ be a totally ordered set of indeterminates
  equipped with a group action $\group \actson \Indets$, that satisfies our \kl{computability assumptions}
  and is \kl{well-structured}.
  Then, one can
  compute a \kl{equivariant Gröbner bases} of \kl{equivariant ideals}.
\end{theorem}

Using standard techniques on polynomial ideals, we then use our
\cref{thm:compute-egb} to provide an effective representation of
\kl{equivariant ideals} under the same assumptions.

\begin{corollary}[name={},restate=cor:equivariant-ideals-computations]
  \label{cor:equivariant-ideals-computations}
  Assume that $\group \actson \Indets$
  is \kl{effectively oligomorphic}
  and \kl{well-structured}.
  Then one has an \emph{effective representation} of
  the \kl{equivariant ideals} of $\poly{\K}{\Indets}$,
  such that:
  \begin{enumerate}
    \item One can obtain a representation from an orbit-finite set of generators,
    \item One can effectively decide the \kl{equivariant ideal membership problem}
      given a representation,
    \item The following operations are computable at the level of representations:
      the union of two \kl{equivariant ideals}, 
      the product of two \kl{equivariant ideals},
      the intersection of two \kl{equivariant ideals},
      and checking whether two \kl{equivariant ideals} are equal.
  \end{enumerate}
  \proofref{cor:equivariant-ideals-computations}
\end{corollary}

\AP Then, we illustrate how our positive results find applications in
numerous situations. This is done by providing families of indeterminates that
satisfy our \kl{computability assumptions} and are \kl{well-structured},
and show that these are closed
under \kl{disjoint sums} and \kl{lexicographic products}. Furthermore,
we circumvent the requirement that a total ordering is present on 
the indeterminates by defining \kl{nicely orderable} actions
(\cref{thm:reducts-computable}). Examples of 
indeterminates that we can therefore 
deal with are:
\begin{enumerate}
  \item \kl{Equality Atoms}: the indeterminates are an infinite set 
    and $\group$ is all permutations.
  \item \kl{Dense Linear Orders}: the indeterminates are $\Q$,
    and $\group$ is all order-preserving bijections.
  \item \kl{Dense Meet Tree}: the indeterminates are elements of the 
    infinite dense meet tree, and $\group$ is its group of automorphisms.
\end{enumerate}

\AP We then leverage our positive results (\cref{thm:compute-egb,cor:equivariant-ideals-computations})
to obtain decision procedures for the following problems,
where $\group \actson \Indets$ is a \kl{nicely orderable} group action:
\begin{enumerate}
  \item \cref{cor:orbit-finite-polynomial-automata-zeroness}:
    The \kl{zeroness problem} for \kl{orbit finite polynomial automata},
  \item \cref{cor:rev data VAS}:
    The \kl(revdatapn){reachability problem} for \kl{reversible Petri nets with data},
  \item \cref{cor:lin solv}:
    The \kl(ofeq){solvability problem} for \kl{orbit-finite systems of equations}.
\end{enumerate}

\AP Finally, we provide undecidability results for the \kl{equivariant ideal
membership problem} in the case where our effective assumptions are satisfied,
but the action is not \kl{well-structured}. This aims at illustrating the fact
that our assumptions are close to optimal. One classical obstruction for a
group action to be \kl{well-structured} is to have the ability to represent an
\emph{infinite path} (a formal definition will be given in
\cref{sec:undecidability}). We prove that whenever one can (effectively)
represent an infinite path in the set of \kl{monomials} $\mon{\Indets}$, then
the \kl{equivariant ideal membership problem} is undecidable.

\begin{theorem}[name={Undecidability of Equivariant Ideal Membership},restate=thm:undecidable-paths]
  \label{thm:undecidable-paths}
  Let $\Indets$ be a totally ordered set of indeterminates
  equipped with a group action $\group \actson \Indets$, under our \kl{computability assumptions}.
  If $\Indets$ contains an \kl(of){infinite path}
  then the \kl{equivariant ideal membership problem} is undecidable.
\end{theorem}

\subsection{Related Research} Let us call \kl{Equality Atoms} the infinite set
of indeterminates with all permutations acting on them. The fact that
\kl{Hilbert's basis property} holds for polynomials with indeterminates being
the \kl{Equality Atoms} is a frequently rediscovered fact
\cite{AH07,AH08,HS12,HKL18}. Recently, Ghosh and Lasota provided a general
answer to characterize which group actions enjoy \kl{Hilbert's basis property}
\cite[Theorem 11 and 12]{GHOLAS24}, and provided in some limited setting a
version of \kl{Buchberger's algorithm} \cite[Section 6]{GHOLAS24}.
Let us recall their precise statements in order to compare it with our contributions.

\begin{theorem}[name={\cite[Theorem 11 and 12]{GHOLAS24}}]
  \label{thm:equiv-hilbert-property}
  Let $\Indets$ be a set of indeterminates equipped with a
  group action $\group \actson \Indets$.
  Then,
  \cref{item:equiv-hilb-om-ord} implies 
  \cref{item:equiv-hilb-hbp} implies 
  \cref{item:equiv-hilb-om}, where
  \begin{enumerate}
    \item \label{item:equiv-hilb-om-ord}
      The action is \kl{$\omega$-well-structured} and the indeterminates are equipped with a total order
      \kl(ord){compatible} with the group action,
    \item \label{item:equiv-hilb-hbp}
      The \kl{equivariant Hilbert basis property} holds for  $\poly{\K}{\Indets}$,
    \item \label{item:equiv-hilb-om}
      The action is \kl{$\omega$-well-structured}.
  \end{enumerate}
\end{theorem}

Let us briefly state that being \kl{$\omega$-well-structured} is \emph{a
priori} a weaker condition than being \kl{well-structured}, but that the two
are conjectured to be equivalent \cite[Problems 9]{POUZ24}. Similarly, it is
conjectured that \cref{item:equiv-hilb-om} and \cref{item:equiv-hilb-om-ord}
are equivalent\footnote{Up to modifying the group action to respect the
ordering.} \cite[Problems 12]{POUZ24}. As a consequence, our
\cref{thm:compute-egb} is conjectured to hold whenever the \kl{equivariant
Hilbert property} does. Beware that
\cref{thm:equiv-hilbert-property,thm:compute-egb} are incomparable: the former
does not talk about decidability, while the latter only considers
\kl{equivariant ideals} that are already finitely presented, and we will show
in \cref{ex:non-wqo-undecidable} an example where \kl{equivariant Gröbner
bases} are computable, but the \kl{equivariant Hilbert basis property} fails.

Let us now comment on the decision procedures provided in the literature.
First, most results focus on \kl{Dense Linear Orders} or \kl{Equality Atoms},
which are only special cases of our general result. A reason why this happens
is that, until this paper, the only way to provide a decision procedure was to
assume that the ordering on the indeterminates was \emph{well-founded}
\cite[Section 6]{GHOLAS24}, or to encode the behaviour of the indeterminates in
a set with a \kl{well-founded total ordering} \cite[Section 7, Reduction
Game]{GHOLAS24}. We provide the first result that gets rid of the assumption
that the ordering is \emph{well-founded}. As a consequence, we can deal with
the \kl{Dense Linear Order} without using any encoding tricks. Furthermore, we
provided with the \kl{Dense Meet Tree} an example of group action that was not
shown to have decidable \kl{equivariant ideal membership problem} prior to this
work. Our applications to the decidability of other problems in theoretical
computer science strictly extend those given in \cite[Section 4, 8, and
9]{GHOLAS24}. Indeed, their encoding of \emph{orbit finite weighted automata}
did not require the ability to test inclusion of \kl{equivariant ideals}, while
it is central to our result on \kl{orbit finite polynomial automata} (that
strictly generalise weighted automata). Furthermore, their solutions to the
problems concerning \kl{reversible Petri-nets with data} and \kl{orbit-finite
linear systems of equations} only apply when the indeterminates are equipped
with a \emph{well-founded} total ordering, which we do not require.

Finally, our results are part of a larger research direction that aims at
establishing an algorithmic theory of computation with orbit-finite sets. For
instance, \cite{BFKM24} studies equivariant subspaces of vector spaces
generated by orbit-finite sets, \cite{GHL22,GHL25} study solvability of
orbit-finite systems of linear equations and inequalities, and
\cite{BFKM24,GHL22,Prz23} study duals of vector spaces generated by
orbit-finite sets.


\paragraph{Organisation.} \AP The rest of the paper is organised as follows. In
\cref{sec:preliminaries}, we introduce formally the notions of \kl{Gröbner
bases}, \kl{effectively oligomorphic} actions, and \kl{well-quasi-orderings},
which are the main assumptions of our positive results. After that, we
introduce in \cref{sec:weakgb} an adaptation of \kl{Buchberger's algorithm} to
the equivariant case, that computes a \kl{weak equivariant Gröbner basis} of an
\kl{equivariant ideal}. In \cref{sec:equivariant-grobner-basis}, we use
\kl{weak equivariant Gröbner bases} to prove our main positive
\cref{thm:compute-egb}, and we show that it provides a way to effectively
represent \kl{equivariant ideals} (\cref{cor:equivariant-ideals-computations}).
We continue by showing in \cref{sec:closure-properties} that the assumptions of
our \cref{thm:compute-egb} are closed under natural operations
(\cref{cor:closure-properties,thm:reducts-computable}). The positive results
regarding the \kl{equivariant ideal membership problem} are then leveraged to
obtain several decision procedures. Finally, in \cref{sec:undecidability}, we
show that our assumptions are close to optimal by proving that the
\kl{equivariant ideal membership problem} is undecidable whenever one can find
\kl(of){infinite paths} in the set of indeterminates
(\cref{thm:undecidable-paths}), which is conjectured to be a complete
characterisation of the undecidability of the \kl{equivariant ideal membership
problem} (\cref{rem:conj-wqo-infinite-path}).
