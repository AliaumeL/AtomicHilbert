%!TEX root = ../atomic.asmart.tex
% LTeX: language=en
\section{Introduction}
\label{sec:intro}
\AP For a field $\K$ and a non-empty set $\X$ of indeterminates, we use
$\poly{\K}{\X}$ to denote the ring of polynomials with coefficients from $\K$
and indeterminates/variables from $\X$. A fundamental result in commutative
algebra is \intro{Hilbert's basis theorem}, stating that when $\X$ is finite,
every ideal in $\poly{\K}{\X}$ is finitely generated \cite{HILB1890}, where an
\intro{ideal} is a non-empty subset of $\poly{\K}{\X}$ that is closed under
addition and multiplication by elements of $\poly{\K}{\X}$. This property can
be rephrased as the fact that the set of polynomials $\poly{\K}{\X}$ is
\intro{Noetherian}. \kl{Hilbert's basis theorem} extends to the case where $\K$
is a ring that is itself \kl{Noetherian} \cite[Theorem 4.1]{Lang02}.

\AP A \intro*\Grb\ is a specific kind of generating set of a polynomial ideal
which allows easy checking of membership of a given polynomial in that ideal.
\kl{Gr\"{o}bner bases} were introduced by Buchberger who showed when $\X$ is
finite, every ideal in $\poly{\K}{\X}$ has a finite \kl{Gr\"{o}bner basis} and
that, for a given a set of polynomials in $\poly{\K}{\X}$, one can compute a
finite \kl{Gr\"{o}bner basis} of the ideal generated by them \cite{BUCH76}. The
existence and computability of \Grbs\ implies the decidability of the
\intro{ideal membership problem}: given a polynomial $f$ and set of polynomial
$H$, decide whether $f$ is in the ideal generated by $H$. The theory of
\kl{Gr\"{o}bner bases} has applications in very diverse areas of computer
science, including integer programming \cite{Sturmfels96}, algebraic proof
systems \cite{algProof}, geometric reasoning \cite{Cox2015chGeom}, fixed
parameter tractability \cite{ACDM22}, program analysis \cite{SSM04} and
constraint satisfaction problems \cite{Mas21}.
In automata theory it has been used for deciding zeroness of polynomial
automata \cite{BEDUSHWO17}, reachability in symmetric Petri nets \cite{MAME82},
equivalence for string-to-string transducers \cite{HONKALA00} and equivalence
of polynomial differential equations \cite{CLEMENTE24}. 

\AP There has been a growing interest in the last few years for computational
models that are manipulating infinite data structures in a finite way, for
instance an automaton reading words on the infinite alphabet $\N$, while
maintaining a finite number of states. While this idea can be traced back to
the 90s with the notion of register automata \cite{KAFR94}, it has been revived
in with the development of the theory of \emph{orbit finite sets}. In this
setting, one would like to consider an infinite set of variables $\X$. As an
example, let us consider the set $\X$ of variables $x_i$ for $i \in \N$, and
the \kl{ideal} $\idlZ$ generated by the set $\setof{x_i}{i \in \N}$. It is
clear that $\idlZ$ is not finitely generated, and we conclude that the
\kl{Hilbert's basis theorem} (and a fortiori, the \kl{Gr\"{o}bner basis}
theory) does not extend to the case of infinite sets of indeterminates.

\AP However, in the applications mentionned above, the infinite set of
variables (data) comes with an extra structure: the behaviour of the considered
systems are invariant under the action of a group $\group$ on $\X$. The action
of this $\group$ on $\X$ naturally induces an action on $\poly{\K}{\X}$, by
renaming the variables. The typical example is the group of all permutations of
$\X$, which corresponds to seeing $\X$ as a set of \emph{indistinguishable}
names: one is not interested in the ideal $\idlZ$ generated by the set
$\setof{x_i}{i \in \N}$, but rather in the \kl{equivariant ideal} generated by
the set $\setof{x_i}{i \in \N}$, which is the smallest ideal that contains it
and is invariant under the action of $\group$. In this case, this ideal is
finitely generated by a single indeterminate, e.g. $x_1$. Please note that
equivariance does not imply finite generation in general: for instance, the
ideal $\idlZ$ is not finitely generated as an equivariant ideal with respect to
the trivial group.

\AP There has been a growing interest in understanding which groups $\group$
and sets of variables $\X$ allow one to extend the \kl{Hilbert's basis theorem}
to the equivariant case, and to adapt the theory of \kl{Gr\"{o}bner bases} to
this setting \cite{BRDR11,HISU12,HIKRLE18,GHOLAS24}, and there is an almost
complete characterisation of the pairs $(\X,\group)$ for which the
\kl{Equivariant Hilbert basis property} \cite[Theorems 11 and 12]{GHOLAS24}.
But to obtain decision procedures, one still lacks a generalisation of
\kl{Buchberger's algorithm} to the equivariant case, except under very strong
extra assumptions: the set of indeterminates should come equipped with a
\emph{total} and \emph{well-founded} ordering \cite[Section 6]{GHOLAS24}.

\AP In this paper, we develop an equivariant generalisation of \kl{Buchberger's
algorithm}, to effectively compute \kl{Gr\"{o}bner bases} of \kl{equivariant
ideals} under mild assumptions. We also show that common examples violating
these assumptions have undecidable \kl{equivariant ideal membership problems}.

\paragraph{Contributions.} \AP In this paper, we continue the work of
\cite{GHOLAS24} and show that the \kl{equivariant ideal membership problem} is
decidable under mild assumptions. We will state our main results without
defining all the concepts appearing in them to aid readability.
Our first theorem is that the \kl{equivariant ideal membership problem} is decidable
under very weak effectivity assumptions.

\begin{theorem}
  \label{thm:decide-equiv-ideal-mem}
  Let $(\X, \group, \leq)$ be a set of indeterminates, a group acting \kl{effectively
  oligomorphically}
  on $\X$ and an
  \kl{effective total ordering} $\leq$ on $\X$ that is compatible with the action of
  $\group$. 
  If $(\mon[\omega + 1]{\X}, \gdivleq)$ is a \kl{WQO}, then one can decide the
  \kl{equivariant ideal membership problem}.
\end{theorem}

Next, we show that one can strengthen the conclusion an obtain a fully-fledged
\kl{Gr\"{o}bner basis} algorithm, under the same effectivity assumptions, but
with a slightly stronger structural requirement on the set of indeterminates.

\begin{theorem}
  \label{thm:compute-egb}
  Let $(\X, \group, \leq)$ be a set of indeterminates, a group acting \kl{effectively
  oligomorphically}
  on $\X$ and an
  \kl{effective total ordering} $\leq$ on $\X$ that is compatible with the action of
  $\group$. 
  If $(\mon[\omega + \omega]{\X}, \gdivleq)$ is a \kl{WQO}, then one can
  compute an \kl{equivariant Gröbner basis} from a finite set $H$ of generators.
\end{theorem}

Informally, we require that the set of indeterminates comes equipped with a
total ordering (not necessarily well-founded) that is compatible with the
action of the group, and that one can effectively compute on the representation
(compute the order, and compute representatives of orbits). In addition to
these natural assumptions, we also require a \kl{well-quasi-ordering} property
on the set of \kl{monomials}. Note that it is shown in \cite{GHOLAS24} that for
the \kl{equivariant Hilbert basis property} to hold, it is necessary that
$(\mon[\omega]{\X}, \gdivleq)$ is a \kl{WQO}. Furthermore, it is conjectured
that whenever $(\mon[\omega]{\X}, \gdivleq)$ is a \kl{WQO}, then $(\mon[\omega
+ 1]{\X}, \gdivleq)$ and $(\mon[\omega + \omega]{\X}, \gdivleq)$ are also
\kl{WQOs}.

Finally, we show that if one can interpret paths, then the 
equivariant membership problem is undecidable.

\begin{theorem}
  \label{thm:undecidable-paths}
  todo.
\end{theorem}

\paragraph{Organisation.} \AP
The rest of the paper is organised as follows.
In \cref{sec:preliminaries}, we introduce formally what a
\kl{Gröbner basis} is, and present the background on 
how to compute them using \kl{Buchberger's algorithm}.
Then, we will present in \cref{sec:weakgb} an adaptation of 
\kl{Buchberger's algorithm} to the \kl{equivariant case}, that computes
a \kl{weak equivariant Gröbner basis} of an \kl{equivariant ideal}.
In \cref{sec:algorithm}, we use \kl{weak equivariant Gröbner bases} to devise a decision
procedure for the \kl{equivariant ideal membership problem}, and prove
our \cref{thm:decide-equiv-ideal-mem}.
The procedure will be based on the following idea:
given a finite set $S \subfin \X$ of indeterminates, and a finite set $H$ of generators, we will compute a \kl{weak equivariant Gröbner basis} 
$\Basis_S$ that has the property that 
$\EqIdlGen{H} \cap \poly{\K}{S} = \IdlGen{\Basis_S \cap \poly{\K}{S}}$, for this 
particular set $S$.
We will call such sets \kl{$S$-strong equivariant Gröbner bases} for 
$\EqIdlGen{H}$.
Then, in \cref{sec:equivariant-grobner-basis},
we will show how to transform our algorithm that computes 
\kl{$S$-strong equivariant Gröbner bases} into an algorithm that computes
\kl{equivariant Gröbner bases} for \kl{equivariant ideals}.
In \cref{sec:examples}
we will show how to use our algorithm to solve some non-trivial examples. This will also be a way
to show that our restrictions are very mild, and that one can really apply our theoretical
results.
Finally, in \cref{sec:conclusion}, we will discuss on the need 
for a total ordering on the set of indeterminates, and the possibility
to relax the hypotheses of our results.

