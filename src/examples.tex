% LTeX: language=en
%!TEX root = ../atomic.asmart.tex
%
\section{Examples of group actions}\label{sec:act ex}

\AP Many of the common examples of group actions $\group\actson\Indets$ are
obtained by considering $\Indets$  as set with some structure, described by
some relations and functions on that set, and $\group$ is the group
$\aut{\Indets}$ of all automorphisms (i.e.\ bijections that preserve and
reflect the structure) of $\Indets$. A monomial $\monelt[p] \in
\mon[Q]{\Indets}$ can be thought as a labelling of a finite substructure of
$\X$ using elements of $Y$. If the structure $\Indets$ is \intro{homogenous},
that is, if isomorphisms between finite induced substructures extends to
automorphisms of the whole structure, then $\gdivleq$ is same as embedding of
labelled finite induced substructures of $\X$. \footnote{ We refer the reader
to \cite[Chapter 7]{BOJAN16inf} and \cite{homsurvey} for more details on
homogeneous structures.} Let us now give some examples of such structures and
whether they satisfy our \kl{computability assumptions}, and whether the
\kl{divisibility relation up-to-$\group$} is a \kl{well-quasi-ordering} for
\kl{generalised monomials}.


\begin{example}[Equality Atoms]
  \label{ex:eq atoms}
Let $\intro*\EqualityAtoms$ be an infinite set without any additional structure other than the equality relation.
Up to isomorphism, finite induced substructures of $\EqualityAtoms$ are finite sets,
monomials in $\mon[Y]{\EqualityAtoms}$ are finite multisets of elements in $Y$,
and $\gdivleq[\aut{\EqualityAtoms}]$ is the multiset ordering \cite[Section 1.5]{SCSC17},
which is a \kl{WQO} \cite[Corollary 1.21]{SCSC17}.
\end{example}

\begin{example}[Dense linear order]
  \label{ex:dlo}
Let $\intro*\OrderAtoms$ be the set of rational numbers ordered by the usual ordering.
Note that under this ordering, $\OrderAtoms$ is a dense linear order without endpoints.
We write $\OrderAtoms$ instead of $\Q$ to emphasise that we use its elements as indeterminates and not as coefficients of polynomials. 
Up to isomorphism, finite induced substructures of $\OrderAtoms$ are finite linear orders,
monomials in $\mon[Y]{\OrderAtoms}$ are words in $Y^*$ (i.e.\ finite linear order labelled with elements of $Y$)
and $\gdivleq[\aut{\OrderAtoms}]$ is the scattered subword ordering, which is a \kl{WQO} due to Higman's lemma \cite{HIG52}.
\end{example}

\begin{example}[The Rado graph]
  \label{ex:rado}
Let $\intro*\RadoAtoms$ be the Rado graph (\cite[Section 7.3.1]{BOJAN16inf},\cite[Example 2.2.1]{homsurvey}).
Up to isomorphism,
finite induced substructures of $\RadoAtoms$ are finite undirected graphs,
monomials in $\mon[Y]{\RadoAtoms}$ are graphs with vertices labelled with $Y$,
and $\gdivleq[\aut{\RadoAtoms}]$ is the labelled induced subgraph ordering even when $Y$ is a singleton.
For example, cycles of length more than three form an infinite antichain.
\end{example}

\begin{example}[Infinite dimensional vector space]
  \label{ex:bit vector}
Let $\intro*\BitVectorAtoms$ be an infinite dimensional vector space over $\ftwo$.
Up to isomorphism,
finite induced substructures of $\BitVectorAtoms$ are finite dimensional vector spaces over $\ftwo$.
These are well-quasi-ordered in the absence of labelling.
However, even when $Y = \N$,
$(\mon[Y]{\BitVectorAtoms},\gdivleq[\aut{\BitVectorAtoms}])$ is not a \kl{WQO} as illustrated by the following antichain.
Let $\{v_1,v_2,\dots\}\subseteq \BitVectorAtoms$ be a countable set of linearly independent vectors in $\BitVectorAtoms$.
Let $\oplus$ denote the addition operation of $\BitVectorAtoms$.
For $n \geq 3$ define the monomial 
$
\monelt[p]_n \defined v^2_1 \ldots v^2_n  (v_1\oplus v_2)(v_2\oplus v_3) \ldots (v_{n-1}\oplus v_n)  (v_{n}\oplus v_1)
$.
Then, $\setof{\monelt[p]_n}{n = 3,4,\dots}$ forms an infinite antichain.
\end{example}

The previous \cref{ex:eq atoms,ex:dlo,ex:rado,ex:bit vector}
are
well known examples in the theory of \emph{sets with atoms} \cite{BOJAN16inf}.
Let us now give a new example of \kl{well-quasi-ordered} \kl{generalised
monomials}, by extending \Cref{ex:dlo} that
relied on Higman's lemma \cite{HIG52} via Kruskal's tree theorem
\cite{Kruskal60}.


\begin{example}[Dense Tree]
  \label{ex:dense tree}
Let $\intro*\TreeAtoms$ denote the universal countable dense meet-tree, as
defined in 
\cite[Page 2]{KRS21} or \cite[Section 7.3.3]{BOJAN16inf}.
Note that the tree structure is given by the \intro{least common ancestor} (\reintro{meet})
operation, and not by its edges.
For a subset $S\subset \TreeAtoms$,
define its \intro(meet-tree){closure} to be the smallest subtree of $\TreeAtoms$ containing $S$.
Up to isomorphism, finite induced substructures of $\TreeAtoms$ are finite meet-trees.
Monomials in $\mon[Y]{\TreeAtoms}$ are finite meet-trees labelled with $1 + Y$.
Here $1 + Y$ is the \kl{WQO} containing one more element than $Y$ which is incomparable to elements in $Y$,
and is used to label nodes that are in the \kl(meet-tree){closure} of the set of variable of a monomial, but not in the monomial itself.
The relation $\gdivleq[\aut{\TreeAtoms}]$ is the embedding of labelled meet-trees,
which is a \kl{WQO} due to Kruskal's tree theorem \cite{Kruskal60}.
\end{example}

Even though the above examples using \kl{homogeneous} 
structures nicely illustrate the correspondance between 
monomials and labelled finite substructures, 
we can also consider \kl{non-homogeneous} structures,
such as in \Cref{ex:int} below.
%
\begin{example}\label{ex:int}
Let $\calZ$ be the set of integers ordered by the usual ordering.
Then $\aut{\calZ}$ is the set of all order preserving bijections of $\D$.
Note that every order preserving bijection of the set $\calZ$ is a translation $n \mapsto n + c$ for some constant $c\in\calZ$.
By definition, the action $\aut{\calZ} \actson \calZ$ preserves the linear order on $\Z$.
However, $(\mon[Y]{\calZ}, \gdivleq[\aut{\calZ}])$ is not a \kl{WQO} even when $Y$ is a singleton.
An example of an infinite antichain is the set $\setof{a b}{b\in\calZ\setminus\{a\}}$, for any fixed $a\in\calZ$.
\end{example}

\paragraph{Reducts} \AP As mentioned in the introduction, some examples of
group actions $\group\actson\Indets$ do not preserve a linear order on
$\Indets$, such as the set $\EqualityAtoms$ of indistinguishable names without
any relations. However, there are techniques that allow us to reduce the
problem of computing \kl{equivariant Gröbner bases} to the case where the
action preserves a linear ordering, using what is called a \kl{reduct} of the
action. A group action $\group\actson\Indets$ is said to be a \intro{reduct} of
another group action $\calH\actson\Y$ if there exists a bijection $f \colon
\Indets\to\Y$ such that $f^{-1}\circ \gelem \circ f \in\group$ for every
$\gelem \in \calH$. It is called an \intro{effective reduct} if $f$ is
computable.

\begin{lemma}\label{lem:reducts-equiv-hilbert}
  Let $\group\actson\Indets$ be an \kl{effective reduct} of $\calH\actson\Y$ 
  via a computable function $f\colon\Indets\to\Y$. Then,
  \begin{enumerate}
    \item for every $\group$-equivariant ideal $\idl[I]\subseteq\poly{\K}{\X}$,
    $f(\idl[I])\subseteq\poly{\K}{\Y}$ is a $\calH$-equivariant ideal,
    \item if $\Basis\subseteq\poly{\K}{\Y}$ is a $\calH$-equivariant Gr\"{o}bner basis then $\orbit[\group]{f^{-1}(\Basis)}$ is a $\group$-equivariant Gr\"{o}bner basis.
  \end{enumerate}
\end{lemma}

\AP We say that an action $\group\actson\Indets$ is \intro{nicely orderable} if
it is an \kl{effective reduct} of some  \kl{effectively oligomorphic} action
$\calH\actson\Y$ that satisfies the conditions mentioned in
\Cref{thm:compute-egb},
i.e.\ $\calH\actson\Y$ preserves an effective linear order and
$\gdivleq[\calH]$ is a \kl{WQO}. As an immediate corollary of
\Cref{thm:compute-egb},
\Cref{lem:reducts-equiv-hilbert} we get the desired \cref{cor:egb}.
  
\begin{corollary}\label{cor:egb} If $\group\actson\Indets$ is
  \kl{nicely orderable} then
  one can compute an \kl{equivariant Gröbner basis} of any 
  \kl{equivariant ideal} given by an \kl{orbit finite} set of
  generators.
\end{corollary}

\begin{remark}\label{rem:reduct}

\Cref{lem:reducts-equiv-hilbert} implies that one can apply our
results to an action $\group\actson\Indets$ that does not preserve a linear
order, as soon as it is  a reduct of some another action $\calH\actson\Indets$ which
does preserves a linear order. 

For example, $\aut{\EqualityAtoms}\actson\EqualityAtoms$ is a reduct of
$\aut{\OrderAtoms}\actson\OrderAtoms$ assuming $\EqualityAtoms$ is countable.
Similarly, let $\T_<$ be the countable dense-meet tree with a lexicographic
ordering, as defined in \cite[Remark 6.14]{KRS21}.\footnote{The remark says
that finite meet-trees expanded with a lexicographic ordering is a Fra\"{i}sse
class, from which it follows that there exists a Fra\"{i}sse limit $\T_<$ for
that class.} Let $\group$ be the group of bijections of $\T_<$ which do not
necessarily preserve the lexicographic ordering. Then $\group\actson\T_<$ is
isomorphic to $\aut{\TreeAtoms}\actson\TreeAtoms$, and hence
$\aut{\TreeAtoms}\actson\TreeAtoms$ is a reduct of $\aut{\T_<}\actson\T_<$.
\end{remark}


\todo[inline]{Rework this paragraph (aliaume)}

\paragraph{Effective oligomorphicity}
%
\cite[Theorem 7.6]{BOJAN16inf} implies that the structures mentioned in  \Cref{ex:eq atoms,ex:dlo,ex:rado,ex:bit vector,ex:dense tree,rem:reduct} are \kl{oligomorphic}.
They are also \kl{effectively oligomorphic}.
To show effectivity we need to give a way to represent the element of the structures and show point (2) and (3) in the definition of \kl{effective oligomorphicity} holds under the representations.
This is possible for all the structures mentioned in \Cref{ex:eq atoms,ex:dlo,ex:rado,ex:bit vector,ex:dense tree,rem:reduct}.
For example, $\A$ can be identified with the set of strings over a fixed finite alphabet.
Elements of $\calQ$ can be represented in the usual way.
A representation of elements of $\T$ is given in \cite[Page 244-245]{BOJAN16inf}.
\cite{CompFraisse} gives a general way to represent Fra\"{i}sse limits.
