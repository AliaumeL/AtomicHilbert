% LTeX: language=en
%%!TEX root = ../atomic.asmart.tex
%
\section{Relation to Existing Results and Examples}
\label{sec:examples}

In this section, we are interested in the consequences of our decidability
results. First, we will provide numerous examples of sets of indeterminates
that satisfy our \kl{computability assumptions} as well as our
\kl{well-quasi-ordering} conditions. Then, we will discuss how our results can
be applied to solve various decidability problems in theoretical computer
science.

\subsection{Crafting sets of indeterminates}


\paragraph{Sets with atoms.}


\paragraph{Relational structures.} Let $\mathbb{A}$ be an infinite relational
structure with finitely many relations. Then, one can consider the set of
polynomials $\poly{\K}{\mathbb{A}}$, where indeterminates are elements of the
universe of $\mathbb{A}$. The group of all automorphisms of $\mathbb{A}$ (i.e.,
bijections of the universe that preserve the relations) acts on
$\poly{\K}{\mathbb{A}}$ by permuting the indeterminates.

Natural examples are polynomials whose indeterminates are indexed by the
natural numbers (with inequality), or the rationals (with inequality). In this
setting, \kl{effective oligomorphcity} means that \todo{do it}. The fact that
$(\mon{\mathbb{A}}, \gdivleq)$ is a well-quasi-ordering corresponds to ordering
\emph{finite substructures} of $\mathbb{A}$ by the \emph{labelled induced
substructure} relation, and asking whether the class obtained is
well-quasi-ordered. This is a well-studied question in graph theory, where a
conjecture of Pouzet states that this holds with two labels if and only if it
holds for every ordinal. In particular, for such structures, it is therefore
conjectured that $(\mon{\Indets}, \gdivleq)$ is a well-quasi-ordering if and
only if $(\mon[\om \ordplus \ordfin{1}]{\Indets}, \gdivleq)$, $(\mon[\om
\ordplus \om]{\Indets}, \gdivleq)$, and $(\mon[\om^2]{\Indets}, \gdivleq)$ are
well-quasi-orderings too.

\subsection{Decision procedures}

\paragraph{Topological Well-Structured Transition Systems.} \AP The notion of
\kl{topological well-structured transition system} was introduced by
Goubault-Larrecq in \cite{JGL07}, noticing that the pre-existing notion of
\kl{Noetherian space} could serve as a topological generalisation of
\kl{well-quasi-orderings}, for which the celebrated decision procedures on
\kl{well-structured transition systems} can be applied. In particular,
Goubault-Larrecq used such systems to verify properties of \emph{polynomial
programs} computing over the rationals, that can communicate over lossy
channels using a finite alphabet \cite{JGL10}. The fact that polynomial
computations could be verified followed from the theory of \kl{Gröbner bases}
on finitely many indeterminates.

\AP A \intro{topological space} is a set $X$ equipped with a collection $\tau$
of subsets of $X$ that is stable under finite intersections and arbitrary
unions.\footnote{In particular, $\tau$ contains the empty set and $X$ itself.}
In a \kl{topological space}, elements of $\tau$ are called \intro{open
subsets}, while their complements (in $X$) are called \intro{closed subsets}. A
\kl{topological space} is \intro(space){Noetherian} when, for every sequence
$\seqof{U_i}[i \in \N]$ of \kl{open subsets}, there exists $n \in \N$ such that
$\bigcup_{i \in \N} U_i = \bigcup_{i \leq n} U_i$. We refer the readers to the
book \cite{JGL13} for a comprehensive introduction to \kl{Noetherian spaces}
and their usage in theoretical computer science. Let us briefly argue that
\kl{Noetherian spaces} generalize \kl{well-quasi-orders} in
\cref{ex:well-quasi-orders-are-noeth}, and encode the
\kl{Hilbert basis property} in \cref{ex:polynomials-noetherian}.

\begin{example}[ see \cite{JGL13}]
  \label{ex:well-quasi-orders-are-noeth}
  Let $(X, \leq)$ be a quasi-ordered set.
  Then, the set $X$ equipped with the \kl{topology} having 
  as \kl{open sets} the upwards-closed subsets of $X$ is \kl(space){Noetherian}
  if and only if $(X, \leq)$ is \kl{well-quasi-ordered}.
\end{example}

\begin{example}[ see \cite{JGL13}]
  \label{ex:polynomials-noetherian}
  Let $\K$ be a field, and let $n \in \N$.
  The space $\K^n$ equipped with the \kl{Zariski topology}
  \kl(space){Noetherian}; where the \intro{Zariski topology}
  is the topology whose \kl{closed sets} are 
  of the form $\setof{ \vec{x} \in \K^n}{ \forall p \in \idl, p(\vec{x}) = 0}$,
  where $\idl$ is an \kl{ideal} of $\poly{\K}{x_1, \dots, x_n}$.
\end{example}

\AP The advantage of \kl{Noetherian spaces} over \kl{well-quasi-orderings} and
\kl{Noetherian rings} is that they generalize both and can be \emph{combined}:
\kl{Noetherian spaces} are closed under finite sums, finite products,
considering finite words, considering finite trees, and many more \todo{cite}.
As a consequence, they provide a versatile tool to express the set of states of
a system, ensuring that a strong termination property holds.

\AP A \intro{topological well-structured transition system} is a
\kl{topological space} $(X, \tau)$, equipped with a finite number of transition
functions

\paragraph{Consequences for polynomial automata.}

\paragraph{Consequences for vector addition systems with states.}
