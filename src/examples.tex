% LTeX: language=en
%%!TEX root = ../atomic.asmart.tex
%
\section{Relation to Existing Results and Examples}
\label{sec:examples}
%In this section, we are interested in the consequences of our decidability
%results. First, we will provide numerous examples of sets of indeterminates
%that satisfy our \kl{computability assumptions} as well as our
%\kl{well-quasi-ordering} conditions. Then, we will discuss how our results can
%be applied to solve various decidability problems in theoretical computer
%science.
%
%\todo[inline]{for arka: integrate these examples}
%\begin{example}
%  \label{ex:q-is-super-wqo}
%  The set $(\mon[Y]{\Indets_\calQ}, \gdivleq)$ is a \kl{well-quasi-ordering} whenever $Y$ is
%  one, and in particular $\Indets_\calQ$ satisfies the computability assumptions and
%  the termination assumptions of both \cref{thm:compute-egb}
%  and
%  \cref{thm:decide-equiv-ideal-mem}.
%\end{example}
%\begin{proof}
%  Let $\seqof{\monelt_i}[i \in \N]$ be a sequence of monomials in
%  $\mon[Y]{\Indets_\calQ}$. Let us write each monomial $\monelt_i$ as
%  a finite word $w_i$ over the alphabet $Y$, by writing all the exponents in the order 
%  prescribed by the indeterminates.
%  Because $Y$ is a \kl{well-quasi-ordering}, the set of all finite words over $Y$ is
%  \kl{well-quasi-ordered} by the \emph{scattered subword} relation \cite{HIG52}.
%  Now, if $w_i$ is a scattered subword of $w_j$, then
%  $\monelt_i \gdivleq \monelt_j$, by choosing a suitable $\gelem \in \group$.
%\end{proof}
%
%\begin{example}
%  \label{ex:z-is-not-wqo}
%  The set $(\mon[Y]{\Indets_\Z}, \gdivleq)$ is not a \kl{well-quasi-ordering} whenever $Y$ is
%  contains two distinct elements.
%  In particular, $\Indets_\Z$ does not satisfy the termination assumptions of
%  \cref{thm:compute-egb} and \cref{thm:decide-equiv-ideal-mem}.
%\end{example}
%\begin{proof}
%  Assume that $Y$ has two distinct elements $a$ and $b$, and let us assume without loss of generality
%  that $a \leq b$. The sequence of monomials 
%  $\monelt_i \defined x_1^{b} x_2^{a} \cdots x_{i-1}^{a} x_i^{b}$
%  forms an infinite antichain in $(\mon[Y]{\Indets_\Z}, \gdivleq)$.
%  Indeed, if $\monelt_i \gdivleq \monelt_j$ for some $i < j$, then
%  without loss of generality, $\gelem_i (x_1) = x_1$, and 
%  $\gelem_i (x_i) = x_j$, because these are the only ones that can be 
%  equipped with a large enough exponent.
%  Therefore, $\gelem_i = \mathrm{id}$ since the group only contains translations.
%  However, this implies that the exponent of $x_j$ in $\monelt_j$ is at most $a$,
%  which contradicts the fact that it is $b$.
%\end{proof}

\subsection{Crafting sets of indeterminates}
%
Many of the common examples of group actions $\group\actson\Indets$ are of the form where $\Indets$ is a set with some structure, described by some relations and functions on that set,
and $\group$ is the group $\aut{\Indets}$ of all automorphisms (i.e.\ bijections that preserve and reflect the structure) of $\Indets$.
A monomial $\monelt[p] \in \mon[Q]{\Indets}$ can be thought as a labelling of a finite substructure of $\X$ using elements of $Y$.
If the structure $\Indets$ is \intro{homogenous} :
isomorphism between finite induced substructures extends to an automorphism of the whole structure,
then $\gdivleq$ is same as embedding of labelled finite induced substructures of $\X$.
\footnote{
We refer the reader to \cite[Chapter 7]{BOJAN16inf} and \cite{homsurvey} for more details on homogeneous structures.}
We give some examples of homogeneous structures and discuss whether they satisfy the WQO condition of \Cref{thm:compute-egb} :
$(\mon[Y]{\Indets},\gdivleq[\aut{\Indets}])$ is a \kl{WQO} for every \kl{WQO} $Y$. 
%
\begin{example}\label{ex:eq atoms}
Let $\A$ be an infinite set without any additional structure other than the equality relation.
%Then $\aut{\A}$ is the set of all bijections of the set $\A$.
Up to isomorphism, finite induced substructures of $\A$ are finite sets,
monomials in $\mon[Y]{\A}$ are finite multisets of elements in $Y$,
and $\gdivleq[\aut{\A}]$ is the multiset ordering \cite[Section 1.5]{SCSC17},
which is a \kl{WQO} \cite[Corollary 1.21]{SCSC17}.
\end{example}
%
\begin{example}\label{ex:dlo}
Let $\calQ$ be the set of rational numbers ordered by the usual ordering.
Note that under this ordering, $\calQ$ is a dense linear order without endpoints.
We write $\calQ$ instead of $\Q$ to emphasise that we use its elements as indeterminates and not as coefficients of polynomials. 
Up to isomorphism, finite induced substructures of $\calQ$ are finite linear orders,
monomials in $\mon[Y]{\calQ}$ are words in $Y^*$ (i.e.\ finite linear order labelled with elements of $Y$)
and $\gdivleq[\aut{\calQ}]$ is the scattered subword ordering, which is a \kl{WQO} due to Higman's lemma \cite{HIG52}.
\end{example}
%
\begin{example}\label{ex:rado}
Let $\G$ be the Rado graph (\cite[Section 7.3.1]{BOJAN16inf},\cite[Example 2.2.1]{homsurvey}).
Up to isomorphism,
finite induced substructures of $\G$ are finite undirected graphs,
monomials in $\mon[Y]{\G}$ are graphs with vertices labelled with $Y$,
and $\gdivleq[\aut{\G}]$ is the labelled induced subgraph ordering even when $Y$ is a singleton.
For example, cycles of length more than three form an infinite antichain.
\end{example}
%
\begin{example}\label{ex:bit vector}
Let $\V$ be an infinite dimensional vector space over $\ftwo$.
Up to isomorphism,
finite induced substructures of $\V$ are finite dimensional vector spaces over $\ftwo$.
These are well-quasi-ordered in the absence of labelling.
However, even when $Y = \N$,
$(\mon[Y]{\V},\gdivleq[\aut{\V}])$ is not a \kl{WQO} as illustrated by the following antichain.
Let $\{v_1,v_2,\dots\}\subseteq \V$ be a countable set of linearly independent vectors in $\V$.
Let $\oplus$ denote the addition operation of $\V$.
For $n \geq 3$ define the monomial 
$
\monelt[p]_n \defined v^2_1 \cdot \ldots \cdot v^2_n \cdot (v_1\oplus v_2) \cdot (v_2\oplus v_3) \ldots \cdot (v_{n-1}\oplus v_n) \cdot (v_{n}\oplus v_1)
$.
Then, $\setof{\monelt[p]_n}{n = 3,4,\dots}$ forms an infinite antichain.
\end{example}
%
\begin{example}\label{ex:dense tree}
Let $\T$ denote the universal countable dense meet-tree
\cite[Page 2]{KRS21}\cite[Section 7.3.3]{BOJAN16inf}.
Note that the tree structure is given by the \intro{meet} operation (and not by edges),
which takes as input a pair of nodes of $\T$ and returns their closest common ancestor.
For a subset $S\subset \T$,
define its \intro{closure} to be the smallest subtree of $\T$ containing $S$.
Note that the \kl{closure} of a finite subset is finite. 
Up to isomorphism, finite induced substructures of $\calQ$ are finite meet-trees.
monomials in $\mon[Y]{\calQ}$ are finite meet-trees labelled with $1 + Y$.
Here $1 + Y$ is the \kl{WQO} containing one more element than $Y$ which is incomparable to elements in $Y$,
and is used to label nodes that are in the \kl{closure} of the set of variable of a monomial, but not in the monomial itself.
The relation $\gdivleq[\aut{\T}]$ is the embedding of labelled meet-trees,
which is a \kl{WQO} due to Kruskal's tree theorem \cite{Kruskal60}.
\end{example}
%
Even though the above examples using \kl{homogeneous} 
structures nicely illustrate the correspondance between 
monomials and labelled finite substructures, 
we can also consider \kl{non-homogeneous} structures,
such as in \Cref{ex:int} below.
%
\begin{example}\label{ex:int}
Let $\calZ$ be the set of integers ordered by the usual ordering.
Then $\aut{\calZ}$ is the set of all order preserving bijections of $\D$.
Note that every order preserving bijection of the set $\calZ$ is a translation $n \mapsto n + c$ for some constant $c\in\calZ$.
By definition, the action $\aut{\calZ} \actson \calZ$ preserves the linear order on $\Z$.
However, $(\mon[Y]{\calZ}, \gdivleq[\aut{\calZ}])$ is not a \kl{WQO} even when $Y$ is a singleton.
An example of an infinite antichain is the set $\setof{a b}{b\in\calZ\setminus\{a\}}$, for any fixed $a\in\calZ$.
\end{example}
%
\paragraph{On reducts of structures.} \AP Let $\sigma$ be a finite relational
signature, and $\tau \subseteq \sigma$ be another finite relational signature.
Let $\mathbb{A}, \mathbb{B}$ be respectively a $\sigma$ and a $\tau$ structure. We say
that \intro{$\mathbb{B}$ is a reduct of $\mathbb{A}$} when $\mathbb{B}$ is
obtained from $\mathbb{A}$ by keeping the same universe, and relations. It was noted by \cite[Lemma 13]{GHOLAS24} that in
this case, the \kl{equivariant Hilbert basis property} transfers from
$\mathbb{A}$ to $\mathbb{B}$. Let us briefly argue that this transfer holds too
for our \cref{thm:decide-equiv-ideal-mem,thm:compute-egb}.

\arka{The next lemma should be rewritten, right?}
\todo[inline]{yes, it should be rephrased}

\begin{lemma}\label{lem:reducts-equiv-hilbert}
  Let $\mathbb{A}$ be a relational structure, let $\mathbb{B}$ be a 
  \kl(struct){reduct} of $\mathbb{A}$. Then, if $\mathbb{A}$ satisfies the
  hypotheses of \cref{thm:decide-equiv-ideal-mem},
  then one can decide the \kl{equivariant ideal membership problem} for
  $\poly{\K}{\mathbb{B}}$. Similarly, 
  if $\mathbb{A}$ satisfies the hypotheses of
  \cref{thm:compute-egb}, then one can compute an
  \kl{equivariant Gröbner basis} of an
  \kl{equivariant ideal} of $\poly{\K}{\mathbb{B}}$.
\end{lemma}
\begin{proof}
  \todo[inline]{Just write it, and it works.}
\end{proof}

\AP 
As a consequence, one can apply our results to structures that are not equipped 
with an ordering, because one can always consider the 

\todo[inline]{Talk about $\N$ I guess.}

\paragraph{Computable oligomorphicity}
%
The structures mentioned in \Cref{ex:eq atoms,ex:dlo,ex:int,ex:rado,ex:bit vector,ex:dense trr} are \kl{homogenous} over a finite signature and every and hence \kl{oligomorphic} ().
%

\subsection{Applications}


\paragraph{Polynomial computations.} \AP The fact that (finite control) systems
performing polynomial computations can be verified follows from the theory of
\kl{Gröbner bases} on finitely many indeterminates \cite{MULSEI02,BEDUSHWO17}.
There were also numerous applications to automata theory, such as deciding
whether a weighted automaton could be determinised (resp. desambiguated)
\cite{BESM23,PUSM24}. We refer the readers to a nice survey recapitulating the
successes of the so-called ``Hilbert method'' automata theory \cite{BOJAN19}. A
natural consequence of the effective computations of \kl{equivariant Gröbner
bases} is that one can apply the same decision techniques to \emph{orbit finite
polynomial computations}. For simplicity and clarity, we will focus on
\kl{polynomial automata} without states or zero-tests \cite{BEDUSHWO17}, but
the same reasoning would apply to more general systems as we will discuss in
\cref{rem:topological-wsts}.


\AP Before discussing the case of orbit finite polynomial automata, let us
recall in \cref{ex:polynomial-automata} the setting of \kl{polynomial automata}
in the classical case, as studied by \cite{BEDUSHWO17}, with techniques that
dates back to \cite{MULSEI02}. A \intro{polynomial automaton} is a tuple $A
\defined (Q, \Sigma, \delta, q_0, F)$, where $Q = \K^n$ for some finite $n \in
\N$, $\Sigma$ is a finite alphabet, $\delta \colon Q \times \Sigma \to Q$ is a
transition function such that $\delta(\cdot,a)_i$ is a polynomial in the
indeterminates $q_1, \dots, q_n$ for every $a \in \Sigma$ and every $i \in
\set{1, \dots, n}$, $q_0 \in Q$ is the initial state, and $F \colon Q \to \K$
is a polynomial function describing the final result of the automaton. The
\intro{zeroness problem for polynomial automata} is the following decision
problem: given a \kl{polynomial automaton} $A$, is it true that for all words
$w \in \Sigma^*$, the polynomial $F(\delta^*(q_0, w))$ is zero? It is known
that the \kl{zeroness problem for polynomial automata} is decidable
\cite{BEDUSHWO17}, using the theory of \kl{Gröbner bases} on finitely many
indeterminates. Let us now propose a new model of computation called \kl{orbit
finite polynomial automata}, and prove an analogue decidability result.

\AP Let us fix a group $\group$ that acts on the set of indeterminates
$\Indets$, and on an alphabet $\Sigma$ in an \kl{effectively oligomorphic}
fashion. We write $\K^{(\Indets)}$ for the set of finitely supported functions
from $\Indets$ to $\K$, i.e., the set of functions $f \colon \Indets \to \K$
such that there exists a finite set $S \subseteq \Indets$ such that $f(x) = 0$
for every $x \notin S$. Given an element $f \in \K^{(\Indets)}$, and given a
polynomial $p \in \poly{\K}{\Indets}$, we write $p(f)$ for the evaluation of
$p$ on $f$.

\todo[inline]{for arka: we never defined finitely supported functions.}

\begin{definition}
  \label{def:orbit-finite-polynomial-automaton}
  An \reintro{orbit finite polynomial
  automaton} is a tuple $A \defined (Q, \Sigma, \delta, q_0, F)$, where $Q =
  \K^{(\Indets)}$, $\Sigma$ is an \kl{orbit finite} alphabet, $\delta \colon
  \Sigma \to (\Indets \to \poly{\K}{\Indets})$ is a \kl(func){finitely supported}
  polynomial update function, and $F \in \poly{\K}{\Indets}$ is a polynomial
  computing the result of the automaton. 

  Given a letter $a \in \Sigma$ and a
  state $q \in Q$, the update $\delta^*(q,a)$ is defined as the function from
  $\Indets$ to $\K$ defined by $\delta^*(q,a) \colon x \mapsto \delta(a,x)[ q ]$,
  which is well-defined because $\delta(a,x)$ is a \kl{finitely supported}
  polynomial. The update function is naturally extended to words. Finally, the
  output of an \kl{orbit finite polynomial automaton} on a word $w \in \Sigma^*$
  is defined as $F(\delta^*(q_0, w))$.
\end{definition}


\begin{example}
  \label{ex:orbit-finite-polynomial-automata}
  Let $\Indets = \calQ$, and let $\group$ be the group of all
  order-preserving bijections of $\calQ$.
  Let $\Sigma \defined \calQ \times \calQ$.
  Then, the following function are computable by 
  \kl{orbit finite polynomial automata}:
  the number $\mathrm{inc}(w)$ of letters $(a,b)$ such that $a < b$ in a word $w \in \Sigma^*$,
  the number $\mathrm{dec}(w)$ of letters $(a,b)$ such that $a > b$ in a word $w \in \Sigma^*$,
  and the number $(\mathrm{inc}(w) - \mathrm{dec}(w))^2$.
\end{example}

\AP As for \kl{polynomial automata}, the \intro(ofpa){zeroness problem} for
orbit finite polynomial automata is the following decision problem: decide if
for every input word $w$, the output $F(\delta^*(q_0, w))$ is zero. Solving the
\kl(ofpa){zeroness problem} for orbit finite polynomial automata allows us to
decide the equality of two such automata, by computing their difference. Let us
prove that the \kl(ofpa){zeroness problem} is decidable for \kl{orbit finite
polynomial automata}.

\todo[inline]{the proof of the following is incorrect and I do not know
  how to fix it easily}

\begin{theorem}
  \label{cor:orbit-finite-polynomial-automata-zeroness}
  Let $\Indets$ be a set of indeterminates that satisfies the
  \kl{computability assumptions} and such that $(\mon[Y]{\Indets}, \gdivleq)$ is a
  \kl{well-quasi-ordering}, for every \kl{well-quasi-ordered} set $(Y, \leq)$.
  Then, the \kl(ofpa){zeroness problem} is decidable for all \kl{orbit finite polynomial automata}
  with register names in $\Indets$, for every \kl{orbit finite} alphabet $\Sigma$,
  that is \kl{effectively oligomorphic} with respect to the action of $\group$.
\end{theorem}
\begin{proof}
  Let $A = (Q, \Sigma, \delta, q_0, F)$ be an \kl{orbit finite polynomial
  automaton}. Following the classical \emph{backward procedure} for such
  systems, we will compute a sequence of sets $E_0 \defined \setof{ q \in Q }{
  F(q) = 0 }$, and $E_{i+1} \defined \mathrm{pre}^\forall(E_i) \cap E_i$, where
  $\mathrm{pre}^\forall(E)$ is the set of states $q \in Q$ such that for every
  $a \in \Sigma$, $\delta^*(q,a) \in E$. We will prove that the sequence of
  sets $E_i$ stabilises, and that it is computable. As an immediate
  consequence, it suffices to check that $q_0 \in E_{\infty}$, where $E_\infty$
  is the limit of the sequence $(E_i)_{i \in \N}$, to decide the
  \kl(ofpa){zeroness problem}.

  The only idea of the proof is to notice that all the sets $E_i$ are
  representable as zero-sets of \kl{equivariant ideals} in
  $\poly{\K}{\Indets}$, allowing us to leverage the effective computations of
  \cref{cor:equivariant-ideals-computations}. Given a set $H$ of polynomials,
  we write $\mathcal{V}(H)$ the collections of states $q \in Q$ such that $p(q)
  = 0$ for all $p \in H$.
  It is easy to see that $E_0 = \mathcal{V}(\set{F}) = \mathcal{V}(\idl_0)$,
  where $\idl_0$ is the \kl{equivariant ideal} generated by $F$. 
  Furthermore, assuming that $E_i = \mathcal{V}(\idl_i)$, we can
  see that 
  \begin{align*}
    \mathrm{pre}^\forall(E_i) 
    & = \setof{ q \in Q }{ \forall a \in \Sigma, \delta^*(q,a) \in E_i } \\
    & = \setof{ q \in Q }{ \forall a \in \Sigma, \forall p \in \idl_i, p(\delta^*(q,a)) = 0 } \\
    & = \setof{ q \in Q }{ \forall p' \in \idl[J], p'(q) = 0 }
  \end{align*}
  Where, the \kl{equivariant ideal} $\idl[J]$ is generated by the
  polynomials $\mathrm{pullback}(p,a) \defined p [ x \mapsto \delta(a)(x)]$
  for every pair $(p, a) \in \idl_i \times \Sigma$. 
  As a consequence, we have $E_{i+1} = \mathcal{V}(\idl_{i+1})$, where
  $\idl_{i+1} = \idl_i + \idl[J]$.

  Because the sequence $\seqof{ \idl_i }[ i \in \N]$ is increasing, and thanks
  to the \kl{equivariant Hilbert basis property} of $\poly{\K}{\Indets}$, there
  exists an $n_0 \in \N$ such that $\idl_{n_0} = \idl_{n_0 + 1} = \idl_{n_0 +
  2} = \cdots$. In particular, we do have $E_{n_0} = E_{n_0 + 1} = E_{n_0 + 2}
  = \cdots$.

  Let us argue that we can compute the sequence $\idl_i$ effectively.
  First,  $\idl_0 = \EqIdlGen{F}$ is finitely represented.
  Now, 
  given an \kl{equivariant ideal} $\idl$, represented by an \kl{orbit finite}
  set of generators $H$,
  we can compute the \kl{equivariant ideal} $\idl[J]$ generated by the
  polynomials $\mathrm{pullback}(p,a) \defined p [ x_i \mapsto \delta(a)(x_i)]$
  for every pair $(p, a) \in H \times \Sigma$. Indeed, $H \times \Sigma$ is
  \kl{orbit finite} because the action of $\group$ on $\Indets$ is
  \kl{effectively oligomorphic}, and the function $\mathrm{pullback}$ is
  computable and \kl(func){equivariant}: indeed, given $\gelem \in \group$, we can
  show that
  \begin{align*}
    \gelem \cdot \mathrm{pullback}(p, a) & = 
    \gelem \cdot (p [ x_i \mapsto \delta(a)(x_i)]) \\
    & = p [ x_i \mapsto (\gelem \cdot \delta(a, x_i))] \\
    & = p [ x_i \mapsto \delta(\gelem \cdot a, \gelem \cdot x_i))] \\
    & = (\gelem \cdot p) [ x_i \mapsto \delta(\gelem \cdot a, x_i)] \\
    & = \mathrm{pullback}(\gelem \cdot p, \gelem \cdot a).
  \end{align*}
  
  Finally, one can detect when the sequence stabilises, by checking whether
  $\idl_i = \idl_{i+1}$, which is decidable because the
  \kl{equivariant ideal membership problem} is decidable 
  by \cref{thm:compute-egb}.

  To conclude, it remains to check whether $q_0 \in E_\infty$,
  which amounts to check that $q_0 \in \mathcal{V}(\idl_\infty)$.
  This is equivalent to checking whether for every element $p \in \Basis$
  where $\Basis$ is an \kl{equivariant Gröbner basis} of $\idl_\infty$, we have
  $p(q_0) = 0$, which can be done by enumerating relevant orbits.
\end{proof}


\begin{remark}
  \label{rem:topological-wsts}
  The notion of
  \intro{topological well-structured transition system} was introduced by
  Goubault-Larrecq in \cite{JGL07}, noticing that the pre-existing notion of
  \kl{Noetherian space} could serve as a topological generalisation of
  \kl{Noetherian rings} (where ideal-based method can be applied),
  and 
  \kl{well-quasi-orderings}, for which the celebrated decision procedures on
  \kl{well-structured transition systems} can be applied \cite{ABDU96}. In particular,
  Goubault-Larrecq used such systems to verify properties of \emph{polynomial
  programs} computing over the complex numbers, that can communicate over lossy
  channels using a finite alphabet \cite{JGL10}. 
  Because of \cref{cor:equivariant-ideals-computations}, we do have an 
  effective way to compute on the topological spaces at hand, 
  and therefore we can apply the theory of
  \kl{topological well-structured transition systems} to verify systems
  such as \emph{orbit finite polynomial automata communicating using a finite alphabet
  over lossy channels}.
  We refer to \cite[Chapter 9]{JGL13} for a survey on the theory of 
  Noetherian spaces.
\end{remark}

\paragraph{Reachability problem of symmetric data Petri nets.}
%
Define a Petri net with data $\group\actson\Indets$ to be a Petri nets which uses elements of $\Indets$ to label its tokens, and its transitions are invariant under the action of $\group$.
As observed in \cite[Section 8]{GHOLAS24},
a reversible data Petri nets are same as a monomial rewrite system.
Using \Cref{thm:compute-egb}, \Cref{lem:reducts-equiv-hilbert} and \cite[Theorem 64]{GHOLAS24} we can get the following corollary:
%
\begin{corollary}
The reachability problem is decidable for reversible Petri net with data $\group\actson\Indets$ if it is $\group\actson\Indets$ is a reduct of $\calH\actson\Indets$ such that ...
\end{corollary}
%
\paragraph{Orbit-finite systems of equations}

\todo[inline]{for arka: do it}