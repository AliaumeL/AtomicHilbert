% LTeX: language=en
%%!TEX root = ../atomic.asmart.tex
%
\section{Relation to Existing Results and Examples}
\label{sec:examples}
%In this section, we are interested in the consequences of our decidability
%results. First, we will provide numerous examples of sets of indeterminates
%that satisfy our \kl{computability assumptions} as well as our
%\kl{well-quasi-ordering} conditions. Then, we will discuss how our results can
%be applied to solve various decidability problems in theoretical computer
%science.
%
%\todo[inline]{for arka: integrate these examples}
%\begin{example}
%  \label{ex:q-is-super-wqo}
%  The set $(\mon[Y]{\Indets_\calQ}, \gdivleq)$ is a \kl{well-quasi-ordering} whenever $Y$ is
%  one, and in particular $\Indets_\calQ$ satisfies the computability assumptions and
%  the termination assumptions of both \cref{thm:compute-egb}
%  and
%  \cref{thm:decide-equiv-ideal-mem}.
%\end{example}
%\begin{proof}
%  Let $\seqof{\monelt_i}[i \in \N]$ be a sequence of monomials in
%  $\mon[Y]{\Indets_\calQ}$. Let us write each monomial $\monelt_i$ as
%  a finite word $w_i$ over the alphabet $Y$, by writing all the exponents in the order 
%  prescribed by the indeterminates.
%  Because $Y$ is a \kl{well-quasi-ordering}, the set of all finite words over $Y$ is
%  \kl{well-quasi-ordered} by the \emph{scattered subword} relation \cite{HIG52}.
%  Now, if $w_i$ is a scattered subword of $w_j$, then
%  $\monelt_i \gdivleq \monelt_j$, by choosing a suitable $\gelem \in \group$.
%\end{proof}
%
%\begin{example}
%  \label{ex:z-is-not-wqo}
%  The set $(\mon[Y]{\Indets_\Z}, \gdivleq)$ is not a \kl{well-quasi-ordering} whenever $Y$ is
%  contains two distinct elements.
%  In particular, $\Indets_\Z$ does not satisfy the termination assumptions of
%  \cref{thm:compute-egb} and \cref{thm:decide-equiv-ideal-mem}.
%\end{example}
%\begin{proof}
%  Assume that $Y$ has two distinct elements $a$ and $b$, and let us assume without loss of generality
%  that $a \leq b$. The sequence of monomials 
%  $\monelt_i \defined x_1^{b} x_2^{a} \cdots x_{i-1}^{a} x_i^{b}$
%  forms an infinite antichain in $(\mon[Y]{\Indets_\Z}, \gdivleq)$.
%  Indeed, if $\monelt_i \gdivleq \monelt_j$ for some $i < j$, then
%  without loss of generality, $\gelem_i (x_1) = x_1$, and 
%  $\gelem_i (x_i) = x_j$, because these are the only ones that can be 
%  equipped with a large enough exponent.
%  Therefore, $\gelem_i = \mathrm{id}$ since the group only contains translations.
%  However, this implies that the exponent of $x_j$ in $\monelt_j$ is at most $a$,
%  which contradicts the fact that it is $b$.
%\end{proof}

\subsection{Crafting sets of indeterminates}
%
Many of the common examples of group actions $\group\actson\Indets$ are of the form where $\Indets$ is a set with some structure, described by some relations and functions on that set,
and $\group$ is the group $\aut{\Indets}$ of all automorphisms (i.e.\ bijections that preserve and reflect the structure) of $\Indets$.
A monomial $\monelt[p] \in \mon[Q]{\Indets}$ can be thought as a labelling of a finite substructure of $\X$ using elements of $Y$.
If the structure $\Indets$ is \intro{homogenous} :
isomorphism between finite induced substructures extends to an automorphism of the whole structure,
then $\gdivleq$ is same as embedding of labelled finite induced substructures of $\X$.
\footnote{
We refer the reader to \cite[Chapter 7]{BOJAN16inf} and \cite{homsurvey} for more details on homogeneous structures.}
We give some examples of homogeneous structures and discuss whether they satisfy the WQO condition of \Cref{thm:compute-egb} :
$(\mon[Y]{\Indets},\gdivleq[\aut{\Indets}])$ is a \kl{WQO} for every \kl{WQO} $Y$. 
%
\begin{example}\label{ex:eq atoms}
Let $\A$ be an infinite set without any additional structure other than the equality relation.
%Then $\aut{\A}$ is the set of all bijections of the set $\A$.
Up to isomorphism, finite induced substructures of $\A$ are finite sets,
monomials in $\mon[Y]{\A}$ are finite multisets of elements in $Y$,
and $\gdivleq[\aut{\A}]$ is the multiset ordering \cite[Section 1.5]{SCSC17},
which is a \kl{WQO} \cite[Corollary 1.21]{SCSC17}.
\end{example}
%
\begin{example}\label{ex:dlo}
Let $\calQ$ be the set of rational numbers ordered by the usual ordering.
Note that under this ordering, $\calQ$ is a dense linear order without endpoints.
We write $\calQ$ instead of $\Q$ to emphasise that we use its elements as indeterminates and not as coefficients of polynomials. 
Up to isomorphism, finite induced substructures of $\calQ$ are finite linear orders,
monomials in $\mon[Y]{\calQ}$ are words in $Y^*$ (i.e.\ finite linear order labelled with elements of $Y$)
and $\gdivleq[\aut{\calQ}]$ is the scattered subword ordering, which is a \kl{WQO} due to Higman's lemma \cite{HIG52}.
\end{example}
%
\begin{example}\label{ex:rado}
Let $\G$ be the Rado graph (\cite[Section 7.3.1]{BOJAN16inf},\cite[Example 2.2.1]{homsurvey}).
Up to isomorphism,
finite induced substructures of $\G$ are finite undirected graphs,
monomials in $\mon[Y]{\G}$ are graphs with vertices labelled with $Y$,
and $\gdivleq[\aut{\G}]$ is the labelled induced subgraph ordering even when $Y$ is a singleton.
For example, cycles of length more than three form an infinite antichain.
\end{example}
%
\begin{example}\label{ex:bit vector}
Let $\V$ be an infinite dimensional vector space over $\ftwo$.
Up to isomorphism,
finite induced substructures of $\V$ are finite dimensional vector spaces over $\ftwo$.
These are well-quasi-ordered in the absence of labelling.
However, even when $Y = \N$,
$(\mon[Y]{\V},\gdivleq[\aut{\V}])$ is not a \kl{WQO} as illustrated by the following antichain.
Let $\{v_1,v_2,\dots\}\subseteq \V$ be a countable set of linearly independent vectors in $\V$.
Let $\oplus$ denote the addition operation of $\V$.
For $n \geq 3$ define the monomial 
$
\monelt[p]_n \defined v^2_1 \cdot \ldots \cdot v^2_n \cdot (v_1\oplus v_2) \cdot (v_2\oplus v_3) \ldots \cdot (v_{n-1}\oplus v_n) \cdot (v_{n}\oplus v_1)
$.
Then, $\setof{\monelt[p]_n}{n = 3,4,\dots}$ forms an infinite antichain.
\end{example}
%
\begin{example}\label{ex:dense tree}
Let $\T$ denote the universal countable dense meet-tree
\cite[Page 2]{KRS21}\cite[Section 7.3.3]{BOJAN16inf}.
Note that the tree structure is given by the \intro{meet} operation (and not by edges),
which takes as input a pair of nodes of $\T$ and returns their closest common ancestor.
For a subset $S\subset \T$,
define its \intro{closure} to be the smallest subset of $\T$ that is closed under the \kl{meet} operation.
Note that the \kl{closure} of a finite subset is finite. 
Up to isomorphism, finite induced substructures of $\calQ$ are finite meet-trees.
monomials in $\mon[Y]{\calQ}$ are finite meet-trees labelled with $1 + Y$.
Here $1 + Y$ is the \kl{WQO} containing one more element than $Y$ which is incomparable to elements in $Y$,
and is used to label nodes that are in the \kl{closure} of the set of variable of a monomial, but not in the monomial itself.
The relation $\gdivleq[\aut{\T}]$ is the embedding of labelled meet-trees,
which is a \kl{WQO} due to Kruskal's tree theorem \cite{Kruskal60}.
\end{example}
%
Even though the above examples using \kl{homogeneous} 
structures nicely illustrate the correspondance between 
monomials and labelled finite substructures, 
we can also consider \kl{non-homogeneous} structures,
such as in \Cref{ex:int} below.
%
\begin{example}\label{ex:int}
Let $\calZ$ be the set of integers ordered by the usual ordering.
Then $\aut{\calZ}$ is the set of all order preserving bijections of $\D$.
Note that every order preserving bijection of the set $\calZ$ is a translation $n \mapsto n + c$ for some constant $c\in\calZ$.
By definition, the action $\aut{\calZ} \actson \calZ$ preserves the linear order on $\Z$.
However, $(\mon[Y]{\calZ}, \gdivleq[\aut{\calZ}])$ is not a \kl{WQO} even when $Y$ is a singleton.
An example of an infinite antichain is the set $\setof{a b}{b\in\calZ\setminus\{a\}}$, for any fixed $a\in\calZ$.
\end{example}
%
\paragraph{On reducts of structures.} \AP Let $\sigma$ be a finite relational
signature, and $\tau \subseteq \sigma$ be another finite relational signature.
Let $\mathbb{A}, \mathbb{B}$ be respectively a $\sigma$ and a $\tau$ structure. We say
that \intro{$\mathbb{B}$ is a reduct of $\mathbb{A}$} when $\mathbb{B}$ is
obtained from $\mathbb{A}$ by keeping the same universe, and relations. It was noted by \cite[Lemma 13]{GHOLAS24} that in
this case, the \kl{equivariant Hilbert basis property} transfers from
$\mathbb{A}$ to $\mathbb{B}$. Let us briefly argue that this transfer holds too
for our \cref{thm:decide-equiv-ideal-mem,thm:compute-egb}.

\arka{The next lemma should be rewritten, right?}
\todo[inline]{yes, it should be rephrased}

\begin{lemma}
  \label{lem:reducts-equiv-hilbert}
  Let $\mathbb{A}$ be a relational structure, let $\mathbb{B}$ be a 
  \kl(struct){reduct} of $\mathbb{A}$. Then, if $\mathbb{A}$ satisfies the
  hypotheses of \cref{thm:decide-equiv-ideal-mem},
  then one can decide the \kl{equivariant ideal membership problem} for
  $\poly{\K}{\mathbb{B}}$. Similarly, 
  if $\mathbb{A}$ satisfies the hypotheses of
  \cref{thm:compute-egb}, then one can compute an
  \kl{equivariant Gröbner basis} of an
  \kl{equivariant ideal} of $\poly{\K}{\mathbb{B}}$.
\end{lemma}
\begin{proof}
  \todo[inline]{Just write it, and it works.}
\end{proof}

\AP 
As a consequence, one can apply our results to structures that are not equipped 
with an ordering, because one can always consider the 

\todo[inline]{Talk about $\N$ I guess.}

\paragraph{Computable oligomorphicity}
%
The structures mentioned in \Cref{ex:eq atoms,ex:dlo,ex:int,ex:rado,ex:bit vector,ex:dense trr} are \kl{homogenous} over a finite signature and every and hence \kl{oligomorphic} ().
%

