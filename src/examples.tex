% LTeX: language=en
%%!TEX root = ../atomic.asmart.tex
%
\section{Relation to Existing Results and Examples}
\label{sec:examples}

In this section, we are interested in the consequences of our decidability
results. First, we will provide numerous examples of sets of indeterminates
that satisfy our \kl{computability assumptions} as well as our
\kl{well-quasi-ordering} conditions. Then, we will discuss how our results can
be applied to solve various decidability problems in theoretical computer
science.

\subsection{Crafting sets of indeterminates}


\paragraph{Sets with atoms.}

\todo[inline]{Say that if one starts with atoms and equality, then we can only 
  have dimension 1, and that this is the case of the rationals.}

\paragraph{Relational structures.} Let $\mathbb{A}$ be an infinite relational
structure with finitely many relations. Then, one can consider the set of
polynomials $\poly{\K}{\mathbb{A}}$, where indeterminates are elements of the
universe of $\mathbb{A}$. The group of all automorphisms of $\mathbb{A}$ (i.e.,
bijections of the universe that preserve the relations) acts on
$\poly{\K}{\mathbb{A}}$ by permuting the indeterminates.

Natural examples are polynomials whose indeterminates are indexed by the
natural numbers (with inequality), or the rationals (with inequality). In this
setting, \kl{effective oligomorphcity} means that \todo{do it}. The fact that
$(\mon{\mathbb{A}}, \gdivleq)$ is a well-quasi-ordering corresponds to ordering
\emph{finite substructures} of $\mathbb{A}$ by the \emph{labelled induced
substructure} relation, and asking whether the class obtained is
well-quasi-ordered. This is a well-studied question in graph theory, where a
conjecture of Pouzet states that this holds with two labels if and only if it
holds for every ordinal. In particular, for such structures, it is therefore
conjectured that $(\mon{\Indets}, \gdivleq)$ is a well-quasi-ordering if and
only if $(\mon[\om \ordplus \ordfin{1}]{\Indets}, \gdivleq)$, $(\mon[\om
\ordplus \om]{\Indets}, \gdivleq)$, and $(\mon[\om^2]{\Indets}, \gdivleq)$ are
well-quasi-orderings too.

\subsection{Topological Well-Structured Transition Systems}

The fact that (finite control) systems performing polynomial computations can
be verified folloms from the theory of \kl{Gröbner bases} on finitely many
indeterminates \cite{MULSEI02,BEDUSHWO17}. There were also numerous
applications to automata theory, suchz as deciding whether a weighted automaton
could be determinised (resp. desambiguated) \cite{BESM23,PUSM24}. We refer the
readers to a nice survey recapitulating the successes of the so-called
``Hilbert method'' automata theory \cite{BOJAN19}. In this section, we will
show how our results can be used in the framework of \kl{topological
well-structured transition systems}, that is an abstract setting to verify
infinite state systems that is more general than computing polynomial
invariants \cite{JGL07,JGL10}. As a consequence, we will show how our results
can be used to decide the zeroness of \kl{orbit finite polynomial automata}, a
new model of computation that generalizes \kl{polynomial automata}
\cite{BEDUSHWO17} and \kl{orbit finite weighted automata} \cite{BOKLMO21}.

\paragraph{Topological Well-Structured Transition Systems.} \AP The notion of
\kl{topological well-structured transition system} was introduced by
Goubault-Larrecq in \cite{JGL07}, noticing that the pre-existing notion of
\kl{Noetherian space} could serve as a topological generalisation of
\kl{well-quasi-orderings}, for which the celebrated decision procedures on
\kl{well-structured transition systems} can be applied. In particular,
Goubault-Larrecq used such systems to verify properties of \emph{polynomial
programs} computing over the rationals, that can communicate over lossy
channels using a finite alphabet \cite{JGL10}. 
\AP A \intro{topological space} is a set $X$ equipped with a collection $\tau$
of subsets of $X$ that is stable under finite intersections and arbitrary
unions.\footnote{In particular, $\tau$ contains the empty set and $X$ itself.}
In a \kl{topological space}, elements of $\tau$ are called \intro{open
subsets}, while their complements (in $X$) are called \intro{closed subsets}. A
\kl{topological space} is \intro(space){Noetherian} when, for every sequence
$\seqof{U_i}[i \in \N]$ of \kl{open subsets}, there exists $n \in \N$ such that
$\bigcup_{i \in \N} U_i = \bigcup_{i \leq n} U_i$. We refer the readers to the
book \cite{JGL13} for a comprehensive introduction to \kl{Noetherian spaces}
and their usage in theoretical computer science. Let us briefly argue that
\kl{Noetherian spaces} generalize \kl{well-quasi-orders} in
\cref{ex:well-quasi-orders-are-noeth}, and encode the
\kl{Hilbert basis property} in \cref{ex:polynomials-noetherian}.

\begin{example}[ see \cite{JGL13}]
  \label{ex:well-quasi-orders-are-noeth}
  Let $(X, \leq)$ be a quasi-ordered set.
  Then, the set $X$ equipped with the \kl{topology} having 
  as \kl{open subsets} the upwards-closed subsets of $X$ is \kl(space){Noetherian}
  if and only if $(X, \leq)$ is \kl{well-quasi-ordered}.
\end{example}

\begin{example}[ see \cite{JGL13}]
  \label{ex:polynomials-noetherian}
  Let $\K$ be a field, and let $n \in \N$.
  The space $\K^n$ equipped with the \kl{Zariski topology}
  \kl(space){Noetherian}; where the \intro{Zariski topology}
  is the topology whose \kl{closed subsets} are finite unions of sets
  of the form $\setof{ \vec{x} \in \K^n}{ \forall p \in \idl, p(\vec{x}) = 0}$,
  where $\idl$ is an \kl{ideal} of $\poly{\K}{x_1, \dots, x_n}$.
\end{example}

\AP The advantage of \kl{Noetherian spaces} over \kl{well-quasi-orderings} and
\kl{Noetherian rings} is that they generalize both and can be \emph{combined}:
\kl{Noetherian spaces} are closed under finite sums, finite products,
considering finite words, considering finite trees, and many more \todo{cite}.
As a consequence, they provide a versatile tool to express the set of states of
a system, ensuring that a strong termination property holds.

\AP A \intro{topological well-structured transition system} with alphabet
$\Sigma$ is a \kl{topological space} $(X, \tau)$, equipped with a transition
function $\delta \colon X \times \Sigma \to X$, such that the following
properties hold: for every $U \in \tau$, $\mathrm{pre}^\exists(U)$, the set of
states $x \in X$ such that there exists $a \in \Sigma$ with $\delta(x, a) \in
U$, is an \kl{open subset}. Equivalently, the set $\mathrm{pre}^\forall(E)$ of
states $x \in X$ such that for every $a \in \Sigma$, $\delta(x, a) \in E$ is a
\kl{closed subset} of $X$ whenever $E$ is itself a \kl{closed subset} of $X$.
The natural decition problem for \kl{topological well-structured transition
systems} is the following \intro{open reachability problem} is decidable: given
an initial state $x_0 \in X$ and an \kl{open subset} $U \in \tau$, is it true that
there exists a word $w \in \Sigma^*$ such that $\delta^*(x_0, w) \in U$? The
prototypical algorithm to solve this problem is the following \intro{backward
algorithm}: start with $U_0 \defined U$, and iteratively compute $U_{i+1}
\defined U_i \cup \mathrm{pre}^\exists(U_i)$ until $U_i = U_{i+1}$, then check
whether $x_0 \in U_\text{last}$.
There are easy-to-state sufficient conditions  for such an algorithm to be computable and terminate:
\begin{enumerate}
  \item One is equipped with an effective representation of open subsets,
    where one is able to test equality of open subsets, compute unions of open subsets, and test 
    membership of a point in an open subset.
  \item The pre-image function $\mathrm{pre}^\exists$ is computable, i.e., one can
    compute the set $\mathrm{pre}^\exists(U)$ for every open subset $U$.
  \item The space $(X, \tau)$ is \kl{Noetherian}. 
\end{enumerate}

\AP Our \cref{cor:equivariant-ideals-computations} shows that
under some assumptions on $\Indets$, the set of finitely supported functions
$\Indets \to \K$ is a \kl{Noetherian space} with respect to the
\intro{equivariant Zariski topology}, i.e., the topology whose \kl{closed subsets}
are finite unions of sets of the form $E_{\idl} \defined \setof{f \in
\K^{(\Indets)}}{\forall p \in \idl, p(f) = 0}$, where $\idl$ is an
\kl{equivariant ideal} of $\poly{\K}{\Indets}$. Furthermore, we have an
effective representation of the \kl{closed subsets} in this topology, using
\kl{equivariant Gröbner bases} of \kl{equivariant ideals}. In particular, the
theory of \kl{topological well-structured transition systems} can be applied to
systems whose state space contains ``named registers'' that contain numbers and
are updated by polynomial functions.


\paragraph{Consequences for orbit-finite polynomial automata.} Before
discussing the case of orbit finite polynomial automata, let us recall in
\cref{ex:polynomial-automata}
the
setting of \kl{polynomial automata} in the classical case, as studied by
\cite{BEDUSHWO17}, with techniques that dates back to \cite{MULSEI02}. TWe will
formally state in \cref{lem:zeroness-problem-polynomial-automata}
how the classical problem of deciding the \kl(pa){zeroness} of a
\kl{polynomial automaton} is a special case of the \kl{open reachability
problem} for \kl{topological well-structured transition systems}. Beware that
these are consequences of \cite{JGL10}.

\begin{definition}[Polynomial automata, as described in \cite{BEDUSHWO17}]
  \label{ex:polynomial-automata}
  A \intro{polynomial automaton} is a tuple $A \defined (Q, \Sigma, \delta, q_0, F)$,
  where $Q = \K^n$ for some finite $n \in \N$, $\Sigma$ is a finite alphabet,
  $\delta \colon Q \times \Sigma \to Q$ is a transition function such that 
  $\delta(\cdot,a)_i$ is a polynomial in the indeterminates $q_1, \dots, q_n$ for every
  $a \in \Sigma$ and every $i \in \set{1, \dots, n}$, $q_0 \in Q$ is the initial state,
  and $F \colon Q \to \K$ is a polynomial function describing the final result of the 
  automaton.
  The \intro{zeroness problem for polynomial automata} is the following decision problem:
  given a \kl{polynomial automaton} $A$, is it true that 
  for all words $w \in \Sigma^*$, the polynomial $F(\delta^*(q_0, w))$ is zero?
\end{definition}

\begin{lemma}
  \label{lem:zeroness-problem-polynomial-automata}
  The \kl{zeroness problem for polynomial automata} is a special case of the
  \kl{open reachability problem} for \kl{topological well-structured transition systems}.
\end{lemma}
\begin{proof}
  Let $A = (Q, \Sigma, \delta, q_0, F)$ be a \kl{polynomial automaton}.
  We consider the topological space $(Q, \tau)$, where $\tau$ is the
  \kl{Zariski topology} on $\K^n$.
  Let $\idl$ be an \kl{ideal} of $\poly{\K}{x_1,\dots,x_n}$ generated by the polynomials
  $p_1, \dots, p_m$,
  and let $E \defined \setof{q \in Q}{\forall p \in \idl, p(q) = 0}$,
  a \kl{closed subset} of $Q$.
  Then,
  \begin{align*}
    q \in \mathrm{pre}^\forall(E) & \iff 
    \forall a \in \Sigma, \forall p \in \idl, p(\delta(q, a)) = 0 \\
                                  & \iff 
    \forall a \in \Sigma, \forall p \in \idl, p(\delta(q, a)) = 0 \\
                                  & \iff 
                                  \forall p \in \idl[J], p(q) = 0
  \end{align*}
  where $\idl[J] \defined \IdlGen{ \setof{ p_i[ x_i \mapsto \delta(\cdot, a)_i] }{ i \in \set{1, \dots, m}, a \in \Sigma } }$.
  In particular, one can represent \kl{closed subsets} of $Q$ as finite 
  lists of \kl{ideals} using their \kl{Gröbner bases}, and we showed that 
  one can effectively compute the pre-image of \kl{closed subsets} of $Q$
  via $\mathrm{pre}^\forall$ by substituting polynomials.
  In this representation, it is very easy to compute the union 
  of two \kl{closed subsets}, which is simply concatenating the two lists 
  of \kl{ideals} reperesenting them.
  To compute the intersection of two \kl{closed subsets} $E_1$ and $E_2$,
  one can assume without loss of generality that both are represented by a 
  single ideal (i.e., that they are irreducible closed subsets), respectively 
  $\idl_1$ and $\idl_2$.
  Then, an easy computation shows that 
  $E_1 \cap E_2 = \setof{q \in Q}{\forall p \in \idl_1 + \idl_2, p(q) = 0}$,
  where $\idl_1 + \idl_2$ is the sum of the two ideals.
  Whether a point $q \in Q$ is in a \kl{closed subset} $E$ is decidable
  because one can evaluate the generating polynomials on $q$ and check that 
  it is indeed $0$.
  The equality check is more complicated, and can be done by first 
  normalizing the list of ideals so that their intersection is trivial,
  which requires computing the intersection of ideals
  and performing equality checks on the resulting \kl{ideals}.

  As a consequence, it suffices to test the \kl{open reachability problem} for
  the \kl{topological well-structured transition system} $(Q, \tau)$ with the
  initial state $q_0$ and the \kl{open subset} $U = Q \setminus E_\text{final}$,
  where $E_\text{final} \defined \setof{q \in Q}{F(q) = 0}$ is the \kl{closed subset}
  of states where the automaton outputs zero.
\end{proof}

\AP Let us fix a group $\group$ that acts on the set of indeterminates
$\Indets$, and on an alphabet $\Sigma$ in an \kl{effectively oligomorphic}
fashion. Let us now consider the case of \intro{orbit finite polynomial
automata}, that we define as follows: an \reintro{orbit finite polynomial
automaton} is a tuple $A \defined (Q, \Sigma, \delta, q_0, F)$, where $Q =
\K^{(\Indets)}$, $\Sigma$ is an \kl{orbit finite} alphabet, $\delta \colon
\Sigma \to (\Indets \to \poly{\K}{\Indets})$ is a \kl(func){finitely supported}
polynomial update function, and $F \in \poly{\K}{\Indets}$ is a polynomial
computing the result of the automaton. Given a letter $a \in \Sigma$ and a
state $q \in Q$, the update $\delta^*(q,a)$ is defined as the function from
$\Indets$ to $\K$ defined by $\delta^*(q,a) \colon x \mapsto \delta(a,x)[ q ]$,
which is well-defined because $\delta(a,x)$ is a \kl{finitely supported}
polynomial. The update function is naturally extended to words. Finally, the
output of an \kl{orbit finite polynomial automaton} on a word $w \in \Sigma^*$
is defined as $F(\delta^*(q_0, w))$.

\begin{example}
  \label{ex:orbit-finite-polynomial-automata}
  Let $\Indets = \Q$, and let $\group$ be the group of all
  order-preserving bijections of $\Q$.
  Let $\Sigma \defined \Q \times \Q$.
  Then, the following function are computable by 
  \kl{orbit finite polynomial automata}:
  the number $\mathrm{inc}(w)$ of letters $(a,b)$ such that $a < b$ in a word $w \in \Sigma^*$,
  the number $\mathrm{dec}(w)$ of letters $(a,b)$ such that $a > b$ in a word $w \in \Sigma^*$,
  and the number $(\mathrm{inc}(w) - \mathrm{dec}(w))^2$.
\end{example}

\AP As for \kl{polynomial automata}, the \intro(ofpa){zeroness problem} for
orbit finite polynomial automata is the following decision problem: decide if
for every input word $w$, the output $F(\delta^*(q_0, w))$ is zero. Solving the
\kl(ofpa){zeroness problem} for orbit finite polynomial automata allows us to
decide the equality of two such automata, by computing their substraction. 

It follows from the same reasoning as in
\cref{lem:zeroness-problem-polynomial-automata} that the
\kl(ofpa){zeroness problem} for orbit finite polynomial automata reduces to the
\kl{open reachability problem} for \kl{topological well-structured transition
systems} with the \kl{equivariant Zariski topology} on $\K^{(\Indets)}$.

\begin{corollary}
  \label{cor:orbit-finite-polynomial-automata-zeroness}
  Let $Y \defined (\N \ordplus \ordfin{1})^2$ be the set of pairs 
  of potentially infinite natural numbers ordered pointwise.
  Let $\Indets$ be a set of indeterminates that satisfies the
  \kl{computability assumptions} and such that $(\mon[Y]{\Indets}, \gdivleq)$ is a
  \kl{well-quasi-ordering}.
  Then, the \kl(ofpa){zeroness problem} is decidable for all \kl{orbit finite polynomial automata}
  with register names in $\Indets$, for every \kl{orbit finite} alphabet $\Sigma$,
  that is \kl{effectively oligomorphic} with respect to the action of $\group$.
\end{corollary}
\begin{proof}
  It is the same proof  
\end{proof}


\paragraph{Consequences for vector addition systems with states.}


\paragraph{Communicating orbit finite polynomial automata.} Following the lines
of \cite{JGL10}, one can consider the case of communicating orbit finite
polynomial automata, where we have a collection of processes that communicate
letters using lossy channels, and can perform polynomial updates on their local
state. Furthermore, we can add \emph{ifs} and finite state to the automata.
Deciding whether such a system can reach a state where one process fails to
satisfy a given polynomial invariant is a special case of the \kl{open
reachability problem}, and is decidable.
