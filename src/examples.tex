\section{Relation to Existing Results and Examples}
\label{sec:examples}

\paragraph{Sets with atoms.}


\paragraph{Relational structures.} Let $\mathbb{A}$ be an infinite relational
structure with finitely many relations. Then, one can consider the set of
polynomials $\poly{\K}{\mathbb{A}}$, where indeterminates are elements of the
universe of $\mathbb{A}$. The group of all automorphisms of $\mathbb{A}$ (i.e.,
bijections of the universe that preserve the relations) acts on
$\poly{\K}{\mathbb{A}}$ by permuting the indeterminates.

Natural examples are polynomials whose indeterminates are indexed by the
natural numbers (with inequality), or the rationals (with inequality). In this
setting, \kl{effective oligomorphcity} means that \todo{do it}. The fact that
$(\mon{\mathbb{A}}, \gdivleq)$ is a well-quasi-ordering corresponds to ordering
\emph{finite substructures} of $\mathbb{A}$ by the \emph{labelled induced
substructure} relation, and asking whether the class obtained is
well-quasi-ordered. This is a well-studied question in graph theory, where a
conjecture of Pouzet states that this holds with two labels if and only if it
holds for every ordinal. In particular, for such structures, it is therefore
conjectured that $(\mon{\Indets}, \gdivleq)$ is a well-quasi-ordering if and
only if $(\mon[\om \ordplus \ordfin{1}]{\Indets}, \gdivleq)$, $(\mon[\om
\ordplus \om]{\Indets}, \gdivleq)$, and $(\mon[\om^2]{\Indets}, \gdivleq)$ are
well-quasi-orderings too.

\paragraph{Decision procedures.}

\begin{itemize}
  \item zeroness of polynomial automata
  \item topological well-structured transition systems  
  \item VASS?
  \item etc
\end{itemize}
