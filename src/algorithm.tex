% LTeX: language=en
\section{Decision Procedure for the Equivariant Ideal Membership Problem}
\label{sec:algorithm}

\todo[inline]{
    It is known that an equivariant Gröbner basis exists.
}

In \cite{GHOLAS24}, an \intro{equivariant Gröbner basis} 
is a finite set $\Basis$
satisfying:
\begin{equation}
    \label{eq:grob-eq-bas}
    \forall S \subfin \X, 
    \forall p \in \poly{\K}{S},
    p \in \gen{\Basis}{\group}
    \iff
    p \in \gen{\Basis \cap \poly{\K}{S}}
        \quad .
\end{equation}
In particular, once we have an \kl{equivariant Gröbner basis} $\Basis$,
the \kl{equivariant ideal membership problem} is decidable: given a polynomial
$p$, we can check whether $p$ belongs to $\gen{\Basis \cap \poly{\K}{S}}{}$
where $S$ is the set of indeterminates appearing in $p$, which is possible
using classical tools for Gröbner basis computations such as 
the Buchberger algorithm \cite{BUCH76}.

The key idea of this section is to swap quantifiers in \cref{eq:grob-eq-bas}
to obtain a weaker notion of Gröbner basis that is somehow
\emph{local}.  Instead of computing a Gröbner basis that works
``for all $S$'', we will compute, given $S$, a
Gröbner basis that works ``for this $S$'', as formalised hereafter.
\begin{definition}
    \label{def:weak-grob}
    Let $S \subfin \X$ be a finite set of indeterminates.
    A set $\Basis \subseteq \poly{\K}{\X}$ is an \intro{$S$-local Gröbner basis}
    if it satisfies the following property:
    \begin{equation}
        \label{eq:weak-grob}
        \forall p \in \poly{\K}{S},
        p \in \gen{\Basis}{\group}
        \iff
        p \in \gen{\Basis \cap \poly{\K}{S}}
        \quad .
    \end{equation}
    A \reintro{local Gröbner basis} 
    is a set $\Basis$ that is an $S$-local Gröbner basis
    for some finite set $S \subfin \X$.
\end{definition}

The following lemma states that computing \kl{local Gröbner bases} is enough to
decide the \kl{equivariant ideal membership problem}. It is not known whether
one can use the computability of \kl{local Gröbner bases} to compute an
\kl{equivariant Gröbner basis}, even in simple cases.

\begin{lemma}
    Assume that for all finite set $S$ and 
    finitely presented \kl{equivariant ideal} $I$,
    one can compute an 
    \kl{$S$-local Gröbner basis} $\Basis$ of $I$.
    Then, the
    \kl{equivariant ideal membership problem}
    is decidable.
\end{lemma}
\begin{proof}
    Let $p \in \poly{\K}{S}$, 
    and $\Basis$ be an \kl{$S$-local Gröbner basis}.
    To decide whether $p \in \gen{\Basis}{\group}$,
    it suffices to compute $\Basis \cap \poly{\K}{S}$
    and check whether 
    $p \in \gen{\Basis \cap \poly{\K}{S}}{}$, which is possible
    because $S$ is a finite set and one can therefore use 
    the classical ideal membership decision procedure.
\end{proof}

\paragraph{Finitely supported ideal membership problem.}
In order to compute a \kl{local Gröbner basis}, we will first compute
a smaller set of polynomials.
We will consider the subgroup $\group_S$ of $\group$ that fixes
all indeterminates in $S$, that is, the set of all $g \in \group$
such that $g \cdot x = x$ for all $x \in S$.
We say that
an \intro{$S$-supported Gröbner basis} is a set $\Basis$
satisfying:
\begin{equation*}
    \label{eq:supp-grob}
    \forall p \in \poly{\K}{S},
    p \in \gen{\Basis}{\group_S}
    \iff
    p \in \gen{\Basis \cap \poly{\K}{S}}{}
\end{equation*}
Fixing all indeterminates in $S$ will allow us to compute 
such bases using an adaptation of the Buchberger algorithm
\cite{BUCH76}.

\begin{lemma}
    Assume that $\X$ is \kl{effectively oligomorphic}.
    Then,
    from an \kl{$S$-supported Gröbner basis}, we can compute
    an \kl{$S$-local Gröbner basis}.
\end{lemma}
\begin{proof}
    Let $\Basis$ be an \kl{$S$-supported Gröbner basis}, and let $p \in \Basis$
    be a polynomial. We write $p_S \defined \setof{p_i}{1 \leq i \leq k_p}$ the
    set of representatives of \kl{$\group_S$-orbits} inside $\poly{\K}{\X}$
    that are included in the \kl{$\group$-orbit} of $p$. 

    Given a polynomial $p$, $p_S$ is effectively computable because $\X$ is
    \kl{effectively oligomorphic}: one can compute representatives of all the
    \kl{$\group_S$-orbits} of $p$, and then check which ones belong to $O$.

    Let us write $\Basis_S \defined \bigcup_{p \in \Basis} p_S$. It is clear
    that $\gen{\Basis_S}{\group_S} \subseteq \gen{\Basis}{\group}$. Conversely,
    let $q \in \gen{\Basis}{\group}$, then $q = \sum_{1 \leq j \leq \ell} q_j
    \pi(p_j)$, with $p_j \in \Basis$, and $\pi \in \group$. By definition,
    $\pi(p_j)$ is in the \kl{$\group$-orbit} of $p_j$, hence there exists $p_j'
    \in \Basis_S$ such that $\pi(p_j)$ and $p_j'$ are in the same
    \kl{$\group_S$-orbit}. Therefore, $q \in \gen{\Basis_S}{\group_S}$.

    We have proven that 
    \begin{equation*}
        \gen{\Basis}{\group} = \gen{\Basis_S}{\group_S} \quad ,
    \end{equation*}
    hence effectively transformed an \kl{$S$-supported Gröbner basis}
    into an \kl{$S$-local Gröbner basis}.
\end{proof}


\subsection{Existence of $S$-supported Bases}
\label{subsec:weakgb}

In this subsection, we fix $S \subfin \X$, and will work only with the group
$\group_S$ that fixes all indeterminates in $S$. We furthermore assume that
$\X$ is equipped with an (effective) linear order $\order$ that is
\kl{respected by the group action}. Our goal is to prove the existence of an
\kl{$S$-supported basis}, without caring (yet) for effective
procedures. The main technical content of this subsection consists in the
understanding of the action of $\group_S$ on $\poly{\K}{S}$.

\AP Let us recall that for any \kl{group action} $\group \actson \X$ yields a
notion of \kl{divisibility up-to $\group$} that we write $p \gdivleq[\group]
q$. As $\group_S$ is a subgroup of $\group$, the action of $\group$ on $\X$
induces an action of $\group_S$ on $S$, and therefore a notion of divisibility
up-to $\group_S$ that we write $p \gdivleq[\group_S] q$. Remark that a
necessary condition for the existence of \kl{$S$-supported Gröbner bases} is
that the \kl{divisibility up-to $\group_S$} is a \kl{well-quasi-ordering} on
$\mon{S}$. This follows the usual pattern (see \cite{GHOLAS24}). 

\begin{lemma}
    \label{lem:wqo-needed-g-s}
    Assume that every \kl{equivariant ideal} $I$
    for $\group_S$ in $\poly{\K}{\X}$
    has an \kl{$S$-supported Gröbner basis}.
    Then, the \kl{divisibility up-to $\group_S$} is a \kl{well-quasi-ordering}.
\end{lemma}
\begin{proof}[Proof Sketch]
    Assume that there exists an infinite bad sequence 
    $\seqof{m_i}[i \in \N]$ for the \kl{divisibility up-to $\group_S$}.
    Then, let us write $I_i \defined \gen{\setof{m_j}{j \leq i}}{\group_S}$
    for $i \in \N$. By assumption, there exists a \kl{$S$-supported Gröbner basis}
    $\Basis$ of $\bigcup_{i \in \N} I_i$. This means that
    the union stabilises, that is, there exists $N \in \N$ such that
    $I_i = I_N$ for all $i \geq N$. 
    In particular, $m_i \in I_N$ for all $i \geq N$.
    We conclude that $m_{N+1} = \sum_{j \leq N} q_j \pi_j(m_j)$ for some 
    $q_j \in \poly{\K}{\X}$,
    $\pi_j \in \group_S$, and $m_j \in \setof{m_j}{j \leq N}$.
    But this immediately shows that $m_{N+1}$ is \kl{divisible up-to
    $\group_S$} by some $m_j$ for $j \leq N$, which is a contradiction.
\end{proof}

In order to derive the fact that \kl{divisibility up-to $\group_S$} is a
\kl{well-quasi-ordering}, we enforce that the $\mon[\omega+1]{\X}$ is a
\kl{well-quasi-ordering} with respect to the \kl{divisibility up-to $\group$}.

\begin{lemma}
    \label{def:wqo-g-implies-g-s}
    Assume that $\mon[\omega+1]{\X}$ is a \kl{well-quasi-ordering} with respect to
    the \kl{divisibility up-to $\group$}. Then, the \kl{divisibility up-to
    $\group_S$} is a \kl{well-quasi-ordering}.
\end{lemma}
\begin{proof}

    Let $\seqof{m_i}[i \in \N]$ be an infinite sequence of \kl{monomials}. Let
    us write $s_i$ to be the \kl{monomial} obtained by restricting $m_i$ to the
    set of indeterminates in $S$, an $r_i$ be such that $m_i = s_i r_i$. We can
    build $s'$ to be the monomial $\prod_{x \in S} x^{\omega+1}$, which belongs
    to $\mon[\omega+1]{\X}$. Using this monomial, one can build a new sequence
    $\seqof{m_i'}[i \in \N]$ of monomials, defined by $m_i' \defined s' r_i$.

    Because $\mon[\omega+1]{\X}$ is a \kl{well-quasi-ordering} for the
    \kl{divisibility up-to $\group$}, one can assume without loss of generality
    that for all $i < j$, there exists an element $\pi_{i,j} \in \group$ and
    $q_{i,j} \in \mon[\omega+1]{\X}$ such that $q_{i,j} \pi_{i,j}(m_i') =
    m_j'$. Let us first remark that $q_{i,j}$ belongs to $\mon{\X}$, i.e., does
    not use the exponent $\omega+1$. Furthermore, we claim that for all $i <
    j$, $\pi_{i,j}(S) = S$. Indeed, indeterminates of $S$ are exactly the ones
    with exponent $\omega+1$ in the monomials $m_i'$ for all $i \in \N$.

    Using a Ramsey argument, one can assume that $\pi_{i,j}$ acts as the
    identity over $S$. In particular, $\pi_{i,j} \in \group_S$ for all $i < j$.
    Because $\N^{|S|}$ is a \kl{well-quasi-ordering} for the pointwise
    comparison relation, one can also assume that for all $i < j$, $s_i \divleq
    s_j$. In particular, one obtains monomials $p_{i,j} \in \mon{S}$ such that
    $s_i p_{i,j} = s_j$.

    Finally, we have 
    \begin{align*}
        m_j'      &= q_{i,j} \cdot \pi_{i,j}(m_i')
                  = q_{i,j} \cdot \pi_{i,j}(s' r_i) \\
        s' r_j    &= q_{i,j} s' \cdot \pi_{i,j}(r_i) \\
        r_j    &= q_{i,j} \cdot \pi_{i,j}(r_i) \\
        s_j r_j    &= s_j q_{i,j} \cdot \pi_{i,j}(r_i) \\
        m_j    &= p_{i,j} q_{i,j} s_i \cdot \pi_{i,j}(r_i) \\
        m_j    &= p_{i,j} q_{i,j} \cdot \pi_{i,j}(s_i r_i) \\
        m_j    &= p_{i,j} q_{i,j} \cdot \pi_{i,j}(m_i) 
    \end{align*}
    Hence, $m_i \gdivleq[\group_S] m_j$ for all $i < j$, which concludes the proof.
\end{proof}

\AP
Provided that one has a total ordering $\order$ on $\X$ that is 
\kl{respected by the group action} (of $\group$), we can adapt $\order$
by conserving the same relative ordering for elements that are in $S$
(resp. not in $S$) but placing all elements of $S$ \emph{below} all
elements not in $S$:
\begin{equation}
    x \sOrderLt y \iff 
    \begin{cases}
        x \order y & \text{if } x,y \in S \lor x,y \not\in S  \\
        \top       & \text{if } x \in S \land y \not \in S \\
        \bot       & \text{if } x \not\in S \land y \in S \\
    \end{cases}
\end{equation}

\begin{lemma}
    \label{lem:total-ordering}
    The ordering $\sOrderLt$ is a total ordering on $\X$,
    that is \kl{respected by} $\group_S$,
    and is \kl(order){effective} whenever $\order$ is.
\end{lemma}
\begin{proof}
    Because $\order$ is a \kl{total ordering}, 
    so is $\sOrderLt$. 
    Assume that $x \sOrderLt y$, and let $\pi \in \group_S$.
    If $x,y \in S$, then $\pi(x) = x$ and $\pi(y) = y$, hence
    $\pi(x) \sOrderLt \pi(y)$, the same happens if
    both elements are not in $S$. The last case is when
    $x \in S$ and $y \not \in S$, in which case $\pi(x) = x$
    and $\pi(y) \not \in S$ (because $\pi$ is a bijection that \kl{fixes} $S$).

    Finally,
    assuming that one can test equality of elements in $\X$ and that
    $\order$ is \kl{effective}, it is clear that $\sOrderLt$ is \kl{effective}
    too.
\end{proof}

\AP The construction of $\sOrderLeq$ is such that its corresponding
\kl{monomial ordering} $\revlex[S]$ ensures that whenever the \kl{leading
monomial} of a polynomial $p$ belongs to $\mon{S}$, then $p$ belongs to
$\poly{\K}{S}$ (\cref{lem:g-s-lm-in-S}), and that the action of $\group_S$ on
$\poly{\K}{S}$ preserves \kl{leading monomials}, \kl{leading coefficients},
\kl{leading terms}, and \kl{characteristic monomials}
(\cref{lem:g-s-acts-preserves-lead}).

\begin{lemma}
    \label{lem:g-s-acts-preserves-lead}
    Let $\pi \in \group_S$ and $p \in \poly{\K}{\X}$.
    Then, $\lm[S](\pi(p)) = \pi(\lm[S](p))$,
    and similarly for $\cm[S]$, $\lt[S]$, and $\lc[S]$.
\end{lemma}
\begin{proof}
    This follows from the fact that for every pair of monomials
    $m,m' \in \mon{\X}$,
    $m \revlex[S] m'$ if and only if $\pi(m) \revlex[S] \pi(m')$.
\end{proof}

\begin{lemma}
    \label{lem:g-s-lm-in-S}
    Let $p \in \poly{\K}{\X}$,
    be such that $\lm[S](p) \in \mon{S}$.
    Then, $p \in \poly{\K}{S}$.
\end{lemma}
\begin{proof}
    We prove that if $m \revlex[S] m'$ and $m' \in \mon{S}$, then $m \in
    \mon{S}$. Let us order the indeterminates appearing in $m$ and $m'$ $x_1
    \sOrderLt x_2 \sOrderLt \cdots \sOrderLt x_n$. Let us write $m = \prod_{1
    \leq i \leq n} x_i^{\ell_i}$ and $m' = \prod_{1 \leq i \leq n} x_i^{k_i}$.
    Because $m \revlex[S] m'$, there exists $i$ such that $k_i < \ell_i$ and
    $k_j = \ell_j$ for all $i < j$. In particular, $k_i \neq 0$, hence $x_i \in
    S$. As a consequence, $x_j \in S$ for all $j \leq i$ by definition of
    $\sOrderLt$. Assume by contraction that $x_j \not\in S$ for some $j > i$,
    then $k_j = l_j$, but since $x_j \not\in S$, we have $k_j = 0$, hence $l_j
    = 0$, which is absurd because $x_j$ appears in $m$ or $m'$.
\end{proof}

\begin{lemma}
    \label{lem:g-s-gb-is-gb}
    Let $\Basis$ be a finite set such that,
    for every $p \in \gen{\Basis}{\group_S}$,
    there exists $q \in \Basis$
    satisfying $\lm[S](q) \gdivleq[\group_S] \lm[S](q)$.
    Then, $\Basis$ is an \kl{$S$-supported Gröbner basis}.
\end{lemma}
\begin{proof}
    Let $p \in \poly{\K}{S} \cap \gen{\Basis}{\group_S}$.
    Let us remark that $\lm[S](p) \in \mon{S}$,
    and that $\revlex[S]$ is \kl{well-founded} on $\mon{S}$,
    since $S$ is finite. Assume by contradiction that 
    $p \not \in \gen{\Basis \cap \poly{\K}{S}}{}$,
    and that $p$ is a counter-example with a $\revlex[S]$ minimal
    \kl{leading monomial}.

    By construction, there exists $q \in \Basis$ such that $\lm[S](q)
    \gdivleq[\group_S] \lm[S](p)$. This provides a $\pi \in \group_S$ and an $r
    \in \mon{\X}$ such that $\pi(\lm[S](q)) \cdot r = \lm[S](p)$.
    Because $\lm[S](p) \in \mon{S}$, we conclude 
    that $\pi(\lm[S](q)) \in \mon{S}$ and $r \in \mon{S}$.
    Since, $\pi$ \kl{fixes} $S$, we conclude that $\lm[S](q) \in \mon{S}$,
    hence $q \in \poly{\K}{S}$ by \cref{lem:g-s-lm-in-S}.
    Finally, 
    $(p - r \cdot q) \in \gen{\Basis}{\group_S}$,
    has a \kl{leading monomial} that is strictly smaller than $\lm[S](p)$,
    and belongs to $\poly{\K}{S}$, which is a contradiction.
\end{proof}

\subsection{Computing $S$-supported Gröbner Bases}
\label{subsec:weakgbcomp}

\AP In this subsection, we prove that one can compute given a finite set
$\Basis$ of polynomials, a finite set $\Basis_S$ that has the following
property: for every $p \in \gen{\Basis}{\group_S}$, there exists $q \in
\Basis_S$ such that $\lm[S](q) \gdivleq[\group_S] \lm[S](p)$. This will prove
the (effective) existence of an \kl{$S$-supported Gröbner basis}
thanks to \cref{lem:g-s-gb-is-gb}.

\AP The algorithm is a variation of the Buchberger algorithm \cite{BUCH76},
which we will briefly comment on. First, any finite set $\Basis$ of polynomials
can be used to define a rewriting relation $\reducstep{\Basis}$, and its
reflexive transitive closure $\reduc{\Basis}$.
Namely, we write $p \reducstep{\Basis} q$
if there exists $\pi \in \group_S$, $r \in \Basis$ and $s \in \poly{\K}{\X}$
such that:
\begin{itemize}
    \item $q = p - s \cdot \pi(r)$,
    \item $\dom(\pi(r)) \subseteq \dom(p)$,
    \item $\lt[S](\pi(r)) = s \cdot \lt[S](p)$.
\end{itemize}
That is, we can use an element of $\Basis$ (up to $\group_S$),
to cancel the \kl{leading monomial}
of a polynomial $p$, without the introduction of new indeterminates.
We write $p \reduc{\Basis} q$ if there exists a
sequence $p = p_0 \reducstep{\Basis} p_1 \reducstep{\Basis} \cdots
\reducstep{\Basis} p_n = q$. 

The inability to introduce new indeterminates is crucial
to ensure that this rewriting system is terminating.
\begin{lemma}
    \label{lem:finite-remainders}
    Let $p \in \poly{\K}{\X}$,
    the set of polynomials $q$ such that $p \reduc{\Basis} q$
    is finite.
\end{lemma}
\begin{proof}
    Let us first prove that there are no infinite rewriting sequences.
    To that end, let $\seqof{p_i}[i \in \N]$ be a sequence of 
    polynomial such that $p_i \reducstep{\Basis} p_{i+1}$
    for all $i \in \N$.
    Remark that $\lm[S](p_i)$ is a strictly decreasing 
    sequence of \kl{monomials} in $\mon{\dom(p_0)}$
    for $\revlex[S]$. Because $\dom(p_0)$ is finite,
    this is absurd.

    Let us now prove that there are finitely many 
    possible \kl{rewrite steps} for a given polynomial $p$.
    Indeed, such a step $p \reducstep{\Basis} q$ is
    uniquely determined by a polynomial $r \in \Basis$
    and a bijection $\pi \in \group_S$ such that
    $\dom(\pi(r)) \subseteq \dom(p)$. Because $\Basis$ is finite,
    and $\dom(p)$ is finite, there are only finitely many such $\pi(r)$
    to choose from.

    We conclude thanks to König's lemma that there are only finitely
    many polynomials $q$ such that $p \reduc{\Basis} q$.
\end{proof}

\AP As a consequence of \cref{lem:finite-remainders}, starting polynomial $p$,
we can compute the (finite) set of all \kl{reduced polynomials} that can be obtained
from $p$ by rewriting with $\Basis$, that we write $\rem{\Basis}{p}$. 

\begin{lemma}
    \label{lem:rewrite-equivariant}
    Let $p,q \in \poly{\K}{\X}$, and $\pi \in \group_S$.
    Then,
    $p \reducstep{\Basis} q$ if and only if
    $\pi(p) \reducstep{\Basis} \pi(q)$.
    In particular, $\pi(\rem{\Basis}{p}) = \rem{\Basis}{\pi(p)}$,
    i.e., the set of remainders is \kl{equivariant}
    under the action of $\group_S$.
\end{lemma}

\AP It is clear that every element $p$ such that $0 \in \rem{\Basis}{p}$
belongs to $\gen{\Basis}{\group_S}$. The Buchberger algorithm extends $\Basis$
so that the converse holds, that is, every element in $\gen{\Basis}{\group_S}$
has $0$ in its set of remainders. The saturation step consists in adding to
$\Basis$ all the \intro{$\mathsf{D}$-polynomials} 
\begin{equation}
    \label{eq:d-polynomial}
    \spoly{p}{q} \defined
    \frac{\lcm(\lm[S](p), \lm[S](q))}{\lm[S](p)} \cdot p
    - \frac{\lcm(\lm[S](p), \lm[S](q))}{\lm[S](q)} \cdot q
    \quad .
\end{equation}

These polynomials clearly belong to $\gen{\Basis}{\group_S}$.
Given a set $H$, we write $\spolyset(H) \defined \bigcup_{p,q \in H}
\rem{\Basis}{\spoly{p}{q}}$.

\begin{figure}
    \centering
    \begin{algorithm}[H]
        \caption{Computing $S$-supported Gröbner bases}
        \label{alg:weakgb}
        \KwIn{A finite set $\Basis$ of polynomials}
        \KwOut{A finite set $\Basis_S$ such that for every $p \in \gen{\Basis}{\group_S}$,
        there exists $q \in \Basis_S$ such that 
        $\lm[S](q) \gdivleq[\group_S] \lm[S](p)$}
        \Begin{
            $\Basis_S \gets \Basis$\;
            \Repeat{$\Basis_S$ stabilizes}{
                $\Basis_S \gets \Basis_S \cup \spolyset(\Basis_S)$\;
                $\Basis_S \gets \textrm{ orbit representatives of } \Basis_S 
                \textrm{ under } \group_S$\;
            }
            \Return{$\Basis_S$}\;
        }
    \end{algorithm}
    \caption{The algorithm for computing $S$-supported Gröbner bases}
    \label{fig:weakgb}
\end{figure}

\begin{lemma}
    \label{lem:weakgb-terminates}
    The \cref{alg:weakgb} terminates on every input.
\end{lemma}
\begin{proof}
    Let us write $H_0 \defined \Basis$,
    $R_0 \defined H_0$,
    $H_{i+1} \defined H_i \cup \spolyset(R_i)$
    $R_{i+1} \defined \textrm{ orbit representatives of } H_{i+1} \textrm{ under } \group_S$.

    It is clear that $\seqof{H_i}[i \in \N]$ is an increasing sequence, 
    because we assume that computing the orbit representatives is idempotent.


\end{proof}

\begin{lemma}
    \label{lem:weakgb-correct}
    Let $\Basis$ be a finite set of polynomials
    and $\Basis_S$ be the output of \cref{alg:weakgb}.
    Then, for every $p \in \gen{\Basis}{\group_S} \setminus \set{ 0 }$,
    there exists
    a transition $p \reducstep{\Basis_S} p'$.
\end{lemma}
\begin{proof}
    TODO.
\end{proof}

\begin{theorem}
    \label{thm:weakgb}
    Let $\Basis$ be a finite set of polynomials.
    Then, the output of \cref{alg:weakgb} is an \kl{$S$-supported Gröbner basis}.
\end{theorem}
