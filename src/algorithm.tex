% LTeX: language=en
%%!TEX root = ../atomic.asmart.tex
%
\section{Decision Procedure for the Equivariant Ideal Membership Problem}
\label{sec:algorithm}

In this section we provide a decision procedure for the \kl{equivariant ideal
membership problem}, we will leverage the results of the previous section
to compute what we call \kl{$V$-strong equivariant Gröbner bases} for a finite set
$V \subfin \Indets$ of indeterminates.

\begin{definition}
  \label{def:strong-equiv-grob}
  Let $V \subseteq \X$ be a set of indeterminates.
  An orbit finite set $\Basis_V$
  of polynomials is called an \intro{$V$-strong equivariant Gröbner basis}
  of an \kl{equivariant ideal} $\idl$ with respect to $\pmonleq$
  whenever $\Basis_V \cap \poly{\K}{V}$ is a
  \kl{Gröbner basis} of the ideal $\idl \cap \poly{\K}{V}$ with respect to
  $\pmonleq$.
\end{definition}

\AP To compute an \kl{$S$-strong equivariant Gröbner basis} of an
\kl{equivariant ideal} $\idl$, we will compute a \kl{weak equivariant Gröbner
basis} of the ideal $\idl$ but with respect to a new ordering $\varleq_S$ and a
new group action $\group_S \actson \X$ that is the restriction of $\group$ to
the operations that fix the indeterminates in $S$, i.e., such that for all
$\gelem \in \group_S$ and $x \in S$, we have $\gelem(x) = x$. The ordering
$\varleq_S$ is defined as $\varleq$ on $S$ and $\X \setminus S$, and orders all
elements of $S$ below all elements of $\X \setminus S$.
\begin{equation}
  \label{eq:order-S}
  x \varleq_S y \iff
  \begin{cases}
    x \varleq y & \text{if } x,y \in S \lor x,y \not\in S  \\
    \top        & \text{if } x \in S \land y \not\in S \\
    \bot        & \text{if } x \not\in S \land y \in S
  \end{cases}
\end{equation}

\AP Let us first show that if one can compute \kl{weak equivariant Gröbner
bases} in $(\X, \group_S, \varleq_S)$, then one can compute \kl{$S$-strong
equivariant Gröbner bases} in $(\X, \group, \varleq)$. This will crucially rely
on the fact that whenever $q$ has indeterminates smaller (for $\varleq_S$) than
$p \in \poly{\K}{S}$, then $q$ is in $\poly{\K}{S}$ too, because of the choice
to put all elements of $S$ below all elements not in $S$.

\begin{lemma}
  \label{lem:local-to-supported}
  Let $\X$ be a set of indeterminates, $\group$ a group acting
  on $\X$ \kl{effectively oligomorphically}, and $\varleq$ a total ordering
  on $\X$ that is \kl{respected by the group action}.
  Let $S$ be a finite set of indeterminates, and 
  assume that one can compute \kl{weak equivariant Gröbner bases}
  in $(\X, \group_S, \varleq_S)$.
  Then, one can compute an \kl{$S$-strong equivariant Gröbner basis}
  for any finitely presented \kl{equivariant ideal} $\idl$ in $(\X, \group, \varleq)$.
\end{lemma}
\begin{proof}
  Let $H$ be a finite set of polynomials
  that generates the \kl{equivariant ideal} $\idl$.
  We will compute a new finite set $H_S$ of polynomials
  such that $\gen{H}{\group} = \gen{H_S}{\group_S}$.
  To do so, we will use the fact that $\X$ is \kl{effectively oligomorphic}:
  for every polynomial $p \in H$, we can compute a finite set
  $p_S$ of representatives of the \kl{$\group_S$-orbits} of $p$.
  We define $H_S$ to be the union of all the $p_S$ for $p \in H$.

  It is clear that $\EqIdlGen[\group_S]{H_S} \subseteq \EqIdlGen{H}$.
  Conversely, let $p \in \EqIdlGen{H}$, then $p$ is in the \kl{$\group$-orbit}
  of some $p' \in H$. By definition, there exists a $\pi \in \group$ such that
  $p = \pi(p')$. We can compute a representative $p_S$ of the
  \kl{$\group_S$-orbit} of $p'$, and we can compute a $\pi_S \in \group_S$
  such that $\pi_S(p_S) = p$. We conclude that $p$ is in the \kl{$\group_S$-orbit}
  of $p_S$, hence $p \in \EqIdlGen[\group_S]{H_S}$.
  Hence, $\EqIdlGen{H} = \EqIdlGen[\group_S]{H_S}$.


  
  Let us now compute  \kl{weak equivariant Gröbner basis} of
  $\EqIdlGen[\group_S]{H_S}$, that we call $\Basis_S$. Remark that for every
  polynomial $p \in \poly{\K}{S} \cap \EqIdlGen[\group_S]{H_S}$, there exists
  some $q \in \Basis_S$ such that $\lm[S](q) \divleq \lm[S](p)$, and such that
  the variables appearing in $\lm[S](q)$ are all smaller than some variable of
  $\lm[S](p)$. In particular, one concludes that $q$ is in $\poly{\K}{S}$.
  Furthermore, because of the way one defined $\varleq_S$, we have that
  $\lm[S](p) = \lm(p)$ and $\lm[S](q) = \lm(q)$.

  Hence, we conclude that $\Basis_S$ is a \kl{Gröbner basis} of the ideal
  $\EqIdlGen{H} \cap \poly{\K}{S}$, as expected.
  \todo[inline]{this is actually false, we have a problem because 
  we still care about the domains in our order $\pmonleq$, but I think
  we should just handwave this.}
\end{proof}

We will now focus on proving that one can compute \kl{weak equivariant
Gröbner bases} in $(\X, \group_S, \varleq_S)$, under mild assumptions 
on the original group action $\group$.
The first step is to remark that decidability condictions are preserved
by the construction of $\varleq_S$ and $\group_S$.

\begin{lemma}
  \label{lem:decidability-preserved}
  Let $(\X, \group, \leq)$ be a set of indeterminates, a group acting
  \kl{effectively oligomorphically} on $\X$ and an
  \kl{effective total ordering} $\leq$ on $\X$ that is compatible with the
  action of $\group$.
  Then, the set $(\X, \group_S, \varleq_S)$ is a set of indeterminates, a group acting
  \kl{effectively oligomorphically} on $\X$ and an
  \kl{effective total ordering} $\varleq_S$ on $\X$ that is compatible with the
  action of $\group_S$.
\end{lemma}
\begin{proof}
  It is clear that $\varleq_S$ is an \kl{effective total ordering} on $\X$, and that it is
  compatible with the action of $\group_S$.
  Let us now show that $\group_S$ is \kl{effectively oligomorphic}.

  \todo[inline]{what we want to show is that 
    the orbits of tuples of elements in
    $\group_S$ can be computed.
    }
\end{proof}

Then, we will prove that a mild assumption on the monomials allows us to
conclude that $(\mon[\omega]{\X}, \gdivleq[\group_S])$ is a
\kl{well-quasi-ordering}: it suffices to assume that $(\mon[\omega + 1]{\X},
\gdivleq[\group])$ is a \kl{well-quasi-ordering} itself.

\begin{lemma}
  \label{lem:wqo-mon-S}
  Let $\X$ be a set of indeterminates, $\group$ a group acting
  on $\X$ \kl{effectively oligomorphically}, and $\varleq$ a total ordering
  on $\X$ that is \kl{respected by the group action}.
  Assume that $(\mon[\omega + 1]{\X}, \gdivleq[\group])$ is a \kl{well-quasi-ordering}.
  Then, $(\mon[\omega]{\X}, \gdivleq[\group_S])$ is a \kl{well-quasi-ordering}.
\end{lemma}
\begin{proof}
  Let $\seqof{\monelt_i}[i \in \N]$ be an infinite sequence of elements of
  $\mon[\omega]{\X}$. We will show that there exists $i < j$ such that
  $\monelt_i \gdivleq[\group_S] \monelt_j$.
  We will assume without loss of generality that the elements of 
  $\monelt_i$ contain all variables in $S$.

  To that end, let us define the function $h \colon \mon[\omega]{\X} \to
  \mon[\omega + 1]{\X}$ that maps a monomial $m$ to itself but with the
  elements of $S$ having exponent $\omega$.
  Because $(\mon[\omega + 1]{\X}, \gdivleq[\group])$ is a
  \kl{well-quasi-ordering},
  there exists an infinite subsequence of indices $J$ such that 
  for all $i < j$ in $J$, we have $h(\monelt_i) \gdivleq[\group] h(\monelt_j)$.
  By definition, this provides us with a sequence of 
  group elements $\gelem_{i,j} \in \group$ such that
  for all  $i < j$ in $J$, we have
  \begin{equation*}
    \gelem_{i,j}(h(\monelt_i)) \divleq h(\monelt_j)
  \end{equation*}
  In particular, 
  the elements $\gelem_{i,j}$ must send elements of $S$ to elements of $S$.
  Because $S$ is a finite set, one can further refine the extraction
  so that $\gelem_{i,j}$ is the identity fuction on $S$ for all $i < j$ in $J$. 
  Therefore, we have $\gelem_{i,j} \in \group_S$.
  Finally,
  consider the sequence of monomials
  $\monelt[s]_i$ defined as the restriction of $\monelt_i$ to the elements
  of $S$. Since $S$ is finite, $(\mon{S}, \divleq)$ is a \kl{well-quasi-ordering},
  and one can further extract $J$ to assume that for all 
  $i < j$ in $J$, we have $\monelt[s]_i \divleq \monelt[s]_j$.
  Finally,
  we conclude that $\gelem_{i,j}(\monelt_i) \divleq \monelt_j$,
  for all $i < j$ in $J$, and in particular for one such pair.
\end{proof}


We can now conclude with our first main result, which is the
decidability of the \kl{equivariant ideal membership problem}.

\todo[inline]{restate}

