% LTeX: language=en
%%!TEX root = ../atomic.asmart.tex
%
\section{Decision Procedure for the Equivariant Ideal Membership Problem}
\label{sec:refinements}

In the previous
\cref{sec:equivariant-grobner-basis},
we relied on a strong assumption on the set of indeterminates. Let us prove
that one can safely weaken this assumption to obtain a decision procedure for
the \kl{equivariant ideal membership problem}, leaving a very tight gap between
the hypothesis needed for the existence of \kl{equivariant Gröbner bases} and
the decidability of the \kl{equivariant ideal membership problem}.

The key idea of this section is to focus on a finite set $V \subfin \Indets$ of
indeterminates, and to consider what we call \kl{$V$-strong equivariant Gröbner
bases}, that are intuitively good enough to decide the \kl{equivariant ideal
membership problem} for all polynomials in $\poly{\K}{V}$, that is, for all
polynomials that only depend on the indeterminates in $V$.

\begin{definition}
  \label{def:strong-equiv-grob}
  Let $V \subseteq \X$ be a set of indeterminates.
  An orbit finite set $\Basis_V$
  of polynomials is called a \intro{$V$-strong equivariant Gröbner basis}
  of an \kl{equivariant ideal} $\idl$ 
  whenever $\Basis_V \cap \poly{\K}{V}$ is a
  \kl{Gröbner basis} of the ideal $\idl \cap \poly{\K}{V}$.
\end{definition}

Let us note that a set in as \kl{equivariant Gröbner basis}
if and only if it is a \kl{$V$-strong equivariant Gröbner basis} for
every finite set $V \subseteq \Indets$ of indeterminates.

\AP To compute a \kl{$V$-strong equivariant Gröbner basis} of an
\kl{equivariant ideal} $\idl$, we will compute a \kl{weak equivariant Gröbner
basis} of the ideal $\idl$ with respect to a new ordering of the indeterminates
and a new group action. Let us write $\IndetsV[V] \defined V \ordplus (\Indets
\setminus V)$, that is the totally ordered set of indeterminates having the
same elements as $\Indets$, but with the elements of $V$ ordered below all
other elements of $\Indets$. We will define a new group $\group_V$ that acts on
$\IndetsV[V]$ by restricting the group to elements $\gelem$ such that $\gelem
\cdot V = V$. In particular, the action of $\group_V$ on $\IndetsV[V]$ is
\kl(ord){compatible} with the ordering on $\IndetsV[V]$.

Let us call $\iota_V \colon \Indets \to \IndetsV[V]$ the identity map that maps
every element of $\Indets$ to itself. We extend this map as a morphism from
$\poly{\K}{\Indets}$ to $\poly{\K}{\IndetsV[V]}$. Beware that $\iota_V$ is not
an order-preserving map. Similarly, let us introduce $\upsilon_V \colon
\IndetsV[V] \to \Indets$ that maps every element of $\IndetsV[V]$ to itself,
and extends as a morphism from $\poly{\K}{\IndetsV[V]}$ to
$\poly{\K}{\Indets}$. Again, $\upsilon_V$ is not an order-preserving map. Using
these two constructions, we are ready to define the algorithm that computes a
\kl{$V$-strong equivariant Gröbner basis} of an \kl{equivariant ideal}:
$\kl{vsgb}(H, V) \defined \upsilon_V(\kl{weakgb}(\iota_V(H)))$.

Let us proceed similarly as in \cref{sec:equivariant-grobner-basis}
to prove that
$\mathsf{vsgb}$ is a computable \kl{equivariant function} that outputs a
\kl{$V$-strong equivariant Gröbner basis}, whenever $\Indets$ satisfies the
\kl{computability assumptions} and $(\mon[\om \ordplus \ordfin{1}]{\Indets},
\gdivleq[\group])$ is a \kl{well-quasi-ordering}.

\begin{lemma}
  \label{lem:strong-v-gb-algorithm}
  Let $H \subseteq \poly{\K}{\Indets}$,
  and $V \subfin \Indets$ be a finite set of indeterminates.
  Then, $\kl{vsgb}(H,V)$
  is a computable function that calls
  \kl{weakgb} on with valid inputs.
\end{lemma}
\begin{proof}
  For the algorithm to be computable, we will rely on the fact that 
  $\poly{\K}{\IndetsV[V]}$ is \kl{effectively oligomorphic} 
  and that the ordering on $\IndetsV[V]$ is also effective and 
  \kl(ord){compatible} with the action of $\group_V$.
  Then, following the same reasoning as in \cref{lem:colored-hypothesis-sat},
  we conclude that $\iota_V(H)$ is a computable orbit finite set
  of polynomials in $\poly{\K}{\IndetsV[V]}$, and that 
  $\upsilon_V(\kl{weakgb}(\iota_V(H)))$ is also computable and ouputs an 
  \kl{orbit finite set}.

  The only non-trivial check being that the set $(\mon{\IndetsV[V]},
  \gdivleq[\group_V])$ is a \kl{well-quasi-ordering}. To that end, let us
  consider an infinite sequence $\seqof{\monelt_i}[i \in \N]$ of elements of
  $\mon{\IndetsV[V]}$. Let us convert each $\monelt_i$ into a monomial
  $\monelt[n]_i$ in $\mon[\om \ordplus \ordfin{1}]{\Indets}$, by replacing the
  exponents of elements in $V$ by $\om$. Because $(\mon[\om \ordplus
  \ordfin{1}]{\Indets}, \gdivleq[\group])$ is a \kl{well-quasi-ordering}, there
  exists an infinite subsequence $\seqof{\monelt[n]_j}[j \in J]$ such that for
  all $i < j$ in $J$, we have a group element $\gelem_{i,j} \in \group$ such
  that $\gelem_{i,j}(\monelt[n]_i) \divleq \monelt[n]_j$. Let us now consider
  the action of $\gelem_{i,j}$ on the elements of $V$. Because $\gelem_{i,j}$
  must send an element of $V$ (having exponent $\om$ in $\monelt[n]_i$) to an
  element of $V$ (having exponent $\om$ in $\monelt[n]_j$), and since $V$ is a
  finite set, we can without loss of generality extract the subsequence $J$ so
  that $\gelem_{i,j}$ is the identity function on the variables of $V$ that
  appear in the monomials, for all $i < j$ in $J$.
  Now, because $\N^V$ with the pointwise ordering is a \kl{well-quasi-ordering}
  (as a finite product of $\N$ with the usual ordering), there exists $i < j$
  such that $\monelt[m]_i (x) \leq \monelt[m]_j (x)$ for all $x \in V$.
  As a consequence, 
  $\gelem_{i,j} (\monelt_i) \divleq \monelt_j$, and we have shown
  that $(\mon{\IndetsV[V]}, \gdivleq[\group_V])$ is a
  \kl{well-quasi-ordering}.
\end{proof}

\begin{lemma}
  \label{lem:correct-v-strong-gb}
  Let $H$ be an orbit finite set of polynomials in $\poly{\K}{\Indets}$,
  and let $V \subseteq \Indets$ be a finite set of indeterminates.
  Then, $\kl{vsgb}(H,V)$ is a
  \kl{$V$-strong equivariant Gröbner basis} of $\EqIdlGen{H}$.
\end{lemma}
\begin{proof}
  Let $H_V = \iota_V(H)$, $\Basis_V
  = \mathsf{weakgb}(H_V)$, and $\Basis = \upsilon_V(\Basis_V)$.
  Let us first check that $\Basis$ is indeed a
  generating set of the ideal $\EqIdlGen{H}$. It is clear that
  $\upsilon_V(\IdlGen{\iota_V(H)}) = \IdlGen{H}$, because $\upsilon_V$ is a
  morphism of polynomial rings. Furthermore, because $\Basis_S$ is a \kl{weak
  equivariant Gröbner basis} of $\EqIdlGen[\group_V]{H_V}$, we have that
  $\IdlGen{\Basis_S} = \IdlGen{H_S}$, and therefore
  $\upsilon_V(\IdlGen{\Basis_S}) = \IdlGen{H}$.

  Let us prove that $\Basis$ is a \kl{$V$-strong equivariant Gröbner basis} of
  $\EqIdlGen{H}$. Let $p \in \poly{\K}{V} \cap \EqIdlGen{H}$, and let us
  consider a decomposition $\mathfrak{d} = \seqof{a_i \monelt_i h_i}[i \in I]$
  of $p$ such that $p = \sum_{i \in I} a_i \monelt_i h_i$, where $a_i \in \K$,
  $\monelt_i \in \mon{\Indets}$, and $h_i \in H$. Then, $\iota_V(p) = \sum_{i
  \in I} a_i \iota_V(\monelt_i) \iota_V(h_i)$, and $\iota_V(h_i) \in H_V$ for
  all $i \in I$, because $\iota_V$ is a morphism.
  Because $\Basis_S$ is a \kl{weak equivariant Gröbner basis} of
  $\IdlGen{H_V}$, there exists a decomposition
  $\mathfrak{d}'$ of $\iota_V(p)$ such that:
  \begin{equation*}
    \iota_V(p) = \sum_{j \in J} b_j \monelt'_j h'_j
    \quad ,
  \end{equation*}
  where $b_j \in \K$, $\monelt'_j \in \mon{\IndetsV[V]}$, and $h'_j \in
  \Basis_V$, and such that
  $\lm[\IndetsV](\iota_V(p)) = \lmdec[\IndetsV](\mathfrak{d}')$,
  and $\domdec(\mathfrak{d}') \subseteq \domdec(\iota_V(\mathfrak{d}))$.

  Remark that $\lm[\IndetsV](\iota_V(p)) = \iota_V(\lm[\Indets](p)) \in
  \mon{V}$, because all variables of $p$ are in $V$. As a consequence, we know
  that all the variables appearing in $\mathfrak{d}'$ are in $V$, since
  \kl{leading monomials} are computed using the \kl{reverse lexicographic
  ordering}, and variables in $V$ are ordered below all other variables in
  $\IndetsV[V]$. As a consequence, for the rest of the proof, we will assume
  that $\mathfrak{d}'$, $p$ and $\iota_V(p)$ are all in $\poly{\K}{V}$, where
  $V$ equipped with the ordering induced by $\Indets$ (which is the same as the
  one induced by $\IndetsV[V]$).

  Finally, let us consider a term $\monelt_i' h_i'$ appearing in
  $\mathfrak{d}'$, with maximal \kl{leading monomial}. We know that $\lm(p) =
  \lm(\monelt_i' h_i') = \monelt_i' \cdot \lm(h_i')$. In particular, $\lm(h_i')
  \divleq \lm(p)$, and $h_i' \in \poly{\K}{V}$.
  We have shown that $\Basis \cap \poly{\K}{V}$ is a \kl{Gröbner basis} of the
  ideal $\IdlGen{H} \cap \poly{\K}{V}$,
  i.e., that $\Basis$ is a \kl{$V$-strong equivariant Gröbner basis} of
  $\EqIdlGen{H}$.
\end{proof}


As a consequence, we have proven that \cref{thm:decide-equiv-ideal-mem}
holds
using the algorithm \kl{vsgb}. Furthermore, notice that the assumption of
\cref{thm:decide-equiv-ideal-mem}
is
that $(\mon[\om \ordplus \ordfin{1}]{\Indets}, \gdivleq[\group])$ is a
\kl{well-quasi-ordering}, which very close to the necessary condition that
$(\mon{\Indets}, \gdivleq[\group])$ being a \kl{well-quasi-ordering}, and is
weaker than the assumption used in
\cref{thm:compute-egb}.
In most applications manipulating \kl{equivariant ideals}, one is mostly
interested in computing \emph{inclusion queries}, or the \emph{union} of two
\kl{equivariant ideals}, all of which can be done by representing such ideals
as generated by an \kl{orbit finite set} of polynomials, and performing
finitely many \kl{equivariant ideal membership queries}. However, to compute
\emph{intersections} of \kl{equivariant ideals}, the known methods require the
use of \kl{Gröbner bases} (see
\cref{cor:equivariant-ideals-computations}).
