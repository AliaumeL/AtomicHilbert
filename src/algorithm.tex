% LTeX: language=en
%%!TEX root = ../atomic.asmart.tex
%
\section{Decision Procedure for the Equivariant Ideal Membership Problem}
\label{sec:algorithm}

In this section we provide a decision procedure for the \kl{equivariant ideal
membership problem}, we will leverage the results of the previous section
to compute what we call \kl{$V$-strong equivariant Gröbner bases} for a finite set
$V \subfin \Indets$ of indeterminates.

\begin{definition}
  \label{def:strong-equiv-grob}
  Let $V \subseteq \X$ be a set of indeterminates.
  An orbit finite set $\Basis_V$
  of polynomials is called a \intro{$V$-strong equivariant Gröbner basis}
  of an \kl{equivariant ideal} $\idl$ 
  whenever $\Basis_V \cap \poly{\K}{V}$ is a
  \kl{Gröbner basis} of the ideal $\idl \cap \poly{\K}{V}$.
\end{definition}

Let us note that a set in as \kl{equivariant Gröbner basis}
if and only if it is a \kl{$V$-strong equivariant Gröbner basis} for
every finite set $V \subseteq \Indets$ of indeterminates.

\AP To compute a \kl{$V$-strong equivariant Gröbner basis} of an
\kl{equivariant ideal} $\idl$, we will compute a \kl{weak equivariant Gröbner
basis} of the ideal $\idl$ with respect to a new ordering of the indeterminates
and a new group action. Let us write $\IndetsV[V] \defined V \ordplus (\Indets
\setminus V)$, that is the totally ordered set of indeterminates having the
same elements as $\Indets$, but with the elements of $V$ ordered below all
other elements of $\Indets$.
We will define a new group $\group_V$ that acts on $\IndetsV[V]$ by
restricting the action of $\group$ to the operations that fix the
indeterminates in $V$, i.e., such that for all $\gelem \in \group_V$ and
$x \in V$, we have $\gelem(x) = x$.

Let us call $\iota_V \colon \Indets \to \IndetsV[V]$ the identity map that maps
every element of $\Indets$ to itself. We extend this map as a morphism from
$\poly{\K}{\Indets}$ to $\poly{\K}{\IndetsV[V]}$. Beware that $\iota_V$ is not
an order-preserving map. Similarly, let us introduce $\upsilon_V \colon
\IndetsV[V] \to \Indets$ that maps every element of $\IndetsV[V]$ to itself,
and extends as a morphism from $\poly{\K}{\IndetsV[V]}$ to
$\poly{\K}{\Indets}$. Again, $\upsilon_V$ is not an order-preserving map.
Using these two constructions, we are ready to define the algorithm that computes
a \kl{$V$-strong equivariant Gröbner basis} of an \kl{equivariant ideal}
in \cref{alg:v-stronggb}.

\begin{algorithm}
    \caption{Computing a \kl{$V$-strong equivariant Gröbner bases}}
    \label{alg:v-stronggb}
    \KwIn{An orbit finite set $H$ of polynomials, a finite set $V$ of indeterminates}
    \KwOut{A \kl{$V$-strong equivariant Gröbner basis} of
      $\EqIdlGen{H}$}
    \Begin{
        $H_V \gets \iota_V(H)$\;
        $\Basis_V \gets \mathsf{weakgb}(H_V)$\;
        $\Basis \gets \upsilon_V(\Basis_V)$\;
        \Return{$\Basis$}\;
    }
\end{algorithm}

For \cref{alg:v-stronggb} to be valid, it suffices to check that: $\iota_V(H)$
is an orbit finite set of polynomials in $\poly{\K}{\IndetsV[V]}$ under the
action of $\group_V$, that the hypotheses of
  \cref{thm:weakgb-comput}
are satisfied for $(\IndetsV[V],
\group_V)$, and that the result $\upsilon_V(\Basis_V)$ is indeed a
\kl{$V$-strong equivariant Gröbner basis} of $\EqIdlGen{H}$.

Let us first check that $\iota_V(H)$ is an orbit finite set of
polynomials in $\poly{\K}{\IndetsV[V]}$ under the action of $\group_V$,
and that it is effectively computable from $V$ and $H$.

\begin{lemma}
  \label{lem:iota-V-orbit-finite}
  Let $H$ be an orbit finite set of polynomials in $\poly{\K}{\Indets}$.
  Then, $\iota_V(H)$ is an orbit finite set of polynomials in
  $\poly{\K}{\IndetsV[V]}$ under the action of $\group_V$.
\end{lemma}
\begin{proof}
  Let $H$ be a finite set of polynomials.
  Because $\group \actson \Indets$ is \kl{effectively oligomorphic}:
  for every polynomial $p \in H$, we can compute a finite set
  $P_V$ of representatives of the \kl{$\group_V$-orbits} of $p$.
  We define $H_V$ to be the union of all the $P_V$ for $p \in H$.
  This union is \kl{orbit finite}.

  \todo[inline]{Is it just not that 
    $\iota_V$ is finitely supported by $V$? Maybe there is a
    one liner argument here.}

\end{proof}

\begin{lemma}
  \label{lem:decidability-preserved}
  Let $(\Indets, \group)$ be a set of indeterminates, a group acting
  \kl{effectively oligomorphically} on $\Indets$ and an
  \kl{effective total ordering} $\varleq$ on $\Indets$ that is compatible with the
  action of $\group$.
  Then, the set $(\IndetsV, \group_V)$ is a set of indeterminates, a group acting
  \kl{effectively oligomorphically} on $\X$ and an
  \kl{effective total ordering} on $\IndetsV$ that is compatible with the
  action of $\group_V$.
\end{lemma}
\begin{proof}
  It is clear that $\IndetsV$ 
  is equipped with an \kl{effective total ordering}, and it is 
  \kl{compatible} with the action of $\group_V$ because the latter fixes
  the elements of $V$.
  Let us now show that $\group_V$ is \kl{effectively oligomorphic}.

  \todo[inline]{what we want to show is that 
    the orbits of tuples of elements in
    $\group_V$ can be computed.
    }
\end{proof}

Then, we will prove that a mild assumption on the monomials allows us to
conclude that $(\mon[\omega]{\IndetsV}, \gdivleq[\group_V])$ is a
\kl{well-quasi-ordering}: it suffices to assume that $(\mon[\omega + 1]{\Indets},
\gdivleq[\group])$ is a \kl{well-quasi-ordering} itself.

\begin{lemma}
  \label{lem:wqo-mon-S}
  Assume that $(\mon[\omega + 1]{\X}, \gdivleq[\group])$ is a \kl{well-quasi-ordering}.
  Then, the set 
  $(\mon[\omega]{\IndetsV}, \gdivleq[\group_V])$ is \kl{well-quasi-ordered}.
\end{lemma}
\begin{proof}
  Let $\seqof{\monelt_i}[i \in \N]$ be an infinite sequence of elements of
  $\mon[\omega]{\X}$. We will show that there exists $i < j$ such that
  $\monelt_i \gdivleq[\group_S] \monelt_j$.
  We will assume without loss of generality that the elements of 
  $\monelt_i$ contain all variables in $S$.

  To that end, let us define the function $h \colon \mon[\omega]{\X} \to
  \mon[\omega + 1]{\X}$ that maps a monomial $m$ to itself but with the
  elements of $S$ having exponent $\omega$.
  Because $(\mon[\omega + 1]{\X}, \gdivleq[\group])$ is a
  \kl{well-quasi-ordering},
  there exists an infinite subsequence of indices $J$ such that 
  for all $i < j$ in $J$, we have $h(\monelt_i) \gdivleq[\group] h(\monelt_j)$.
  By definition, this provides us with a sequence of 
  group elements $\gelem_{i,j} \in \group$ such that
  for all  $i < j$ in $J$, we have
  \begin{equation*}
    \gelem_{i,j}(h(\monelt_i)) \divleq h(\monelt_j)
  \end{equation*}
  In particular, 
  the elements $\gelem_{i,j}$ must send elements of $S$ to elements of $S$.
  Because $S$ is a finite set, one can further refine the extraction
  so that $\gelem_{i,j}$ is the identity fuction on $S$ for all $i < j$ in $J$. 
  Therefore, we have $\gelem_{i,j} \in \group_S$.
  Finally,
  consider the sequence of monomials
  $\monelt[s]_i$ defined as the restriction of $\monelt_i$ to the elements
  of $S$. Since $S$ is finite, $(\mon{S}, \divleq)$ is a \kl{well-quasi-ordering},
  and one can further extract $J$ to assume that for all 
  $i < j$ in $J$, we have $\monelt[s]_i \divleq \monelt[s]_j$.
  Finally,
  we conclude that $\gelem_{i,j}(\monelt_i) \divleq \monelt_j$,
  for all $i < j$ in $J$, and in particular for one such pair.
\end{proof}

Now that we have all the ingredients to show that 
\cref{alg:v-stronggb} is an algorithm, let us prove that it is correct.

\begin{lemma}
  \label{lem:correct-v-strong-gb}
  Let $H$ be an orbit finite set of polynomials in $\poly{\K}{\Indets}$,
  and let $V \subseteq \Indets$ be a finite set of indeterminates.
  Then, the result of \cref{alg:v-stronggb} is a
  \kl{$V$-strong equivariant Gröbner basis} of $\EqIdlGen{H}$.
\end{lemma}
\begin{proof}
  Let $H_V = \iota_V(H)$, $\Basis_V
  = \mathsf{weakgb}(H_V)$, and $\Basis = \upsilon_V(\Basis_V)$.
  Let us first check that $\Basis$ is indeed a
  generating set of the ideal $\EqIdlGen{H}$. It is clear that
  $\upsilon_V(\IdlGen{\iota_V(H)}) = \IdlGen{H}$, because $\upsilon_V$ is a
  morphism of polynomial rings. Furthermore, because $\Basis_S$ is a \kl{weak
  equivariant Gröbner basis} of $\EqIdlGen[\group_V]{H_V}$, we have that
  $\IdlGen{\Basis_S} = \IdlGen{H_S}$, and therefore
  $\upsilon_V(\IdlGen{\Basis_S}) = \IdlGen{H}$.

  Let us prove that $\Basis$ is a \kl{$V$-strong equivariant Gröbner basis} of
  $\EqIdlGen{H}$. Let $p \in \poly{\K}{V} \cap \EqIdlGen{H}$, and let us
  consider a decomposition $\mathfrak{d} = \seqof{a_i \monelt_i h_i}[i \in I]$
  of $p$ such that $p = \sum_{i \in I} a_i \monelt_i h_i$, where $a_i \in \K$,
  $\monelt_i \in \mon{\Indets}$, and $h_i \in H$. Then, $\iota_V(p) = \sum_{i
  \in I} a_i \iota_V(\monelt_i) \iota_V(h_i)$, and $\iota_V(h_i) \in H_V$ for
  all $i \in I$, because $\iota_V$ is a morphism.
  Because $\Basis_S$ is a \kl{weak equivariant Gröbner basis} of
  $\IdlGen{H_V}$, there exists a decomposition
  $\mathfrak{d}'$ of $\iota_V(p)$ such that:
  \begin{equation*}
    \iota_V(p) = \sum_{j \in J} b_j \monelt'_j h'_j
    \quad ,
  \end{equation*}
  where $b_j \in \K$, $\monelt'_j \in \mon{\IndetsV[V]}$, and $h'_j \in
  \Basis_V$, and such that
  $\lm[\IndetsV](\iota_V(p)) = \lmdec[\IndetsV](\mathfrak{d}')$,
  and $\domdec(\mathfrak{d}') \subseteq \domdec(\iota_V(\mathfrak{d}))$.

  Remark that $\lm[\IndetsV](\iota_V(p)) = \iota_V(\lm[\Indets](p)) \in
  \mon{V}$, because all variables of $p$ are in $V$. As a consequence, we know
  that all the variables appearing in $\mathfrak{d}'$ are in $V$, since
  \kl{leading monomials} are computed using the \kl{reverse lexicographic
  ordering}, and variables in $V$ are ordered below all other variables in
  $\IndetsV[V]$. As a consequence, for the rest of the proof, we will assume
  that $\mathfrak{d}'$, $p$ and $\iota_V(p)$ are all in $\poly{\K}{V}$, where
  $V$ equipped with the ordering induced by $\Indets$ (which is the same as the
  one induced by $\IndetsV[V]$).

  Finally, let us consider a term $\monelt_i' h_i'$ appearing in
  $\mathfrak{d}'$, with maximal \kl{leading monomial}. We know that $\lm(p) =
  \lm(\monelt_i' h_i') = \monelt_i' \cdot \lm(h_i')$. In particular, $\lm(h_i')
  \divleq \lm(p)$, and $h_i' \in \poly{\K}{V}$.
  We have shown that $\Basis \cap \poly{\K}{V}$ is a \kl{Gröbner basis} of the
  ideal $\IdlGen{H} \cap \poly{\K}{V}$,
  i.e., that $\Basis$ is a \kl{$V$-strong equivariant Gröbner basis} of
  $\EqIdlGen{H}$.
\end{proof}

As a corollary, we obtain that \cref{alg:v-stronggb} is a decision procedure for the
\kl{equivariant ideal membership problem}. 

\csname thm:decide-equiv-ideal-mem\endcsname*
