%!TEX root = ../atomic.asmart.tex
% LTeX: language=en
\section{Preliminaries}
\label{sec:preliminaries}

\paragraph{Partial orders, ordinals, well-founded sets, and well-quasi-ordered
sets.} \AP We assume basic familiarity with partial orders, well-founded sets,
and ordinals. We will use the notation $\intro*\om$ for the first infinite
ordinal (that is, $(\N, \leq)$), and write $X \intro*\ordplus Y$ for the
lexicographic sum of two partial orders $X$ and $Y$. We will also use the usual
notations for finite ordinals, writing $\intro*\ordfin{n}$ for the finite
ordinal of size $n$. For instance, $\om \ordplus \ordfin{1}$ is the total order
$\N \uplus \set{+\infty}$, where $+\infty$ is the new largest element.

\AP In order to guarantee the termination of the algorithms presented in this
paper, a key ingredient will be the notion of \intro{well-quasi-ordering}
(WQO), that are sets $(X, \leq)$ such that every infinite sequence
$\seqof{x_i}[i \in \N]$ of elements of $X$ contains a pair $i < j$ such that
$x_i \leq x_j$. Examples of \kl{well-quasi-orderings} include finite sets with
any ordering, or $\N \times \N$ with the product ordering. We refer the reader
to \cite{SCSC12} for a comprehensive introduction to \kl{well-quasi-orderings}
and their applications in computer science.

\paragraph*{Polynomials, monomials, divisibility.} \AP 
We assume basic familiarity with the theory of
commutative algebra, and polynomials. We will use the notation $\poly{\K}{\Indets}$
for the ring of polynomials with coefficients from a field $\K$ and
indeterminates/variables from a set $\Indets$, and $\mon{\Indets}$ for the set of
monomials in $\poly{\K}{\Indets}$. Letters $p,q,r$ are used to denote polynomials,
$\monelt,\monelt[n]$ are used to denote monomials, and $a,b,\alpha,\beta$ are
used to denote coefficients in $\K$.

A classical example of a \kl{WQO} is the set of monomials $\mon{\Indets}$,
endowed with the \kl{divisibility} relation $\intro*\divleq$ whenever $\Indets$
is finite. We recall that a monomial $\monelt[m]$ \intro{divides} a monomial
$\monelt[n]$ if there exists a monomial $\monelt[l]$ such that $\monelt[m]
\times \monelt[l] = \monelt[n]$. In this case, we write $\monelt[m]
\reintro*\divleq \monelt[n]$. Note that monomials can be seen as functions from
$\Indets$ to $\N$ having a finite support, and that the \kl{divisibility}
relation can be extended to monomials that are functions from $\Indets$ to
$(Y,\leq)$, where $Y$ is any partially ordered set. In this case, we write
$\monelt[m] \divleq \monelt[n]$ if for every $x \in \Indets$, we have
$\monelt[m](x) \leq \monelt[n](x)$. We will write $\intro*\mon[\om \ordplus
1]{\Indets}$ (resp. $\mon[\om^2]{\Indets}$) for the set of monomials that are
functions from $\Indets$ to $\om \ordplus \ordfin{1}$ (resp. $\om^2$).

\AP Unless otherwise specified, we will assume that the set of indeterminates
$\Indets$ comes equipped with a total ordering $\varleq$. Using this order, we
define the \intro{reverse lexicographic} (revlex) ordering on monomials as
follows: $\monelt[n] \intro*\revlexlt \monelt[m]$ if there exists an
indeterminate $x \in \Indets$ such that $\monelt[n](x) < \monelt[m](x)$, and such
that for every $y \in \Indets$, if $x \varlt y$ then $\monelt[n](y) =
\monelt[m](y)$. Remark that if $\monelt[n] \revlexleq \monelt[m]$, then in
particular $\monelt[n] \divleq \monelt[m]$. 

\AP We can now use the \kl{revlex} ordering to identify particular elements in
a given polynomial. Namely, for a polynomial $p \in \poly{\K}{\X}$, we define
the \intro{leading monomial} $\intro*\lm(p)$ of $p$ as the largest monomial
appearing in $p$ with respect to the \kl{revlex} ordering, and the
\intro{leading coefficient} $\intro*\lc(p)$ of $p$ as the coefficient of
$\lm(p)$ in $p$. We can then define the \intro{leading term} $\intro*\lt(p)$ of
$p$ as the product of its \kl{leading monomial} and its \kl{leading
coefficient}, and the \intro{characteristic monomial} $\intro*\cm(p)$ of $p$ as
the product of its \kl{leading monomial} and all the indeterminates appearing
in $p$. We also define the \intro(monomial){domain} of $\monelt[m]$ as the set
$\intro*\dom(\monelt[m])$ of indeterminates $x \in \X$ such that $\monelt[m](x) \neq
0$. Because the coefficients and monomial in question are highly dependent on
the ordering $\varleq$, we allow ourselves to write $\lm[\Indets](p)$ to
highlight the precise ordered set of variables that was used to compute the
\kl{leading monomial} of $p$.

Let us briefly argue in favor of the \kl{reverse lexicographic} ordering. In
the case of a finite set of indeterminates, one can choose any total ordering
on $\mon{\Indets}$, as long as it contains the \kl{divisibility}
quasi-ordering, and is compatible with the product of monomials.\footnote{This
is often called a \emph{monomial ordering}, see \cite{CLO15}.} In our case,
having an infinite number of indeterminates, we rely on a connection between
$\lm(p)$ and $\dom(p)$: $\dom(p) \subseteq \dwset{\dom(\lm(p))}$, where
$\dwset{S}$ is the downward closure of a set $S \subseteq \Indets$, i.e. the
set of all indeterminates $x \in \Indets$ such that $y \leq x$ for some $y \in
S$. This means that the \kl{leading monomial} encodes a \emph{global property}
of the polynomial, and it will be crucial in our termitation arguments. This is
already at the core of the \emph{elimination theorems} \cite[Chapter 3, Theorem
2]{CLO15}.


\paragraph{Ideals, and Gröbner Bases.} \AP An \intro{ideal} $\idl$ of
$\poly{\K}{\X}$ is a non-empty subset of $\poly{\K}{\X}$ that is closed under
addition and multiplication by elements of $\poly{\K}{\X}$. Given a set $H
\subseteq \poly{\K}{\X}$, we denote by $\intro*\IdlGen{H}$ the ideal generated
by $H$, i.e. the smallest ideal that contains $H$. The \intro{ideal membership
problem} is the following decision problem: given a polynomial $p \in
\poly{\K}{\X}$ and a set of polynomials $H \subseteq \poly{\K}{\X}$, decide
whether $p$ belongs to the ideal $\IdlGen{H}$ generated by $H$. We know that
this problem is decidable when $\X$ is finite, and that it is even
$\EXPTIME$-complete \cite{MAME82}. The classical approach to the \kl{ideal
membership problem} is to use the \kl{Gröbner basis} theory that was developed
in the 70s by Buchberger~\cite{BUCH76}. 
A set $\Basis$ of polynomials is called a \intro{Gröbner basis} of
an ideal $\idl$ if, $\IdlGen{\Basis} = \idl$ and for every polynomial $p \in
\idl$, there exists a polynomial $q \in \Basis$ such that $\lm[\Indets](q)
\divleq \lm[\Indets](p)$.

Given a \kl{Gröbner basis} $\Basis$ of an ideal $\idl$, and a polynomial $p$,
it suffices to iteratively reduce the \kl{leading monomial} of $p$ by
subtracting multiples of elements in $\Basis$, until one cannot apply any
reductions. If the result is $0$, then $p$ belongs to $\idl$, and otherwise it
does not. 



\paragraph{Group actions, equivariance, and orbit finite sets.}  \AP A
\intro{group} $\group$ is a set equipped with a binary operation that is
associative, has an identity element and has inverses. In our setting, we are
interested in infinite sets $\X$ of indeterminates that is equipped with a
\intro{group action} $\group \actson \X$. This means that for each $\gelem \in
\group$, we have a bijection $\X \tobij \X$ that we denote by $x \mapsto \gelem
\cdot x$. A set $S \subseteq \X$ is \intro{equivariant} under the action of
$\group$ if for all $\gelem \in \group$ and $x \in S$, we have $\gelem \cdot x
\in S$. We give in \cref{ex:idl-equiv} an example and a non-example
of \kl{equivariant ideals}.

\arka{Maybe we can simply assume $\group$ is a subgroup of bijections of
$\Indets$. Then we can also use the notation $\pi(x)$ which is less confusing
than $\pi\cdot x$} \todo[inline]{In my opinion, the new notation is more
  confusing because now $p(x)$ is very diffrent from $\pi(x)$: one is
polynomial evaluation/substitution, the other is a group action.
}

\arka{But $\cdot$ is also product}

\begin{example}
    \label{ex:idl-equiv}
    Let $\Indets$ be any infinite set, and $\group$ be the 
    group of all bijections of $\Indets$. 
    Then the set $S_0 \subset \poly{\K}{\Indets}$ of all polynomials 
    whose set of coefficients sums to $0$ is an equivariant ideal.
    Conversely, the set of all polynomials that are multiple
    of $x \in X$ is an \kl{ideal} that is not \kl{equivariant}.
\end{example}
\begin{proof}
    Let $p,q\in S_0$, and $r \in \poly{\K}{\Indets}$.
    Then, $p \times r + q$ is in $S_0$. Remark that 
    $p,r$ and $q$ belong to a subset $\poly{\K}{\Indets}$ of the 
    polynomials that uses only finitely many indeterminates.
    In this subset, the sum of all coefficients is obtained
    by applying the polynomials to the value $1$ for every indeterminate
    $y \in \Indets$. We conclude that
    $(p \times r + q)(1,\dots, 1) 
    = p(1,\dots,1) \times r(1,\dots,1) + q(1,\dots,1)
    = 0 \times r(1, \dots, 1) + 0 = 0$, hence that
    $p \times r + q$ belongs to $S_0$. 
    Because $0$ is in $S_0$, we conclude that $S_0$ is an \kl{ideal}.
    Furthermore, if $\gelem \in \group$ and $p \in S_0$, then
    the sum of the coefficients $\gelem \cdot p$ is exactly
    the sum of the coefficients of $p$, hence is $0$ too.
    This shows that $S_0$ is \kl(ideal){equivariant}.

    It is clear that all multiples of a given polynomial $x \in \Indets$
    is an ideal of $\poly{\K}{\Indets}$. This is not an \kl{equivariant ideal}:
    take any bijection $\gelem \in \group$ that does not map $x$ to $x$ (it
    exists because $\Indets$ is infinite and $\group$ is all permutations),
    then $\gelem \cdot x$ is not a multiple of $x$, and therefore does 
    not belong to the ideal.
\end{proof}

\AP An \kl{equivariant set} is said to be \intro{orbit finite} if it is the
union of finitely many \intro{orbits} under the action of $\group$. We denote
$\intro*\orbit[\group]{E}$ for the set of all elements $\gelem \cdot x$ for
$\gelem \in \group$ and $x \in E$. Equivalently, an \reintro{orbit finite set}
is a set of the form $\orbit[\group]{E}$ for some finite set $E$. Not every
\kl{equivariant subset} is \kl{orbit finite}, as shown in
\cref{ex:orbit-finite}. However, \kl{orbit finite sets} are
robust in the sense that \kl{equivariant subsets} of \kl{orbit finite sets} are
also \kl{orbit finite}, and similarly, an \kl{equivariant subset} of $E^n$ is
\kl{orbit finite} whenever $E$ is \kl{orbit finite} and $n \in \N$ is finite.
For algorithmic purposes, \kl{orbit finite sets} are the ones that can be taken
as input as a finite set of representatives (one for each orbit). The notions
of \kl{equivariance} and \kl{orbit finite sets} from a computational
perspective are discussed in \cite{BOJAN16inf}, and we refer the reader to this
book for a more comprehensive introduction to the topic.

\begin{example}
  \label{ex:orbit-finite}
  Let $\Indets = \N$, and $\group$ be all permutations 
  that fixes prime numbers. The
  set of all polynomials whose coefficients sum to $0$ is an 
  \kl{equivariant ideal}, but it is not \kl{orbit finite},
  since all the polynomials $x_p - x_q$ for $p \neq q$ primes
  are in distinct orbits under the action of $\group$.
\end{example}

\AP A function $f \colon X \to Y$ between two sets $X$ and $Y$ equipped with
actions $\group \actson X$ and $\group \actson Y$ is said to be
\intro(func){equivariant} if for all $\gelem \in \group$ and $x \in X$, we have
$f(\gelem \cdot x) = \gelem \cdot f(x)$. For instance, the
\kl(monomial){domain} of a monomial is an \kl{equivariant function} if $\gelem
\in \group$, then $\gelem \cdot \dom(\monelt[m]) = \dom(\gelem \cdot
\monelt[m])$. Let us point out that the image of an \kl{orbit finite set} under
an \kl{equivariant function} is \kl{orbit finite}, and that the algorithms that
we will develop in this paper will all be \kl(func){equivariant}.

\paragraph*{Computability assumptions.} \AP We say that the action is
\intro{effectively oligomorphic} if :
%
\begin{enumerate}
\item It is \intro{oligomorphic}, i.e.\ for every $n \in \N$ and every \kl{orbit
finite set} $E \subseteq \Indets$,
the set $E^n$ is \kl{orbit finite} under the action of $\group$ on $\Indets^n$.
\item There exists an algorithm that decides whether two elements $\vec{x},
\vec{y} \in \Indets^*$ are in the same orbit under the action of $\group$ on $\Indets^*$.
\item There exists an algorithm which on input $n\in\N$ outputs a set $A\subseteqfin\Indets^n$ such that $|A\cap U| = 1$ for every orbit $U\in\Indets^n$.
\end{enumerate}
%


A group action $\group \actson \X$ is said to be \intro(ord){compatible}
with an ordering $\leq$ on $\X$ if for all $\gelem \in \group$ and $x,y \in
\X$, we have $x \leq y$ if and only if $\gelem \cdot x \leq \gelem \cdot y$.
Let us point out that in this case, $\revlexleq$ is also \kl(ord){compatible} with
the action of $\group$ on $\mon{\X}$, i.e. for all $\gelem \in \group$ and
monomials $\monelt[m], \monelt[n] \in \mon{\X}$, we have $\monelt[m] \revlexleq
\monelt[n]$ if and only if $\gelem \cdot \monelt[m] \revlexleq \gelem \cdot
\monelt[n]$.
Our \intro{computability assumptions} on the tuple $(\Indets, \group,
\leq)$ will therefore be that $\group$ acts \kl{effectively oligomorphic} on
$\Indets$, and that its action is \kl(ord){compatible} with the ordering $\leq$
on $\Indets$.

\begin{example}
  \label{ex:computability-assumptions}
  Let $\Indets \defined \Q$ and $\group$ be the group of all
  order preserving bijections of $\Q$.
  Then, $\group$ acts \kl{effectively oligomorphically} on $\Indets$,
  and its action is \kl(ord){compatible} with the ordering of $\Q$ by definition.
\end{example}

Note that under our \kl{computability assumptions}, the set of polynomials
$\poly{\K}{\Indets}$ is also \kl{effectively oligomorphic} under the action of
$\group$ on $\Indets$. This is because a polynomial $p \in \poly{\K}{\Indets}$
can be seen as an element of $(\K \times \Indets^{\leq d})^n$ where $n$ is the
number of monomials in $p$, and $d$ is the maximal degree of a monomial
appearing in $p$. Beware that the set of polynomials $\poly{\K}{\Indets}$ is
not \kl{orbit finite}, precisely because the orbit of a polynomial $p$ under 
the action of $\group$ cannot change the degree of $p$, and that there are 
polynomials of arbitrarily large degree.

\paragraph{Equivariant Gröbner bases.} \AP We know from \cite{GHOLAS24} that a
necessary condition for the \kl{equivariant Hilbert basis property} to hold is
that the set  $\mon{\X}$  of monomials is a \kl{well-quasi-ordering} when
endowed with the \intro{divisibility up-to $\group$} relation
($\intro*\gdivleq$), which is defined as follows: for $\monelt_1, \monelt_2 \in
\mon{\X}$, we write $\monelt_1 \gdivleq \monelt_2$ if there exists $\gelem \in
\group$ such that $\monelt_1$ \kl{divides} $\gelem \cdot \monelt_2$. This
relation also extends to monomials that are functions from $\Indets$ to
$(Y,\leq)$ with finite support, where $Y$ is any partially ordered set. We say
that a set $\Basis \subseteq \poly{\K}{\X}$ is an \intro{equivariant Gr\"{o}bner
basis} of an equivariant ideal $\idl$ if $\Basis$ is \kl{equivariant},
$\IdlGen{\Basis} = \idl$, and for every polynomial $p \in \idl$, there exists
$q \in \Basis$ such that $\lm[\Indets](q) \gdivleq \lm[\Indets](p)$ and
$\dom(q) \subseteq \dom(p)$, following the definition of \cite{GHOLAS24}.

Beware that even in the case of a finite set of variables, a \kl{Gröbner basis}
is not necessarily an \kl{equivariant Gröbner basis}, because of the
\kl(polynomial){domain} condition. However, every \kl{equivariant Gröbner
basis} is a \kl{Gröbner basis}.

\begin{example}
  \label{ex:equivariant-gb}
  Let $\Indets \defined \set{ x_1, x_2 }$,
  with $x_1 \varleq x_2$,
  and $\group$ be the trivial group.
  Let us furthermore consider the ideal $\idl$ \kl(idl){generated by}
  $\set{ x_1, x_2 }$.
  Then, the set $\Basis \defined \set{ x_2 - x_1, x_1 }$ is a
  \kl{Gröbner basis} of $\idl$, but not an \kl{equivariant Gröbner basis}.
  Indeed, $x_2 \in \idl$, but there is no polynomial $q \in \Basis$
  such that $\lm(q) \divleq x_2$ and $\dom(q) \subseteq \dom(x_2)$.
\end{example}

In the finite case, one can always compute an \kl{equivariant Gröbner basis} by
computing \kl{Gröbner bases} for every possible ordering of the indeterminates,
and taking their union.\footnote{This algorithm is correct because we are
  considering the \kl{reverse lexicographic} ordering.}

\section{Related Research}
\todo[inline]{Check new version. Maybe merge?}
%
The existence of \kl{equivariant Gr\"{o}bner basis} for the case when $\group$ is equal to the group $\symgr{\Indets}$ of all bijections from $\Indets$ to itself,
is proven in \cite[Proposition 2]{COHEN67},
and the computability can be found in \cite[Theorem 10]{COHEN87} and \cite{Emmott87}.
Their definition of Gr\"{o}bner basis is different than ours.
The crucial observation made by them is that ideal invariant under the action of $\symgr{\Indets}$ is also invariant under the action of monoid $\inc{<}$ of one-to-one functions from $\Indets$ to itself which preserve a fixed well-order $<$.
Well-ordering the set of variables allows one to extend intuition from the classical setting of polynomials rings with finitely many variables to the infinitely many variable case.

%Let $<$ be a well-order on $\Indets$.
%Let $\inc{<}$ be the set of all strictly increasing one-to-one function from $\Indets$ to itself.
%An obvious but crucial observation is that,
%an ideal that is invariant under the action of $\symgr{\Indets}$ is also invariant under the action of $\inc{<}$.
%Hence we can consider ideals invariant under the action of $\inc{<}$.
%Using the well-order we define the leading monomial of a polynomial by the short-lex order : \arka{todo:define $\shortlexlt$}.
%Since $<$ is a well-order, $\shortlexlt$ is also a well-order.
%Hence any sequence of reductions that decreases the leading monomial with respect to $\shortlexlt$ terminates.
%Again, due to the fact that $<$ is a well-order,
%the divisibility relation $\gdivleq[\inc{<}]$ upto $\inc{<}$ is a \kl{well-partial-order}.
%As a consequence, for every $\inc{<}$-equivariant ideal $\idl[I]$,
%we can find a finite set of polynomials $B\subseteq \idl[I]$ which is a Gr\"{o}bner basis of $\idl{I}$ in the sense that,
%for every polynomial $f$ in $\idl[I]$,
%there exists a polynomial $b$ in $B$ such that $\lm(b) \gdivleq[\inc{<}] \lm(f)$.
%It can be proven that $B$ is a set of generators of $\idl[I]$.
%This proves that every $\inc{<}$-equivariant ideal, and hence every $\symgr{\Indets}$-equivariant ideal is finitely generated.
%Now for the computation of a Gr\"{o}bner basis,
%one can extend the Buchberger's algorithm to this setting,
%which terminates due to the fact that $\gdivleq[\inc{<}]$ is a \kl{well-partial-order}.

The above mentioned results were rediscovered in \cite{AH07,AH08,HKL18}.
In \cite{HS12} these results were used to prove the Independent Set Conjecture in algebraic statistics.
In \cite{HS12}, the authors also showed that one can even take a submonoid $\calM$ of $\inc{<}$ and prove existence and computability of finite Gr\"{o}bner basis assuming that $\gdivleq[\calM]$ is a well-partial-order.

\arka{Either add \cite{GHOLAS24} here or merge this part into the intro}

\cite[Corollary 59]{GHOLAS24} shows that one can use a similar strategy for the action $\aut{\calQ}\actson\calQ$ (\Cref{ex:dlo}).
However, it becomes difficult to do the same for a general action $\group\actson\Indets$ satisfying the conditions in \Cref{thm:compute-egb}.

Last but not least,
our results are closely related to some of the recent results regarding computation with orbit-finite sets,
in particular on algebraic problems \cite{BFKM24,GHL22,GHL25,KKOT15,Prz23}.
 




