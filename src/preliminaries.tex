\section{Preliminaries}
\label{sec:preliminaries}

\paragraph{Ordinals.} We assume basic familiarity with ordinals, and in
particular with the ordinal $\omega+1$. \todo{define ordinals? really?} 

\paragraph{Polynomials, monomials, divisibility and ideals.} We write
$\poly{\K}{\X}$ for the ring of \intro{polynomials} with indeterminates in $\X$
and coefficients in $\K$, where $\X$ is a set and $\K$ is a field. In this
paper, we will restrict our attention to fields of \intro{characteristic zero},
such as the rationals $\Q$, the reals $\R$ or the complex numbers $\C$. Let us
recall that a \kl{polynomial} $p$ \intro{divides} a polynomial $h$ if and only
if there exists polynomials $q$ and $r$ such that $p = h q + r$. An
\intro{ideal} $I$ of $\poly{\K}{\X}$ is a non-empty subset of $\poly{\K}{\X}$
that is closed under addition and multiplication by elements of
$\poly{\K}{\X}$. Given a subset $S \subseteq \poly{\K}{\X}$, we write
$\gen{S}{\K}$ for the \intro{ideal generated by} $S$ in $\poly{\K}{\X}$, that
is the smallest ideal of $\poly{\K}{\X}$ containing $S$, or equivalently, the
set of all polynomials that can be written as follows: $\,\sum_{i=1}^n q_i s_i$ for some
$n \in \N$, $q_i \in \poly{\K}{\X}$ and $s_i \in S$.


Among \kl{polynomials}, we will be interested in \intro{monomials}, which are
products of indeterminates, the set of which we denote by $\mon{\X}$. Because a
monomial is essentially a multiset of indeterminates, we can see it as a
function from $\X \to \N$ with finite support. In particular, a monomial $m$
\kl{divides} a monomial $n$ if and only if $m(x) \leq n(x)$ for all $x \in \X$.
Given an \kl{ordinal} $\alpha$, we will write $\mon[\alpha]{\X}$ for the
extension of \kl{monomials} to the set of all functions from $\X$ to $\alpha$
with finite support, with the same notion of \kl{divisibility}.


\paragraph{Group actions and equivariant ideals.} Recall that a \intro{group}
$\group$ is a set equipped with a binary operation that is associative, has an
identity element and has inverses. In our setting, we are interested in
infinite sets $\X$ of indeterminates that is equipped with a \intro{group
action} $\group \curvearrowright \X$. This means that for each $g \in \group$,
we have a bijection $\X \tobij \X$ that we denote by $x \mapsto g \cdot x$. A
set $S \subseteq \X$ is \intro{equivariant} under the action of $\group$ if for
all $g \in \group$ and $x \in S$, we have $g \cdot x \in S$.

Given a group action $\group \curvearrowright \X$, we can extend it to a group
action $\group \curvearrowright \poly{\K}{\X}$ by defining $g \cdot p$ as the
polynomial obtained by applying $g$ to each indeterminate of $p$, that is
extending $\group \curvearrowright \X$ by linearity additively, and
multiplicatively for \kl{monomials}. In this case, an \intro{equivariant ideal}
is an ideal $I$ of $\poly{\K}{\X}$ that is \kl{equivariant} under the action of
$\group$.

\begin{example}
    \label{ex:idl-equiv}
    Let $\X$ be any infinite set, and $\group$ be the 
    group of all bijections of $\X$. 
    Then the set $S_0 \subset \poly{\K}{\X}$ of all polynomials 
    whose set of coefficients sums to $0$ is an equivariant ideal.
    Conversely, the set of all polynomials that are multiple
    of $x \in X$ is an \kl{ideal} that is not \kl{equivariant}.
\end{example}
\begin{proof}
    Let $p,q\in S_0$, and $r \in \poly{\K}{\X}$.
    Then, $p \times r + q$ is in $S_0$. Remark that 
    $p,r$ and $q$ belong to a subset $\poly{\K}{\Y}$ of the 
    polynomials that uses only finitely many indeterminates.
    In this subset, the sum of all coefficients is obtained
    by applying the polynomials to the value $1$ for every indeterminate
    $y \in \Y$. We conclude that
    $(p \times r + q)(1,\dots, 1) 
    = p(1,\dots,1) \times r(1,\dots,1) + q(1,\dots,1)
    = 0 \times r(1, \dots, 1) + 0 = 0$, hence that
    $p \times r + q$ belongs to $S_0$. 
    Because $0$ is in $S_0$, we conclude that $S_0$ is an \kl{ideal}.
    Furthermore, if $g \in \group$ and $p \in S_0$, then
    the sum of the coefficients $g \cdot p$ is exactly
    the sum of the coefficients of $p$, hence is $0$ too.
    This shows that $S_0$ is \kl(ideal){equivariant}.

    It is clear that all multiples of a given polynomial $x \in \X$
    is an ideal of $\poly{\K}{\X}$. This is not an \kl{equivariant ideal}:
    take any bijection $g \in \group$ that does not map $x$ to $x$ (it
    exists because $\X$ is infinite and $\group$ is all permutations),
    then $g \cdot x$ is not a multiple of $x$, and therefore does 
    not belong to the ideal.
\end{proof}


\paragraph{Group actions respecting a linear order.}

\paragraph{Oligomorphic structures.}


