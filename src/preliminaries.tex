%!TEX root = ../atomic.tex
% LTeX: language=en
\section{Preliminaries}
\label{sec:preliminaries}

\paragraph{Orders.} We assume basic familiarity with ordinals, and in
particular with the ordinal $\omega+1$. A sequence $\seqof{x_i}[i \in \N]$ of a
quasi-ordered set $(X,\leq)$ is \intro(wqo){good} if it contains an increasing
pair $i < j$ such that $x_i \leq x_j$. A sequence that is not \kl(wqo){good} is
a \reintro{bad sequence}. A set $(X,\leq)$ is \intro{well-quasi-orderded} if it
does not contain any infinite \kl{bad sequence}.

\paragraph{Polynomials, monomials, divisibility and
ideals.} We write $\poly{\K}{\X}$ for the ring of
\intro{polynomials} with indeterminates in $\X$ and
coefficients in $\K$, where $\X$ is a set and $\K$ is a
field. In this paper, we will restrict our attention to
fields of characteristic zero, such as the rationals
$\Q$, the reals $\R$ or the complex numbers $\C$.
\arka{Is this necessary?}
Let us
recall that a \kl{polynomial} $p$ \intro{divides} a
polynomial $h$ if and only if there exists polynomials
$q$ and $r$ such that $p = h q + r$. An \intro{ideal} $I$
of $\poly{\K}{\X}$ is a non-empty subset of
$\poly{\K}{\X}$ that is closed under addition and
multiplication by elements of $\poly{\K}{\X}$. Given a
subset $S \subseteq \poly{\K}{\X}$, we write
$\gen{S}{\K}$ for the \intro{ideal generated by} $S$ in
$\poly{\K}{\X}$, that is the smallest ideal of
$\poly{\K}{\X}$ containing $S$, or equivalently, the set
of all polynomials that can be written as follows:
$\,\sum_{i=1}^n q_i s_i$ for some $n \in \N$, $q_i \in
\poly{\K}{\X}$ and $s_i \in S$.

\AP Among \kl{polynomials}, we will be interested in \intro{monomials}, which
are products of indeterminates, the set of which we denote by $\mon{\X}$. Note
that a monomial is simply a function $M \colon \X \topartial \N$ with finite
domain. We will extend the notion of monomials to functions of the form $M
\colon \X \topartial (X, \leq)$, where $(X, \leq)$ is a quasi-ordered set, and
write $\mon[X]{\X}$ for the set of all such functions. The notion of
\kl{divisibility} is naturally extended: given two generalized monomials $M_1,
M_2 \in \mon[X]{\X}$, we say that $M_1$ divides $M_2$ if for all $x \in
\dom(M_1)$, we have $M_1(x) \leq M_2(x)$. We will mostly be interested in the
case where $(X, \leq)$ is an \kl{ordinal} $\alpha$, or a finite set equipped
with the equality relation.

\paragraph{Group actions and equivariant ideals.} Recall that a \intro{group}
$\group$ is a set equipped with a binary operation that is associative, has an
identity element and has inverses. In our setting, we are interested in
infinite sets $\X$ of indeterminates that is equipped with a \intro{group
action} $\group \actson \X$. This means that for each $g \in \group$,
we have a bijection $\X \tobij \X$ that we denote by $x \mapsto g \cdot x$. A
set $S \subseteq \X$ is \intro{equivariant} under the action of $\group$ if for
all $g \in \group$ and $x \in S$, we have $g \cdot x \in S$.

Given a group action $\group \actson \X$, we can extend it to a group
action $\group \actson \poly{\K}{\X}$ by defining $g \cdot p$ as the
polynomial obtained by applying $g$ to each indeterminate of $p$, that is
extending $\group \actson \X$ by linearity additively, and
multiplicatively for \kl{monomials}. In this case, an \intro{equivariant ideal}
is an ideal $I$ of $\poly{\K}{\X}$ that is \kl{equivariant} under the action of
$\group$. We give in \cref{ex:idl-equiv} an example and a
non-example of equivariant ideals.

From a finite set $S \subseteq \poly{\K}{\X}$, we can define the
\kl{equivariant ideal} \kl(eqidl){generated by} $S$ as the smallest equivariant
ideal containing $S$, that is the set of all polynomials that can be written as
follows: $\,\sum_{i=1}^n q_i \times (g_i \cdot s_i)$ for some $n \in \N$, $g_i
\in \group$, $q_i \in \poly{\K}{\X}$, and $s_i \in S$.


\begin{example}
    \label{ex:idl-equiv}
    Let $\X$ be any infinite set, and $\group$ be the 
    group of all bijections of $\X$. 
    Then the set $S_0 \subset \poly{\K}{\X}$ of all polynomials 
    whose set of coefficients sums to $0$ is an equivariant ideal.
    Conversely, the set of all polynomials that are multiple
    of $x \in X$ is an \kl{ideal} that is not \kl{equivariant}.
\end{example}
\begin{proof}
    Let $p,q\in S_0$, and $r \in \poly{\K}{\X}$.
    Then, $p \times r + q$ is in $S_0$. Remark that 
    $p,r$ and $q$ belong to a subset $\poly{\K}{\Y}$ of the 
    polynomials that uses only finitely many indeterminates.
    In this subset, the sum of all coefficients is obtained
    by applying the polynomials to the value $1$ for every indeterminate
    $y \in \Y$. We conclude that
    $(p \times r + q)(1,\dots, 1) 
    = p(1,\dots,1) \times r(1,\dots,1) + q(1,\dots,1)
    = 0 \times r(1, \dots, 1) + 0 = 0$, hence that
    $p \times r + q$ belongs to $S_0$. 
    Because $0$ is in $S_0$, we conclude that $S_0$ is an \kl{ideal}.
    Furthermore, if $g \in \group$ and $p \in S_0$, then
    the sum of the coefficients $g \cdot p$ is exactly
    the sum of the coefficients of $p$, hence is $0$ too.
    This shows that $S_0$ is \kl(ideal){equivariant}.

    It is clear that all multiples of a given polynomial $x \in \X$
    is an ideal of $\poly{\K}{\X}$. This is not an \kl{equivariant ideal}:
    take any bijection $g \in \group$ that does not map $x$ to $x$ (it
    exists because $\X$ is infinite and $\group$ is all permutations),
    then $g \cdot x$ is not a multiple of $x$, and therefore does 
    not belong to the ideal.
\end{proof}


\paragraph{Group actions respecting a linear order.} In this paper, we will be
interested in sets $\X$ equipped with a \kl{group action} that \intro{respects
a linear order}, i.e., such that $X$ is equipped with a linear order $\leq$
that is \kl{invariant} under the action of $\group$. This means that for all $g
\in \group$ and $x,y \in \X$, we have $x \leq y \iff g \cdot x \leq g \cdot y$.



\paragraph{Gröbner Bases.} \AP Given a total ordering $\order$ on $\X$, one can
consider the \intro{monomial ordering} $\revlex$ on $\mon{\X}$ defined by $M_1
\revlex M_2$ if and only if there exists $x \in \dom(M_1) \cap \dom(M_2)$ such
that $M_1(x) < M_2(x)$ and for all $x \order y$, we have $M_1(y) = M_2(y)$

This is in turned used to define the notions of \intro{leading monomial},
\intro{leading coefficient}, \intro{leading term}, and \intro{characteristic
monomial}.
\begin{description}
    \item[The leading monomial] of a polynomial $p$ is the largest monomial
        appearing in $p$ with respect to $\revlex$.
    \item[The leading coefficient] of a polynomial $p$ is the coefficient of
        its leading monomial.
    \item[The leading term] of a polynomial $p$ is the product of its leading
        coefficient and its leading monomial.
    \item[The characteristic monomial] of a polynomial $p$ is the monomial
        obtained by multiplying its \kl{leading monomial} with all indeterminates
        appearing in $p$.
\end{description}

A \intro{Gröbner basis} $\Basis$ of an ideal $I$ is a set of polynomials such
that for all $p \in I$, there exists $q \in \Basis$ such that the \kl{leading
monomial} of $q$ \kl{divides} the \kl{leading monomial} of $p$.
