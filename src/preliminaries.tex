%!TEX root = ../atomic.asmart.tex
% LTeX: language=en
\section{Preliminaries}
\label{sec:preliminaries}

We assume basic familiarity with the theory of commutative algebra, and
polynomials. We will use the notation $\poly{\K}{\X}$ for the ring of
polynomials with coefficients from a field $\K$ and indeterminates/variables
from a set $\X$, and $\mon{\X}$ for the set of monomials in $\poly{\K}{\X}$.
Letters $p,q,r$ are used to denote polynomials,  $\monelt,\monelt[n]$ are used
to denote monomials, and $a,b,\alpha,\beta$ are used to denote coefficients in
$\K$.

We will also assume basic familiarity with partial orders and ordinals. In
particular, we will use the notation $\intro*\om$ for the first infinite
ordinal (that is, $(\N, \leq)$), and write $X \intro*\ordplus Y$ for the
lexicographic sum of two partial orders $X$ and $Y$. We will also use the usual
notations for finite ordinals, writing $\intro*\ordfin{n}$ for the finite
ordinal of size $n$. For instance, $\om \ordplus \ordfin{1}$ is the total order
$\N \uplus \set{+\infty}$, where $+\infty$ is a (new) largest element.

\paragraph{Gröbner Bases.} \AP An \intro{ideal} $\idl$ of $\poly{\K}{\X}$ is a
non-empty subset of $\poly{\K}{\X}$ that is closed under addition and
multiplication by elements of $\poly{\K}{\X}$. Given a set $H \subseteq
\poly{\K}{\X}$, we denote by $\intro*\IdlGen{H}$ the ideal generated by $H$,
i.e. the smallest ideal that contains $H$. The \intro{ideal membership problem}
is the following decision problem: given a polynomial $p \in \poly{\K}{\X}$ and
a set of polynomials $H \subseteq \poly{\K}{\X}$, decide whether $p$ belongs to
the ideal $\IdlGen{H}$ generated by $H$. We know that this problem is decidable
when $\X$ is finite, and that it is even $\EXPTIME$-complete \cite{MAME82}.

\AP
The classical approach to the \kl{ideal membership problem} is to use the
\kl{Gröbner basis} theory that was developed in the 70s by Buchberger
\cite{BUCH76}. Given a set $H \subseteq \poly{\K}{\X}$, 
and ordering $\preceq$ on polynomials, one can define 
a rewriting relation $\intro*\toeucl{\preceq}{H}$ 
on $\poly{\K}{\X}$ as follows:
$p \toeucl{\preceq}{H} q$ if and only if there exists $h \in H$,
$a \in \K$,
and $\monelt \in \mon{\X}$ such that
\begin{equation}
    \label{eq:euclidian-ord}
    p = a \monelt h + q \text{ and } q \prec p
\end{equation}
This relation generalises the classical notion of \kl{division} of polynomials
to the case of multiple variables.
In this setting, a set $\Basis$ of polynomials is called a \intro{Gröbner basis} of an ideal $\idl$
for the ordering $\preceq$ if
the following properties hold:
\begin{enumerate}
  \item $\Basis$ is a generating set of $\idl$, i.e. $\IdlGen{\Basis} = \idl$.
  \item For every polynomial $p \in \poly{\K}{\X}$, we have
    $p \toeucl{\preceq}{\Basis}^* 0$ if and only if $p \in \idl$.
  \item The relation $\toeucl{\preceq}{\Basis}$ is \kl{terminating}, i.e. every
    rewriting sequence $p \toeucl{\preceq}{\Basis}^* q$ is finite.
  \item The relation $\toeucl{\preceq}{\Basis}$ is \kl{confluent}, i.e. for all
    $p \toeucl{\preceq}{\Basis}^* r_1$ and $p \toeucl{\preceq}{\Basis}^* r_2$,
    there exists $r_3$ such that $r_1 \toeucl{\preceq}{\Basis}^* r_3$ and
    $r_2 \toeucl{\preceq}{\Basis}^* r_3$.
\end{enumerate}
\AP
We say that $\Basis$ is a \reintro{Gröbner basis} for an ordering $\preceq$ if
it is a \kl{Gröbner basis} of the \kl{generated ideal} $\IdlGen{\Basis}$. It is
clear that given a \kl{Gröbner basis} $\Basis$ of an ideal $\idl$, one can
decide the \kl{ideal membership problem} by computing the normal form of $p$
with respect to $\Basis$ by computing the normal form of $p$ in this rewriting
system and checking if it is equal to $0$.\footnote{
  This is a very abstract definition of \kl{Gröbner bases}. In practice, one
  would use particular orderings, and expect $\preceq$ to be computable 
  in a reasonable time \cite[Chapter 2, Section 2]{CLO15}.
}

\paragraph{Group actions and equivariant ideals.}  A \intro{group} $\group$ is
a set equipped with a binary operation that is associative, has an identity
element and has inverses. In our setting, we are interested in infinite sets
$\X$ of indeterminates that is equipped with a \intro{group action} $\group
\actson \X$. This means that for each $g \in \group$, we have a bijection $\X
\tobij \X$ that we denote by $x \mapsto g \cdot x$. A set $S \subseteq \X$ is
\intro{equivariant} under the action of $\group$ if for all $g \in \group$ and
$x \in S$, we have $g \cdot x \in S$. We give in \cref{ex:idl-equiv}
an example and a non-example of \kl{equivariant ideals}.


\begin{example}
    \label{ex:idl-equiv}
    Let $\X$ be any infinite set, and $\group$ be the 
    group of all bijections of $\X$. 
    Then the set $S_0 \subset \poly{\K}{\X}$ of all polynomials 
    whose set of coefficients sums to $0$ is an equivariant ideal.
    Conversely, the set of all polynomials that are multiple
    of $x \in X$ is an \kl{ideal} that is not \kl{equivariant}.
\end{example}
\begin{proof}
    Let $p,q\in S_0$, and $r \in \poly{\K}{\X}$.
    Then, $p \times r + q$ is in $S_0$. Remark that 
    $p,r$ and $q$ belong to a subset $\poly{\K}{\Y}$ of the 
    polynomials that uses only finitely many indeterminates.
    In this subset, the sum of all coefficients is obtained
    by applying the polynomials to the value $1$ for every indeterminate
    $y \in \Y$. We conclude that
    $(p \times r + q)(1,\dots, 1) 
    = p(1,\dots,1) \times r(1,\dots,1) + q(1,\dots,1)
    = 0 \times r(1, \dots, 1) + 0 = 0$, hence that
    $p \times r + q$ belongs to $S_0$. 
    Because $0$ is in $S_0$, we conclude that $S_0$ is an \kl{ideal}.
    Furthermore, if $g \in \group$ and $p \in S_0$, then
    the sum of the coefficients $g \cdot p$ is exactly
    the sum of the coefficients of $p$, hence is $0$ too.
    This shows that $S_0$ is \kl(ideal){equivariant}.

    It is clear that all multiples of a given polynomial $x \in \X$
    is an ideal of $\poly{\K}{\X}$. This is not an \kl{equivariant ideal}:
    take any bijection $g \in \group$ that does not map $x$ to $x$ (it
    exists because $\X$ is infinite and $\group$ is all permutations),
    then $g \cdot x$ is not a multiple of $x$, and therefore does 
    not belong to the ideal.
\end{proof}

\AP A group action $\group \actson \X$ is said to be \intro{compatible} with an
ordering $\leq$ on $\X$ if for all $g \in \group$ and $x,y \in \X$, we have $x
\leq y$ if and only if $g \cdot x \leq g \cdot y$. We say that the action is
\intro{effectively oligomorphic} if \todo{do it}.

\AP Given a monomial $\monelt[m] \in \mon{\X}$, we define the \intro{domain} of
$\monelt[m]$ as the set $\dom(\monelt[m])$ of indeterminates $x \in \X$ such
that $\monelt[m](x) \neq 0$. Note that the domain of a monomial is equivariant:
if $g \in \group$, then $g \cdot \dom(\monelt[m]) = \dom(g \cdot \monelt[m])$.

\paragraph{Equivariant Gröbner bases.} \AP
We say that a set $\Basis \subseteq \poly{\K}{\X}$ is an
\intro{equivariant Gröbner basis} of an equivariant ideal $\idl$ and 
a partial ordering $\preceq$ on $\poly{\K}{\X}$ that is \intro{compatible with the action}
of $\group$ if
the following properties hold:
\begin{enumerate}
  \item $\Basis$ is a generating set of $\idl$, i.e. $\EqIdlGen{\Basis} = \idl$.
  \item For every polynomial $p \in \poly{\K}{\X}$, we have
    $p \toeucl{\preceq}{\Basis}^* 0$ if and only if $p \in \idl$.
  \item The relation $\toeucl{\preceq}{\Basis}$ is \kl{terminating}.
  \item The relation $\toeucl{\preceq}{\Basis}$ is \kl{confluent}.
\end{enumerate}


\paragraph{Orderings.} \AP In order to guarantee the termination of the
rewriting relation, a key ingredient will be the notion of
\intro{well-quasi-ordering} (WQO), that are sets $(X, \leq)$
such that every infinite sequence $\seqof{x_i}[i \in
\N]$ of elements of $X$ contains a pair $i < j$ such that $x_i \leq x_j$.

A classical example of a \kl{WQO} is the set of monomials $\mon{\X}$, endowed
with the \kl{divisibility} relation $\divleq$ whenever $\X$ is finite. We
recall that a monomial $\monelt[m]$ \intro{divides} a monomial $\monelt[n]$ if
there exists a monomial $\monelt[l]$ such that $\monelt[m] \times \monelt[l] =
\monelt[n]$. In this case, we write $\monelt[m] \divleq \monelt[n]$. Note that
monomials can be seen as functions from $\X$ to $\N$ having a finite support,
and that the \kl{divisibility} relation can be extended to monomials that are
functions from $\X$ to $(X,\leq)$, where $X$ is any partially ordered set. In
this case, we write $\monelt[m] \divleq \monelt[n]$ if for every $x \in \X$, we
have $\monelt[m](x) \leq \monelt[n](x)$. We will write
$\intro*\mon[\omega+1]{\X}$ (resp. $\mon[\omega+\omega]{\X}$) for the set of
monomials that are functions from $\X$ to $\omega + 1$ (resp. $\omega +
\omega$).


We know from \cite{GHOLAS24} that a necessary condition for the \kl{equivariant
Hilbert basis property} to hold is that the set  $\mon{\X}$  of monomials is a
\kl{well-quasi-ordering} when endowed with the \intro{divisibility up-to
$\group$} relation ($\intro*\gdivleq$), which is defined as follows: for
$\monelt_1, \monelt_2 \in \mon{\X}$, we write $\monelt_1 \gdivleq \monelt_2$ if
there exists $\gelem \in \group$ such that $\monelt_1$ \kl{divides} $\gelem
\cdot \monelt_2$. Let us recall that a monomial $\monelt[m]$ \intro{divides} a
monomial $\monelt[n]$ if there exists a monomial $\monelt[l]$ such that
$\monelt[m] \times \monelt[l] = \monelt[n]$. This relation also extends to
monomials that are functions from $\X$ to $(X,\leq)$ with finite support, where
$X$ is any partially ordered set.

