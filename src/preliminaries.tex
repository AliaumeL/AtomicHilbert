%!TEX root = ../atomic.tex
% LTeX: language=en
\section{Preliminaries}
\label{sec:preliminaries}

\paragraph{Orders.}
We assume basic familiarity with partial orders and ordinals.
In particular, we will use the notation $\intro*\om$ 
for the first infinite ordinal (that is, $(\N, \leq)$), and
write $X \intro*\ordplus Y$ for the lexicographic sum of two
partial orders $X$ and $Y$. We will also use the usual 
notations for finite ordinals, writing
$\intro*\ordfin{n}$ for the finite ordinal of size $n$.
For instance, $\om \ordplus \ordfin{1}$ is the 
total order $\N \uplus \set{+\infty}$, where $+\infty$ is a (new) 
largest element.


\paragraph{Group actions and equivariant ideals.}  A \intro{group}
$\group$ is a set equipped with a binary operation that is associative, has an
identity element and has inverses. In our setting, we are interested in
infinite sets $\X$ of indeterminates that is equipped with a \intro{group
action} $\group \actson \X$. This means that for each $g \in \group$,
we have a bijection $\X \tobij \X$ that we denote by $x \mapsto g \cdot x$. A
set $S \subseteq \X$ is \intro{equivariant} under the action of $\group$ if for
all $g \in \group$ and $x \in S$, we have $g \cdot x \in S$.
We give in \cref{ex:idl-equiv} an example and a
non-example of \kl{equivariant} \kl{ideals}.
Given a set $H$ of generators,
the \kl{equivariant ideal generated by} $H$
can alternatively be decribed
as 
 the set of all polynomials that can be written as
follows: $\,\sum_{i=1}^n q_i (\gelem_i \cdot s_i)$ for some $n \in \N$, $\gelem_i
\in \group$, $q_i \in \mon{\X}$, and $s_i \in H$.


\begin{example}
    \label{ex:idl-equiv}
    Let $\X$ be any infinite set, and $\group$ be the 
    group of all bijections of $\X$. 
    Then the set $S_0 \subset \poly{\K}{\X}$ of all polynomials 
    whose set of coefficients sums to $0$ is an equivariant ideal.
    Conversely, the set of all polynomials that are multiple
    of $x \in X$ is an \kl{ideal} that is not \kl{equivariant}.
\end{example}
\begin{proof}
    Let $p,q\in S_0$, and $r \in \poly{\K}{\X}$.
    Then, $p \times r + q$ is in $S_0$. Remark that 
    $p,r$ and $q$ belong to a subset $\poly{\K}{\Y}$ of the 
    polynomials that uses only finitely many indeterminates.
    In this subset, the sum of all coefficients is obtained
    by applying the polynomials to the value $1$ for every indeterminate
    $y \in \Y$. We conclude that
    $(p \times r + q)(1,\dots, 1) 
    = p(1,\dots,1) \times r(1,\dots,1) + q(1,\dots,1)
    = 0 \times r(1, \dots, 1) + 0 = 0$, hence that
    $p \times r + q$ belongs to $S_0$. 
    Because $0$ is in $S_0$, we conclude that $S_0$ is an \kl{ideal}.
    Furthermore, if $g \in \group$ and $p \in S_0$, then
    the sum of the coefficients $g \cdot p$ is exactly
    the sum of the coefficients of $p$, hence is $0$ too.
    This shows that $S_0$ is \kl(ideal){equivariant}.

    It is clear that all multiples of a given polynomial $x \in \X$
    is an ideal of $\poly{\K}{\X}$. This is not an \kl{equivariant ideal}:
    take any bijection $g \in \group$ that does not map $x$ to $x$ (it
    exists because $\X$ is infinite and $\group$ is all permutations),
    then $g \cdot x$ is not a multiple of $x$, and therefore does 
    not belong to the ideal.
\end{proof}

As explained in the introduction, we will assume that the set of indeterminates
is equipped with an \kl{effective linear order} $\leq$, that is \intro{compatible with
the group action}, that is, such that for all $\gelem
\in \group$ and $x,y \in \X$, we have $x \leq y \iff \gelem \cdot x \leq \gelem \cdot y$.



\paragraph{Gröbner Bases.} \AP
Let us now give a brief overview of the theory of Gröbner bases, and their 
adaptations in the equivariant setting. In particular, the following 
will be crucial to understand the algorithm we will present in \cref{sec:algorithm}.

In order to obtain a decision procedure for the ideal membership problem, a natural approach
is to generalise the Euclidean division to the case of polynomials with multiple
indeterminates. Namely, 
we write $p \to_{q} r$ if there exists $\monelt \in \mon{\X}$ such that
\begin{equation}
    \label{eq:euclidian}
    p = q \cdot \monelt + r
  \end{equation}

\begin{itemize}
  \item Generalized euclidian division
  \item Order on polynomials that is wqo
  \item Abstract \kl{Gröbner bases} (rewriting is complete and confluent)
  \item Monomial orderings / admissible orderings
  \item Gröbner bases and their equivalent characterizations
  \item Critical pairs and S-polynomials + Buchberger's algorithm
\end{itemize}

Given a total ordering $\order$ on $\X$, one can
consider the \intro{monomial ordering} $\revlex$ on $\mon{\X}$ defined by $M_1
\revlex M_2$ if and only if there exists $x \in \dom(M_1) \cap \dom(M_2)$ such
that $M_1(x) < M_2(x)$ and for all $x \order y$, we have $M_1(y) = M_2(y)$

This is in turned used to define the notions of \intro{leading monomial},
\intro{leading coefficient}, \intro{leading term}, and \intro{characteristic
monomial}.
\begin{description}
    \item[The leading monomial] of a polynomial $p$ is the largest monomial
        appearing in $p$ with respect to $\revlex$.
    \item[The leading coefficient] of a polynomial $p$ is the coefficient of
        its leading monomial.
    \item[The leading term] of a polynomial $p$ is the product of its leading
        coefficient and its leading monomial.
    \item[The characteristic monomial] of a polynomial $p$ is the monomial
        obtained by multiplying its \kl{leading monomial} with all indeterminates
        appearing in $p$.
\end{description}

A \intro{Gröbner basis} $\Basis$ of an ideal $I$ is a set of polynomials such
that for all $p \in I$, there exists $q \in \Basis$ such that the \kl{leading
monomial} of $q$ \kl{divides} the \kl{leading monomial} of $p$.
