%!TEX root = ../atomic.asmart.tex

\section{Strong Basis}\label{sec:strong}
%
\begin{definition}\label{def:strong}
For $T\subseteqfin\X$, a set $B\subseteq\poly{\K}{\X}$ is called a \kl{$T$-strong basis} for an equivariant ideal $I\subseteqfin\poly{\K}{\X}$ if for all $f\in I$ there exists $b\in B$ such that $\lm[T](b)$ divides $\lm[T](f)$.
The set $B$ is called a \kl{strong basis} for $I$ if it is a $T$-strong basis for $I$ for every $T\subseteqfin \X$.
\end{definition}
%
\begin{definition}
Extend the relation $\gdivleq$ (divisibility up to $\group$) to $\mon{X}{\times}\mon{X}$ as:
$(p,q)\gdivleq (p',q')$ if there exists $\pi\in\group$ such that $\pi(p)$ and $\pi(q)$ divides $p'$ and $q'$, respectively.
\end{definition}
%
\arka{Counters for algorithm and assumption do not follow the same numbering convention as other environments}
%
\begin{lemma}\label{lem:strong exists}
Under \Cref{assume:mon mon wqo}, every equivariant ideal in $\poly{\K}{\X}$ has an orbit-finite strong basis.
\end{lemma}
%
Recall the definitions of $\lm[T]$ and $\var()$ (\Cref{def:LM T})
%
\begin{definition}\label{def:cmp T f}
For $T\subseteqfin \X$ and $f\in \poly{\K}{\X}$ define $\varin{T}(f),\varout{T}(f)\subseteqfin \X$ as
\[
\varin{T}(f) \defined \var(f) \cap T
\qquad
\varout{T}(f) \defined \var(f) \setminus T \ .
\]
Let $\varin{T}(f) = \set{x_1,\dots,x_m} $, $\varout{T}(f) \defined \set{y_1,\dots,y_n}$.
Then,
\[
\lm[T](f) = x_1^{k_1} \cdot\dots\cdot x_m^{k_m} \cdot y_1^{\ell_1} \cdot\dots\cdot y_n^{\ell_n} \ .
\]
for some $k_1,\dots,k_m,\ell_1,\dots,\ell_n\in\N$.
Define $\cmin{T}(f),\cmout{T}(f)\in\mon{\X}$ as
\[
\cmin{T}(f) \defined x_1^{k_1 + 1} \cdot\dots\cdot x_m^{k_m + 1}
\qquad
\cmout{T}(f) \defined y_1^{\ell_1 + 1} \cdot\dots\cdot y_n^{\ell_n + 1} \ ,
\]
Finally, define
\[
\cmp[T](f) \defined (\cmin{T}(f),\cmout{T}(f)) \ .
\]
\end{definition}
%
\begin{lemma}\label{lem:cmp equiv}
For every $T\subseteqfin \X$, $f\in\poly{\K}{\X}$ and $\pi\in\group$ we have
\[
\pi(\cmp[T](f)) = \cmp[\pi(T)](\pi(f)) \ .
\]
\end{lemma}
%
\begin{proof}
Follows from the definition of $\cmp[T](f)$ (\Cref{def:cmp T f}) and the fact that $\pi(\var{f}\cap T) = \pi(\var(f))\cap \pi(T) = \var(\pi(f))\cap \pi(T)$.
\end{proof}
%
\begin{lemma}\label{lem:cmp equiv}
For every $T\subseteqfin \X$, $f\in\poly{\K}{\X}$ we have
\[
\cmp[T](f) = \cmp[T\cap \var(f)](f) \ .
\]
\end{lemma}
%
\begin{proof}
Follows from the definition of $\cmp[T](f)$ (\Cref{def:cmp T f}).
\end{proof}
%
\begin{proof}[Proof of \Cref{lem:strong exists}]
\arka{todo}
\end{proof}
%
\begin{lemma}
Every strong basis is also a Gr\"{o}bner basis.
\end{lemma}
%
\begin{definition}
For every orbit-finite $\Basis\subseteq \poly{\K}{\X}$ define
\[
\width(\Basis) \defined
\max_{b\in \Basis} |\var(f)| \ .
\]
\end{definition}
%
\begin{lemma}
For every orbit-finite $\Basis\subseteq \poly{\K}{\X}$,
$\width(\Basis)$ is finite.
\end{lemma}
%
\begin{proof}
Follows from the fact that for every $\pi\in\group$ and $f\in\poly{\K}{\X}$ we have
\[
\width(\pi(f)) = \width(f)
\]
\end{proof}
%
\begin{definition}
\arka{Define $\reducstep{\Basis,T}, \reduc{\Basis,T}, \rem[T]{\Basis}{f}$}
\end{definition}
%
Fix an orbit-finite set $\Basis\subseteq \poly{\K}{\X}$.
Define a sequence of subsets $(\Basis_n)_{n\in\N}$ of $\poly{\K}{\X}$ as
%
\begin{align*}
\Basis_0 &= \Basis \\
\Basis_{n + 1} &= \bigcup_{f,g\in\Basis_{n},
T\subseteq \var(f)\cup \var(g)}
\rem[T]{\Basis_{n}}{\spolyset(f,g)} \ .
\end{align*}
%
\begin{lemma}
$\IdlGen{\Basis} = \IdlGen{\Basis_0} =
\IdlGen{\Basis_1} = \dots$.
\end{lemma}
%
\begin{lemma}
There exists $N\in\N$ such that
$
\Basis_N = \Basis_{N+1} = \dots
$.
\end{lemma}
%
\begin{lemma}
If $\Basis_n = \Basis_{n+1}$,
then $\Basis_n$ is a strong basis of $\IdlGen{\Basis}$.
\end{lemma}
%
\begin{lemma}
$\Basis_{n}$ is orbit-finite for every $n\in\N$.
\end{lemma}
%