%!TEX root = ../atomic.asmart.tex
% LTeX: language=en
\section{Undecidability Results}
\label{sec:undecidability}

In this section, we will show that under certain conditions,
the \kl{equivariant ideal membership problem} is undecidable. We aim to show
that it is the case when we assume the usual effectiveness conditions on the
group action, but we do not assume that 
$(\mon{\Indets}, \gdivleq)$ is a well-quasi-ordering.
Beware though that there are some pathological cases where
the \kl{equivariant ideal membership problem} is undecidable even when
$(\mon{\Indets}, \gdivleq)$ is not a well-quasi-ordering,
as illustrated by the following example.

\begin{example}
  \label{ex:non-wqo-undecidable}
  Let $\Indets = \{x_1, x_2, \ldots\}$ be an infinite set of indeterminates,
  and let $\group$ be trivial group acting on $\Indets$.
  Then, the \kl{equivariant ideal membership problem} is decidable.
\end{example}
\begin{proof}
  Because the group is trivial, whenever one provides a finite set
  $H$ of generators of an \kl{equivariant ideal} $I$, one can
  in fact work in $\poly{\K}{V}$, where $V$ is the set of indeterminates
  that appear in $H$.
  Then, the \kl{equivariant ideal membership problem} is reduces to 
  the \kl{ideal membership problem} in $\poly{\K}{V}$, which is decidable.
\end{proof}

\subsection{Interpreting Paths}
\label{subsec:paths}

In this section, we will assume that the set of indeterminates $\Indets$
contains an infinite path $P \defined \seqof{x_i}[i \in \N] \subseteq \Indets$,
that is, a set of indeterminates such that $\group$ acts on $\Indets$ as a
translation of the indices, that is, for all $\gelem \in \group$ such that
$\gelem \cdot P \subseteq P$, there exists $n \in \N$ such that $\gelem \cdot
x_i = x_{i + n}$ for all $i \in \N$.

The following two lemmas show that the existence of an infinite path is a
sufficient condition to ensure that $(\mon{\Indets}, \gdivleq)$ is not a
\kl{well-quasi-ordering}, and that it is close to being a sufficient condition
in the case of finite relational structures.

\begin{lemma}
  If $\Indets$ contains an infinite path $P$, then the
  $(\mon{\Indets}, \gdivleq)$ is not a \kl{well-quasi-ordering}.
\end{lemma}
\begin{proof}
  Consider the monomials
  $x_0^2 x_1 \cdots x_n x_{n+1}^2$ for all $n \in \N$.
  They are pairwise incomparable with respect to $\gdivleq$,
  hence form an infinite antichain.
\end{proof}


\begin{conjecture}[Schmitz]
  Let $\mathcal{C}$ be a class of finite relational sturctures
  that is not $2$-well-quasi-ordered, then 
  there exists in existential formula $\phi(x,y)$ that defines
  arbitrarily long paths in the structures of $\mathcal{C}$.
\end{conjecture}

\begin{corollary}
  Assume that $\mathbb{A}$ is an infinite relational structure,
  and that its age is not $2$-well-quasi-ordered.
  Then, there exists an infinite structure $\mathbb{B}$
  that is elementarily equivalent to $\mathbb{A}$,
  and where an existential formula $\phi(x,y)$ defines
  an infinite path.
\end{corollary}
\begin{proof}
  It suffices to use the conjecture above and the compactness theorem 
  for first-order logic.
\end{proof}

\begin{corollary}
  Let $\mathbb{A}$ be an infinite relational structure,
  and assume that its age is not $2$-well-quasi-ordered.
  Then, if $\Indets$ is the elements of $\mathbb{A}$,
  and $\group$ is the automorphism group of $\mathbb{A}$,
  then one can assume that $\mathbb{A}$
  contains in infinite path.
\end{corollary}
\begin{proof}
  We first replace $\mathbb{A}$ by an infinite structure
  $\mathbb{B}$ that is elementarily equivalent to $\mathbb{A}$.
  Because we only ever write finite sets of indeterminates,
  equivariant ideal membership problems are the same in $\mathbb{A}$
  and $\mathbb{B}$.
  Finally, because of $\phi(x,y)$, we can add a binary relation to 
  $\mathbb{B}$ that defines an infinite path, without changing the
  \kl{equivariant ideal membership problem}.
\end{proof}



\begin{definition}
  \label{def:mon-rewrite-system}
  A \intro{monomial rewrite system} is a finite set of pairs of the form
  $(\monelt, \monelt')$ where $\monelt, \monelt' \in \mon{\Indets}$.
  The \intro{monomial reachability problem} is the problem of deciding whether
  there exists a sequence of rewrites that transforms $\monelt_s$ into $\monelt_t$
  using the rules of a monomial rewrite system $R$, where
  a \intro{rewrite step} is a pair of the form
  \begin{equation*}
    \monelt[n] (\gelem \cdot \monelt)
    \to_R 
    \monelt[n] (\gelem \cdot \monelt')
    \text{ if } (\monelt, \monelt') \in R
    \text{ and } \gelem \in \group
    \quad .
  \end{equation*}
\end{definition}


\begin{lemma}
  \label{lem:mon-rewrite-red-membership}
  One can solve the \kl{monomial reachability problem} in polynomial time
  provided that one can solve the \kl{equivariant ideal membership problem}.
\end{lemma}
\begin{proof}
  Let $R$ be a monomial rewrite system, and let $\monelt_s, \monelt_t \in
  \mon{\Indets}$ be two monomials. We can encode the problem of deciding whether
  $\monelt_s$ can be rewritten into $\monelt_t$ using the rules of $R$ as an
  instance of the \kl{equivariant ideal membership problem} as follows:
  \begin{itemize}
    \item Let $H$ be the set of all polynomials of the form $\monelt - \monelt'$
      for all pairs
      $(\monelt, \monelt') \in R$.
    \item Then, we ask whether $\monelt_s - \monelt_t$ belongs to the ideal generated by $H$.
  \end{itemize}

  It is clear that if $\monelt_s$ can be rewritten into $\monelt_t$ using the
  rules of $R$, then $\monelt_s - \monelt_t$ belongs to the equivariant ideal generated by
  $H$. Conversely, if $\monelt_s - \monelt_t$ belongs to the ideal generated by
  $H$, then 
  \begin{equation}
    \label{eq:mon-rewrite-red-membership}
    \monelt_s - \monelt_t 
    = 
    \sum_{i=1}^n a_i \monelt[n]_i (\gelem_i \cdot \monelt_i - \gelem_i \cdot \monelt'_i)
    \quad .
  \end{equation}

  Let us write the (finite) graph $G$ whose vertices are the monomials
  $\monelt[n] (\gelem_i \cdot \monelt_i)$ and $\monelt[n] (\gelem_i \cdot
  \monelt'_i)$, and whose edges are the directed weighted edges labelled by
  $a_i$ (in a direction that makes the weight positive).

  Let us now analyse \cref{eq:mon-rewrite-red-membership}, and notice that
  identifying monomials in the left and right-hand sides of the equation allows
  us to show that $\monelt_s$ and $\monelt_t$ are vertices of $G$. Furthermore,
  we deduce that the sum of the weights of the edges having $\monelt_s$ as a
  source or target equals $1$, and that the sum of the weights of the edges
  having $\monelt_t$ as a source or target equals $-1$. Finally, for every
  vertex $v$ of $G$ that is not $\monelt_s$ or $\monelt_t$, the sum of the
  weights of the edges having $v$ as a source or target is $0$, again because
  of an analysis of the coefficient of the monomial $v$ in the sum of
  \cref{eq:mon-rewrite-red-membership}.

  Hence, the graph $G$ is a flow network, with a flow value of at least $1$
  from $\monelt_s$ to $\monelt_t$. As a consequence, there must exist a path
  from $\monelt_s$ to $\monelt_t$ in $G$, which is a witness
  of the fact that 
  one can rewrite $\monelt_s$ into $\monelt_t$ using the rules of $R$.
\end{proof}

\AP It will be easier for us to work with rewriting systems than the
\kl{equivariant ideal membership problem} to prove undecidability results. In
particular, they will be particularly useful to encode the \intro{Post
Correspondence Problem} (PCP), which is a well-known undecidable problem that
can be stated as follows: given two finite sets of words $A \defined \set{u_1,
\dots, u_n}$ and $B \defined \set{v_1, \dots, v_n}$ over a binary alphabet,
does there exist a sequence of indices $i_1, \ldots, i_n$ such that $u_{i_1}
\cdots u_{i_n} = v_{i_1} \cdots v_{i_n}$. We can further assume that the
$i_1 = 1$.

\AP To simplify our translation, let us first assume that one can use colored
variables, that is, one has access to monomials with exponents in $\N \times Q$
for some finite set $Q$. We will use $Q \defined \set{ \mathsf{u}, \mathsf{v},
\mathsf{endU}, \mathsf{endV} }$. Then, we can encode a word $u$ over the binary
alphabet $\set{0, 1}$ as a monomial $\monelt[w]$ as follows: $\mathsf{encU}_{i}
(\varepsilon) = x_i^{(1, \mathsf{endU})}$ and $\mathsf{encU}_{i} (a u') =
x_i^{(1 + a, \mathsf{u})} \cdot \mathsf{encU}_{i+1} (u')$. Similarly, one can
define a monomial $\mathsf{encV}_{i} (v)$ that encodes a word $v$ over the
binary alphabet $\set{0, 1}$ as a monomial in the same way. Then, given a pair
of words $u \in A$ and $v \in B$, we can encode their pairing as a monomial
as follows: $\mathsf{enc} (u,v) = \mathsf{encU}_0(u) \cdot
\mathsf{encV}_0 (v)$.

Let us now define a rewriting system $R$ that encodes the PCP problem. The
rules of $R$ are separated in two parts:
\begin{itemize}
  \item A forward part: that replaces
    $x_i^{\alpha} \cdot x_{j+1}^{\mathsf{endU}} \cdot x_{j}^{\beta} x_{j+1}^{\mathsf{endV}}$
    by 
    $x_i^{\alpha} \mathsf{encU}_{i+1}(u_k) \cdot x_{j}^{\beta} \mathsf{encV}_{j+1}(v_k)$
    for some $k \in \set{1, \ldots, n}$,
    where $i \neq j$,
    and $\alpha, \beta$ are non-empty exponents.
  \item A backward part: that replaces
    $x_i^{(a, \mathsf{u})}
     x_i^{(a, \mathsf{v})}
     x_{i+1}^{\mathsf{endU}}
     x_{i+1}^{\mathsf{endV}}$
    by 
     $x_{i}^{\mathsf{endU}}
      x_{i}^{\mathsf{endV}}$, 
    where $a \in \set{1, 2}$.
\end{itemize}

\AP The set of rules defined here is infinite, but it is finitely generated,
and a representation of the rules can be computed from the PCP instance because
$\Indets$ is \kl{effectively oligomorphic}. It is clear that if there is a
positive answer to the PCP problem, then there exists a sequence of rewrites
that transforms the monomial $\mathsf{enc} (u_1, v_1)$ into the monomial
$x_0^{(1, \mathsf{endU})} \cdot x_0^{(1, \mathsf{endV})}$, that is obtained by
replacing the end markers of the words until the matching is complete, and then
removing letters by applying the backward rules until both end markers reach
the start of the word.
The converse implication is slightly more involved, 
and is stated in \cref{lem:reachable-correct}.

\begin{lemma}
  \label{lem:reachable-correct}
  Assume that $\mathsf{enc} (u_1, v_1)$ can be rewritten into
  $x_0^{(1, \mathsf{endU})} \cdot x_0^{(1, \mathsf{endV})}$ using the rules of
  the rewriting system $R$ defined above.
  Then there is a positive answer to the PCP problem.
\end{lemma}
\begin{proof}

  We first remark that the backward rules can only be applied when the two end
  markers are on the same indeterminate, and that in this case, the two end
  markers are replaced by the same indeterminate. On the other hand, the
  forward rules can only be applied when the two end markers are on different
  indeterminates. As a consequence, the rewrite sequence must be starting with
  forward rules, and ending with backward rules.

   It is an easy check that reaching the final monomial by applying only the
   backward rules can only be obtained from monomials where every variable
   $x_i$ has the same $u$ and $v$ exponents.

   It suffices now to prove that if the forward rules create a monomial
   where every variable $x_i$ has the same $u$ and $v$ exponents, then
   there is a solution to the PCP problem.

   We will first analyse the shape of the monomials obtained by applying 
   only forward rules, and claim that they are \emph{path-like}
   in the sense that no rule can introduce a variable that was already present
   in the monomial (except for the end markers, that are moved)
   and that the variables in the reachable monomials form paths.

   \todo[inline]{
     There is a problem: we do not ensure that the reachable monomials
     are path like. We can make sure of it by adding a check
     rule that moves a cursor along the monomial starting from
     the beginning of the word.
   }

\end{proof}

\begin{theorem}
  \label{thm:undecidable-pcp}
  One can reduce the \kl{Post Correspondence Problem} to the
  \kl{monomial reachability problem}, assuming that one
  has access to colored variables.
\end{theorem}
\begin{proof}
  
\end{proof}


Now, let us argue that one can get rid of colors by using 
finitely many numbers to encode them. 
The first remark is that some colors (most) appear only once 
in the monomial and can be replaced by a unique (big) number.
The second remark is that the encoding of $u$ and $v$ can be interleaved
instead of using the same variables for both and distinguishing them
by their color. 

\subsection{Infinite Vector Space}
\label{subsec:vector}





