%!TEX root = ../atomic.asmart.tex
% LTeX: language=en
\section{Undecidability Results}
\label{sec:undecidability}

In this section, we will show that under certain conditions, the
\kl{equivariant ideal membership problem} is undecidable. We aim to show that
it is the case when we assume the usual effectiveness conditions on the group
action, but we do not assume that $(\mon{\Indets}, \gdivleq)$ is a
well-quasi-ordering. Beware though that there are some pathological cases where
the \kl{equivariant ideal membership problem} is undecidable even when
$(\mon{\Indets}, \gdivleq)$ is not a well-quasi-ordering, as illustrated by the
following \cref{ex:non-wqo-undecidable}.

\begin{example}
  \label{ex:non-wqo-undecidable}
  Let $\Indets = \{x_1, x_2, \ldots\}$ be an infinite set of indeterminates,
  and let $\group$ be trivial group acting on $\Indets$.
  Then, the \kl{equivariant ideal membership problem} is decidable.
\end{example}
\begin{proof}
  Because the group is trivial, whenever one provides a finite set
  $H$ of generators of an \kl{equivariant ideal} $I$, one can
  in fact work in $\poly{\K}{V}$, where $V$ is the set of indeterminates
  that appear in $H$.
  Then, the \kl{equivariant ideal membership problem} is reduces to 
  the \kl{ideal membership problem} in $\poly{\K}{V}$, which is decidable.
\end{proof}

Avoiding pathological cases such as the one in
\cref{ex:non-wqo-undecidable}, we can give one positive and one
negative examples that are designed to illustrate the results we will present
in this section. On the positive side, one can consider $\Indets_\Q$, the set
of all rationals equipped with their usual ordering, and the group of all
bijections of $\Indets_\Q$ that preserve the ordering
(\cref{ex:q-is-super-wqo}). While on the negitive side, one can
consider $\Indets_\Z$, the set of all integers equipped with their usual
ordering, and the group of all shifts of integers
(\cref{ex:z-is-not-wqo}).

\begin{example}
  \label{ex:q-is-super-wqo}
  The set $(\mon[Y]{\Indets_\Q}, \gdivleq)$ is a \kl{well-quasi-ordering} whenever $Y$ is
  one, and in particular $\Indets_\Q$ satisfies the computability assumptions and
  the termination assumptions of both \cref{thm:compute-egb}
  and
  \cref{thm:decide-equiv-ideal-mem}.
\end{example}
\begin{proof}
  Let $\seqof{\monelt_i}[i \in \N]$ be a sequence of monomials in
  $\mon[Y]{\Indets_\Q}$. Let us write each monomial $\monelt_i$ as
  a finite word $w_i$ over the alphabet $Y$, by writing all the exponents in the order 
  prescribed by the indeterminates.
  Because $Y$ is a \kl{well-quasi-ordering}, the set of all finite words over $Y$ is
  \kl{well-quasi-ordered} by the \emph{scattered subword} relation \cite{HIG52}.
  Now, if $w_i$ is a scattered subword of $w_j$, then
  $\monelt_i \gdivleq \monelt_j$, by choosing a suitable $\gelem \in \group$.
\end{proof}

\begin{example}
  \label{ex:z-is-not-wqo}
  The set $(\mon[Y]{\Indets_\Z}, \gdivleq)$ is not a \kl{well-quasi-ordering} whenever $Y$ is
  contains two distinct elements.
  In particular, $\Indets_\Z$ does not satisfy the termination assumptions of
  \cref{thm:compute-egb} and \cref{thm:decide-equiv-ideal-mem}.
\end{example}
\begin{proof}
  Assume that $Y$ has two distinct elements $a$ and $b$, and let us assume without loss of generality
  that $a \leq b$. The sequence of monomials 
  $\monelt_i \defined x_1^{b} x_2^{a} \cdots x_{i-1}^{a} x_i^{b}$
  forms an infinite antichain in $(\mon[Y]{\Indets_\Z}, \gdivleq)$.
  Indeed, if $\monelt_i \gdivleq \monelt_j$ for some $i < j$, then
  without loss of generality, $\gelem_i (x_1) = x_1$, and 
  $\gelem_i (x_i) = x_j$, because these are the only ones that can be 
  equipped with a large enough exponent.
  Therefore, $\gelem_i = \mathrm{id}$ since the group only contains translations.
  However, this implies that the exponent of $x_j$ in $\monelt_j$ is at most $a$,
  which contradicts the fact that it is $b$.
\end{proof}

\subsection{Monomial Reachability}

The undecidability results we will present in this section regarding the
\kl{equivariant ideal membership problem} will use the polynomials in a very
limited way: we will only need to consider \emph{monomials}, and there will
even be a bound on the maximal exponent used. Before going into the details of
our reductions, let us first introduce an intermediate problem that will be
easier to work with: the \kl{monomial reachability problem}.

\begin{definition}
  \label{def:mon-rewrite-system}
  A \intro{monomial rewrite system} is a finite set of pairs of the form
  $\set{\monelt, \monelt'}$ where $\monelt, \monelt' \in \mon{\Indets}$.
  The \intro{monomial reachability problem} is the problem of deciding whether
  there exists a sequence of rewrites that transforms $\monelt_s$ into $\monelt_t$
  using the rules of a monomial rewrite system $R$, where
  a \intro(monrew){rewrite step} is a pair of the form
  \begin{equation*}
    \monelt[n] (\gelem \cdot \monelt)
    \leftrightarrow_R 
    \monelt[n] (\gelem \cdot \monelt')
    \text{ if } \set{\monelt, \monelt'} \in R
    \text{ and } \gelem \in \group
    \quad .
  \end{equation*}
\end{definition}

\begin{example}
  \label{ex:mon-rewrite-system}
  Let $\Indets = \N$ and $\group$ be the set of all bijections of $\Indets$.
  Then, the rewrite system $x_1^2 x_2^2 \leftrightarrow_R x_1^2$
  satisfies $\monelt \leftrightarrow_R^* x_1^2$ if and only if 
  $\monelt$ has all its exponents that are multiple of $2$.
\end{example}


\begin{lemma}
  \label{lem:mon-rewrite-red-membership}
  One can solve the \kl{monomial reachability problem}
  provided that one can solve the \kl{equivariant ideal membership problem}.
\end{lemma}
\begin{proof}
  Let $R$ be a monomial rewrite system, and let $\monelt_s, \monelt_t \in
  \mon{\Indets}$ be two monomials. We can encode the problem of deciding whether
  $\monelt_s$ can be rewritten into $\monelt_t$ using the rules of $R$ as an
  instance of the \kl{equivariant ideal membership problem} as follows:
  \begin{itemize}
    \item Let $H$ be the set of all polynomials of the form $\monelt - \monelt'$
      for all pairs
      $(\monelt, \monelt') \in R$.
    \item Then, we ask whether $\monelt_s - \monelt_t$ belongs to the ideal generated by $H$.
  \end{itemize}

  It is clear that if $\monelt_s$ can be rewritten into $\monelt_t$ using the
  rules of $R$, then $\monelt_s - \monelt_t$ belongs to the equivariant ideal generated by
  $H$. Conversely, if $\monelt_s - \monelt_t$ belongs to the ideal generated by
  $H$, then 
  \begin{equation}
    \label{eq:mon-rewrite-red-membership}
    \monelt_s - \monelt_t 
    = 
    \sum_{i=1}^n a_i \monelt[n]_i (\gelem_i \cdot \monelt_i - \gelem_i \cdot \monelt'_i)
    \quad .
  \end{equation}

  Let us write the (finite) graph $G$ whose vertices are the monomials
  $\monelt[n] (\gelem_i \cdot \monelt_i)$ and $\monelt[n] (\gelem_i \cdot
  \monelt'_i)$, and whose edges are the directed weighted edges labelled by
  $a_i$ (in a direction that makes the weight positive).

  Let us now analyse \cref{eq:mon-rewrite-red-membership}, and notice that
  identifying monomials in the left and right-hand sides of the equation allows
  us to show that $\monelt_s$ and $\monelt_t$ are vertices of $G$. Furthermore,
  we deduce that the sum of the weights of the edges having $\monelt_s$ as a
  source or target equals $1$, and that the sum of the weights of the edges
  having $\monelt_t$ as a source or target equals $-1$. Finally, for every
  vertex $v$ of $G$ that is not $\monelt_s$ or $\monelt_t$, the sum of the
  weights of the edges having $v$ as a source or target is $0$, again because
  of an analysis of the coefficient of the monomial $v$ in the sum of
  \cref{eq:mon-rewrite-red-membership}.

  Hence, the graph $G$ is a flow network, with a flow value of at least $1$
  from $\monelt_s$ to $\monelt_t$. As a consequence, there must exist a path
  from $\monelt_s$ to $\monelt_t$ in $G$, which is a witness
  of the fact that 
  one can rewrite $\monelt_s$ into $\monelt_t$ using the rules of $R$.
\end{proof}

In order to show that the \kl{equivariant ideal membership problem} is
undecidable, it is therefore enough to show that the \kl{monomial reachability
problem} is undecidable. To that end, we will encode the Halting problem of a
Turing machine. There are two main obstacles to overcome: first, the
reversibility of the rewriting system, which can be (partially) solved by
considering \emph{reversible} Turing machines; and second, the fact that the
configurations of the Turing machine cannot staightforwardly be encoded as
monomials due to the commutativity of the multiplication.
To overcome the second issue, we will use the following notion of 
\kl{word encoding}.


\begin{definition}
  \label{def:word-encoding}
  Let $\Sigma$ be a finite alphabet.
  A \intro{word encoding} is a function $\intro*\wenc{\cdot} \colon \Sigma^* \to \mon{\Indets}$ such
  that:
  \begin{enumerate}
    \item For all words $u, v \in \Sigma^*$,
      there exists a $\gelem \in \group$ such
      that $\pi \cdot \wenc{u} \divleq \wenc{v}$
      if and only if
      there exists $x, y \in \Sigma^*$ such that 
      $x u y = v$. 
    \item 
      Furthermore, for every other word
      $u'$ of the same size as $u$,
      $\wenc{x u' y} (\gelem \cdot \wenc{u}) = \wenc{x u y} (\gelem \cdot \wenc{u'})$.
  \end{enumerate}
\end{definition}

The first condition of \cref{def:word-encoding} ensures that
the \kl{word encoding}  correctly relates the \emph{infix} relation on words
with the divisibility up-to the group action $\group$. The second condition
ensures that the \emph{substitutions} of infixes having the same size can be
modelled using \kl{monomial rewriting}. It is therefore straightforward to
simulate the runs of reversible Turing machines using \kl{monomial rewriting}
when one has access to a suitable \kl{word encoding}.

\begin{remark}
  \label{rem:alphabet-size}
  If there exists a \kl{word encoding} $\wenc{\cdot}$ for a binary alphabet $\Sigma$,
  then one can construct a \kl{word encoding} for any finite alphabet $\Sigma'$
  using a suitable binary encoding. The only difficulty is to ensure unique 
  parsability of the encoding, which can be done using unambiguous codes.
\end{remark}

\begin{remark}
  \label{rem:not-wqo}
  If there exists a \kl{word encoding} for a binary alphabet $\Sigma$, then
  $(\mon{\Indets}, \gdivleq)$ is not a \kl{well-quasi-ordering}.
  Indeed, the infix ordering relation on words over a binary alphabet is
  not a \kl{well-quasi-ordering}, as shown by the infinite antichain
  $\setof{ a b^n a}{n \in \N }$,
  and the \kl{word encoding} $\wenc{\cdot}$ is an order embedding of
  the infix ordering relation on words over $\Sigma$ into $(\mon{\Indets}, \gdivleq)$.
\end{remark}


Because of \cref{rem:alphabet-size}, we will be interested in the
existence of \kl{word encodings} for arbitrary finite alphabets, which is the
same as the existence of a \kl{word encoding} for a binary alphabet. Given a
reversible Turing machine $M$ with a finite set $Q$ of states and tape alphabet
$\Sigma$, we will consider the following alphabet $\Gamma \defined \set{
\triangleleft, \triangleright } \times \set{ \text{pre}, \text{run},
\text{post} } \uplus Q \uplus \Sigma \uplus \set{ \square, \square_1, \square_2}$. The letter
$\square$ is a blank symbol, and the letters $\triangleleft$ and
$\triangleright$ are used to delimit the beginning and the end of the tape,
with some extra ``phase information''.

\begin{lemma}
  \label{lem:rev-turing-machine}
  There exists a
  \kl{monomial rewrite system} $R_M$ such that the following
  are equivalent for every $n \geq 1$:
  \begin{enumerate}
    \item $\wenc{ \triangleright^{\text{pre}} \square^n 
                  \triangleleft^{\text{pre}}
     } \leftrightarrow_{R_M}^* 
     \wenc{ \triangleright^{\text{post}} \square^n 
                  \triangleleft^{\text{post}} }$,
      \item $M$ halts on the empty word using a tape bounded by $n-1$ cells.
  \end{enumerate}
  Furthermore, every monomial that is 
  reachable from $\wenc{ \triangleright^{\text{pre}} \square^n \triangleleft^{\text{pre}} }$
  or $\wenc{ \triangleright^{\text{post}} \square^n \triangleleft^{\text{post}} }$
  is the image of a word of the form
  $\wenc{\triangleright^{\text{run}} u \triangleleft^{\text{run}}}$  
  where $u \in (Q \uplus \Sigma \uplus \square)^n$.
\end{lemma}
\begin{proof}
  The rewrite system simply acts on the tape of the reversible Turing machine 
  using blank symbols. Because transitions of the reversible Turing machine
  are substitutions of strings having the same size if one does not create new
  tape cells, the rewriting system can straigthforwardly simulate the 
  substitutions because of \cref{def:word-encoding}.
  To this monomial rewriting system, we add two rules,
  respectively of the form
  $\wenc{ \triangleright^{\text{pre}} \square^n \triangleleft^{\text{pre}} }
  \leftrightarrow_{R_M}^*
  \wenc{ \triangleright^{\text{run}} q_0 \square^{n-1} \triangleleft^{\text{run}} }$
  and 
  $\wenc{ \triangleright^{\text{post}} \square^n \triangleleft^{\text{post}} }
  \leftrightarrow_{R_M}^*
  \wenc{ \triangleright^{\text{run}} q_f \square^{n-1} \triangleleft^{\text{run}} }$,
  where $q_0$ is the initial state of the Turing machine $M$ and $q_f$ is its
  final state.
  This is not problematic because one can 
  simply write $\wenc{\triangleright^{\text{run}} q_0} (\gelem \cdot \wenc{\triangleleft^{\text{run}}})$
  for a suitable $\gelem \in \group$, in order
  to ignore the number of blank symbols in the tape and assume that the reversible Turing machine
  starts with a clean tape and ends with a clean tape.
\end{proof}

\begin{lemma}
  \label{lem:tape-creation}
  There exists a \kl{monomial rewrite system} $R_\text{pre}$
  such that for every monomial $\monelt \in \mon{\Indets}$, the following are
  equivalent:
  \begin{enumerate}
    \item $\wenc{ \triangleright^{\text{pre}} \square \triangleleft^{\text{pre}}} 
      \leftrightarrow_{R_\text{pre}}^* 
      \monelt$
    \item there exists $n \in \N$ such that
      $\monelt = \wenc{ \triangleright^{\text{pre}} \square^n 
                        \triangleleft^{\text{pre}} }$.
  \end{enumerate}
  Similarly, there exists a \kl{monomial rewrite system} $R_\text{post}$
  with analogue properties.
\end{lemma}
\begin{proof}
  The rewrite system is going to have two phases.
  In a first phase, we will consider the rules
  $\wenc{ \triangleleft^{\text{pre}} } \leftrightarrow_{R_\text{pre}}
   \wenc{ \square \triangleleft^{\text{pre}} }$.
  This will create a long \emph{tape like} monomial.
  However, there is no guarantee that the reachable monomials are linear, i.e., that we do not
  reuse the same indeterminate twice in our reduction.
  To ensure that we do obtain a correct monomial, we will implement the 
  Floyd cycle-finding algorithm, which uses two pointers moving at different speeds.
  To that end, we will use two new blank symbols $\square_1$ and $\square_2$,
  and the rules 
  $\wenc{ \square_1 \square } ( \gelem \cdot \wenc{ \square_2 \square \square })
  \leftrightarrow_{R_\text{pre}}
  \wenc{ \square \square_1 } (\gelem \cdot \wenc{ \square \square \square_2 })$,
  where $\pi$ is a permutation of the indeterminates that ensures that the
  indetermitates of the two words are distinct.
  Finally, one adds the possibility to switch to the sencond phase,
  and end the second phase when pointer $\square_2$ reaches the end of the tape.
\end{proof}

\begin{corollary}
  \label{cor:undecidability}
  Let $\Indets$ be a set of indeterminates, and let $\group$ be a group acting
  on $\Indets$ such that the action is effective. If there exists a
  \kl{word encoding} for a binary alphabet, then the \kl{equivariant ideal
  membership problem} is undecidable.
\end{corollary}
\begin{proof}
  It suffices to combine the rewriting systems $R_M$, $R_\text{pre}$ and 
  $R_\text{post}$ by taking their union.
\end{proof}

Now that we have obtained a general proof scheme to obtain the undecidability
of the \kl{equivariant ideal membership problem}, let us illustrate more
concrete conditions on the group action $\group$ that allow constructing such
\kl{word encodings}.

\subsection{Interpreting Paths}
\label{subsec:paths}

In this section, we will assume that the set of indeterminates $\Indets$
contains an \intro(of){infinite path} $P \defined \seqof{x_i}[i \in \N]
\subseteq \Indets$, that is, a set of indeterminates such that $\group$ acts on
$\Indets$ as a translation of the indices. Formally, we require that, for all
$\gelem \in \group$ and for all segments $J \defined [k,l] \subfin \N$
such that $\gelem \cdot x_j \subseteq P$ for all $j \in J$, there exists $n \in
\N$ such that $\gelem \cdot x_j = x_{j + n}$ for all $j \in J$ ; and there
exists a $\gelem_{+1} \in \group$ such that $\gelem_{+1} \cdot x_i = x_{i + 1}$ for all
$i \in \N$.

Let us now define a \kl{word encoding} for a binary alphabet $\Sigma =
\set{a,b}$, leveraging the existence of such an \kl(of){infinite path} $P$. We
define $\wenc{\varepsilon} \defined 1$, and $\wenc{a u} \defined x_0^{3} x_1^{2} x_2 x_3^3
(\gelem_{+4} \cdot \wenc{u})$ and
$\wenc{b u} \defined x_0^{2} x_1^{3} x_2^2 x_3 (\gelem_{+4} \cdot \wenc{u})$ for all
$u \in \Sigma^*$. 

\begin{lemma}
  \label{lem:word-encoding-paths}
  The function $\wenc{\cdot}$ defined above is a \kl{word encoding} for the
  binary alphabet $\Sigma = \set{a,b}$.
\end{lemma}
\begin{proof}
  Let $u, v \in \Sigma^*$ be two words.
  Assume that there exists $x,y$ such that $x u y = v$.
  Then, it as clear that $\wenc{v} = \monelt (\gelem_{+n} \cdot \wenc{u})$ for some
  $n \in \N$, hence $\wenc{u} \gdivleq \wenc{v}$.
  Furthermore, if $u'$ is another word of the same size as $u$, then
  $\wenc{x u' y} = \monelt' (\gelem_{+n} \cdot \wenc{u'})$ for the same $n \in \N$,
  hence
  $\wenc{x u' y} (\gelem_{+n} \cdot \wenc{u}) = \wenc{x u y} (\gelem_{+n} \cdot \wenc{u'})$.

  Conversely,
  assume that $\wenc{u} \gdivleq \wenc{v}$.
  Then, there exists a $\gelem \in \group$ such that
  $\wenc{v} = \monelt (\gelem \cdot \wenc{u})$ for some monomial $\monelt$.
  In particular, $\gelem$ maps the elements of $\wenc{u}$ to $P$,
  hence, there exists $n \in \N$ such that
  $\gelem \cdot x_i = x_{i + n}$ for all $i$ such that $x_i$ appears in $\wenc{u}$.
  It follows that 
  $\wenc{v} = \monelt (\gelem_{+n} \cdot \wenc{u})$.

  Now, because of the exponents in the definition of $\wenc{\cdot}$, the $n$
  must be a multiple of $4$, and the exponents of the indeterminates in every
  block of four consecutive indeterminates of the form $x_i^3 x_{i+1}^\alpha
  x_{i+2}^\beta x_{i+3}^3$ must be such that $(\alpha, \beta) \in \set{(2,1),
  (1,2)}$. It follows that the letters $a$ and $b$ must be the same in the 
  words $u$ and the word $v[n / 4, n/ 4 + |u| - 1]$. In particular,
  there exists $x, y \in \Sigma^*$ such that $x u y = v$.
\end{proof}

As an immediate consequence of \cref{lem:word-encoding-paths}, we
conclude that the \kl{equivariant ideal membership problem} is undecidable when
variables are $\Indets_\Z$ equipped with the group of shifts, as described in
\cref{ex:z-is-not-wqo}.

\subsection{Infinite Dimensional Vector Space}
\label{subsec:vector}

\AP In this section, we will show that the \kl{equivariant ideal membership
problem} is undecidable for the \intro{infinite dimensional vector space}
$\Indets \defined \mathbb{V}$, that is, an infinite dimensional vector space,
where the group $\group$ acting on $\Indets$ is the group of linear
automorphisms of $\mathbb{V}$.

It is unclear whether the \kl{infinite dimensional vector space} contains an
\kl(of){infinite path} $P$ as defined in
\cref{subsec:paths}. However, we can
construct a \kl{word encoding} for a binary alphabet $\Sigma = \set{a,b}$. To
that end, let us consider an infinite sequence $v_0, v_1, v_2, \ldots$ of
vectors in $\mathbb{V}$ such that the set $\set{v_0, v_1, v_2, \ldots}$ is
linearly independent, which is possible precisely because $\mathbb{V}$ is
infinite dimensional. Let us write $\gelem_{+1} \in \group$ for the linear
automorphism that sends $v_i$ to $v_{i + 1}$ for all $i \in \N$, and acts
trivially on a supplement of the subspace spanned by $\set{v_0, v_1, v_2,
\ldots}$. We then define $\wenc{\varepsilon} \defined v_0^3$, $\wenc{a u}
\defined v_0^3 (v_0 + v_1)^2 v_1^3 (v_1 + v_2) (\gelem_{+2} \cdot \wenc{u})$
and $\wenc{b u} \defined v_0^3 (v_0 + v_1)^1 v_1^3 (v_1 + v_2)^2 (\gelem_{+2}
\cdot \wenc{u})$.

\begin{lemma}
  \label{lem:word-encoding-vector}
  The function $\wenc{\cdot}$ defined above is a \kl{word encoding} for the
  binary alphabet $\Sigma = \set{a,b}$.
\end{lemma}
\begin{proof}
  The proof is similar to the one of \cref{lem:word-encoding-paths}.
  The only difficulty is to ensure that if a linear automorphism $\gelem$ maps
  $\wenc{u}$ to a subset of variables in $\wenc{v}$ for some $u,v$, then
  it acts as a translation of the indices on the vectors $v_i$ that appear in
  $u$.
  The reason is that the indeterminates $v_i$ are the only ones that appear with
  exponent $3$ in the definition of $\wenc{\cdot}$, hence $\gelem$ must 
  send all the $v_i's$ that appear in $\wenc{u}$ to some $v_j$'s that appear 
  in $\wenc{v}$.
  Assume that $\gelem \cdot v_0 = v_k$ for some $k \in \N$,
  and that $\gelem \cdot v_1 = v_l$ for some $l \in \N$.
  Then, 
  $\gelem \cdot (v_0 + v_1) = v_k + v_l$, or the only indeterminates 
  that are sums of two vectors in $\wenc{v}$ are of the form
  $v_j + v_{j+1}$ for some $j \in \N$.
  We conclude that $l = k + 1$.
  Similarly, one can prove that $\gelem \cdot v_2 = v_{k + 2}$, and
  by induction conclude.
\end{proof}

The role of \cref{lem:word-encoding-vector} is to show that the idea
of \kl{word encodings} is versatile enough to obtain undecidability for
structures where actual \kl(of){infinite paths} are not clearly visible.
