%!TEX root = ../atomic.asmart.tex
% LTeX: language=en
\section{Undecidability Results}
\label{sec:undecidability}

In this section, we will show that under certain conditions,
the \kl{equivariant ideal membership problem} is undecidable. We aim to show
that it is the case when we assume the usual effectiveness conditions on the
group action, but we do not assume that 
$(\mon{\Indets}, \gdivleq)$ is a well-quasi-ordering.
Beware though that there are some pathological cases where
the \kl{equivariant ideal membership problem} is undecidable even when
$(\mon{\Indets}, \gdivleq)$ is not a well-quasi-ordering,
as illustrated by the following example.

\begin{example}
  \label{ex:non-wqo-undecidable}
  Let $\Indets = \{x_1, x_2, \ldots\}$ be an infinite set of indeterminates,
  and let $\group$ be trivial group acting on $\Indets$.
  Then, the \kl{equivariant ideal membership problem} is decidable.
\end{example}
\begin{proof}
  Because the group is trivial, whenever one provides a finite set
  $H$ of generators of an \kl{equivariant ideal} $I$, one can
  in fact work in $\poly{\K}{V}$, where $V$ is the set of indeterminates
  that appear in $H$.
  Then, the \kl{equivariant ideal membership problem} is reduces to 
  the \kl{ideal membership problem} in $\poly{\K}{V}$, which is decidable.
\end{proof}

\subsection{Interpreting Paths}
\label{subsec:paths}

In this section, we will assume that the set of indeterminates $\Indets$
contains an infinite path $P \defined \seqof{x_i}[i \in \N] \subseteq \Indets$,
that is, a set of indeterminates such that $\group$ acts on $\Indets$ as a
translation of the indices, that is, for all $\gelem \in \group$ such that
$\gelem \cdot P \subseteq P$, there exists $n \in \N$ such that $\gelem \cdot
x_i = x_{i + n}$ for all $i \in \N$.

\todo[inline]{
  We can encode the PCP problem 
  in this case. The idea is to 
  write the words over a binary alphabet 
  as monomials, we distinguish "last" letters of both words
  and encode the PCP problem as follows:
  $1 \in \IdlGen{H}$, where $H$ is a set of \emph{binomials},
  that encode the fact that one can replace both "last" letters
  by corresponding words simultaneously, and that 
  one can remove "identified letters".
}

\subsection{Infinite Vector Space}
\label{subsec:vector}





