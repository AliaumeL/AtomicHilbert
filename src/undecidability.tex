%!TEX root = ../atomic.asmart.tex
% LTeX: language=en
\section{Undecidability Results}
\label{sec:undecidability}

In this section, we aim to show that the \kl{equivariant ideal membership
problem} is undecidable under the usual \kl{computability assumptions} on the
group action, when we do not assume that $(\mon{\Indets}, \gdivleq)$ is a
\kl{well-quasi-ordering}. In particular, this would show that computing
\kl{equivariant Gröbner bases} is not possible in these settings, proving the
optimality of our decidability
\cref{thm:compute-egb}.
Beware that there are some pathological cases where the \kl{equivariant ideal
membership problem} is easily decidable, even when $(\mon{\Indets}, \gdivleq)$
is not a well-quasi-ordering, as illustrated by the following
\cref{ex:non-wqo-undecidable}, and it is not possible to obtain
such a dichotomy result.

\begin{example}
  \label{ex:non-wqo-undecidable}
  Let $\Indets = \{x_1, x_2, \ldots\}$ be an infinite set of indeterminates,
  and let $\group$ be trivial group acting on $\Indets$.
  Then, the \kl{equivariant ideal membership problem} is decidable.
  Indeed, since the group is trivial, whenever one provides a finite set
  $H$ of generators of an \kl{equivariant ideal} $I$, one can
  in fact work in $\poly{\K}{V}$, where $V$ is the set of indeterminates
  that appear in $H$.
  Then, the \kl{equivariant ideal membership problem} is reduces to 
  the \kl{ideal membership problem} in $\poly{\K}{V}$, which is decidable.
\end{example}


\AP However, one we are able to prove the undecidability of the \kl{equivariant
ideal membership problem} under the assumption that the set of indeterminates
$\Indets$  contains an \intro(of){infinite path} $P \defined \seqof{x_i}[i \in
\N] \subseteq \Indets$, that is, a set of indeterminates such that $(x_i,x_j)
\in P^2$ is in the same orbit as $(x_0, x_1)$ if and only if $|i - j| = 1$, for
all $i,j \in \N$. We similarly define \reintro(of){finite paths} by considering
finitely many elements. The prototypical example of a set of indeterminates
containing an \kl(of){infinite path} is $\Indets = \Z$ equipped with the group
$\group$ of all shifts. The presence of an \kl(of){infinite path} clearly
prevents $(\mon{\Indets}, \gdivleq)$ from being a \kl{well-quasi-ordering}, as
shown by the following \cref{rem:not-wqo}. Furthermore, for
indeterminates obtained by considering \kl{homogeneous structures} and their
automorphism groups
(\cref{sec:act ex}),
the presence of an \kl(of){infinite path} has been conjectured to be a
necessary and sufficient condition for $(\mon{\Indets}, \gdivleq)$ to be a
\kl{well-quasi-ordering}: this follows from a conjecture of Schmitz restated in
\cref{conj:wqo-infinite-path}, that generalises one
of Pouzet (\cref{rem:conj-wqo-pouzet}), as explained in
\cref{rem:conj-wqo-infinite-path}.


\begin{remark}
  \label{rem:not-wqo}
  Assume that $\Indets$ contains an \kl(of){infinite path}
  $P \defined \seqof{x_i}[i \in \N]$.
  Then, the set of monomials $\setof{x_0^3 x_1^1 \cdots x_{n-1}^1 x_n^2}{n \in \N}$
  is an infinite antichain in $(\mon{\Indets}, \gdivleq)$.
  Indeed, assume that there exists $n < m$, and a group element $\gelem \in \group$ such that
  $\gelem \cdot \monelt_n \divleq \monelt_m$.
  Then, $\gelem \cdot x_0 = x_0$, because it is the only indeterminate with 
  exponent $3$ in $\monelt_m$. Furthermore, 
  $\gelem \cdot (x_0,x_1) = (x_i,x_j)$ implies that 
  $|i - j| = 1$, and since $\gelem \cdot x_0 = x_0$, we conclude
  $\gelem \cdot x_1 = x_1$. By an immediate induction, we 
  conclude that $\gelem \cdot x_i = x_i$ for all $0 \leq i \leq n$,
  but then we also have that the degree of $\gelem \cdot x_n$ is less than $2$
  in $\monelt_m$, which contradicts the fact that $\gelem \cdot \monelt_n \divleq \monelt_m$.
\end{remark}

\begin{conjecture}[Schmitz]
  \label{conj:wqo-infinite-path}
  Let $\mathcal{C}$ be a class of finite structures. Then, the following are
  equivalent:
  \begin{enumerate}
    \item The class of structures of $\mathcal{C}$ labelled with 
      any \kl{well-quasi-ordered} set $(Y, \leq)$ is
      itself \kl{well-quasi-ordered} under the
      labelled-induced-substructure relation.
    \item For every existential formula $\varphi(x,y)$,
      there exists $N_\varphi \in \N$, such 
      that $\varphi$ does not \kl(efo){define paths} of length greater than $N_\varphi$
      in the structures of $\mathcal{C}$.
  \end{enumerate}
  Where a formula \intro(efo){defines a path} of length $n$ in a structure
  if there exists $n$ distinct elements $a_0, \ldots, a_{n-1}$ in the structure
  such that $\varphi(a_i, a_j)$ holds if and only if $|i - j| = 1$.
\end{conjecture}

\begin{remark}
  \label{rem:conj-wqo-pouzet}
  The conjecture of Schmitz is a generalization of Pouzet's conjecture
  \cite{POUZ72} that states that a class $\mathcal{C}$  of finite structures is
  \kl{well-quasi-ordered} under the labelled induced-substructure relation for
  every \kl{well-quasi-ordered} set of labels, 
  if and only if it is the case for the set of two incomparable labels.
\end{remark}

\begin{remark}
  \label{rem:conj-wqo-infinite-path}
  Let $\Indets$ be an infinite \kl{homogeneous structure},
  such that $(\mon{\Indets}, \gdivleq)$ is not a \kl{well-quasi-ordering}.
  Then, the collection of finite substructures of $\Indets$
  labelled by $(\N,\leq)$ is not \kl{well-quasi-ordered} under the
  labelled-induced-substructure relation.
  Hence, if one believes that \cref{conj:wqo-infinite-path} holds,
  there exists an existential formula $\varphi(x,y)$ such that
  $\varphi$ defines arbitrarily long paths in $\Indets$.
  Because $\Indets$ is \kl{homogeneous},
  this means that $\varphi$ defines an infinite path in $\Indets$,
  and in particular, 
  $\Indets$ contains an \kl(of){infinite path} $P$, as introduced
  for generic sets of indeterminates.
\end{remark}

\paragraph{Monomial Reachability}
The undecidability results we will present in this section regarding the
\kl{equivariant ideal membership problem} will use the polynomials in a very
limited way: we will only need to consider \emph{monomials}, and there will
even be a bound on the maximal exponent used. Before going into the details of
our reductions, let us first introduce an intermediate problem that will be
easier to work with: the (equivariant) \kl{monomial reachability problem}. 

\begin{definition}
  \label{def:mon-rewrite-system}
  A \intro{monomial rewrite system} is a finite set of pairs of the form
  $\set{\monelt, \monelt'}$ where $\monelt, \monelt' \in \mon{\Indets}$.
  The \intro{monomial reachability problem} is the problem of deciding whether
  there exists a sequence of rewrites that transforms $\monelt_s$ into $\monelt_t$
  using the rules of a monomial rewrite system $R$, where
  a \intro(monrew){rewrite step} is a pair of the form
  \begin{equation*}
    \monelt[n] (\gelem \cdot \monelt)
    \leftrightarrow_R 
    \monelt[n] (\gelem \cdot \monelt')
    \text{ if } \set{\monelt, \monelt'} \in R
    \text{ and } \gelem \in \group
    \quad .
  \end{equation*}
\end{definition}

\begin{example}
  \label{ex:mon-rewrite-system}
  Let $\Indets = \N$ and $\group$ be the set of all bijections of $\Indets$.
  Then, the rewrite system $x_1^2 x_2^2 \leftrightarrow_R x_1^2$
  satisfies $\monelt \leftrightarrow_R^* x_1^2$ if and only if 
  $\monelt$ has all its exponents that are multiple of $2$.
\end{example}

The following \cref{lem:mon-rewrite-red-membership} shows that the \kl{monomial
reachability problem} can be reduced to the \kl{equivariant ideal membership
problem}, and follows the exact same reasoning as in the case of finitely many
indeterminates \cite{MAME82}. This reduction was also noticed in \cite[Theorem
64]{GHOLAS24}.


\begin{lemma}[label=lem:mon-rewrite-red-membership,ref=lem:mon-rewrite-red-membership]
  One can solve the \kl{monomial reachability problem}
  provided that one can solve the \kl{equivariant ideal membership problem}.
\end{lemma}

In order to show that the \kl{equivariant ideal membership problem} is
undecidable, it is therefore enough to show that the \kl{monomial reachability
problem} is undecidable. To that end, we will encode the Halting problem of a
Turing machine. There are two main obstacles to overcome: first, the
reversibility of the rewriting system, which can be (partially) solved by
considering \emph{reversible} Turing machines; and second, the fact that the
configurations of the Turing machine cannot staightforwardly be encoded as
monomials due to the commutativity of the multiplication.


\paragraph{Structures Containing Paths.} \AP Let us assume for the rest of this
section that $\Indets$ is a set of indeterminates that contains an
\kl(of){infinite path}, let us fix a binary alphabet $\Sigma \defined
\set{a,b}$. Given a \kl(of){finite path} $P \defined \seqof{x_i}[0 \leq i <
4n]$, we define a function $\intro*\wenc{ \cdot}_P \colon \Sigma^{\leq n} \to
\mon{\Indets}$, where $\Sigma$ is a finite alphabet, that encodes a word $u \in
\Sigma^{\leq n}$ as a monomial. Namely, we define inductively
$\wenc{\varepsilon} \defined 1$, $\wenc{a u}_P = x_0^4 x_1^2 x_2^1 x_3^3
(\mathsf{shift}_{+4} \cdot \wenc{u}_P)$ and $\wenc{b u}_P = x_0^4 x_1^1 x_2^2
x_3^3 (\mathsf{shift}_{+4} \cdot \wenc{u}_P)$ for all $u \in \Sigma^*$, where
$\mathsf{shift}_{+k}$ acts on $P$ by shifting the indices by
$k$.\footnote{There may be no element $\gelem \in \group$ that acts like
$\mathsf{shift}_{+1}$, we only use it as a function.} Let us remark that
\kl{monomial rewriting} applied on \kl{word encodings} can simulate
(reversible) string rewriting on words of a given size.

\begin{lemma}
  \label{lem:word-encoding-string-subst}
  Let $P,Q$ be two \kl(of){finite paths} in $\Indets$,
  such that $(p_0,p_1)$ is in the same orbit as 
  $(q_0,q_1)$.
  Let $u,v,w \in \Sigma^*$ be three words, such that $|u| = |v| \leq |w|$,
  and let $\monelt[n] \in \mon{\Indets}$ be a monomial.
  Assume that there exists $\gelem \in \group$
      such that $\wenc{w}_P = \monelt[m] (\gelem \cdot \wenc{u}_Q)$,
       $\monelt[n] = \monelt[m] (\gelem \cdot \wenc{v}_Q)$,
  and that $\wenc{w}_P$, $\wenc{u}_Q$ and $\wenc{v}_Q$
  are well-defined.
  Then,
      there exists $x, y \in \Sigma^*$
      such that $x u y = w$ and $\wenc{x v y}_P = \monelt[n]$.
\end{lemma}
\begin{proof}
  Let us write $\gelem \cdot q_0 = p_k$ for some $k \in \N$.
  Because the only indeterminates with degree $4$ in $\wenc{w}_P$ are
  the ones of the form $p_{4i}$, we have that $k$ is a multiple of $4$
  (i.e. at the start of a letter block).
  Since $(q_0, q_1)$ is in the same orbit as $(p_0, p_1)$,
  and both $P$ and $Q$ are \kl(of){finite paths},
  we conclude that $\gelem \cdot (q_0, q_1) = (p_{4i}, p_{4i+1})$
  or $\gelem \cdot (q_0, q_1) = (p_{4i+1}, p_{4i-1})$.
  Applying the same reasoning, thrice, 
  we have either $\gelem \cdot (q_0, q_1, q_2, q_3) = (p_{4i}, p_{4i+1}, p_{4i+2}, p_{4i+3})$
  or $\gelem \cdot (q_0, q_1, q_2, q_3) = (p_{4i}, p_{4i-1}, p_{4i-2}, p_{4i-3})$.
  However, in the second case, the exponent of $p_{4i-3}$ in $\wenc{w}_P$ is at most $2$,
  which is incompatible with the fact that the one of $q_3$ in $\wenc{u}_Q$ is $3$.
  By induction on the length of $u$, we immediately obtain that 
  $\gelem \cdot \wenc{u}_Q = \mathsf{shift}_{+4i} \cdot \wenc{u}_P$ and
  therefore that 
  $w = x u y$ for some $x,y \in \Sigma^*$.
  Finally, because $\wenc{v}_Q$ uses exactly the same indeterminates as 
  $\wenc{u}_Q$, we can also conclude that
  $\wenc{xvy}_P = \monelt[n]$.
\end{proof}

\Cref{lem:word-encoding-string-subst} shows that all encodings
using \kl(of){finite paths} with the same initial orbit are compatible with
each other for the purpose of \kl{monomial rewriting}. Let us now assume that
the alphabet is any finite set of letters, using a suitable unambiguous
encoding of the alphabet in binary \cite{BERST09}. This bigger alphabet size
will simplify the statement and proof of the following
\cref{lem:reversible-machine}, which explains how to simulate a
reversible Turing machine using \kl{monomial rewriting}. Given a reversible
Turing machine $M$ with a finite set $Q$ of states and tape alphabet $\Sigma$,
we will consider the following alphabet $\Gamma \defined \set{ \triangleleft,
  \triangleright } \times \set{ \text{pre}, \text{run}, \text{post} } \uplus Q
  \uplus \Sigma \uplus \set{ \square, \square_1, \square_2}$. The letter
  $\square$ is a blank symbol, and the letters $\triangleleft$ and
  $\triangleright$ are used to delimit the beginning and the end of the tape,
  with some extra ``phase information''. In a first \kl{monomial rewrite
  system}, we will encode a run of a reversible Turing machine $M$ on a fixed
  size input tape (\cref{lem:reversible-machine}), and in a second
  \kl{monomial rewrite system}, we will create a tape of arbitrary size
  (\cref{lem:tape-creation}). The union of these two \kl{monomial
  rewrite systems} will then be used to prove the undecidability of the
  \kl{equivariant ideal membership problem} in \cref{thm:undecidable-paths}.

\begin{lemma}
  \label{lem:reversible-machine}
  Let us fix $(x_0, x_1)$ a pair of indeterminates.
  There exists a
  \kl{monomial rewrite system} $R_M$ such that the following
  are equivalent for every $n \geq 1$,
  and for any \kl(of){finite path} $P$ of length $4(n+2)$ 
  such that $(p_0, p_1)$ is in the same orbit as $(x_0, x_1)$:
  \begin{enumerate}
    \item $\wenc{ \triangleright^{\text{run}} q_0 \square^{n-1}
                  \triangleleft^{\text{run}}
     }_P \leftrightarrow_{R_M}^* 
     \wenc{ \triangleright^{\text{run}} q_f \square^{n-1}
                  \triangleleft^{\text{run}} }_P$,
      \item $M$ halts on the empty word using a tape bounded by $n-1$ cells.
  \end{enumerate}
  Furthermore, every monomial that is 
  reachable from $\wenc{ \triangleright^{\text{run}} q_0 \square^{n-1} \triangleleft^{\text{pre}} }_P$
  or $\wenc{ \triangleright^{\text{run}} q_f \square^{n-1} \triangleleft^{\text{run}} }_P$
  is the image of a word of the form
  $\wenc{\triangleright^{\text{run}} u \triangleleft^{\text{run}}}_P$  
  where $u \in (Q \uplus \Sigma \uplus \square)^n$.
\end{lemma}
\begin{proof}
  Transitions of the reversible Turing machine using bounded tape size can be 
  modelled as a reversible string rewriting system using finitely many rules 
  of the form $u \leftrightarrow v$, where $u$ and $v$ are words
  over $(Q \uplus \Sigma \uplus \square)$ having the same length $\ell$
  For each rule $u \leftrightarrow v$, we create rules 
  $\wenc{u}_P \leftrightarrow_{R_M} \wenc{v}_P$ 
  for every \kl(of){finite path} $P$ of length $4l$.
  Note that there are only orbit finitely many such \kl(of){finite paths} $P$,
  and one can effectively list some representatives,
  because $\Indets$ is \kl{effectively oligomorphic}.
  This system is clearly complete, in the sense that one can perform a substitution
  by applying a monomial rewriting rule, but \cref{lem:word-encoding-string-subst}
  also tells us it is correct, in the sense that it cannot perform anything else
  than string substitutions.
  Furthermore, we can assume 
  that the reversible Turing machine
  starts with a clean tape and ends with a clean tape.
\end{proof}

\Cref{lem:reversible-machine} shows that one can simulate the
runs, provided we know in advance the maximal size of the tape used by the
reversible Turing machine. The key ingredient that remains to be explained is
how one can start from a finite monomial $\monelt$ and create a tape of
arbitrary size using a \kl{monomial rewrite system}. The difficulty is that we
will not be able to ensure that we follow one specific \kl(of){finite path}
when creating the tape.

\begin{lemma}
  \label{lem:tape-creation}
  Let $(x_0, x_1)$ be a pair of indeterminates, $P$ be a \kl(of){finite path}
  such that $(p_0, p_1)$ is in the same orbit as $(x_0, x_1)$.
  There exists a \kl{monomial rewrite system} $R_\text{pre}$
  such that for every monomial $\monelt \in \mon{\Indets}$, the following are
  equivalent:
  \begin{enumerate}
    \item $\wenc{ \triangleright^{\text{pre}} \square \square_1 \square_2 \triangleleft^{\text{pre}}}_P
      \leftrightarrow_{R_\text{pre}}^* 
      \monelt$
      and $\wenc{\triangleright^{\text{run}}}_{P'}
      \gdivleq \monelt$ for some \kl(of){finite path} $P'$ such that
      $(p_0', p_1')$ is in the same orbit as $(x_0, x_1)$.
    \item There exists $n \geq 2$ and a \kl(of){finite path} $P'$ such that 
      $(p_0', p_1')$ is in the same orbit as $(x_0, x_1)$,
      and 
      $\monelt = \wenc{ \triangleright^{\text{run}} q_0 \square^{n}
      \triangleleft^{\text{run}} }_{P'}$.
  \end{enumerate}
  Similarly, there exists a \kl{monomial rewrite system} $R_\text{post}$
  with analogue properties using $q_f$ instead of $q_0$.
\end{lemma}
\begin{proof}
  We create the following rules,
  where $P_1$ and $P_2$ range over \kl(of){finite paths} such that
  their first two elements are in the same orbit as $(x_0, x_1)$,
  and assuming that the indeterminates of $P_1$ and $P_2$ are disjoint:
  \begin{enumerate}
    \item Cell creation: 
      $\wenc{\triangleright^{\text{pre}} \square}_{P_1}
        \wenc{ \square_1 \square_2 \triangleleft^{\text{pre}}}_{P_2}
      \leftrightarrow_{R_\text{pre}}
      \wenc{\triangleright^{\text{pre}} \square_1}_{P_1}
      \wenc{ \square \square \square_2 \triangleleft^{\text{pre}}}_{P_2}$
    \item Linearity checking:
      $\wenc{\square_1 \square}_{P_1} \wenc{\square_2 \triangleleft^{\text{pre}}}_{P_2}
      \leftrightarrow_{R_\text{pre}}
      \wenc{\square \square_1}_{P_1} \wenc{\square_2 \triangleleft^{\text{pre}}}_{P_2}$
    \item Phase transition:
      $\wenc{\triangleright^{\text{pre}} \square}_{P_1}
       \wenc{\square_1 \square_2 \triangleleft^{\text{pre}}}_{P_2}
      \leftrightarrow_{R_\text{pre}}
       \wenc{\triangleright^{\text{run}} q_0}_{P_1}
       \wenc{\square \square \triangleleft^{\text{run}}}_{P_2}$
  \end{enumerate}
  Note that there are only orbit finitely many such pairs of monomials,
  and that we can enumerate representative of these orbits because 
  $\Indets$ is \kl{effectively oligomorphic}.

  Let us first argue that this system is complete. Because there exists an
  infinite path $P_{\infty}$, it is indeed possible to reach
  $\wenc{\triangleright^{\text{run}} q_0 \square^n
  \triangleleft^{\text{run}}}_{P_\infty}$ by repeatedly applying the first
  rule, and then the second rule until $\square_1$ reaches the end of the tape,
  and continuing so until one decides to apply the third rule to reach the
  desired tape configuration.

  We now claim that the system is correct, in the sense that it can only reach
  valid tape encodings. First, let us observe that in a rewrite sequence, one
  can always assume that the rewriting takes the form of applying the first
  rule, then the second rule until one cannot apply it anymore, and repeating
  this process until one applies the third rule. Because rule (2) ensures that
  when we add new indeterminates using rule (1), they were not already present
  in the monomial, and because rule (1) ensures that locally the structure of
  the indeterminates remains a \kl(of){finite path}, we can conclude that the
  whole set of indeterminates used come from a \kl(of){finite path} $P'$. As a
  consequence, if one can reach a state where (1) or (3) are applicable, then
  the tape is of the form $\wenc{ \triangleright^{\text{pre}} \square^n
  \square_1 \square_2 \triangleleft^{\text{pre}} }_{P'}$, with $n \geq 1$. It
  follows that when one can apply rule (3), the monomial obtained is of the
  form $\wenc{ \triangleright^{\text{run}} q_0 \square^n
  \triangleleft^{\text{run}} }_{P'}$, where $P'$ is a \kl(of){finite path} such
  that $(p_0', p_1')$ is in the same orbit as $(x_0, x_1)$. 
\end{proof}

\csname thm:undecidable-paths\endcsname*
\begin{proof}
  It suffices to combine the rewriting systems $R_M$, $R_\text{pre}$ and 
  $R_\text{post}$ by taking their union.
\end{proof}


\begin{remark}
  \label{rem:more-generally}
  The undecidability result of \cref{cor:undecidability} can be generalised to
  any set of indeterminates in which one can encode words over a binary alphabet,
  and for which there is a \kl{monomial rewrite system} that can
  produce arbitrary long words.
  We strongly conjecture that this is the case for 
  the \kl{infinite dimensional vector space}, as defined in 
  \cref{ex:bit vector} of
  \cref{sec:examples}.
\end{remark}
