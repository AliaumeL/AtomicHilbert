%!TEX root = ../atomic.asmart.tex
%
\section{Colourful ideas}
%
\begin{assumption}\label{assume:mon mon wqo}
$(\mon{X}{\times}\mon{X},\gdivleq)$ is a WQO.
\end{assumption}
%
For this section, let $\col \defined \{0,1\}$ and $\paint{\X} \defined \X{\times}\col$.
The order $<$ on $\X$ is extended to $\paint{\X}$ as
\[
(x,c) < (y,d) \quad\iff\quad
\text{$c < d$, or $c = d$ and $x < y$.}
\]
%
\begin{definition}
Define $\forget : \poly{\K}{\paint{\X}} \to \poly{\K}{\X}$ as the extension of the map $(x,c) \mapsto x$ for $x\in\X$ and $c\in\col$.
\end{definition}
%
%\begin{definition}
%A polynomial $f\in\poly{\K}{\paint{\X}}$ is called \kl{consistently coloured} if
%\[
%\var(f) \cap \{(x,0),(x,1)\} \subsetneq \{(x,0),(x,1)\}
%\]
%for every $x\in\X$.
%Similarly, a $S$-polynomial $\spoly{f}{g} \in \poly{\K}{\paint{\X}}$ is called \kl{consistently coloured} if
%\[
%(\var(f)\cup\var(g)) \cap \{(x,0),(x,1)\} \subsetneq \{(x,0),(x,1)\}
%\]
%for every $x\in\X$.
%\end{definition}
%
\arka{How do I add new environments?}
%
\begin{definition}
For a set $A$, let $\symgr{A}$ denote the group of bijections of $A$.
\end{definition}
%
\begin{definition}
Let $\paint{\group}$ denote the \kl{free group} generated by $\group$ and $\symgr{\col}$.
\end{definition}
%
The actions of $\group$ on $\X$ and $\symgr{\col}$ is extended to an action of $\paint{\group}$ of $\paint{\X}$ in the obvious way.
%
\arka{Expand on the definitions and add examples}
%
%\begin{lemma}
%For any \kl{consistently coloured} $f\in\poly{\K}{\X}$ and $\pi\in\paint{\group}$,
%$\pi(f)$ is also \kl{consistently coloured}.
%\end{lemma}
%
Let $\Basis\subseteq\poly{\K}{\X}$ be an orbit-finite set.
Let $\paint{\Basis} = \forget^{-1}(\Basis)$.
%
\begin{lemma}
The set $\paint{\Basis}$ is $\paint{\group}$-equivariant.
\end{lemma}
%
\begin{lemma}
Under \Cref{assume:mon mon wqo},
\Cref{alg:weakgb} terminates on the input $\paint{\Basis}$.
\end{lemma}
%
Let $\paint{\Basis[D]}$ be the output of \Cref{alg:weakgb} on the input $\paint{\Basis}$.
Let $\Basis[D] = \forget(\paint{\Basis[D]})$.
%
\begin{lemma}\label{lem:strong buch}
The set $\Basis[D]$ is an orbit-finite \gr{} basis of $\IdlGen{\Basis}$.
\end{lemma}
%
\subsection{Proof of \Cref{lem:strong buch}}

\qed
%