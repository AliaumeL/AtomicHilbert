% LTeX: language=en
%!TEX root = ../atomic.ieee.tex
%
\section{Equivariant Nullstellensatz}
%
In this section we assume $\K$ is an \kl{algebraically closed} field and $\calH\actson\Indets$ is \kl{nicely orderable}.
Recall that a field $\mathbb{F}$ is called \intro{algebraically closed} if every non-constant univariate polynomial $f$ with coefficients in $\mathbb{F}$ has a root in $\mathbb{F}$.
The most common example of an algebraically closed field is the field $\C$ of complex numbers
\cite[Section 1.1, Theorem 7]{CLO15}.

Thinking a polynomial $f\in\poly{\K}{\Indets}$ as a function from $\Indets \to \K$ to $\K$,
for an element $\iota : \Indets \to \K$ we use $f(\iota)$ to denote the value of $f$ at $\iota$.
%
\begin{example}\label{ex:inst}
    If $f = x^2 + 3y - 2$ and $\iota(x) = \iota(y) = 1$ then $f(\iota) = 2$.
\end{example}
%
\begin{definition}\label{def:variety}
    The \intro{equivariant variety} of an ideal $\idl[I]$ is the set of all equivariant functions $\iota : \Indets \to \K$ such that $f(\iota) = 0$.
\end{definition}
%
\begin{theorem}[Weak Equivariant Nullstellensatz]\label{thm:weak null}
    The set $\zeros{\idl}$ is non-empty for every equivariant ideal $\idl \subsetneq \poly{\K}{\X}$.
\end{theorem}
%
\begin{definition}\label{def:rad}
    The \intro{radical} $\rad{\idl[I]}$ of an ideal $\idl[I]$ of $\poly{\K}{\Indets}$ is defined to be set of polynomials
    \[
    \setof{f\in\poly{\K}{\Indets}}{f^m\in\idl[I] \text{ for some $m\in\N$}} \ .
    \]
\end{definition}
%
\begin{lemma}\label{lem:rad is ideal}
    The \intro{radical} $\rad{\idl[I]}$ of an ideal $\idl[I]$ of $\poly{\K}{\Indets}$ is also an ideal.
    Moreover, if $\idl[I]$ equivariant,
    so is $\rad{\idl[I]}$.
\end{lemma}
%
\begin{theorem}[Strong Equivariant Nullstellensatz]\label{thm:strong null}
    For every polynomial $f$ and equivariant ideal $\idl[I]$,
    $f\in\rad{\idl[I]}$ if and only if $\zeros{f} \supseteq \zeros{\idl[I]}$.
\end{theorem}
%
\begin{definition}\label{def:ideal sum}
    For two ideals $\idl[I]$ and $\idl[J]$,
    let $\idl[I] + \idl[J]$ denote the Minkowski sum of $\idl[I]$ and $\idl[J]$.
    That is,
    \[
    \idl[I] + \idl[J] \defeq
    \setof{f + g}{f\in\idl[I],\ g\in\idl[J]} \ .
    \]
\end{definition}
%
\begin{definition}\label{def:maximal}
    An ideal $\idl[M]$ is called \intro{maximal} if the ideal generated by $\idl[M]\cup\set{f}$ is the whole ring $\poly{\K}{\Indets}$.
\end{definition}
%
\begin{definition}\label{def:equiv maximal}
    For a function $\varphi : \Indets\to\K$ let $\idl[I]_{\varphi}$ denote the ideal generated by the set of polynomials
    \[
    \setof{x - \varphi(x)}{x\in\Indets} \ .
    \]
\end{definition}
%
\begin{lemma}
    For every $\varphi : \Indets \to \K$,
    the ideal $\idl[I]_{\varphi}$ is a maximal ideal.
    It is equivariant if and only if $\varphi$ is equivariant.
\end{lemma}
%
\begin{definition}
    An ideal $\idl[M]$ is called \intro{equivariantly maximal} if the equivariant ideal generated by $\idl[M]\cup\set{f}$ is the whole ring $\poly{\K}{\Indets}$.
\end{definition}
%
\begin{lemma}\label{lem:in equiv maximal}
    Every equivariant ideal $\idl[I]$ is a subset of some equivariantly maximal ideal $\idl[M]$.
\end{lemma}
%
\begin{remark}\label{rem:lem:in equiv max}
    The proof of \Cref{lem:in equiv maximal} does not require the action $\calH\actson\Indets$ to be \kl{nicely orderable}.    
\end{remark}
%
\begin{lemma}\label{lem:equiv max is max}
    Every equivariantly maximal ideal $\idl[M]$ is also a maximal ideal.
\end{lemma}
%
\begin{corollary}
    Every equivariantly maximal ideal $\idl[M]$ is equal to $\idl[I]_{\varphi}$ for some equivariant function $\varphi : \Indets\to\K$.
\end{corollary}
%