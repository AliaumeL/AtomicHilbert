% LTeX: language=en
%!TEX root = ../atomic.ieee.tex
%

\clearpage
\section{Equivariant Nullstellensatz}\label{sec:nullstellensatz}



\newcommand{\variety}{\mathbb{V}}
\newcommand{\eqVariety}{\mathbb{V}}
\newcommand{\ofVariety}{\mathbb{V}_{\mathrm{of}}}
\newcommand{\vanishingPoly}{\mathbb{I}}

In this section we assume $\K$ is an \intro{algebraically closed}, i.e., every
non-constant univariate polynomial with coefficients in $\K$ has a root in
$\K$. The most common example of an algebraically closed field is the field
$\C$ of complex numbers~\cite[Section 1.1, Theorem 7]{CLO15}. Our goal is to
adapt the classical results relating polynomial ideals and algebraic varieties
to the equivariant setting.

In the classical case, having $n$ indeterminates $x_1,\ldots,x_n$, an algebraic
variety is defined as the set of common zeros of a set of polynomials in
$\K[x_1,\ldots,x_n]$, identified with points in $\K^n$. One can 
move between ideals and varieties using the following two adjuctions:
\begin{itemize}
  \item Given a set of polynomials $S\subseteq \K[x_1,\ldots,x_n]$, 
    one can define its variety $\mathcal{V}(S)$
    as the set of points in $\K^n$ that are zeros of all polynomials in $S$.
  \item Given a subset $V \subseteq \K^n$, one can 
    define its ideal $\mathcal{I}(V)$
    as the set of all polynomials in $\K[x_1,\ldots,x_n]$ that vanish on all points in $V$.
\end{itemize} 

The celebrated Nullstellensatz theorems relate these two operations. Its weaker
form states that every proper ideal yields a non-empty variety, while its
stronger form characterizes $\mathcal{I} \circ \mathcal{V}$ as precisely
computing the \intro{radical} $\sqrt{I}$ of an ideal $I$, defined as the set of
polynomials $f$ such that some power of $f$ lies in $I$.

One application of the Nullstellensatz is to decide if a set of equations has a
solution: given polynomials $f_1,\ldots,f_m$, the system of equations $f_1 =
0,\ldots,f_m = 0$ has a solution if and only if the ideal generated by
$f_1,\ldots,f_m$ is proper, i.e., does not contain $1$. Hence, it transforms a
solution existence problem into an ideal membership problem.


Let us now adapt these notions to the equivariant setting. We call
\intro{points} functions from $\Indets$ to $\K$, since any polynomial in
$\poly{\K}{\Indets}$ can be seen as a function from points to $\K$. We will be
interested in \intro{equivariant points}, i.e., functions $\iota : \Indets \to
\K$ that are left invariant by the action of $\calH$, and \intro{orbit-finite
points}, i.e., functions $\iota : \Indets \to \K$ whose orbit under the action
of $\calH$ is finite. To an ideal $\idl[I]$ of $\poly{\K}{\Indets}$, we
associate its \intro{variety} $\variety(\idl[I])$, \intro{equivariant variety}
$\eqVariety(\idl[I])$ and \intro{orbit-finite variety} $\ofVariety(\idl[I])$
respectively as the set of points, equivariant points and orbit-finite points
$\iota : \Indets \to \K$ such that $f(\iota) = 0$ for all $f\in\idl[I]$. 

\begin{definition}
  The \intro{weak Nullstellensatz} (resp. weak equivariant, resp. weak orbit-finite)
  states that for every proper ideal $\idl[I] \subsetneq \poly{\K}{\Indets}$
  (resp. proper equivariant ideal, resp. proper orbit-finite ideal),
  its variety $\variety(\idl[I])$ is non-empty (resp. 
  $\eqVariety(\idl[I])$ is non-empty, resp. $\ofVariety(\idl[I])$ is non-empty).
\end{definition}

Let us remark that if $\idl[I]$ is an equivariant ideal, then its variety
$\variety(\idl[I])$ is stable under the action of $\calH$, but may contain
non-equivariant points.

\begin{example}
  \label{ex:eq-but-non-eq-variety}
  todo
\end{example}

\begin{definition}
  The \intro{strong Nullstellensatz} (resp. strong equivariant, resp. strong orbit-finite)
  states that for every polynomial $\idl[I]$ of $\poly{\K}{\Indets}$
  (resp. equivariant ideal, resp. orbit-finite ideal),
  $\vanishingPoly(\variety(\idl[I])) = \rad{\idl[I]}$
  (resp. $\vanishingPoly(\eqVariety(\idl[I])) = \rad{\idl[I]}$,
  resp. $\vanishingPoly(\ofVariety(\idl[I])) = \rad{\idl[I]}$).
\end{definition}

One easy way to understand the weak Nullstellensatz is to study maximal ideals
(resp. maximal equivariant ideals, resp. maximal orbit-finite ideals), that are
ideals that cannot be extended to larger proper ideals. A typical example of a
maximal ideal is obtained by fixing a \kl{point} $\iota : \Indets \to \K$ and
considering the ideal $\idl[I]_{\iota}$ generated by the polynomials $x -
\iota(x)$ for all $x\in\Indets$. It is easy to prove the converse 
whenever \kl{Hilbert's basis property} holds.

\begin{lemma}
  \label{lem:eq-maximal-charac}
  Assume that $\poly{\K}{\Indets}$ has the \kl{Hilbert basis property}.
  Every maximal equivariant ideal $\idl[M]$ is of the form
  $\idl[I]_{\iota}$ for some equivariant point $\iota : \Indets \to \K$.
\end{lemma}
\begin{proof}
  Take a maximal equivariant ideal $\idl[M]$. Since $\poly{\K}{\Indets}$ has the
  \kl{Hilbert basis property}, $\idl[M]$ is finitely generated, say by
  $f_1,\ldots,f_m$.

  Now, let us consider all indeterminates that appear in the generators of
  $\idl[M]$, and write this set $S$. We define $\idl[N] = \idl[M] \cap
  \poly{\K}{S}$, which is an ideal of $\poly{\K}{S}$. Since $\idl[M]$ is proper,
  then so must be $\idl[N]$.

  Remark that by construction, since $f_1,\ldots,f_m$ belong to $\idl[N]$, we
  know that the equivariant ideal generated by $\idl[N]$ is exactly $\idl[M]$.

  \todo[inline]{Prove that $\idl[N]$ is maximal...}

  Since $\idl[N]$ is maximal, it is of the form 
  $\idl[N] = \idl[I]_{\iota_S}$ for some point $\iota_S : S \to \K$, 
  leveraging the classical Nullstellensatz.

  But $\idl[M]$ is the equivariant ideal generated by $\idl[N]$, hence the
  equivariant ideal generated by the polynomials $x - \iota_S(x)$ for $x\in S$.
  If two variables $x,x'$ are in the same orbit under $\calH$, and $\iota_S(x)
  \neq \iota_S(x')$, then the polynomial $1$ can be obtained in $\idl[M]$,
  contradicting the fact that $\idl[M]$ is proper. As a consequence,
  $\iota_S$ can be extended to an equivariant point $\iota : \Indets \to \K$,
  and $\idl[M] = \idl[I]_{\iota}$.
\end{proof}

The relationship between the weak Nullstellensatz and strong Nullstellensatz
can be obtained using the same techniques as in the classical case, known as
the Rabinowitsch trick. Unfortunately, introducing new indeterminates comes at
a cost in the equivariant setting, forcing us to work with orbit-finite ideals
instead of equivariant ideals.

\begin{lemma}
  \label{lem:nullstellensatz-equivariant-rabinowitsch}
  Assume that the weak orbit-finite Nullstellensatz holds. Then, the strong
  orbit-finite Nullstellensatz holds as well.
\end{lemma}
\begin{proof}

  The proof is similar to the classical case. Let $\idl[I]$ be an orbit-finite
  ideal, and let $f$ be a polynomial vanishing on all orbit-finite points of
  $\idl[I]$. We want to prove that $f$ lies in the radical of $\idl[I]$. Let us
  introduce a new indeterminate $y$, that is invariant under the action of
  $\calH$. We consider the orbit-finite ideal $\idl[J]$ generated by $\idl[I]$
  and the polynomial $1 - yf$, where all the indeterminates of $f$ are
  considered invariant. 

  If $\idl[J]$ were proper, then by the weak orbit-finite Nullstellensatz,
  there would exist an orbit-finite point $\iota$ vanishing on all polynomials
  of $\idl[J]$. In particular, since $1 - yf$ lies in $\idl[J]$, we would have
  $1 - \iota(y) f(\iota) = 0$. This is absurd since $f$ vanishes on all
  orbit-finite points where $\idl[I]$ vanishes, and hence in particular on
  $\iota$.

  As a consequence, $\idl[J] = \poly{\K}{\Indets \cup \{y\}}$, and there
  exist polynomials $g_1,\ldots,g_m$ in $\idl[I]$ and a polynomial $h$ such
  that
  \[
    1 = \sum_{i=1}^m g_i \cdot h_i + h \cdot (1 - yf).
  \]

  Note that here it is very important that $f$ and $y$ are left invariant by
  $\calH$, otherwise there may be multiple occurences of $f$ (under different
  actions of $\calH$) in the expression above, and we would not be able to
  conclude.
  By substituting $y$ by $1/f$ in the expression above, we obtain that some
  power of $f$ lies in $\idl[I]$, concluding the proof.
\end{proof}

We know that in general (orbit-finite) implies (equivariant) implies arbitrary
for both the weak and strong Nullstellensatz. And we proved that the weak
orbit-finite and strong orbit-finite Nullstellensatz are equivalent.

\begin{lemma}
  \label{lem:weak-null-non-equiv}
  The weak Nullstellensatz holds whenever the cardinality of $\K$
  is large enough compared to the cardinality of $\Indets$.
\end{lemma}
\begin{proof}
  todo
\end{proof}

Let us now show that under a classical assumption on the group action, and
assuming that the cardinality of $\K$ is large enough, all variations of the
weak Nullstellensatz and strong Nullstellensatz hold.

\begin{definition}[todo cite]
  A group $\calH$ is \intro{extremely amenable} if
  every continuous action of $\calH$ on a non-empty
  compact Hausdorff space has a fixed point.
\end{definition}


\todo[inline]{Fact check this}
\begin{lemma}
  \label{lem:extremely-amenable-nullstellensatz}
  Assume that $\calH$ is extremely amenable, and that the cardinality of $\K$
  is large enough compared to the cardinality of $\Indets$. Then, the weak
  Nullstellensatz, weak equivariant Nullstellensatz and weak orbit-finite
  Nullstellensatz all hold. 
\end{lemma}
\begin{proof}
  Let us write $Z$ the set of points that vanish on all polynomials of a given
  proper ideal $\idl[I]$. This set is non-empty by
  \cref{lem:weak-null-non-equiv}. 

  We create the Stone–Čech compactification $\beta (\Indets \to \K)$ of the set
  of points. Note that points embed naturally into $\beta (\Indets \to \K)$ via
  a continuous map $e$. 

  \todo[inline]{Are the following even true?}
  \begin{enumerate}
      \item This compactification can be extended so that 
        $\calH$ acts continuously on $\beta (\Indets \to \K)$,
        and $e$ is $\calH$-equivariant.
      \item The closure of $e(Z)$ in $\beta (\Indets \to \K)$, 
        is compact, as a closed subset of a compact space, and 
        is stable under the action of $\calH$.
  \end{enumerate}

  By extreme amenability of $\calH$, there exists a fixed point
  $\Theta \in \overline{e(Z)}$. 
  We can now define an equivariant point 
  $\iota : \Indets \to \K$ as follows: for every $x\in\Indets$,
  we consider the continuous projection 
  $\pi_x : (\Indets \to \K) \to \K$. 

  \todo[inline]{We actually cannot do this, since $\K$ may not be compact.}
  By definition of the Stone–Čech compactification,
  $\pi_x$ can be extended to a continuous map
  $\widehat{\pi_x} : \beta (\Indets \to \K) \to \K$.
  We define $\iota(x) = \widehat{\pi_x}(\Theta)$.

  Since $\Theta$ is a fixed point, for every $h\in\calH$,
  \[
    \iota(h \cdot x)
    = \widehat{\pi_{h \cdot x}}(\Theta)
    = \widehat{\pi_x}(h \cdot \Theta)
    = \widehat{\pi_x}(\Theta)
    = \iota(x),
  \]

  It remains to prove that $\iota$ vanishes on all polynomials of $\idl[I]$.
  Let $f\in\idl[I]$. We consider the continuous map
  $\phi_f : (\Indets \to \K) \to \K$ defined by $\phi_f(\iota) = f(\iota)$.
  By definition of the Stone–Čech compactification,
  $\phi_f$ can be extended to a continuous map
  $\widehat{\phi_f} : \beta (\Indets \to \K) \to \K$.
  Since $\Theta$ lies in the closure of $e(Z)$, and $f$ vanishes on all points
  of $Z$, we have $\widehat{\phi_f}(\Theta) = 0$.

  \todo[inline]{Check this carefully}
  Now, by induction on the structure of $f$, one can prove that
  $\widehat{\phi_f}(\Theta) = f(\iota)$, concluding the proof.
\end{proof}


\begin{corollary}
  \label{cor:extremely-amenable-strong-nullstellensatz}
  Assume that $\calH$ is extremely amenable, and that the cardinality of $\K$
  is large enough compared to the cardinality of $\Indets$. Then, the strong
  orbit-finite Nullstellensatz holds.
\end{corollary}
\begin{proof}
  This is a direct consequence of \cref{lem:extremely-amenable-nullstellensatz}
  and \cref{lem:nullstellensatz-equivariant-rabinowitsch}.
\end{proof}


\todo[inline]{Find an example based on $A^2$ and cycles}
\begin{example}
  \label{ex:nullstellensatz-but-not-hilbert}
  Here is an example of a situation where the strong orbit-finite
  Nullstellensatz holds, but the Hilbert basis property does not.

  Even more, the \kl{equivariant ideal membership problem} is undecidable
  in this example, while the \kl{radical equivariant ideal membership problem}
  is decidable.
\end{example}
