\section{intro}
By leveraging the same proof technique,
we can also show that the \kl{equivariant ideal membership problem} is
decidable under a weaker hypothesis, namely that the set of \kl{monomials}
$\mon[\om \ordplus 1]{\Indets}$ is a \kl{WQO}, which is also believed to be
equivalent to the first condition.

\begin{theorem}[name={Equivariant Ideal Membership},restate=thm:decide-equiv-ideal-mem]
  \label{thm:decide-equiv-ideal-mem}
  Let $\Indets$ be a totally ordered set of indeterminates
  equipped with a group action $\group \actson \Indets$, under our \kl{computability assumptions}.
  If $(\mon[\om \ordplus 1]{\Indets}, \gdivleq)$ is a \kl{WQO}, then one can decide the
  \kl{equivariant ideal membership problem}.
\end{theorem}


\section{Proofs of \cref{sec:examples}}

\AP A \intro{topological space} is a set $X$ equipped with a collection $\tau$
of subsets of $X$ that is stable under finite intersections and arbitrary
unions.\footnote{In particular, $\tau$ contains the empty set and $X$ itself.}
In a \kl{topological space}, elements of $\tau$ are called \intro{open
subsets}, while their complements (in $X$) are called \intro{closed subsets}. A
\kl{topological space} is \intro(space){Noetherian} when, for every sequence
$\seqof{U_i}[i \in \N]$ of \kl{open subsets}, there exists $n \in \N$ such that
$\bigcup_{i \in \N} U_i = \bigcup_{i \leq n} U_i$. We refer the readers to the
book \cite{JGL13} for a comprehensive introduction to \kl{Noetherian spaces}
and their usage in theoretical computer science. Let us briefly argue that
\kl{Noetherian spaces} generalize \kl{well-quasi-orders} in
\cref{ex:well-quasi-orders-are-noeth}, and encode the
\kl{Hilbert basis property} in \cref{ex:polynomials-noetherian}.

\begin{example}[ see \cite{JGL13}]
  \label{ex:well-quasi-orders-are-noeth}
  Let $(X, \leq)$ be a quasi-ordered set.
  Then, the set $X$ equipped with the \kl{topology} having 
  as \kl{open subsets} the upwards-closed subsets of $X$ is \kl(space){Noetherian}
  if and only if $(X, \leq)$ is \kl{well-quasi-ordered}.
\end{example}

\begin{example}[ see \cite{JGL13}]
  \label{ex:polynomials-noetherian}
  Let $\K$ be a field, and let $n \in \N$.
  The space $\K^n$ equipped with the \kl{Zariski topology}
  \kl(space){Noetherian}; where the \intro{Zariski topology}
  is the topology whose \kl{closed subsets} are finite unions of sets
  of the form $\setof{ \vec{x} \in \K^n}{ \forall p \in \idl, p(\vec{x}) = 0}$,
  where $\idl$ is an \kl{ideal} of $\poly{\K}{x_1, \dots, x_n}$.
\end{example}

\AP The advantage of \kl{Noetherian spaces} over \kl{well-quasi-orderings} and
\kl{Noetherian rings} is that they generalize both and can be \emph{combined}:
\kl{Noetherian spaces} are closed under finite sums, finite products,
considering finite words, considering finite trees, and many more \todo{cite}.
As a consequence, they provide a versatile tool to express the set of states of
a system, ensuring that a strong termination property holds.

\AP A \intro{topological well-structured transition system} with alphabet
$\Sigma$ is a \kl{topological space} $(X, \tau)$, equipped with a transition
function $\delta \colon X \times \Sigma \to X$, such that the following
properties hold: for every $U \in \tau$, $\mathrm{pre}^\exists(U)$, the set of
states $x \in X$ such that there exists $a \in \Sigma$ with $\delta(x, a) \in
U$, is an \kl{open subset}. Equivalently, the set $\mathrm{pre}^\forall(E)$ of
states $x \in X$ such that for every $a \in \Sigma$, $\delta(x, a) \in E$ is a
\kl{closed subset} of $X$ whenever $E$ is itself a \kl{closed subset} of $X$.
The natural decition problem for \kl{topological well-structured transition
systems} is the following \intro{open reachability problem} is decidable: given
an initial state $x_0 \in X$ and an \kl{open subset} $U \in \tau$, is it true that
there exists a word $w \in \Sigma^*$ such that $\delta^*(x_0, w) \in U$? The
prototypical algorithm to solve this problem is the following \intro{backward
algorithm}: start with $U_0 \defined U$, and iteratively compute $U_{i+1}
\defined U_i \cup \mathrm{pre}^\exists(U_i)$ until $U_i = U_{i+1}$, then check
whether $x_0 \in U_\text{last}$.
There are easy-to-state sufficient conditions  for such an algorithm to be computable and terminate:
\begin{enumerate}
  \item One is equipped with an effective representation of open subsets,
    where one is able to test equality of open subsets, compute unions of open subsets, and test 
    membership of a point in an open subset.
  \item The pre-image function $\mathrm{pre}^\exists$ is computable, i.e., one can
    compute the set $\mathrm{pre}^\exists(U)$ for every open subset $U$.
  \item The space $(X, \tau)$ is \kl{Noetherian}. 
\end{enumerate}

\AP Our \cref{cor:equivariant-ideals-computations} shows that
under some assumptions on $\Indets$, the set of finitely supported functions
$\Indets \to \K$ is a \kl{Noetherian space} with respect to the
\intro{equivariant Zariski topology}, i.e., the topology whose \kl{closed subsets}
are finite unions of sets of the form $E_{\idl} \defined \setof{f \in
\K^{(\Indets)}}{\forall p \in \idl, p(f) = 0}$, where $\idl$ is an
\kl{equivariant ideal} of $\poly{\K}{\Indets}$. Furthermore, we have an
effective representation of the \kl{closed subsets} in this topology, using
\kl{equivariant Gröbner bases} of \kl{equivariant ideals}. In particular, the
theory of \kl{topological well-structured transition systems} can be applied to
systems whose state space contains ``named registers'' that contain numbers and
are updated by polynomial functions.



\AP Let us fix a group $\group$ that acts on the set of indeterminates
$\Indets$, and on an alphabet $\Sigma$ in an \kl{effectively oligomorphic}
fashion. Let us now consider the case of \intro{orbit finite polynomial
automata}, that we define as follows: an \reintro{orbit finite polynomial
automaton} is a tuple $A \defined (Q, \Sigma, \delta, q_0, F)$, where $Q =
\K^{(\Indets)}$, $\Sigma$ is an \kl{orbit finite} alphabet, $\delta \colon
\Sigma \to (\Indets \to \poly{\K}{\Indets})$ is a \kl(func){finitely supported}
polynomial update function, and $F \in \poly{\K}{\Indets}$ is a polynomial
computing the result of the automaton. Given a letter $a \in \Sigma$ and a
state $q \in Q$, the update $\delta^*(q,a)$ is defined as the function from
$\Indets$ to $\K$ defined by $\delta^*(q,a) \colon x \mapsto \delta(a,x)[ q ]$,
which is well-defined because $\delta(a,x)$ is a \kl{finitely supported}
polynomial. The update function is naturally extended to words. Finally, the
output of an \kl{orbit finite polynomial automaton} on a word $w \in \Sigma^*$
is defined as $F(\delta^*(q_0, w))$.

While all of these reasosing could be done outside the realm of (effective)
\kl{topological well-structured transition systems}, we can use the modularity
of the theory to obtain more complex verification properties. Following the
lines of \cite[Theorem 6]{JGL10}, one can consider the case of communicating
orbit finite polynomial automata, where we have a collection processes that
communicate letters over a finite alphabet using lossy channels, and can
perform polynomial updates on their local state. Deciding whether such a system
can reach a state where one process fails to satisfy a given polynomial
invariant is a special case of the \kl{open reachability problem}, and is
decidable.


\begin{lemma}
  \label{lem:zeroness-problem-polynomial-automata}
  The \kl{zeroness problem for polynomial automata} is a special case of the
  \kl{open reachability problem} for \kl{topological well-structured transition systems}.
\end{lemma}
\begin{proof}
  Let $A = (Q, \Sigma, \delta, q_0, F)$ be a \kl{polynomial automaton}.
  We consider the topological space $(Q, \tau)$, where $\tau$ is the
  \kl{Zariski topology} on $\K^n$.
  Let $\idl$ be an \kl{ideal} of $\poly{\K}{x_1,\dots,x_n}$ generated by the polynomials
  $p_1, \dots, p_m$,
  and let $E \defined \setof{q \in Q}{\forall p \in \idl, p(q) = 0}$,
  a \kl{closed subset} of $Q$.
  Then,
  \begin{align*}
    q \in \mathrm{pre}^\forall(E) & \iff 
    \forall a \in \Sigma, \forall p \in \idl, p(\delta(q, a)) = 0 \\
                                  & \iff 
    \forall a \in \Sigma, \forall p \in \idl, p(\delta(q, a)) = 0 \\
                                  & \iff 
                                  \forall p \in \idl[J], p(q) = 0
  \end{align*}
  where $\idl[J] \defined \IdlGen{ \setof{ p_i[ x_i \mapsto \delta(\cdot, a)_i] }{ i \in \set{1, \dots, m}, a \in \Sigma } }$.
  In particular, one can represent \kl{closed subsets} of $Q$ as finite 
  lists of \kl{ideals} using their \kl{Gröbner bases}, and we showed that 
  one can effectively compute the pre-image of \kl{closed subsets} of $Q$
  via $\mathrm{pre}^\forall$ by substituting polynomials.
  In this representation, it is very easy to compute the union 
  of two \kl{closed subsets}, which is simply concatenating the two lists 
  of \kl{ideals} reperesenting them.
  To compute the intersection of two \kl{closed subsets} $E_1$ and $E_2$,
  one can assume without loss of generality that both are represented by a 
  single ideal (i.e., that they are irreducible closed subsets), respectively 
  $\idl_1$ and $\idl_2$.
  Then, an easy computation shows that 
  $E_1 \cap E_2 = \setof{q \in Q}{\forall p \in \idl_1 + \idl_2, p(q) = 0}$,
  where $\idl_1 + \idl_2$ is the sum of the two ideals.
  Whether a point $q \in Q$ is in a \kl{closed subset} $E$ is decidable
  because one can evaluate the generating polynomials on $q$ and check that 
  it is indeed $0$.
  The equality check is more complicated, and can be done by first 
  normalizing the list of ideals so that their intersection is trivial,
  which requires computing the intersection of ideals
  and performing equality checks on the resulting \kl{ideals}.

  As a consequence, it suffices to test the \kl{open reachability problem} for
  the \kl{topological well-structured transition system} $(Q, \tau)$ with the
  initial state $q_0$ and the \kl{open subset} $U = Q \setminus E_\text{final}$,
  where $E_\text{final} \defined \setof{q \in Q}{F(q) = 0}$ is the \kl{closed subset}
  of states where the automaton outputs zero.
\end{proof}


\section{Proofs of \cref{sec:undecidability}}

\begin{proofof}{lem:mon-rewrite-red-membership}
  Let $R$ be a monomial rewrite system, and let $\monelt_s, \monelt_t \in
  \mon{\Indets}$ be two monomials. We can encode the problem of deciding whether
  $\monelt_s$ can be rewritten into $\monelt_t$ using the rules of $R$ as an
  instance of the \kl{equivariant ideal membership problem} as follows:
  \begin{itemize}
    \item Let $H$ be the set of all polynomials of the form $\monelt - \monelt'$
      for all pairs
      $(\monelt, \monelt') \in R$.
    \item Then, we ask whether $\monelt_s - \monelt_t$ belongs to the ideal generated by $H$.
  \end{itemize}

  It is clear that if $\monelt_s$ can be rewritten into $\monelt_t$ using the
  rules of $R$, then $\monelt_s - \monelt_t$ belongs to the equivariant ideal generated by
  $H$. Conversely, if $\monelt_s - \monelt_t$ belongs to the ideal generated by
  $H$, then 
  \begin{equation}
    \label{eq:mon-rewrite-red-membership}
    \monelt_s - \monelt_t 
    = 
    \sum_{i=1}^n a_i \monelt[n]_i (\gelem_i \cdot \monelt_i - \gelem_i \cdot \monelt'_i)
    \quad .
  \end{equation}

  Let us write the (finite) graph $G$ whose vertices are the monomials
  $\monelt[n] (\gelem_i \cdot \monelt_i)$ and $\monelt[n] (\gelem_i \cdot
  \monelt'_i)$, and whose edges are the directed weighted edges labelled by
  $a_i$ (in a direction that makes the weight positive).

  Let us now analyse \cref{eq:mon-rewrite-red-membership}, and notice that
  identifying monomials in the left and right-hand sides of the equation allows
  us to show that $\monelt_s$ and $\monelt_t$ are vertices of $G$. Furthermore,
  we deduce that the sum of the weights of the edges having $\monelt_s$ as a
  source or target equals $1$, and that the sum of the weights of the edges
  having $\monelt_t$ as a source or target equals $-1$. Finally, for every
  vertex $v$ of $G$ that is not $\monelt_s$ or $\monelt_t$, the sum of the
  weights of the edges having $v$ as a source or target is $0$, again because
  of an analysis of the coefficient of the monomial $v$ in the sum of
  \cref{eq:mon-rewrite-red-membership}.

  Hence, the graph $G$ is a flow network, with a flow value of at least $1$
  from $\monelt_s$ to $\monelt_t$. As a consequence, there must exist a path
  from $\monelt_s$ to $\monelt_t$ in $G$, which is a witness
  of the fact that 
  one can rewrite $\monelt_s$ into $\monelt_t$ using the rules of $R$.
\end{proofof}
