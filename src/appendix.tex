% LTeX: language=en
\clearpage
\section{Proofs of \cref{sec:weakgb}}
\label{sec:weakgb:proof}


\begin{proofof}{lem:spoly}
  Let $p,q \in \poly{\K}{\X}$, and let $r \in \CancelPoly{p}{q}$.
  By definition, there exists $\alpha,\beta \in \K$ and $\monelt[n], \monelt[m]
  \in \mon{\X}$ such that $r = \alpha \monelt[n] p + \beta \monelt[m] q$ and
  $\lm(r) < \max(\monelt[n] \lm(p), \monelt[m] \lm(q))$.
  In particular,
  we conclude that $\lm(\monelt[n] p) = \lm(\monelt[m] q)$, and that 
  $\alpha \lc(\monelt[n] p) + \beta \lc(\monelt[m] q) = 0$.

  Let us write $\Delta = \lcm(\lm(p), \lm(q))$.
  Because $\lm(\monelt[n] p) = \lm(\monelt[m] q)$, there exists a monomial 
  $\monelt[l] \in \mon{\X}$ such that 
  $\lm(\monelt[n] p) = \monelt[l] \Delta = \lm(\monelt[m] q)$.
  Furthermore,
  we know that $\lc(p) \beta = - \lc(q) \alpha$.
  As a consequence, one can rewrite $r$ as follows:
  \begin{equation*}
    r = 
    \monelt[l] \alpha \lc(p) 
    \left[
      \frac{\Delta}{\lt(p)} \times p
      - \frac{\Delta}{\lt(q)} \times q
    \right]
    = 
    \monelt[l] \alpha \lc(p) \times \spoly{p}{q} \ .
  \end{equation*}
  We have concluded.
\end{proofof}

\begin{proofof}{lem:normalisation}
  Let us write $H = \orbit[\group]{H'}$, where $H'$ is a finite set of
  polynomials.
  Because the relation $\toeucl{H}$ is \kl{terminating}, it suffices to 
  show that for every polynomial $p$, there are finitely many polynomials $r$ 
  such that $p \toeucl{H} r$, leveraging König's lemma. This is because 
  $p \toeucl{H} r$ implies that 
  $p = \alpha \monelt[n] (\gelem\cdot q) + r$ for some $q \in H'$, 
  $\alpha \in \K$, $\monelt[n] \in \mon{\X}$, and $\gelem \in \group$.
  Because, $\lm(r) \revlexlt \lm(p)$, we  
  conclude that $\lm(p) = \lm(\alpha \monelt[n] (\gelem\cdot q))$, and 
  therefore $r$ is uniquely determined by the choice of $q \in H'$ and the
  choice of $\gelem \in \group$ that maps the \kl(poly){domain} of $q$ to
  the \kl(poly){domain} of
  $p$. There are finitely elements in $H'$ and finitely many such functions
  from $\dom(q)$ to $\dom(p)$
  because both domains are finite.
\end{proofof}

\begin{proofof}{lem:weakgb-termination}
  Let $\seqof{H_n}[n \in \N]$ be the sequence of (orbit finite) sets of polynomials
  computed by \cref{alg:weakgb}. 
  We associate to each set $H_n$ the set $L_n$ of \kl{characteristic monomials} of the
  polynomials in $H_n$. Because the set of monomials is a \kl{WQO}, and because 
  the sequences are non-decreasing for inclusion, there exists an 
  $n \in \N$ such that, for every $\monelt \in L_{n+1}$, there exists
  $\monelt[n] \in L_n$, such that $\monelt[n] \gdivleq \monelt$.

  We will prove that $H_{n+1} = H_n$ by contradiction. Assume towards this
  contradiction that there exists some $r \in H_{n+1} \setminus H_n$. By
  definition of $H_{n+1}$, there exists $p,q \in H_n$ such that $r \in
  \rem{H_n}{\spoly{p}{q}}$. In particular, $r$ is \kl{normalised} with respect
  to $H_n$. However, because $r \in H_{n+1}$, $\cm(r) \in L_{n+1}$, and
  therefore there exists $\monelt[n] \in L_n$ such that $\monelt[n] \gdivleq
  \cm(r)$. This provides us with a polynomial $t \in H_n$
  and an element $\gelem \in \group$
  such that $\cm(t) \divleq \gelem \cdot
  \cm(r)$. Because $H_n$ is \kl{equivariant}, we can assume that $\gelem$ is
  the identity. 
  Hence, there exists $\monelt[n] \in
  \mon{\X}$ such that $\cm(t) \times \monelt[n] = \cm(r)$. 
  This means that for every indeterminate $x \in \dom(t)$
  we have $x \in \dom(r)$, and then that
  $\lm(t) \divleq \lm(r)$ by definition of the \kl{characteristic monomial}.
  Therefore, one can find some $\alpha \in \K$ such
  that the polynomial $r' \defined r - \alpha \monelt[n] t$ satisfies $r'
  \pmonlt r$, and in particular, $r \toeucl{H_n} r'$.
  This contradicts the fact that $r$ is \kl{normalised} with respect to $H_n$.
\end{proofof}


\section{Proofs of \cref{sec:equivariant-grobner-basis}}
\label{sec:egb:proof}

\begin{proofof}{lem:colored-hypothesis-sat}
  We need to prove that the set $\freeColor(H)$ is computable and 
  \kl{orbit finite}, that $\poly{\K}{\IndetsCol}$ satistfies 
  the \kl{computability assumptions} of \kl{weakgb},
  and that the set $(\mon{\IndetsCol}, \gdivleq)$ is a
  \kl{well-quasi-ordered} set.
  Finally, we also need to prove that if $H$ is \kl{orbit finite},
  $\forgetCol(H)$ is computable and \kl{orbit finite}. 

  Let us start by proving that $\freeColor(H)$ is computable and \kl{orbit
  finite}. Because $H$ is \kl{orbit finite}, there exists a finite set $H_0
  \subseteq H$ of polynomials such that $\orbit{H_0} = \orbit{H}$. Then, let us
  remark that $\freeColor(H_0)$ can be obtained by considering all finite
  subsets $V$ of variables that appear in $H_0$, which is a computable finite
  set. As a consequence, $\freeColor(H_0)$ is computable, and since
  $\freeColor$ is \kl(func){equivariant}, $\orbit{\freeColor(H_0)} =
  \freeColor(\orbit{H_0}) = \freeColor(H)$.

  Let us now focus on the set $\poly{\K}{\IndetsCol}$. First, it is clear that
  $\group$ is \kl(ord){compatible} with the ordering on $\IndetsCol$ by
  definition of the action, and because $\group$ was compatible with the
  ordering on $\Indets$. Then, the action of $\group$ on $\IndetsCol$ is
  \kl{effectively oligomorphic}
  since orbits of tuples of $\IndetsCol$ can be identified with
  orbits of tuples of $\Indets$ together with a coloring
  in two colors, which is a finite amount of extra information.

  Let us now prove that $(\mon{\IndetsCol}, \gdivleq)$ is a
  \kl{well-quasi-ordered} set. A monomial
  in $\mon{\IndetsCol}$ naturally corresponds to a monomial in $\mon[\N \times
  \N]{\Indets}$, where the two exponents are respectively the one of the lower
  copy and the one of the upper copy of the variable.
  Because $(\mon[\N \times \N]{\Indets}, \gdivleq)$ is a
  \kl{well-quasi-ordered} set, we immediately conclude that $(\mon{\IndetsCol}, \gdivleq)$ is a
  \kl{well-quasi-ordered} set.

  Finally, let us prove that $\forgetCol(H)$ is computable and \kl{orbit
  finite}. This is clear because $\forgetCol$ simply consists in forgetting
  the color of the variables.
\end{proofof}

\begin{proofof}{lem:correct-gen-set}
  Let us remark that
  \begin{equation}
  \forgetCol(\freeColor(H)) = H \quad .
  \end{equation}
  Since $\kl{weakgb}(\freeColor(H))$
  generates the same ideal as $\freeColor(H)$,
  and since $\forgetCol$ is a morphism,
  we conclude that 
  the set of polynomials
  $\forgetCol(\kl{weakgb}(\freeColor(H)))$
  generates the same ideal as
  $\forgetCol(\freeColor(H)) = H$.
\end{proofof}

\begin{proofof}{cor:equivariant-ideals-computations}
  Most of this statement follows from \cref{thm:compute-egb}, using
  \kl{equivariant Gröbner bases} as a representation of \kl{equivariant ideals}.
  Indeed, because $\N \times \N$ is a \kl{well-quasi-ordered} set,
  we conclude $(\mon[\N \times \N]{\Indets}, \gdivleq)$ is a 
  \kl{well-quasi-ordered} set too.
  The only non-trivial part is the fact that one can compute an
  \kl{equivariant Gröbner basis} of the
  \emph{intersection} of two \kl{equivariant ideals}.
  To that end, we will adapt the classical argument using 
  \kl{Gröbner bases} to the case of \kl{equivariant Gröbner bases}
  \cite[Chapter 4, Theorem 11]{CLO15}.

  Let $I$ and $J$ be two \kl{equivariant ideals} of $\poly{\K}{\Indets}$,
  respectively represented by \kl{equivariant Gröbner bases} $\Basis_I$ and
  $\Basis_J$. Let $t$ be a fresh indeterminate, and let us consider $\IndetsCol
  \defined \Indets \ordplus \set{t}$, that is, the disjoint union of $\Indets$
  and $\set{t}$, where $t$ is greater than all the variables in $\Indets$.
  
  We construct the \kl{equivariant ideal} $T$ of $\poly{\K}{\IndetsCol}$,
  generated by all the polynomials $t \times h_i$, and $(1-t) \times h_j$,
  where $h_i$ ranges over $\Basis_I$ and $h_j$ ranges over $\Basis_J$. It is
  clear that $T \cap \poly{\K}{\Indets} = I \cap J$.
  Now, because of the hypotheses on $\Indets$, we know that 
  one can compute the \kl{equivariant Gröbner basis} $\Basis_T$ of $T$
  by applying \kl{egb} to the generating set of $T$.
  Finally, we can obtain the \kl{equivariant Gröbner basis} of $I \cap J$ by
  considering $\Basis_T \cap \poly{\K}{\Indets}$, that is, 
  selecting the polynomials of $\Basis_T$ that do not contain the
  indeterminate $t$, which is possible because $\Basis_T$ is an 
  \kl{orbit-finite set}
  and $\poly{\K}{\IndetsCol}$ is \kl{effectively oligomorphic}.
\end{proofof}


\section{Proofs of \cref{sec:undecidability}}
\label{sec:undecidability:proof}

\begin{proofof}{lem:word-encoding-string-subst}
  Let us write $\gelem \cdot q_0 = p_k$ for some $k \in \N$.
  Because the only indeterminates with degree $4$ in $\wenc{w}_P$ are
  the ones of the form $p_{4i}$, we have that $k$ is a multiple of $4$
  (i.e. at the start of a letter block).
  Since $(q_0, q_1)$ is in the same orbit as $(p_0, p_1)$,
  and both $P$ and $Q$ are \kl(of){finite paths},
  we conclude that $\gelem \cdot (q_0, q_1) = (p_{4i}, p_{4i+1})$
  or $\gelem \cdot (q_0, q_1) = (p_{4i+1}, p_{4i-1})$.
  Applying the same reasoning, thrice, 
  we have either $\gelem \cdot (q_0, q_1, q_2, q_3) = (p_{4i}, p_{4i+1}, p_{4i+2}, p_{4i+3})$
  or $\gelem \cdot (q_0, q_1, q_2, q_3) = (p_{4i}, p_{4i-1}, p_{4i-2}, p_{4i-3})$.
  However, in the second case, the exponent of $p_{4i-3}$ in $\wenc{w}_P$ is at most $2$,
  which is incompatible with the fact that the one of $q_3$ in $\wenc{u}_Q$ is $3$.
  By induction on the length of $u$, we immediately obtain that 
  $\gelem \cdot \wenc{u}_Q = \mathsf{shift}_{+4i} \cdot \wenc{u}_P$ and
  therefore that 
  $w = x u y$ for some $x,y \in \Sigma^*$.
  Finally, because $\wenc{v}_Q$ uses exactly the same indeterminates as 
  $\wenc{u}_Q$, we can also conclude that
  $\wenc{xvy}_P = \monelt[n]$.
\end{proofof}

\begin{proofof}{lem:reversible-machine}
  Transitions of the deterministic reversible Turing machine using bounded tape size can be 
  modelled as a reversible string rewriting system using finitely many rules 
  of the form $u \leftrightarrow v$, where $u$ and $v$ are words
  over $(Q \uplus \Sigma \uplus \square)$ having the same length $\ell$.
  For each rule $u \leftrightarrow v$, we create rules 
  $\wenc{u}_P \leftrightarrow_{R_M} \wenc{v}_P$ 
  for every \kl(of){finite path} $P$ of length $4\ell$.
  Note that there are only orbit finitely many such \kl(of){finite paths} $P$,
  and one can effectively list some representatives,
  because $\Indets$ is \kl{effectively oligomorphic}.
  This system is clearly complete, in the sense that one can perform a substitution
  by applying a monomial rewriting rule, but \cref{lem:word-encoding-string-subst}
  also tells us it is correct, in the sense that it cannot perform anything else
  than string substitutions.
  Furthermore, we can assume 
  that the reversible Turing machine
  starts with a clean tape and ends with a clean tape.
\end{proofof}

\begin{proofof}{lem:tape-creation}
  We create the following rules,
  where $P_1$ and $P_2$ range over \kl(of){finite paths} such that
  their first two elements are in the same orbit as $(x_0, x_1)$,
  and assuming that the indeterminates of $P_1$ and $P_2$ are disjoint:
  \begin{enumerate}
    \item Cell creation: 
      \[
        \wenc{\triangleright^{\text{pre}} \square}_{P_1}
        \wenc{ \square_1 \square_2 \triangleleft^{\text{pre}}}_{P_2}
      \leftrightarrow_{R_\text{pre}}
      \wenc{\triangleright^{\text{pre}} \square_1}_{P_1}
      \wenc{ \square \square \square_2 \triangleleft^{\text{pre}}}_{P_2}
      \]
    \item Linearity checking:
      \[ \wenc{\square_1 \square}_{P_1} \wenc{\square_2 \triangleleft^{\text{pre}}}_{P_2}
      \leftrightarrow_{R_\text{pre}}
    \wenc{\square \square_1}_{P_1} \wenc{\square_2 \triangleleft^{\text{pre}}}_{P_2} \]
    \item Phase transition:
      \[ \wenc{\triangleright^{\text{pre}} \square}_{P_1}
       \wenc{\square_1 \square_2 \triangleleft^{\text{pre}}}_{P_2}
      \leftrightarrow_{R_\text{pre}}
       \wenc{\triangleright^{\text{run}} q_0}_{P_1}
     \wenc{\square \square \triangleleft^{\text{run}}}_{P_2} \]
  \end{enumerate}
  Note that there are only orbit finitely many such pairs of monomials,
  and that we can enumerate representative of these orbits because 
  $\Indets$ is \kl{effectively oligomorphic}.

  Let us first argue that this system is complete. Because there exists an
  infinite path $P_{\infty}$, it is indeed possible to reach
  $\wenc{\triangleright^{\text{run}} q_0 \square^n
  \triangleleft^{\text{run}}}_{P_\infty}$ by repeatedly applying the first
  rule, and then the second rule until $\square_1$ reaches the end of the tape,
  and continuing so until one decides to apply the third rule to reach the
  desired tape configuration.

  We now claim that the system is correct, in the sense that it can only reach
  valid tape encodings. First, let us observe that in a rewrite sequence, one
  can always assume that the rewriting takes the form of applying the first
  rule, then the second rule until one cannot apply it anymore, and repeating
  this process until one applies the third rule. Because rule (2) ensures that
  when we add new indeterminates using rule (1), they were not already present
  in the monomial, and because rule (1) ensures that locally the structure of
  the indeterminates remains a \kl(of){finite path}, we can conclude that the
  whole set of indeterminates used come from a \kl(of){finite path} $P'$. As a
  consequence, if one can reach a state where (2) or (3) are applicable, then
  the tape is of the form $\wenc{ \triangleright^{\text{pre}} \square^n
  \square_1 \square_2 \triangleleft^{\text{pre}} }_{P'}$, with $n \geq 1$. It
  follows that when one can apply rule (3), the monomial obtained is of the
  form $\wenc{ \triangleright^{\text{run}} q_0 \square^n
  \triangleleft^{\text{run}} }_{P'}$, where $P'$ is a \kl(of){finite path} such
  that $(p_0', p_1')$ is in the same orbit as $(x_0, x_1)$. 
\end{proofof}

\begin{proofof}{rem:indeterminates-infinite-path}
  Assume that there are arbitrarily long finite paths in $\Indets$.
  Then, one can create an infinite tree whose nodes 
  are representatives of (distinct) orbits of finite paths, whose root is the empty path, and 
  where the ancestor relation is obtained by projecting on a subset of
  indeterminates.
  Because $\Indets$ is \kl{oligomorphic}, there are finitely many 
  nodes at each depth in the tree (i.e. at each length of the finite path).
  Hence, there exists an infinite branch in the tree due to König's lemma,
  and this branch is a witness for the existence of an \kl(of){infinite path}
  in $\Indets$.
\end{proofof}


\begin{proofof}{ex:product-indets}
  Let $\seqof{x_i}[i \in \N]$
  and $\seqof{y_i}[i \in \N]$ be two infinite sets of distinct indeterminates
  in $\Indets$.
  Let us define $P \defined (x_0, y_0), (x_1, y_0), (x_1, y_1), (x_2, y_1), \ldots$.
  The orbits of pairs that define the successor relation 
  are the orbits of $((x_i, y_j), (x_k, y_l))$,
  where $x_i = x_k$ and $y_j \neq y_l$, or where $x_i \neq x_k$ and $y_j = y_l$.
  Because $\Indets$ is \kl{oligomorphic}, there are finitely many such orbits.
  Let us sketch the fact that this defines a generalised path.
  Consider that
  $((x_i, y_j), (x_k, y_l))$ is in the same orbit as $((x_0, y_0), (x_1, y_0))$,
  then there exists $\gelem \in \group$ such that
  $\gelem \cdot (x_i, y_j) = (x_0, y_0)$ and $\gelem \cdot (x_k, y_l) = (x_1, y_0)$,
  but then $\gelem \cdot y_j = \gelem \cdot y_l = y_0$, and because $\gelem$ is 
  invertible, $y_j = y_l$. Similarly, we conclude that $x_i \neq x_k$.
  The same reasoning shows that if
  $((x_i, y_j), (x_k, y_l))$ is in the same orbit as $((x_0, y_0), (x_0, y_1))$,
  then $y_j \neq y_l$ and $x_i = x_k$.
\end{proofof}



\section{Proofs of \cref{sec:applications}}
\label{sec:applications:proof}

\begin{proofof}{cor:orbit-finite-polynomial-automata-zeroness}
  Let us consider an \kl{orbit finite polynomial automaton} 
  $A = (Q, \delta, q_0, F)$. Following the classical \emph{backward procedure} for such
  systems, we will compute a sequence of sets $E_0 \defined \setof{ q \in Q }{
  F(q) = 0 }$, and $E_{i+1} \defined \mathrm{pre}^\forall(E_i) \cap E_i$, where
  $\mathrm{pre}^\forall(E)$ is the set of states $q \in Q$ such that for every
  $a \in \Sigma$, $\delta^*(q,a) \in E$. We will prove that the sequence of
  sets $E_i$ stabilises, and that it is computable. As an immediate
  consequence, it suffices to check that $q_0 \in E_{\infty}$, where $E_\infty$
  is the limit of the sequence $(E_i)_{i \in \N}$, to decide the
  \kl(ofpa){zeroness problem}.

  The only idea of the proof is to notice that all the sets $E_i$ are
  representable as zero-sets of \kl{equivariant ideals} in
  $\poly{\K}{\Indets}$, allowing us to leverage the effective computations of
  \cref{cor:equivariant-ideals-computations}. Given a set $H$ of polynomials,
  we write $\mathcal{V}(H)$ the collections of states $q \in Q$ such that $p(q)
  = 0$ for all $p \in H$.
  It is easy to see that $E_0 = \mathcal{V}(\set{F}) = \mathcal{V}(\idl_0)$,
  where $\idl_0$ is the \kl{equivariant ideal} generated by $F$, since 
  $F \in \poly{\K}{V}$ and $V$ is invariant under the action of $\group$.
  Furthermore, assuming that $E_i = \mathcal{V}(\idl_i)$, we can
  see that 
  \begin{align*}
    \mathrm{pre}^\forall(E_i) 
    & = \setof{ q \in Q }{ \forall a \in \Indets, \delta^*(a,q) \in E_i } \\
    & = \setof{ q \in Q }{ \forall a \in \Indets, \forall p \in \idl_i, p(\delta^*(a,q)) = 0 } \\
    & = \setof{ q \in Q }{ \forall p' \in \idl[J], p'(q) = 0 }
  \end{align*}
  Where, the \kl{equivariant ideal} $\idl[J]$ is generated by the
  polynomials $\mathrm{pullback}(p,a) \defined p [ x \mapsto \delta(a,x)]$
  for every pair $(p, a) \in \idl_i \times \Indets$. 
  As a consequence, we have $E_{i+1} = \mathcal{V}(\idl_{i+1})$, where
  $\idl_{i+1} = \idl_i + \idl[J]$.
  Because the sequence $\seqof{ \idl_i }[ i \in \N]$ is increasing, and thanks
  to the \kl{equivariant Hilbert basis property} of $\poly{\K}{\Indets}$, there
  exists an $n_0 \in \N$ such that $\idl_{n_0} = \idl_{n_0 + 1} = \idl_{n_0 +
  2} = \cdots$. In particular, we do have $E_{n_0} = E_{n_0 + 1} = E_{n_0 + 2}
  = \cdots$.

  Let us argue that we can compute the sequence $\idl_i$.
  First,  $\idl_0 = \EqIdlGen{F}$ is finitely represented.
  Now, 
  given an \kl{equivariant ideal} $\idl$, represented by an \kl{orbit finite}
  set of generators $H$,
  we can compute the \kl{equivariant ideal} $\idl[J]$ generated by the
  polynomials $\mathrm{pullback}(p,a) \defined p [ x_i \mapsto \delta(a)(x_i)]$
  for every pair $(p, a) \in H \times \Indets$. Indeed, $H \times \Indets$ is
  \kl{orbit finite}, and the function $\mathrm{pullback}$ is
  computable and \kl(func){equivariant}: given $\gelem \in \group$, we can
  show that
  \begin{align*}
    & \phantom{=.}\gelem \cdot \mathrm{pullback}(p, a) \\
    & = 
    \gelem \cdot (p [ x_i \mapsto \delta(a,x_i)]) & \text{ by definition }\\
                                                  & = p [ x_i \mapsto (\gelem \cdot \delta(a, x_i))] 
                                                  & \text{ $\gelem$ acts as a morphism } \\
    & = p [ x_i \mapsto \delta(\gelem \cdot a, \gelem \cdot x_i))] 
    & \text{ $\delta$ is \kl(func){equivariant} } \\
    & = (\gelem \cdot p) [ x_i \mapsto \delta(\gelem \cdot a, x_i)] 
    & \text{ definition of substitution }
    \\
    & = \mathrm{pullback}(\gelem \cdot p, \gelem \cdot a).
    & \text{ by definition.}
  \end{align*}
%  
  Finally, one can detect when the sequence stabilises, by checking whether
  $\idl_i = \idl_{i+1}$, which is decidable because the
  \kl{equivariant ideal membership problem} is decidable 
  by \cref{thm:compute-egb}.

  To conclude, it remains to check whether $q_0 \in E_\infty$,
  which amounts to check that $q_0 \in \mathcal{V}(\idl_\infty)$.
  This is equivalent to checking whether for every element $p \in \Basis$
  where $\Basis$ is an \kl{equivariant Gröbner basis} of $\idl_\infty$, we have
  $p(q_0) = 0$, which can be done by enumerating relevant orbits.
\end{proofof}
