% LTeX: language=en
%!TEX root = ../atomic.sigconf.tex
\section{Computing the Equivariant Gröbner Basis}
\label{sec:equivariant-grobner-basis}

\AP The goal of this section is to prove
\cref{thm:compute-egb},
that is, to show that one can effectively compute an \kl{equivariant Gröbner
basis} of an \kl{equivariant ideal}. To that end, we will apply the algorithm
\kl{weakgb} on a slightly modified set of polynomials, and then show that the
result is indeed an \kl{equivariant Gröbner basis}.

\AP
Let us fix a set $\Indets$ of indeterminates equipped with a total ordering
$\varleq$. We define $\IndetsCol \defined \Indets \ordplus \Indets$, that is, the
disjoint union of two copies of $\Indets$, ordered. It will be useful to refer
to the first copy (lower copy) and the second copy (upper copy), noting the
isomorphism between $\IndetsCol$ and $\set{\mathsf{first}, \mathsf{second}}
\times \Indets$, ordered lexicographically, where $\mathsf{first} <
\mathsf{second}$. We will also define $\intro*\forgetCol \colon \IndetsCol \to
\Indets$ that maps a colored variable to its underlying variable.
Beware that $\forgetCol$ is not an order preserving map.
We extend $\forgetCol$ as a morphism from polynomials in
$\poly{\K}{\IndetsCol}$ to polynomials in $\poly{\K}{\Indets}$.

\AP
Given a subset $V \subfin \Indets$, we build the injection
$\intro*\colorWith{V} \colon \Indets \to \IndetsCol$ that maps variables $x$ in
$V$ to $(\mathsf{first}, x)$, and variables $x$ not in $V$ to
$(\mathsf{second}, x)$. Again, we extend these maps as morphisms from
$\poly{\K}{\Indets}$ to $\poly{\K}{\IndetsCol}$. We say that a polynomial $p
\in \poly{\K}{\IndetsCol}$ is \intro{$V$-compatible} if $p \in
\colorWith{V}(\poly{\K}{\Indets})$. Using these definitions, we create
$\intro*\freeColor$ that maps a set $H$ of polynomials to the union over all
finite subsets $V$ of $\Indets$ of the set $\colorWith{V}(H)$. Beware that
$\freeColor$ does not equal $\forgetCol^{-1}$, since we only consider
\kl{$V$-compatible} polynomials (for some finite set $V$).

\AP
We are now ready to write our algorithm to compute an \kl{equivariant Gröbner
basis} by computing the ``congugacy'' 
\[
\intro{egb} \defined \forgetCol \circ
\mathop{\kl{weakgb}} \circ \freeColor \ .
\]
To prove the correctness of our algorithm,
let us first argue that one can indeed compute the \kl{weak equivariant Gröbner basis} algorithm.

\begin{lemma}[name={},restate={lem:colored-hypothesis-sat}]
  \label{lem:colored-hypothesis-sat}
  Assume that $\group \actson \Indets$ is
  is \kl{effectively oligomorphic},
  and that $(\mon[\N \times \N]{\Indets}, \gdivleq)$
  is a \kl{well-quasi-order}.
  Then $\kl{egb}$ is a computable function,
  and the function $\kl{weakgb}$ is called 
  on correct inputs.
  \proofref{lem:colored-hypothesis-sat}
\end{lemma}


Let us now argue that the result of \kl{egb} is indeed a generating set of the
ideal (\cref{lem:correct-gen-set}), and then refine our analysis to
prove that it is an \kl{equivariant Gröbner basis}
(\cref{lem:strong-gb-correct}).

\begin{lemma}[name={},restate={lem:correct-gen-set}]
  \label{lem:correct-gen-set}
  Let $H \subseteq \poly{\K}{\Indets}$,
  then $\mathsf{egb}(H)$
  \kl(idl){generates}
  $\EqIdlGen{H}$.
  \proofref{lem:correct-gen-set}
\end{lemma}


\begin{lemma}
  \label{lem:strong-gb-correct}
  Let $H \subseteq \poly{\K}{\Indets}$,
  then $\mathsf{egb}(H)$
  is an \kl{equivariant Gröbner basis}
  of $\EqIdlGen{H}$.
\end{lemma}
\begin{proof}
  Let $H_\star = \freeColor(H)$,
  $\Basis_\star = \kl{weakgb}(H_\star)$,
  and $\Basis = \forgetCol(\Basis_\star)$.
  We want to prove that $\Basis$ is an \kl{equivariant Gröbner basis} of
  $\IdlGen{H}$.
  Let us consider an arbitrary polynomial
  $p \in \EqIdlGen{H}$, our goal is to construct an 
  $h \in \Basis$ such that $\lm(h) \divleq \lm(p)$
  and $\dom(h) \subseteq \dom(p)$.

  Let us define 
  $V \defined \dom(p)$
  and 
  $H_V \defined \colorWith{V}(H)$.
  It is clear that $\colorWith{V}(p)$
  belongs to $\IdlGen{H_V}$.
  Let us write 
  \begin{equation*}
    \colorWith{V}(p) = \sum_{i=1}^n a_i \monelt_i h_i
  \end{equation*}
  Where $a_i \in \K$, $\monelt_i \in \mon{\IndetsCol}$,
  and $h_i \in \Basis_\star$ is \kl{$V$-compatible}.
  Such a decomposition $\mathfrak{d}$ exists
  because $H_V \subseteq H_\star \subseteq \Basis_\star$.

  Now, because $\Basis_\star$ is a \kl{weak equivariant Gröbner basis} of $\IdlGen{H_\star}$,
  there exists a decomposition $\mathfrak{d}'$ of $\colorWith{V}(p)$
  such that
  $\lm(\colorWith{V}(p)) = \lmdec(\mathfrak{d}') \revlexleq \lmdec(\mathfrak{d})$,
  and 
  $\domdec(\mathfrak{d}') \subseteq \domdec(\mathfrak{d})$.
  In particular, $\mathfrak{d}'$ is a decomposition of $\colorWith{V}(p)$
  using only \kl{$V$-compatible} polynomials in $\Basis_\star$.

  Let us consider some element  $(a_i', \monelt[m]_i', h_i')$
  of the \kl{decomposition} $\mathfrak{d}'$ 
  such that 
  $\lm(\monelt[m]_i' h_i') = \lm(\colorWith{V}(p))$, which exists 
  by assumption on $\mathfrak{d}'$. Since 
  $\dom(\monelt[m]_i' h_i) \subseteq 
  \dwset{\dom(\lm(\colorWith{V}(p)))}$,
  we conclude that all variables of $\monelt[m]_i' h_i'$
  are in the first copy of $\IndetsCol$.
  Furthermore, since $h_i'$ is \kl{$V$-compatible},
  we conclude that all variables of $h_i'$ 
  correspond to variables in $V$ in the first copy of $\IndetsCol$.
  Similarly, all variables of $\monelt[m]_i'$
  correspond to variables in $V$ in the first copy of $\IndetsCol$.

  Therefore, $\colorWith{V}(\forgetCol(h_i')) = h_i'$
  and $\colorWith{V}(\forgetCol(\monelt[m]_i')) = \monelt[m]_i'$.
  If we define $h \defined \forgetCol(h_i')$ and
  $\monelt \defined \forgetCol(\monelt[m]_i')$,
  we conclude that 
  $\lm(p) = \lm(\monelt h)$.
  We have proven that $\forgetCol(\Basis_\star)$ is 
  an \kl{equivariant Gröbner basis} of $\EqIdlGen{H}$.
\end{proof}

As a consequence, \kl{egb} is the algorithm of
\cref{thm:compute-egb},
and in particular obtain as a corollary that one can decide the \kl{equivariant
ideal membership problem} under our \kl{computability assumptions}, if the set
of indeterminates satisfies that $(\mon[\N \times \N]{\Indets}, \gdivleq)$ is a
\kl{well-quasi-ordered} set. We can leverage these decidability results to
obtain effective representations of \kl{equivariant ideals}, which can then be
used in algorithms as we will see in \cref{sec:applications}.

\begin{corollary}[name={},restate={cor:equivariant-ideals-computations}]
  \label{cor:equivariant-ideals-computations}
  Assume that $\group \actson \Indets$
  is \kl{effectively oligomorphic},
  and that $(\mon[Y]{\Indets}, \gdivleq)$
  is a \kl{well-quasi-ordered} set
  for every \kl{well-quasi-ordered} set $(Y,\leq)$.
  Then one has an \emph{effective representation} of
  the \kl{equivariant ideals} of $\poly{\K}{\Indets}$,
  such that:
  \begin{enumerate}
    \item One can obtain a representation from an orbit-finite set of generators,
    \item One can effectively decide the \kl{equivariant ideal membership problem}
      given a representation,
    \item The following operations are computable at the level of representations:
      the union of two \kl{equivariant ideals}, 
      the product of two \kl{equivariant ideals},
      the intersection of two \kl{equivariant ideals},
      and checking whether two \kl{equivariant ideals} are equal.
  \end{enumerate}
  \proofref{cor:equivariant-ideals-computations}
\end{corollary}

