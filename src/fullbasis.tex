% LTeX: language=en
\section{Computing the Equivariant Gröbner Basis}
\label{sec:equivariant-grobner-basis}

The goal of this section is to strengthen the results of \cref{sec:algorithm},
and instead of being able to answer to the \kl{equivariant ideal membership
problem}, to compute an \kl{equivariant Gröbner basis} of an equivariant ideal.
Let us recall that a Gröbner basis is known to exist, but that it's
computability was an open question of \cite{GHOLAS24}.

The proof will closely follow the one of \cref{sec:algorithm}, where one starts
from a generating set $H$, and constructs a new set $H'$ together with a group
action, over which one computes a \kl{weak equivariant Gröbner basis}
(\cref{sec:weakgb}). The result is then used to derive an \kl{equivariant
Gröbner basis} of the \kl{equivariant ideal generated by} $H$. Informally, one
wants to apply the technique of isolating finite sets of variables
\emph{uniformly}, that is, compute a set of polynomials that is a
\kl{$V$-strong equivariant Gröbner basis} for every finite subset $V \subfin
\Indets$ of indeterminates.

Let us fix a set $\Indets$ of indeterminates equipped with a total ordering
$\varleq$. We define $\IndetsCol \defined \Indets \ordplus \Indets$, that is, the
disjoint union of two copies of $\Indets$, ordered. It will be useful to refer
to the first copy (lower copy) and the second copy (upper copy), noting the
isomorphism between $\IndetsCol$ and $\set{\mathsf{first}, \mathsf{second}}
\times \Indets$, ordered lexicographically, where $\mathsf{first} <
\mathsf{second}$. We will also define $\forgetCol \colon \IndetsCol \to
\Indets$ that maps a colored variable to its underlying variable.
Beware that $\forgetCol$ is not an order preserving map.
We extend $\forgetCol$ as a morphism from polynomials in
$\poly{\K}{\IndetsCol}$ to polynomials in $\poly{\K}{\Indets}$.

Given a subset $V \subfin \Indets$, we build the injection $\colorWith{V}
\colon \Indets \to \IndetsCol$ that maps variables $x$ in $V$ to
$(\mathsf{fisrt}, x)$, and variables $x$ not in $V$ to $(\mathsf{second}, x)$.
Again, we extend these maps as morphisms from $\poly{\K}{\Indets}$ to
$\poly{\K}{\IndetsCol}$. We say that a polynomial $p \in \poly{\K}{\IndetsCol}$
is \intro{$V$-compatible} if $p \in \colorWith{V}(\poly{\K}{\Indets})$.

\begin{lemma}
  \label{lem:v-saturation-computable}
  Let $H$ be an \kl{orbit finite} subset of $\poly{\K}{\Indets}$.
  Then, the free coloring of $H$, 
  defined as
  $\freeColor(H) \defined \bigcup_{V \subfin \Indets} \colorWith{V}(H)$,
  is a computable \kl{orbit finite} subset of $\poly{\K}{\IndetsCol}$.
\end{lemma}


We are now ready to write our algorithm to compute 
an \kl{equivariant Gröbner basis}.

\begin{algorithm}
    \caption{Computing \kl{equivariant Gröbner bases}}
    \label{alg:stronggb}
    \KwIn{An orbit finite set $H$ of polynomials}
    \KwOut{An \kl{equivariant Gröbner basis} of
      $\EqIdlGen{H}$}
    \Begin{
        $H_C \gets \freeColor(H)$\;
        $\Basis_C \gets \mathsf{weakgb}(H_C)$\;
        $\Basis \gets \forgetCol(\Basis_C)$\;
        \Return{$\Basis$}\;
    }
\end{algorithm}

To prove the correctness of our algorithm, let us first argue that one can
indeed compute the \kl{weak equivariant Gröbner basis} algorithm.

\begin{lemma}
  \label{lem:colored-hypothesis-sat}
  Assume that $(\Indets, \varleq, \group)$
  is \kl{effectively oligomorphic},
  and that $(\mon[\om^2]{\Indets}, \gdivleq)$
  is a \kl{well-quasi-order}.
  Then,
  $\IndetsCol$ with its ordering and the 
  action of $\group$ acting on both components 
  simultaneously is \kl{effectively oligomorphic},
  and $(\mon{\IndetsCol}, \gdivleq)$ is a
  \kl{well-quasi-ordered} set.
\end{lemma}
\begin{proof}
  \todo[inline]{WQO is obvious, but effectively oligomorphic
  should be written down.}
\end{proof}

Now, let us argue that the result of our algorithm
is a generating set of the desired ideal, which follows
from the fact that $\forgetCol$ and $\colorWith{\cdot}$
are morphisms that preserve variable names.

\begin{lemma}
  \label{lem:correct-gen-set}
  Let $H$ be an \kl{orbit finite} subset of $\poly{\K}{\Indets}$,
  then the result of \cref{alg:stronggb}
  is an \kl{orbit finite} generating set
  of $\EqIdlGen{H}$.
\end{lemma}
\begin{proof}
  Let us remark that $\forgetCol(\freeColor(H)) = H$.
  Because we know that $\mathsf{weakgb}(\freeColor(H))$
  generates the same ideal as $\freeColor(H)$,
  and since $\forgetCol$ is a morphism,
  we conclude that 
  the set of polynomial
  $\forgetCol(\mathsf{weakgb}(\freeColor(H)))$
  generates the same ideal as
  $\forgetCol(\freeColor(H)) = H$.
\end{proof}

Let us now prove that the resulting set in indeed
an \kl{equivariant Gröbner basis} of $\EqIdlGen{H}$.

\begin{lemma}
  \label{lem:strong-gb-correct}
  \Cref{alg:stronggb} is correct.
\end{lemma}
\begin{proof}
  Let $p \in \IdlGen{H}$,
  $H_\star = \freeColor(H)$,
  $V \defined \dom(p)$,
  $H_V \defined \colorWith{V}(H)$.
  We let $\Basis_\star = \mathsf{weakgb}(H_\star)$.
  Finally, $\Basis = \forgetCol(\Basis_\star)$.

  It is clear that $\colorWith{V}(p)$
  belongs to $\IdlGen{H_V}$.
  
  Let us write 
  \begin{equation*}
    \colorWith{V}(p) = \sum_{i=1}^n a_i \monelt_i h_i
  \end{equation*}
  Where $a_i \in \K$, $\monelt_i \in \mon{\IndetsCol}$,
  and $h_i \in \Basis_\star$ is \kl{$V$-compatible}.
  Such a decomposition $\mathfrak{d}$ exists
  because $H_V \subseteq H_\star \subseteq \Basis_\star$.

  Now, because $\Basis_\star$ is a \kl{weak equivariant Gröbner basis} of $\IdlGen{H_\star}$,
  there exists a decomposition $\mathfrak{d}'$ of $\colorWith{V}(p)$
  such that
  $\lm(\colorWith{V}(p)) = \lmdec(\mathfrak{d}') \revlexleq \lmdec(\mathfrak{d})$,
  and 
  $\domdec(\mathfrak{d}') \subseteq \domdec(\mathfrak{d})$.
  In particular, $\mathfrak{d}'$ is a decomposition of $\colorWith{V}(p)$
  using only \kl{$V$-compatible} polynomials in $\Basis_\star$.

  This proves that there exists $\monelt \in \mon{\IndetsCol}$ and $h \in
  \Basis_\star$ such that $\lm(\colorWith{V}(p)) = \lm(\monelt h)$ and $\monelt
  h$ is \kl{$V$-compatible}. In particular, $\monelt h$ must have all its
  variables in $V$, and therefore $\lm(\colorWith{V}(p)) =
  \colorWith{V}(\lm(p)) = \lm(\forgetCol(\monelt h))$. We conclude that
  $\lm(\forgetCol(h)) \divleq \lm(p)$, and that $\forgetCol(h)$ has all its
  variables in $V$.

  We have proven that $\forgetCol(\Basis_\star)$ is 
  an \kl{equivariant Gröbner basis} of $\EqIdlGen{H}$.
\end{proof}



We can now state our main result of this section.

\csname thm:compute-equiv-gb\endcsname*
