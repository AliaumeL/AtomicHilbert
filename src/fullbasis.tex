% LTeX: language=en
\section{Computing the Equivariant Gröbner Basis}
\label{sec:equivariant-grobner-basis}

\AP The goal of this section is to prove
\cref{thm:compute-egb},
that is, to show that one can effectively compute an \kl{equivariant Gröbner
basis} of an \kl{equivariant ideal}. To that end, we will apply the algorithm
\kl{weakgb} on a slightly modified set of polynomials, and then show that the
result is indeed an \kl{equivariant Gröbner basis}.

\AP
Let us fix a set $\Indets$ of indeterminates equipped with a total ordering
$\varleq$. We define $\IndetsCol \defined \Indets \ordplus \Indets$, that is, the
disjoint union of two copies of $\Indets$, ordered. It will be useful to refer
to the first copy (lower copy) and the second copy (upper copy), noting the
isomorphism between $\IndetsCol$ and $\set{\mathsf{first}, \mathsf{second}}
\times \Indets$, ordered lexicographically, where $\mathsf{first} <
\mathsf{second}$. We will also define $\intro*\forgetCol \colon \IndetsCol \to
\Indets$ that maps a colored variable to its underlying variable.
Beware that $\forgetCol$ is not an order preserving map.
We extend $\forgetCol$ as a morphism from polynomials in
$\poly{\K}{\IndetsCol}$ to polynomials in $\poly{\K}{\Indets}$.

\AP
Given a subset $V \subfin \Indets$, we build the injection
$\intro*\colorWith{V} \colon \Indets \to \IndetsCol$ that maps variables $x$ in
$V$ to $(\mathsf{first}, x)$, and variables $x$ not in $V$ to
$(\mathsf{second}, x)$. Again, we extend these maps as morphisms from
$\poly{\K}{\Indets}$ to $\poly{\K}{\IndetsCol}$. We say that a polynomial $p
\in \poly{\K}{\IndetsCol}$ is \intro{$V$-compatible} if $p \in
\colorWith{V}(\poly{\K}{\Indets})$. Using these definitions, we create
$\intro*\freeColor$ that maps a set $H$ of polynomials to the union over all
finite subsets $V$ of $\Indets$ of the set $\colorWith{V}(H)$. Beware that
$\freeColor$ does not equal $\forgetCol^{-1}$, since we only consider
\kl{$V$-compatible} polynomials (for some finite set $V$).

\AP
We are now ready to write our algorithm to compute an \kl{equivariant Gröbner
basis} by computing the ``congugacy'' $\intro{egb} \defined \forgetCol \circ
\mathop{\kl{weakgb}} \circ \freeColor$. To prove the correctness of our
algorithm, let us first argue that one can indeed compute the \kl{weak
equivariant Gröbner basis} algorithm.

\begin{lemma}
  \label{lem:colored-hypothesis-sat}
  Assume that $(\Indets, \varleq, \group)$
  is \kl{effectively oligomorphic},
  and that $(\mon[\N \times \N]{\Indets}, \gdivleq)$
  is a \kl{well-quasi-order}.
  Then $\kl{egb}$ is a computable function,
  and the function $\kl{weakgb}$ is called 
  on correct inputs.
\end{lemma}
\begin{proof}
  We need to prove that the set $\freeColor(H)$ is computable and 
  \kl{orbit finite}, that $\poly{\K}{\IndetsCol}$ satistfies 
  the \kl{computability assumptions} of \kl{weakgb},
  and that $(\mon{\IndetsCol}, \gdivleq)$ is a
  \kl{well-quasi-ordered} set.
  Finally, we also need to prove that if $H$ is \kl{orbit finite},
  $\forgetCol(H)$ is computable and \kl{orbit finite}. 

  Let us start by proving that $\freeColor(H)$ is computable and \kl{orbit
  finite}. Because $H$ is \kl{orbit finite}, there exists a finite set $H_0
  \subseteq H$ of polynomials such that $\orbit{H_0} = \orbit{H}$. Then, let us
  remark that $\freeColor(H_0)$ can be obtained by considering all finite
  subsets $V$ of variables that appear in $H_0$, which is a computable finite
  set. As a consequence, $\freeColor(H_0)$ is computable, and since
  $\freeColor$ is \kl(func){equivariant}, $\orbit{\freeColor(H_0)} =
  \freeColor(\orbit{H_0}) = \freeColor(H)$.

  Let us now focus on the set $\poly{\K}{\IndetsCol}$. First, it is clear that
  $\group$ is \kl(ord){compatible} with the ordering on $\IndetsCol$ by
  definition of the action, and because $\group$ was compatible with the
  ordering on $\Indets$. Then, $\poly{\K}{\IndetsCol}$ is an \kl{effectively
  oligomorphic} because a polynomial $p$ can be represented in $(\K \times
  (\Indets)^{\leq d} \times (\Indets)^{\leq d})^n$, where $n$ is the number of
  monomials in $p$, and $d$ is the maximal degree of a monomial in $p$.

  Let us now prove that $(\mon{\IndetsCol}, \gdivleq)$ is a
  \kl{well-quasi-ordered} set. To that end, consider a sequence
  $\seqof{\monelt_i}{i \in \N}$ of monomials in $\mon{\IndetsCol}$. A monomial
  in $\mon{\IndetsCol}$ naturally corresponds to a monomial in $\mon[\N \times
  \N]{\Indets}$, where the two exponents are respectively the one of the lower
  copy and the one of the upper copy of the variable.
  Because $(\mon[\N \times \N]{\Indets}, \gdivleq)$ is a
  \kl{well-quasi-ordered} set, we immediately conclude that $(\mon{\IndetsCol}, \gdivleq)$ is a
  \kl{well-quasi-ordered} set.

  Finally, let us prove that $\forgetCol(H)$ is computable and \kl{orbit
  finite}. This is clear because $\forgetCol$ simply consists in forgetting
  the color of the variables.
\end{proof}

Let us now argue that the result of \kl{egb} is indeed a generating set of the
ideal (\cref{lem:correct-gen-set}), and then refine our analysis to
prove that it is an \kl{equivariant Gröbner basis}
(\cref{lem:strong-gb-correct}).

\begin{lemma}
  \label{lem:correct-gen-set}
  Let $H \subseteq \poly{\K}{\Indets}$,
  then $\mathsf{egb}(H)$
  \kl(idl){generates}
  $\EqIdlGen{H}$.
\end{lemma}
\begin{proof}
  Let us remark that $\forgetCol(\freeColor(H)) = H$.
  Because we know that $\kl{weakgb}(\freeColor(H))$
  generates the same ideal as $\freeColor(H)$,
  and since $\forgetCol$ is a morphism,
  we conclude that 
  the set of polynomial
  $\forgetCol(\kl{weakgb}(\freeColor(H)))$
  generates the same ideal as
  $\forgetCol(\freeColor(H)) = H$.
\end{proof}

\begin{lemma}
  \label{lem:strong-gb-correct}
  Let $H \subseteq \poly{\K}{\Indets}$,
  then $\mathsf{egb}(H)$
  is an \kl{equivariant Gröbner basis}
  of $\EqIdlGen{H}$.
\end{lemma}
\begin{proof}
  Let $p \in \IdlGen{H}$,
  $H_\star = \freeColor(H)$,
  $V \defined \dom(p)$,
  $H_V \defined \colorWith{V}(H)$.
  We let $\Basis_\star = \kl{weakgb}(H_\star)$.
  Finally, $\Basis = \forgetCol(\Basis_\star)$.
  It is clear that $\colorWith{V}(p)$
  belongs to $\IdlGen{H_V}$.
  Let us write 
  \begin{equation*}
    \colorWith{V}(p) = \sum_{i=1}^n a_i \monelt_i h_i
  \end{equation*}
  Where $a_i \in \K$, $\monelt_i \in \mon{\IndetsCol}$,
  and $h_i \in \Basis_\star$ is \kl{$V$-compatible}.
  Such a decomposition $\mathfrak{d}$ exists
  because $H_V \subseteq H_\star \subseteq \Basis_\star$.

  Now, because $\Basis_\star$ is a \kl{weak equivariant Gröbner basis} of $\IdlGen{H_\star}$,
  there exists a decomposition $\mathfrak{d}'$ of $\colorWith{V}(p)$
  such that
  $\lm(\colorWith{V}(p)) = \lmdec(\mathfrak{d}') \revlexleq \lmdec(\mathfrak{d})$,
  and 
  $\domdec(\mathfrak{d}') \subseteq \domdec(\mathfrak{d})$.
  In particular, $\mathfrak{d}'$ is a decomposition of $\colorWith{V}(p)$
  using only \kl{$V$-compatible} polynomials in $\Basis_\star$.

  This proves that there exists $\monelt \in \mon{\IndetsCol}$ and $h \in
  \Basis_\star$ such that $\lm(\colorWith{V}(p)) = \lm(\monelt h)$ and $\monelt
  h$ is \kl{$V$-compatible}. In particular, $\monelt h$ must have all its
  variables in $V$, and therefore $\lm(\colorWith{V}(p)) =
  \colorWith{V}(\lm(p)) = \lm(\forgetCol(\monelt h))$. We conclude that
  $\lm(\forgetCol(h)) \divleq \lm(p)$, and that $\forgetCol(h)$ has all its
  variables in $V$.

  We have proven that $\forgetCol(\Basis_\star)$ is 
  an \kl{equivariant Gröbner basis} of $\EqIdlGen{H}$.
\end{proof}

As a consequence, \kl{egb} is the algorithm of
\cref{thm:compute-egb},
and in particular obtain as a corollary that one can decide the \kl{equivariant
ideal membership problem} under our \kl{computability assumptions}, if the set
of indeterminates satisfies that $(\mon[\N \times \N]{\Indets}, \gdivleq)$ is a
\kl{well-quasi-ordered} set. We can leverage these decidability results to
obtain effective representations of \kl{equivariant ideals}, which can then be
used in algorithms as we will see in \cref{sec:applications}.

\begin{corollary}
  \label{cor:equivariant-ideals-computations}
  Assume that $(\Indets, \varleq, \group)$
  is \kl{effectively oligomorphic},
  and that $(\mon[Y]{\Indets}, \gdivleq)$
  is a \kl{well-quasi-ordered} set
  for every \kl{well-quasi-ordered} set $(Y,\leq)$.
  Then one has an \emph{effective representation} of
  the \kl{equivariant ideals} of $\poly{\K}{\Indets}$,
  such that:
  \begin{enumerate}
    \item One can obtain a representation from a finite set of generators,
    \item One can effectively decide the \kl{equivariant ideal membership problem}
      given a representation,
    \item The following operations are computable at the level of representations:
      the union of two \kl{equivariant ideals}, 
      the product of two \kl{equivariant ideals},
      the intersection of two \kl{equivariant ideals},
      and checking whether two \kl{equivariant ideals} are equal.
  \end{enumerate}
\end{corollary}
\begin{proof}
  Most of this statement follows from \cref{thm:compute-egb}, using
  \kl{equivariant Gröbner bases} as a representation of \kl{equivariant ideals}.
  Indeed, because $\N \times \N$ is a \kl{well-quasi-ordered} set,
  we conclude $(\mon[\N \times \N]{\Indets}, \gdivleq)$ is a 
  \kl{well-quasi-ordered} set too.
  The only non-trivial part is the fact that one can compute an
  \kl{equivariant Gröbner basis} of the
  \emph{intersection} of two \kl{equivariant ideals}.
  To that end, we will adapt the classical argument using 
  \kl{Gröbner bases} to the case of \kl{equivariant Gröbner bases}
  \cite[Chapter 4, Theorem 11]{CLO15}.

  Let $I$ and $J$ be two \kl{equivariant ideals} of $\poly{\K}{\Indets}$,
  respectively represented by \kl{equivariant Gröbner bases} $\Basis_I$ and
  $\Basis_J$. Let $t$ be a fresh indeterminate, and let us consider $\IndetsCol
  \defined \Indets \ordplus \set{t}$, that is, the disjoint union of $\Indets$
  and $\set{t}$, where $t$ is greater than all the variables in $\Indets$.
  
  We construct the \kl{equivariant ideal} $T$ of $\poly{\K}{\IndetsCol}$,
  generated by all the polynomials $t \times h_i$, and $(1-t) \times h_j$,
  where $h_i$ ranges over $\Basis_I$ and $h_j$ ranges over $\Basis_J$. It is
  clear that $T \cap \poly{\K}{\Indets} = I \cap J$.
  Now, because of the hypotheses on $\Indets$, we know that 
  one can compute the \kl{equivariant Gröbner basis} $\Basis_T$ of $T$
  by applying \kl{egb} to the generating set of $T$.
  Finally, we can obtain the \kl{equivariant Gröbner basis} of $I \cap J$ by
  considering $\Basis_T \cap \poly{\K}{\Indets}$, that is, 
  selecting the polynomials of $\Basis_T$ that do not contain the
  indeterminate $t$, which is possible because $\Basis_T$ is an 
  \kl{orbit finite set}
  and $\poly{\K}{\IndetsCol}$ is \kl{effectively oligomorphic}.
\end{proof}
