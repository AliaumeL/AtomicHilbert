%!TEX root = ../atomic.ieee.tex
% LTeX: language=en
%
\section{Nullstellensatz without Hilbert's basis property}
%
\arka{Merge with the previous section.}
%
\begin{enumerate}
    \item Recall Serge Langs results on Nullstellensatz with infinitely many variables.
    \item Give example of proper ideal which does not have an equivariant zero set.
    \item As a consequence show that it is not contained in any maximal ideal that is equivariant.
    \item As a consequence, the equivariantly maximal ideal that contains it is not maximal.
\end{enumerate}
%
In this section we give an example of an action $\calH\actson\Indets$ which does not satisfy the \kl{equivariant Hilbert basis property} and does not satisfy equivariant weak Nullstellensatz (cf. \Cref{thm:weak null}) and hence also equivariant strong Nullstellensatz as it is a stronger statement.
We also show that there are equivariantly maximal ideals in $\poly{\K}{\Indets}$ that are not maximal. 
%

Let $\calH = \aut{\A}$ and $\Indets = \A^{(2)}$.
The action is given by
$\pi((a,b)) = (\pi(a),\pi(b))$ for $(a,b)\in\A^{(2)}$.
By \Cref{lem:product-not-wqo},
\arka{I think here we also have to cite \cite[Theorem 11]{GHOLAS24}. Probably better to cite in the intro}
$\aut{\A}\actson\A^{(2)}$ does have the \kl{equivariant Hilbert's basis property}.
Pick an enumeration $a_1,a_2,\dots$ of the elements of $\A$.
For $n \geq 3$ define $c_n$ to be the monomial
\[
c_n = (a_1,a_2)(a_2,a_3)\dots (a_{n-1},a_n)(a_n,a_1) \ .
\]
\arka{Maybe give this as an example in one of the previous section?}
For $n\geq 3$, Let $\idl[C]_n$ be the equivariant ideal generated by $\setof{c_m + 1}{m\in\set{3,\dots,n}}$.
%
\arka{Easier example}
\begin{example}
$X = \A^{(2)}$ and $G = \A^{(2)}$.
$\idl[I]$ is generated by
\[
\setof{(a,b)(1 - (a,b)),
(a,b) + (b,a) - 1}{(a,b)\in\A^{(2)}}
\]
\end{example}
%
\begin{lemma}\label{lem:C4 no equiv zero}
    The equivariant variety $\zeros{\idl[C]_4}$ is empty.
\end{lemma}
%
\begin{lemma}\label{lem:C4 proper}
    The ideal $\zeros{\idl[C]_4}$ is a proper ideal.
\end{lemma}
%
\arka{I think the best way to show the above is to show that $\orbit{c_3}\cup\orbit{c_4}$ is a \Grb{} by showing that it is stable under Buchberger. We can replace $\A$ with $\D$ if we want order}

%
\begin{lemma}
    Every maximal ideal is of the form $\idl[I]_{\varphi}$ for some function $\varphi : \Indets \to \K$.
\end{lemma}
%
By \Cref{lem:in equiv maximal,rem:lem:in equiv max}, $\idl[C]_4$ is contained in a equivariantly maximal ideal $\idl[J]$.
There exists some maximal ideal $\idl[I]_{\varphi}$ containing $\idl[J]$.
If $\idl[J]$ is also maximal then $\idl[J] = \idl[I]_{\varphi}$ and $\varphi$ is equivariant,
but this contradicts \Cref{lem:C4 no equiv zero}.
%