%!TEX root = ../atomic.sigconf.tex
% LTeX: language=en
%
\section{Old stuff}
\section{Nullstellensatz without Hilbert's basis property}
%
\arka{Merge with the previous section.}
%
\begin{enumerate}
    \item Recall Serge Langs results on Nullstellensatz with infinitely many variables.
    \item Give example of proper ideal which does not have an equivariant zero set.
    \item As a consequence show that it is not contained in any maximal ideal that is equivariant.
    \item As a consequence, the equivariantly maximal ideal that contains it is not maximal.
\end{enumerate}
%
In this section we give an example of an action $\calH\actson\Indets$ which does not satisfy the \kl{equivariant Hilbert basis property} and does not satisfy equivariant weak Nullstellensatz (cf. \Cref{thm:weak null}) and hence also equivariant strong Nullstellensatz as it is a stronger statement.
We also show that there are equivariantly maximal ideals in $\poly{\K}{\Indets}$ that are not maximal. 
%

Let $\calH = \aut{\A}$ and $\Indets = \A^{(2)}$.
The action is given by
$\pi((a,b)) = (\pi(a),\pi(b))$ for $(a,b)\in\A^{(2)}$.
By \Cref{lem:product-not-wqo},
\arka{I think here we also have to cite \cite[Theorem 11]{GHOLAS24}. Probably better to cite in the intro}
$\aut{\A}\actson\A^{(2)}$ does have the \kl{equivariant Hilbert's basis property}.
Pick an enumeration $a_1,a_2,\dots$ of the elements of $\A$.
For $n \geq 3$ define $c_n$ to be the monomial
\[
c_n = (a_1,a_2)(a_2,a_3)\dots (a_{n-1},a_n)(a_n,a_1) \ .
\]
\arka{Maybe give this as an example in one of the previous section?}
For $n\geq 3$, Let $\idl[C]_n$ be the equivariant ideal generated by $\setof{c_m + 1}{m\in\set{3,\dots,n}}$.
%
\arka{Easier example}
\begin{example}
$X = \A^{(2)}$ and $G = \A^{(2)}$.
$\idl[I]$ is generated by
\[
\setof{(a,b)(1 - (a,b)),
(a,b) + (b,a) - 1}{(a,b)\in\A^{(2)}}
\]
\end{example}
%
\begin{lemma}\label{lem:C4 no equiv zero}
    The equivariant variety $\zeros{\idl[C]_4}$ is empty.
\end{lemma}
%
\begin{lemma}\label{lem:C4 proper}
    The ideal $\zeros{\idl[C]_4}$ is a proper ideal.
\end{lemma}
%
\arka{I think the best way to show the above is to show that $\orbit{c_3}\cup\orbit{c_4}$ is a \Grb{} by showing that it is stable under Buchberger. We can replace $\A$ with $\D$ if we want order}

%
\begin{lemma}
    Every maximal ideal is of the form $\idl[I]_{\varphi}$ for some function $\varphi : \Indets \to \K$.
\end{lemma}
%
By \Cref{lem:in equiv maximal,rem:lem:in equiv max}, $\idl[C]_4$ is contained in a equivariantly maximal ideal $\idl[J]$.
There exists some maximal ideal $\idl[I]_{\varphi}$ containing $\idl[J]$.
If $\idl[J]$ is also maximal then $\idl[J] = \idl[I]_{\varphi}$ and $\varphi$ is equivariant,
but this contradicts \Cref{lem:C4 no equiv zero}.
%
\section{Homogeneous intro}
%
In this paper we study ideals in polynomial rings with infinitely many variables.
In Particular, we aim to extend the notions of \Grbs{}, Buchberger's algorithm and Nullstellensatz in this setting.

Let's review these concepts for ideals in polynomial rings with finitely many variables.
The classical Hilbert's basis theorem says that every ideal in a polynomial ring with finitely many variables is finitely generated.
A \Grb{} is a specific kind of generator of an ideal that allows us to decide whether a polynomial is in the ideal generated by it,
using a multivariate version of the division algorithm known as the reduction algorithm.
The Buchberger's algorithm computes a \Grb{} for an ideal from a given set of its generators.
Finally, (weak) Nullstellensatz says that every proper ideal has an algebraic zero.
This is an extension of the fundamental theorem of algebra to the multivariate setting.
Recall that the fundamental theorem of algebra says that every non-constant univariate polynomial has an algebraic root.

The Nullstellensatz implies that a system of polynomial equations (with finitely variables) has a solution if and only if the ideal generated by the corresponding set of polynomials does not contains the unit polynomial $1$.
The latter is true if and only if any(equivalently every) \Grb{} basis of the ideal contains $1$.
Which can be checked by simply computing a \Grb{} using the Buchberger's algorithm.

The above results have interesting applications in automata theory.
For example, the Hilbert's basis theorem is used to prove decidability of zero-ness of polynomial automata.
The reachability problem for reversible Petri nets is solved by reducing it to the ideal membership problem.

To extend the above results to the setting of models of computations with infinite alphabets,
such as register automata and data Petri nets,
we have to study ideals in polynomial rings with infinitely many variables.
In this setting,
we assume $\Indets$ is the domain of a relational structure $(\Indets,R_1,R_2,\dots)$.
We consider ideals that are equivariant,
i.e.\ invariant under the variable-wise action of the group $\aut{\Indets}$ of automorphisms (i.e.\ structure preserving bijections) of $\Indets$.
%
\begin{example}\label{ex:zero sum idl}
Consider the structure of \kl{ordered atoms} $(\D,<)$ where $<$ is a dense-linear order on $\D$ without endpoints.
One can think $\D$ as the set of rationals with the usual ordering.
Let $\idl[Z]_1$ be the ideal of polynomials inside $\poly{\Q}{\D}$ whose coefficients add up to $0$.
Then, $\idl[Z]_1$ is an equivariant ideal.
To explain the equivariance,
for $x,y,z\in\D$ and an automorphism $\pi$ of $\D$ (which in this case is a order preserving bijection of $\D$),
\[
\pi(x^2 - 2yz + z) = \pi(x)^2 - 2\pi(y)\pi(z) + z\ .
\]

For any subset $P$ of $\D$ that is both infinite and co-infinite,
the ideal generated by $x - 2$ for $x\in P$ is not equivariant.
\end{example}
%
Equivariant ideals are not finitely generated in the usual sense.
For example, it is easy to see that the ideal $\idl[Z]_1$ in \Cref{ex:zero sum idl} is generated by the set $\setof{x - 1}{x\in \D}$,
which is not finite.
With a bit of effort one can prove:
%
\begin{lemma}
The ideal $\idl[Z]_1$ is not finitely generated.
\end{lemma}
%
However, for any $x\in\D$,
$\idl[Z]_1$ is the smallest equivariant ideal in $\idl[Z]_1$ containing $x - 1$.
This is because any equivariant ideal $\idl$ in $\poly{\Q}{\D}$ containing $x - 1$ for some $x\in\D$,
must contain $y - 1$ for every $y\in\D$ because it is equivariant.
And thus $\idl$ must contain $\idl[Z]_1$ that is generated by $\poly{\Q}{\D}$. 

However, finite generation of equivariant ideals does not hold in general.
For example, let $\A$ be an infinite set and $(\A^2,=_1,=_2)$ be the structure with domain $\A^2$ and with relations $=_1$ and $=_2$ which respectively check equality in the first and second co-ordinate.
Automorphisms of $\A^2$ are essentially bijections of the set $\A$ acting on $\A^2$ point-wise.
Let $\idl[C]$ be the equivariant ideal generated by polynomials of the form
\[
(a_1,a_2)(a_2,a_3)\cdot\ldots\cdot(a_{n-1},a_n)(a_n,a_1) \ .
\]
%
\begin{lemma}
The ideal $\idl[C]$ is not finitely generated.
\end{lemma}
%
Therefore one is interested in the \kl{equivariant Hilbert's basis property} of $\Indets$:
every equivariant ideal in $\poly{\Q}{\Indets}$ is finitely generated.
This property has been studied independently by \cite{LauSno23} and \cite{GHOLAS24}.
In particular, \cite[Theorem 11]{GHOLAS24} says that if $\Indets$ has the \kl{equivariant Hilbert's basis property},
then it also has the \kl{$\N$-WQO-property}:
$\N$-labelled finite substructures $\Indets$ are well-quasi-ordered under labelled embeddings.
Moreover, \cite[Proposition 3.2]{LauSno23} and \cite[Theorem 12]{GHOLAS24} both say that if $\Indets$ has the \kl{$\N$-WQO-property} and it is linear ordered (i.e.\ its structure contains a binary relation which is a linear order on $\Indets$),
then it has the \kl{equivariant Hilbert's basis property}.
Such structures are called \intro{nicely ordered} in this paper.

\arka{add references to proper sections/definitions and theorems in the paragraph below.}

In this paper, we bridge the gap between the theoretical understanding of the \intro{equivariant Hilbert basis property},
and the computational aspects of \kl{equivariant ideals}.
In particular, we define the notion of equivariant \Grb{}.
\cite[Theorem 12]{GHOLAS24} implies that every equivariant ideal in $\poly{\Q}{\Indets}$ has a finite \Grb{} when $\Indets$ is nicely ordered.
\arka{We can move to orbit-finite notation just by changing the text after \Cref{ex:zero sum idl}}
We also define the equivariant reduction algorithm and show that given a equivariant \Grb{} $\Basis$ we can use the equivariant reduction algorithm to decide whether a polynomial $f$ is in the equivariant ideal generated by $\Basis$.
Then we define the equivariant Buchberger's algorithm and show when $\Indets$ is nicely ordered,
given a finite set $\Basis[C]$ of polynomials as input,
it terminates and outputs a \Grb{} for the ideal generated by $\Basis[C]$.
We compliment these positive results by showing that the membership problem for equivariant ideals in undecidable if $\Indets$ contains an infinite path (\arka{add ref to def}).
This condition is satisfied by many structures $\Indets$ which do not satisfy the \kl{$\N$-WQO-property}.

Then we turn our attention to the \kl{equivariant weak Nullstellensatz}:
every equivariant ideal $\idl[I]\subsetneq \poly{\Q}{\X}$ has an equivariant zero.
We show that the equivariant weak Nullstellensatz holds for $\poly{\Q}{\X}$ if and only if the automorphism group $\aut{\Indets}$ of $\Indets$ is \kl{extremely amenable}.

These results imply the decidability of reachability of Petri nets with nicely ordered data and the 
solvability of the membership problem for equivariant vector subspaces in $\Indets^n$ when $\Indets$ is nicely ordered.
They also imply solvability of orbit-finite systems of polynomial equations in $\poly{\Q}{\Indets}$ when $\Indets$ is both nicely ordered and extremely amenable.

%

%